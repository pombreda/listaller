% Generated by http://pasdoc.sourceforge.net/PasDoc 0.11.0 on 2009-01-29 16:57:19
\documentclass{report}
\usepackage{hyperref}
% WARNING: THIS SHOULD BE MODIFIED DEPENDING ON THE LETTER/A4 SIZE
\oddsidemargin 0cm
\evensidemargin 0cm
\marginparsep 0cm
\marginparwidth 0cm
\parindent 0cm
\setlength{\textwidth}{\paperwidth}
\addtolength{\textwidth}{-2in}


% Conditional define to determine if pdf output is used
\newif\ifpdf
\ifx\pdfoutput\undefined
\pdffalse
\else
\pdfoutput=1
\pdftrue
\fi

\ifpdf
  \usepackage[pdftex]{graphicx}
\else
  \usepackage[dvips]{graphicx}
\fi

% Write Document information for pdflatex/pdftex
\ifpdf
\pdfinfo{
 /Author     (Pasdoc)
 /Title      ()
 /CreationDate (20090129165719)
}
\fi


\begin{document}
\label{toc}\tableofcontents
\newpage
% special variable used for calculating some widths.
\newlength{\tmplength}
\chapter{Unit dgunit}
\label{dgunit}
\index{dgunit}
\section{Description}
This unit contains the stuff neded for installing AutoGET{-}IPK{-}types
\section{uses}
\begin{itemize}
\item \begin{ttfamily}Classes\end{ttfamily}\item \begin{ttfamily}SysUtils\end{ttfamily}\item \begin{ttfamily}LResources\end{ttfamily}\item \begin{ttfamily}Forms\end{ttfamily}\item \begin{ttfamily}Controls\end{ttfamily}\item \begin{ttfamily}Graphics\end{ttfamily}\item \begin{ttfamily}Dialogs\end{ttfamily}\item \begin{ttfamily}StdCtrls\end{ttfamily}\item \begin{ttfamily}Buttons\end{ttfamily}\item \begin{ttfamily}ComCtrls\end{ttfamily}\item \begin{ttfamily}LCLType\end{ttfamily}\item \begin{ttfamily}LCLIntf\end{ttfamily}\item \begin{ttfamily}mainunit\end{ttfamily}(\ref{mainunit})\item \begin{ttfamily}ExtCtrls\end{ttfamily}\item \begin{ttfamily}process\end{ttfamily}\item \begin{ttfamily}HTTPSend\end{ttfamily}(\ref{httpsend})\item \begin{ttfamily}blcksock\end{ttfamily}\item \begin{ttfamily}FTPSend\end{ttfamily}(\ref{ftpsend})\item \begin{ttfamily}IniFiles\end{ttfamily}\item \begin{ttfamily}utilities\end{ttfamily}(\ref{utilities})\item \begin{ttfamily}trstrings\end{ttfamily}(\ref{trstrings})\item \begin{ttfamily}gtk2\end{ttfamily}\item \begin{ttfamily}gtkint\end{ttfamily}\item \begin{ttfamily}gtkdef\end{ttfamily}\item \begin{ttfamily}gtkproc\end{ttfamily}\item \begin{ttfamily}ipkhandle\end{ttfamily}(\ref{ipkhandle})\end{itemize}
\section{Overview}
\begin{description}
\item[\texttt{\begin{ttfamily}TDGForm\end{ttfamily} Class}]
\end{description}
\section{Classes, Interfaces, Objects and Records}
\ifpdf
\subsection*{\large{\textbf{TDGForm Class}}\normalsize\hspace{1ex}\hrulefill}
\else
\subsection*{TDGForm Class}
\fi
\label{dgunit.TDGForm}
\index{TDGForm}
\subsubsection*{\large{\textbf{Hierarchy}}\normalsize\hspace{1ex}\hfill}
TDGForm {$>$} TForm
%%%%Description
\subsubsection*{\large{\textbf{Fields}}\normalsize\hspace{1ex}\hfill}
\begin{list}{}{
\settowidth{\tmplength}{\textbf{GetOutPutTimer}}
\setlength{\itemindent}{0cm}
\setlength{\listparindent}{0cm}
\setlength{\leftmargin}{\evensidemargin}
\addtolength{\leftmargin}{\tmplength}
\settowidth{\labelsep}{X}
\addtolength{\leftmargin}{\labelsep}
\setlength{\labelwidth}{\tmplength}
}
\label{dgunit.TDGForm-BitBtn1}
\index{BitBtn1}
\item[\textbf{BitBtn1}\hfill]
\ifpdf
\begin{flushleft}
\fi
\begin{ttfamily}
public BitBtn1: TBitBtn;\end{ttfamily}

\ifpdf
\end{flushleft}
\fi


\par  \label{dgunit.TDGForm-FinBtn1}
\index{FinBtn1}
\item[\textbf{FinBtn1}\hfill]
\ifpdf
\begin{flushleft}
\fi
\begin{ttfamily}
public FinBtn1: TBitBtn;\end{ttfamily}

\ifpdf
\end{flushleft}
\fi


\par  \label{dgunit.TDGForm-GetOutPutTimer}
\index{GetOutPutTimer}
\item[\textbf{GetOutPutTimer}\hfill]
\ifpdf
\begin{flushleft}
\fi
\begin{ttfamily}
public GetOutPutTimer: TIdleTimer;\end{ttfamily}

\ifpdf
\end{flushleft}
\fi


\par  \label{dgunit.TDGForm-Image1}
\index{Image1}
\item[\textbf{Image1}\hfill]
\ifpdf
\begin{flushleft}
\fi
\begin{ttfamily}
public Image1: TImage;\end{ttfamily}

\ifpdf
\end{flushleft}
\fi


\par  \label{dgunit.TDGForm-Label1}
\index{Label1}
\item[\textbf{Label1}\hfill]
\ifpdf
\begin{flushleft}
\fi
\begin{ttfamily}
public Label1: TLabel;\end{ttfamily}

\ifpdf
\end{flushleft}
\fi


\par  \label{dgunit.TDGForm-Label2}
\index{Label2}
\item[\textbf{Label2}\hfill]
\ifpdf
\begin{flushleft}
\fi
\begin{ttfamily}
public Label2: TLabel;\end{ttfamily}

\ifpdf
\end{flushleft}
\fi


\par  \label{dgunit.TDGForm-Label3}
\index{Label3}
\item[\textbf{Label3}\hfill]
\ifpdf
\begin{flushleft}
\fi
\begin{ttfamily}
public Label3: TLabel;\end{ttfamily}

\ifpdf
\end{flushleft}
\fi


\par  \label{dgunit.TDGForm-Memo1}
\index{Memo1}
\item[\textbf{Memo1}\hfill]
\ifpdf
\begin{flushleft}
\fi
\begin{ttfamily}
public Memo1: TMemo;\end{ttfamily}

\ifpdf
\end{flushleft}
\fi


\par  \label{dgunit.TDGForm-Memo2}
\index{Memo2}
\item[\textbf{Memo2}\hfill]
\ifpdf
\begin{flushleft}
\fi
\begin{ttfamily}
public Memo2: TMemo;\end{ttfamily}

\ifpdf
\end{flushleft}
\fi


\par  \label{dgunit.TDGForm-Memo3}
\index{Memo3}
\item[\textbf{Memo3}\hfill]
\ifpdf
\begin{flushleft}
\fi
\begin{ttfamily}
public Memo3: TMemo;\end{ttfamily}

\ifpdf
\end{flushleft}
\fi


\par  \label{dgunit.TDGForm-PageControl1}
\index{PageControl1}
\item[\textbf{PageControl1}\hfill]
\ifpdf
\begin{flushleft}
\fi
\begin{ttfamily}
public PageControl1: TPageControl;\end{ttfamily}

\ifpdf
\end{flushleft}
\fi


\par  \label{dgunit.TDGForm-DlProgress}
\index{DlProgress}
\item[\textbf{DlProgress}\hfill]
\ifpdf
\begin{flushleft}
\fi
\begin{ttfamily}
public DlProgress: TProgressBar;\end{ttfamily}

\ifpdf
\end{flushleft}
\fi


\par  \label{dgunit.TDGForm-MainProgress}
\index{MainProgress}
\item[\textbf{MainProgress}\hfill]
\ifpdf
\begin{flushleft}
\fi
\begin{ttfamily}
public MainProgress: TProgressBar;\end{ttfamily}

\ifpdf
\end{flushleft}
\fi


\par  \label{dgunit.TDGForm-Process1}
\index{Process1}
\item[\textbf{Process1}\hfill]
\ifpdf
\begin{flushleft}
\fi
\begin{ttfamily}
public Process1: TProcess;\end{ttfamily}

\ifpdf
\end{flushleft}
\fi


\par  \label{dgunit.TDGForm-TabSheet1}
\index{TabSheet1}
\item[\textbf{TabSheet1}\hfill]
\ifpdf
\begin{flushleft}
\fi
\begin{ttfamily}
public TabSheet1: TTabSheet;\end{ttfamily}

\ifpdf
\end{flushleft}
\fi


\par  \label{dgunit.TDGForm-TabSheet2}
\index{TabSheet2}
\item[\textbf{TabSheet2}\hfill]
\ifpdf
\begin{flushleft}
\fi
\begin{ttfamily}
public TabSheet2: TTabSheet;\end{ttfamily}

\ifpdf
\end{flushleft}
\fi


\par  \label{dgunit.TDGForm-TabSheet3}
\index{TabSheet3}
\item[\textbf{TabSheet3}\hfill]
\ifpdf
\begin{flushleft}
\fi
\begin{ttfamily}
public TabSheet3: TTabSheet;\end{ttfamily}

\ifpdf
\end{flushleft}
\fi


\par  \label{dgunit.TDGForm-IIconPath}
\index{IIconPath}
\item[\textbf{IIconPath}\hfill]
\ifpdf
\begin{flushleft}
\fi
\begin{ttfamily}
public IIconPath: String;\end{ttfamily}

\ifpdf
\end{flushleft}
\fi


\par Path to iconfiles\label{dgunit.TDGForm-IDesktopFiles}
\index{IDesktopFiles}
\item[\textbf{IDesktopFiles}\hfill]
\ifpdf
\begin{flushleft}
\fi
\begin{ttfamily}
public IDesktopFiles: String;\end{ttfamily}

\ifpdf
\end{flushleft}
\fi


\par Path to .desktop files\end{list}
\subsubsection*{\large{\textbf{Methods}}\normalsize\hspace{1ex}\hfill}
\paragraph*{BitBtn1Click}\hspace*{\fill}

\label{dgunit.TDGForm-BitBtn1Click}
\index{BitBtn1Click}
\begin{list}{}{
\settowidth{\tmplength}{\textbf{Description}}
\setlength{\itemindent}{0cm}
\setlength{\listparindent}{0cm}
\setlength{\leftmargin}{\evensidemargin}
\addtolength{\leftmargin}{\tmplength}
\settowidth{\labelsep}{X}
\addtolength{\leftmargin}{\labelsep}
\setlength{\labelwidth}{\tmplength}
}
\item[\textbf{Declaration}\hfill]
\ifpdf
\begin{flushleft}
\fi
\begin{ttfamily}
public procedure BitBtn1Click(Sender: TObject);\end{ttfamily}

\ifpdf
\end{flushleft}
\fi

\end{list}
\paragraph*{FinBtn1Click}\hspace*{\fill}

\label{dgunit.TDGForm-FinBtn1Click}
\index{FinBtn1Click}
\begin{list}{}{
\settowidth{\tmplength}{\textbf{Description}}
\setlength{\itemindent}{0cm}
\setlength{\listparindent}{0cm}
\setlength{\leftmargin}{\evensidemargin}
\addtolength{\leftmargin}{\tmplength}
\settowidth{\labelsep}{X}
\addtolength{\leftmargin}{\labelsep}
\setlength{\labelwidth}{\tmplength}
}
\item[\textbf{Declaration}\hfill]
\ifpdf
\begin{flushleft}
\fi
\begin{ttfamily}
public procedure FinBtn1Click(Sender: TObject);\end{ttfamily}

\ifpdf
\end{flushleft}
\fi

\end{list}
\paragraph*{FormCloseQuery}\hspace*{\fill}

\label{dgunit.TDGForm-FormCloseQuery}
\index{FormCloseQuery}
\begin{list}{}{
\settowidth{\tmplength}{\textbf{Description}}
\setlength{\itemindent}{0cm}
\setlength{\listparindent}{0cm}
\setlength{\leftmargin}{\evensidemargin}
\addtolength{\leftmargin}{\tmplength}
\settowidth{\labelsep}{X}
\addtolength{\leftmargin}{\labelsep}
\setlength{\labelwidth}{\tmplength}
}
\item[\textbf{Declaration}\hfill]
\ifpdf
\begin{flushleft}
\fi
\begin{ttfamily}
public procedure FormCloseQuery(Sender: TObject; var CanClose: boolean);\end{ttfamily}

\ifpdf
\end{flushleft}
\fi

\end{list}
\paragraph*{FormCreate}\hspace*{\fill}

\label{dgunit.TDGForm-FormCreate}
\index{FormCreate}
\begin{list}{}{
\settowidth{\tmplength}{\textbf{Description}}
\setlength{\itemindent}{0cm}
\setlength{\listparindent}{0cm}
\setlength{\leftmargin}{\evensidemargin}
\addtolength{\leftmargin}{\tmplength}
\settowidth{\labelsep}{X}
\addtolength{\leftmargin}{\labelsep}
\setlength{\labelwidth}{\tmplength}
}
\item[\textbf{Declaration}\hfill]
\ifpdf
\begin{flushleft}
\fi
\begin{ttfamily}
public procedure FormCreate(Sender: TObject);\end{ttfamily}

\ifpdf
\end{flushleft}
\fi

\end{list}
\paragraph*{FormShow}\hspace*{\fill}

\label{dgunit.TDGForm-FormShow}
\index{FormShow}
\begin{list}{}{
\settowidth{\tmplength}{\textbf{Description}}
\setlength{\itemindent}{0cm}
\setlength{\listparindent}{0cm}
\setlength{\leftmargin}{\evensidemargin}
\addtolength{\leftmargin}{\tmplength}
\settowidth{\labelsep}{X}
\addtolength{\leftmargin}{\labelsep}
\setlength{\labelwidth}{\tmplength}
}
\item[\textbf{Declaration}\hfill]
\ifpdf
\begin{flushleft}
\fi
\begin{ttfamily}
public procedure FormShow(Sender: TObject);\end{ttfamily}

\ifpdf
\end{flushleft}
\fi

\end{list}
\paragraph*{GetOutPutTimerTimer}\hspace*{\fill}

\label{dgunit.TDGForm-GetOutPutTimerTimer}
\index{GetOutPutTimerTimer}
\begin{list}{}{
\settowidth{\tmplength}{\textbf{Description}}
\setlength{\itemindent}{0cm}
\setlength{\listparindent}{0cm}
\setlength{\leftmargin}{\evensidemargin}
\addtolength{\leftmargin}{\tmplength}
\settowidth{\labelsep}{X}
\addtolength{\leftmargin}{\labelsep}
\setlength{\labelwidth}{\tmplength}
}
\item[\textbf{Declaration}\hfill]
\ifpdf
\begin{flushleft}
\fi
\begin{ttfamily}
public procedure GetOutPutTimerTimer(Sender: TObject);\end{ttfamily}

\ifpdf
\end{flushleft}
\fi

\end{list}
\section{Variables}
\ifpdf
\subsection*{\large{\textbf{DGForm}}\normalsize\hspace{1ex}\hrulefill}
\else
\subsection*{DGForm}
\fi
\label{dgunit-DGForm}
\index{DGForm}
\begin{list}{}{
\settowidth{\tmplength}{\textbf{Description}}
\setlength{\itemindent}{0cm}
\setlength{\listparindent}{0cm}
\setlength{\leftmargin}{\evensidemargin}
\addtolength{\leftmargin}{\tmplength}
\settowidth{\labelsep}{X}
\addtolength{\leftmargin}{\labelsep}
\setlength{\labelwidth}{\tmplength}
}
\item[\textbf{Declaration}\hfill]
\ifpdf
\begin{flushleft}
\fi
\begin{ttfamily}
DGForm: TDGForm;\end{ttfamily}

\ifpdf
\end{flushleft}
\fi

\end{list}
\chapter{Unit distri}
\label{distri}
\index{distri}
\section{Description}
Unit to research information about the current distribution (LSB{-}conform)
\section{uses}
\begin{itemize}
\item \begin{ttfamily}SysUtils\end{ttfamily}\item \begin{ttfamily}Classes\end{ttfamily}\item \begin{ttfamily}Process\end{ttfamily}\item \begin{ttfamily}BaseUnix\end{ttfamily}\item \begin{ttfamily}pwd\end{ttfamily}\item \begin{ttfamily}Dialogs\end{ttfamily}\end{itemize}
\section{Overview}
\begin{description}
\item[\texttt{\begin{ttfamily}TDistroInfo\end{ttfamily} record}]
\end{description}
\begin{description}
\item[\texttt{GetDistro}]
\item[\texttt{IsRoot}]
\end{description}
\section{Classes, Interfaces, Objects and Records}
\ifpdf
\subsection*{\large{\textbf{TDistroInfo record}}\normalsize\hspace{1ex}\hrulefill}
\else
\subsection*{TDistroInfo record}
\fi
\label{distri.TDistroInfo}
\index{TDistroInfo}
\subsubsection*{\large{\textbf{Description}}\normalsize\hspace{1ex}\hfill}
Contains information about the current Linux distribution\subsubsection*{\large{\textbf{Fields}}\normalsize\hspace{1ex}\hfill}
\begin{list}{}{
\settowidth{\tmplength}{\textbf{PackageSystem}}
\setlength{\itemindent}{0cm}
\setlength{\listparindent}{0cm}
\setlength{\leftmargin}{\evensidemargin}
\addtolength{\leftmargin}{\tmplength}
\settowidth{\labelsep}{X}
\addtolength{\leftmargin}{\labelsep}
\setlength{\labelwidth}{\tmplength}
}
\label{distri.TDistroInfo-DName}
\index{DName}
\item[\textbf{DName}\hfill]
\ifpdf
\begin{flushleft}
\fi
\begin{ttfamily}
DName: String;\end{ttfamily}

\ifpdf
\end{flushleft}
\fi


\par Name of the distro\label{distri.TDistroInfo-Release}
\index{Release}
\item[\textbf{Release}\hfill]
\ifpdf
\begin{flushleft}
\fi
\begin{ttfamily}
Release: String;\end{ttfamily}

\ifpdf
\end{flushleft}
\fi


\par Release number\label{distri.TDistroInfo-PackageSystem}
\index{PackageSystem}
\item[\textbf{PackageSystem}\hfill]
\ifpdf
\begin{flushleft}
\fi
\begin{ttfamily}
PackageSystem: String;\end{ttfamily}

\ifpdf
\end{flushleft}
\fi


\par Package management system\label{distri.TDistroInfo-InstallCom}
\index{InstallCom}
\item[\textbf{InstallCom}\hfill]
\ifpdf
\begin{flushleft}
\fi
\begin{ttfamily}
InstallCom:   String;\end{ttfamily}

\ifpdf
\end{flushleft}
\fi


\par  \label{distri.TDistroInfo-InstallDBCom}
\index{InstallDBCom}
\item[\textbf{InstallDBCom}\hfill]
\ifpdf
\begin{flushleft}
\fi
\begin{ttfamily}
InstallDBCom: String;\end{ttfamily}

\ifpdf
\end{flushleft}
\fi


\par  \label{distri.TDistroInfo-DoRecheck}
\index{DoRecheck}
\item[\textbf{DoRecheck}\hfill]
\ifpdf
\begin{flushleft}
\fi
\begin{ttfamily}
DoRecheck:    Boolean;\end{ttfamily}

\ifpdf
\end{flushleft}
\fi


\par  \label{distri.TDistroInfo-Desktop}
\index{Desktop}
\item[\textbf{Desktop}\hfill]
\ifpdf
\begin{flushleft}
\fi
\begin{ttfamily}
Desktop: String;\end{ttfamily}

\ifpdf
\end{flushleft}
\fi


\par The desktop environment (KDE/GNOME)\end{list}
\section{Functions and Procedures}
\ifpdf
\subsection*{\large{\textbf{GetDistro}}\normalsize\hspace{1ex}\hrulefill}
\else
\subsection*{GetDistro}
\fi
\label{distri-GetDistro}
\index{GetDistro}
\begin{list}{}{
\settowidth{\tmplength}{\textbf{Description}}
\setlength{\itemindent}{0cm}
\setlength{\listparindent}{0cm}
\setlength{\leftmargin}{\evensidemargin}
\addtolength{\leftmargin}{\tmplength}
\settowidth{\labelsep}{X}
\addtolength{\leftmargin}{\labelsep}
\setlength{\labelwidth}{\tmplength}
}
\item[\textbf{Declaration}\hfill]
\ifpdf
\begin{flushleft}
\fi
\begin{ttfamily}
function GetDistro: TDistroInfo;\end{ttfamily}

\ifpdf
\end{flushleft}
\fi

\par
\item[\textbf{Description}]
Get the distro{-}infos

\end{list}
\ifpdf
\subsection*{\large{\textbf{IsRoot}}\normalsize\hspace{1ex}\hrulefill}
\else
\subsection*{IsRoot}
\fi
\label{distri-IsRoot}
\index{IsRoot}
\begin{list}{}{
\settowidth{\tmplength}{\textbf{Description}}
\setlength{\itemindent}{0cm}
\setlength{\listparindent}{0cm}
\setlength{\leftmargin}{\evensidemargin}
\addtolength{\leftmargin}{\tmplength}
\settowidth{\labelsep}{X}
\addtolength{\leftmargin}{\labelsep}
\setlength{\labelwidth}{\tmplength}
}
\item[\textbf{Declaration}\hfill]
\ifpdf
\begin{flushleft}
\fi
\begin{ttfamily}
function IsRoot: Boolean;\end{ttfamily}

\ifpdf
\end{flushleft}
\fi

\par
\item[\textbf{Description}]
Check if user is root \par
\item[\textbf{Returns}]If user is root (Bool)


\end{list}
\chapter{Unit editor}
\label{editor}
\index{editor}
\section{Description}
IPS source graphical editor
\section{uses}
\begin{itemize}
\item \begin{ttfamily}Classes\end{ttfamily}\item \begin{ttfamily}SysUtils\end{ttfamily}\item \begin{ttfamily}LResources\end{ttfamily}\item \begin{ttfamily}Forms\end{ttfamily}\item \begin{ttfamily}Controls\end{ttfamily}\item \begin{ttfamily}Graphics\end{ttfamily}\item \begin{ttfamily}Dialogs\end{ttfamily}\item \begin{ttfamily}SynEdit\end{ttfamily}\item \begin{ttfamily}ComCtrls\end{ttfamily}\item \begin{ttfamily}Menus\end{ttfamily}\item \begin{ttfamily}StdCtrls\end{ttfamily}\item \begin{ttfamily}fwiz\end{ttfamily}\item \begin{ttfamily}FileUtil\end{ttfamily}\item \begin{ttfamily}SynHighlighterXML\end{ttfamily}\item \begin{ttfamily}ExtCtrls\end{ttfamily}\item \begin{ttfamily}process\end{ttfamily}\item \begin{ttfamily}SynHighlighterTeX\end{ttfamily}\item \begin{ttfamily}SynHighlighterAny\end{ttfamily}\item \begin{ttfamily}MD5\end{ttfamily}\end{itemize}
\section{Overview}
\begin{description}
\item[\texttt{\begin{ttfamily}TfrmEditor\end{ttfamily} Class}]
\end{description}
\section{Classes, Interfaces, Objects and Records}
\ifpdf
\subsection*{\large{\textbf{TfrmEditor Class}}\normalsize\hspace{1ex}\hrulefill}
\else
\subsection*{TfrmEditor Class}
\fi
\label{editor.TfrmEditor}
\index{TfrmEditor}
\subsubsection*{\large{\textbf{Hierarchy}}\normalsize\hspace{1ex}\hfill}
TfrmEditor {$>$} TForm
%%%%Description
\subsubsection*{\large{\textbf{Fields}}\normalsize\hspace{1ex}\hfill}
\begin{list}{}{
\settowidth{\tmplength}{\textbf{mnuBuildCreatePackage}}
\setlength{\itemindent}{0cm}
\setlength{\listparindent}{0cm}
\setlength{\leftmargin}{\evensidemargin}
\addtolength{\leftmargin}{\tmplength}
\settowidth{\labelsep}{X}
\addtolength{\leftmargin}{\labelsep}
\setlength{\labelwidth}{\tmplength}
}
\label{editor.TfrmEditor-mmMain}
\index{mmMain}
\item[\textbf{mmMain}\hfill]
\ifpdf
\begin{flushleft}
\fi
\begin{ttfamily}
public mmMain: TMainMenu;\end{ttfamily}

\ifpdf
\end{flushleft}
\fi


\par  \label{editor.TfrmEditor-MainScriptEdit}
\index{MainScriptEdit}
\item[\textbf{MainScriptEdit}\hfill]
\ifpdf
\begin{flushleft}
\fi
\begin{ttfamily}
public MainScriptEdit: TSynEdit;\end{ttfamily}

\ifpdf
\end{flushleft}
\fi


\par  \label{editor.TfrmEditor-FilesEdit}
\index{FilesEdit}
\item[\textbf{FilesEdit}\hfill]
\ifpdf
\begin{flushleft}
\fi
\begin{ttfamily}
public FilesEdit: TSynEdit;\end{ttfamily}

\ifpdf
\end{flushleft}
\fi


\par  \label{editor.TfrmEditor-memLog}
\index{memLog}
\item[\textbf{memLog}\hfill]
\ifpdf
\begin{flushleft}
\fi
\begin{ttfamily}
public memLog: TMemo;\end{ttfamily}

\ifpdf
\end{flushleft}
\fi


\par  \label{editor.TfrmEditor-mnuFile}
\index{mnuFile}
\item[\textbf{mnuFile}\hfill]
\ifpdf
\begin{flushleft}
\fi
\begin{ttfamily}
public mnuFile: TMenuItem;\end{ttfamily}

\ifpdf
\end{flushleft}
\fi


\par  \label{editor.TfrmEditor-mnuEditPaste}
\index{mnuEditPaste}
\item[\textbf{mnuEditPaste}\hfill]
\ifpdf
\begin{flushleft}
\fi
\begin{ttfamily}
public mnuEditPaste: TMenuItem;\end{ttfamily}

\ifpdf
\end{flushleft}
\fi


\par  \label{editor.TfrmEditor-mnuBuild}
\index{mnuBuild}
\item[\textbf{mnuBuild}\hfill]
\ifpdf
\begin{flushleft}
\fi
\begin{ttfamily}
public mnuBuild: TMenuItem;\end{ttfamily}

\ifpdf
\end{flushleft}
\fi


\par  \label{editor.TfrmEditor-mnuBuildCreatePackage}
\index{mnuBuildCreatePackage}
\item[\textbf{mnuBuildCreatePackage}\hfill]
\ifpdf
\begin{flushleft}
\fi
\begin{ttfamily}
public mnuBuildCreatePackage: TMenuItem;\end{ttfamily}

\ifpdf
\end{flushleft}
\fi


\par  \label{editor.TfrmEditor-mnuEditAddFilePath}
\index{mnuEditAddFilePath}
\item[\textbf{mnuEditAddFilePath}\hfill]
\ifpdf
\begin{flushleft}
\fi
\begin{ttfamily}
public mnuEditAddFilePath: TMenuItem;\end{ttfamily}

\ifpdf
\end{flushleft}
\fi


\par  \label{editor.TfrmEditor-mnuEditUndo}
\index{mnuEditUndo}
\item[\textbf{mnuEditUndo}\hfill]
\ifpdf
\begin{flushleft}
\fi
\begin{ttfamily}
public mnuEditUndo: TMenuItem;\end{ttfamily}

\ifpdf
\end{flushleft}
\fi


\par  \label{editor.TfrmEditor-mnuFileSave}
\index{mnuFileSave}
\item[\textbf{mnuFileSave}\hfill]
\ifpdf
\begin{flushleft}
\fi
\begin{ttfamily}
public mnuFileSave: TMenuItem;\end{ttfamily}

\ifpdf
\end{flushleft}
\fi


\par  \label{editor.TfrmEditor-mnuFileNewWizard}
\index{mnuFileNewWizard}
\item[\textbf{mnuFileNewWizard}\hfill]
\ifpdf
\begin{flushleft}
\fi
\begin{ttfamily}
public mnuFileNewWizard: TMenuItem;\end{ttfamily}

\ifpdf
\end{flushleft}
\fi


\par  \label{editor.TfrmEditor-mnuFileNewBlank}
\index{mnuFileNewBlank}
\item[\textbf{mnuFileNewBlank}\hfill]
\ifpdf
\begin{flushleft}
\fi
\begin{ttfamily}
public mnuFileNewBlank: TMenuItem;\end{ttfamily}

\ifpdf
\end{flushleft}
\fi


\par  \label{editor.TfrmEditor-mnuBuildFast}
\index{mnuBuildFast}
\item[\textbf{mnuBuildFast}\hfill]
\ifpdf
\begin{flushleft}
\fi
\begin{ttfamily}
public mnuBuildFast: TMenuItem;\end{ttfamily}

\ifpdf
\end{flushleft}
\fi


\par  \label{editor.TfrmEditor-mnuEdit}
\index{mnuEdit}
\item[\textbf{mnuEdit}\hfill]
\ifpdf
\begin{flushleft}
\fi
\begin{ttfamily}
public mnuEdit: TMenuItem;\end{ttfamily}

\ifpdf
\end{flushleft}
\fi


\par  \label{editor.TfrmEditor-mnuFileSaveAs}
\index{mnuFileSaveAs}
\item[\textbf{mnuFileSaveAs}\hfill]
\ifpdf
\begin{flushleft}
\fi
\begin{ttfamily}
public mnuFileSaveAs: TMenuItem;\end{ttfamily}

\ifpdf
\end{flushleft}
\fi


\par  \label{editor.TfrmEditor-mnuFileLoadIPS}
\index{mnuFileLoadIPS}
\item[\textbf{mnuFileLoadIPS}\hfill]
\ifpdf
\begin{flushleft}
\fi
\begin{ttfamily}
public mnuFileLoadIPS: TMenuItem;\end{ttfamily}

\ifpdf
\end{flushleft}
\fi


\par  \label{editor.TfrmEditor-mnuFileClose}
\index{mnuFileClose}
\item[\textbf{mnuFileClose}\hfill]
\ifpdf
\begin{flushleft}
\fi
\begin{ttfamily}
public mnuFileClose: TMenuItem;\end{ttfamily}

\ifpdf
\end{flushleft}
\fi


\par  \label{editor.TfrmEditor-mnuEditFileWizard}
\index{mnuEditFileWizard}
\item[\textbf{mnuEditFileWizard}\hfill]
\ifpdf
\begin{flushleft}
\fi
\begin{ttfamily}
public mnuEditFileWizard: TMenuItem;\end{ttfamily}

\ifpdf
\end{flushleft}
\fi


\par  \label{editor.TfrmEditor-mnuEditCut}
\index{mnuEditCut}
\item[\textbf{mnuEditCut}\hfill]
\ifpdf
\begin{flushleft}
\fi
\begin{ttfamily}
public mnuEditCut: TMenuItem;\end{ttfamily}

\ifpdf
\end{flushleft}
\fi


\par  \label{editor.TfrmEditor-mnuEditCopy}
\index{mnuEditCopy}
\item[\textbf{mnuEditCopy}\hfill]
\ifpdf
\begin{flushleft}
\fi
\begin{ttfamily}
public mnuEditCopy: TMenuItem;\end{ttfamily}

\ifpdf
\end{flushleft}
\fi


\par  \label{editor.TfrmEditor-mnuEditFind}
\index{mnuEditFind}
\item[\textbf{mnuEditFind}\hfill]
\ifpdf
\begin{flushleft}
\fi
\begin{ttfamily}
public mnuEditFind: TMenuItem;\end{ttfamily}

\ifpdf
\end{flushleft}
\fi


\par  \label{editor.TfrmEditor-OpenDialog1}
\index{OpenDialog1}
\item[\textbf{OpenDialog1}\hfill]
\ifpdf
\begin{flushleft}
\fi
\begin{ttfamily}
public OpenDialog1: TOpenDialog;\end{ttfamily}

\ifpdf
\end{flushleft}
\fi


\par  \label{editor.TfrmEditor-OpenDialog2}
\index{OpenDialog2}
\item[\textbf{OpenDialog2}\hfill]
\ifpdf
\begin{flushleft}
\fi
\begin{ttfamily}
public OpenDialog2: TOpenDialog;\end{ttfamily}

\ifpdf
\end{flushleft}
\fi


\par  \label{editor.TfrmEditor-PageControl1}
\index{PageControl1}
\item[\textbf{PageControl1}\hfill]
\ifpdf
\begin{flushleft}
\fi
\begin{ttfamily}
public PageControl1: TPageControl;\end{ttfamily}

\ifpdf
\end{flushleft}
\fi


\par  \label{editor.TfrmEditor-Process1}
\index{Process1}
\item[\textbf{Process1}\hfill]
\ifpdf
\begin{flushleft}
\fi
\begin{ttfamily}
public Process1: TProcess;\end{ttfamily}

\ifpdf
\end{flushleft}
\fi


\par  \label{editor.TfrmEditor-SaveDialog1}
\index{SaveDialog1}
\item[\textbf{SaveDialog1}\hfill]
\ifpdf
\begin{flushleft}
\fi
\begin{ttfamily}
public SaveDialog1: TSaveDialog;\end{ttfamily}

\ifpdf
\end{flushleft}
\fi


\par  \label{editor.TfrmEditor-SaveDialog2}
\index{SaveDialog2}
\item[\textbf{SaveDialog2}\hfill]
\ifpdf
\begin{flushleft}
\fi
\begin{ttfamily}
public SaveDialog2: TSaveDialog;\end{ttfamily}

\ifpdf
\end{flushleft}
\fi


\par  \label{editor.TfrmEditor-SynAnySyn1}
\index{SynAnySyn1}
\item[\textbf{SynAnySyn1}\hfill]
\ifpdf
\begin{flushleft}
\fi
\begin{ttfamily}
public SynAnySyn1: TSynAnySyn;\end{ttfamily}

\ifpdf
\end{flushleft}
\fi


\par  \label{editor.TfrmEditor-SynEdit1}
\index{SynEdit1}
\item[\textbf{SynEdit1}\hfill]
\ifpdf
\begin{flushleft}
\fi
\begin{ttfamily}
public SynEdit1: TSynEdit;\end{ttfamily}

\ifpdf
\end{flushleft}
\fi


\par  \label{editor.TfrmEditor-SynTeXSyn1}
\index{SynTeXSyn1}
\item[\textbf{SynTeXSyn1}\hfill]
\ifpdf
\begin{flushleft}
\fi
\begin{ttfamily}
public SynTeXSyn1: TSynTeXSyn;\end{ttfamily}

\ifpdf
\end{flushleft}
\fi


\par  \label{editor.TfrmEditor-SynXMLSyn1}
\index{SynXMLSyn1}
\item[\textbf{SynXMLSyn1}\hfill]
\ifpdf
\begin{flushleft}
\fi
\begin{ttfamily}
public SynXMLSyn1: TSynXMLSyn;\end{ttfamily}

\ifpdf
\end{flushleft}
\fi


\par  \label{editor.TfrmEditor-TabSheet1}
\index{TabSheet1}
\item[\textbf{TabSheet1}\hfill]
\ifpdf
\begin{flushleft}
\fi
\begin{ttfamily}
public TabSheet1: TTabSheet;\end{ttfamily}

\ifpdf
\end{flushleft}
\fi


\par  \label{editor.TfrmEditor-TabSheet2}
\index{TabSheet2}
\item[\textbf{TabSheet2}\hfill]
\ifpdf
\begin{flushleft}
\fi
\begin{ttfamily}
public TabSheet2: TTabSheet;\end{ttfamily}

\ifpdf
\end{flushleft}
\fi


\par  \label{editor.TfrmEditor-FileInfo}
\index{FileInfo}
\item[\textbf{FileInfo}\hfill]
\ifpdf
\begin{flushleft}
\fi
\begin{ttfamily}
public FileInfo: TStringList;\end{ttfamily}

\ifpdf
\end{flushleft}
\fi


\par  \end{list}
\subsubsection*{\large{\textbf{Methods}}\normalsize\hspace{1ex}\hfill}
\paragraph*{Button1Click}\hspace*{\fill}

\label{editor.TfrmEditor-Button1Click}
\index{Button1Click}
\begin{list}{}{
\settowidth{\tmplength}{\textbf{Description}}
\setlength{\itemindent}{0cm}
\setlength{\listparindent}{0cm}
\setlength{\leftmargin}{\evensidemargin}
\addtolength{\leftmargin}{\tmplength}
\settowidth{\labelsep}{X}
\addtolength{\leftmargin}{\labelsep}
\setlength{\labelwidth}{\tmplength}
}
\item[\textbf{Declaration}\hfill]
\ifpdf
\begin{flushleft}
\fi
\begin{ttfamily}
public procedure Button1Click(Sender: TObject);\end{ttfamily}

\ifpdf
\end{flushleft}
\fi

\end{list}
\paragraph*{FormActivate}\hspace*{\fill}

\label{editor.TfrmEditor-FormActivate}
\index{FormActivate}
\begin{list}{}{
\settowidth{\tmplength}{\textbf{Description}}
\setlength{\itemindent}{0cm}
\setlength{\listparindent}{0cm}
\setlength{\leftmargin}{\evensidemargin}
\addtolength{\leftmargin}{\tmplength}
\settowidth{\labelsep}{X}
\addtolength{\leftmargin}{\labelsep}
\setlength{\labelwidth}{\tmplength}
}
\item[\textbf{Declaration}\hfill]
\ifpdf
\begin{flushleft}
\fi
\begin{ttfamily}
public procedure FormActivate(Sender: TObject);\end{ttfamily}

\ifpdf
\end{flushleft}
\fi

\end{list}
\paragraph*{FormCreate}\hspace*{\fill}

\label{editor.TfrmEditor-FormCreate}
\index{FormCreate}
\begin{list}{}{
\settowidth{\tmplength}{\textbf{Description}}
\setlength{\itemindent}{0cm}
\setlength{\listparindent}{0cm}
\setlength{\leftmargin}{\evensidemargin}
\addtolength{\leftmargin}{\tmplength}
\settowidth{\labelsep}{X}
\addtolength{\leftmargin}{\labelsep}
\setlength{\labelwidth}{\tmplength}
}
\item[\textbf{Declaration}\hfill]
\ifpdf
\begin{flushleft}
\fi
\begin{ttfamily}
public procedure FormCreate(Sender: TObject);\end{ttfamily}

\ifpdf
\end{flushleft}
\fi

\end{list}
\paragraph*{FormShow}\hspace*{\fill}

\label{editor.TfrmEditor-FormShow}
\index{FormShow}
\begin{list}{}{
\settowidth{\tmplength}{\textbf{Description}}
\setlength{\itemindent}{0cm}
\setlength{\listparindent}{0cm}
\setlength{\leftmargin}{\evensidemargin}
\addtolength{\leftmargin}{\tmplength}
\settowidth{\labelsep}{X}
\addtolength{\leftmargin}{\labelsep}
\setlength{\labelwidth}{\tmplength}
}
\item[\textbf{Declaration}\hfill]
\ifpdf
\begin{flushleft}
\fi
\begin{ttfamily}
public procedure FormShow(Sender: TObject);\end{ttfamily}

\ifpdf
\end{flushleft}
\fi

\end{list}
\paragraph*{IdleTimer1Timer}\hspace*{\fill}

\label{editor.TfrmEditor-IdleTimer1Timer}
\index{IdleTimer1Timer}
\begin{list}{}{
\settowidth{\tmplength}{\textbf{Description}}
\setlength{\itemindent}{0cm}
\setlength{\listparindent}{0cm}
\setlength{\leftmargin}{\evensidemargin}
\addtolength{\leftmargin}{\tmplength}
\settowidth{\labelsep}{X}
\addtolength{\leftmargin}{\labelsep}
\setlength{\labelwidth}{\tmplength}
}
\item[\textbf{Declaration}\hfill]
\ifpdf
\begin{flushleft}
\fi
\begin{ttfamily}
public procedure IdleTimer1Timer(Sender: TObject);\end{ttfamily}

\ifpdf
\end{flushleft}
\fi

\end{list}
\paragraph*{mnuBuildCreatePackageClick}\hspace*{\fill}

\label{editor.TfrmEditor-mnuBuildCreatePackageClick}
\index{mnuBuildCreatePackageClick}
\begin{list}{}{
\settowidth{\tmplength}{\textbf{Description}}
\setlength{\itemindent}{0cm}
\setlength{\listparindent}{0cm}
\setlength{\leftmargin}{\evensidemargin}
\addtolength{\leftmargin}{\tmplength}
\settowidth{\labelsep}{X}
\addtolength{\leftmargin}{\labelsep}
\setlength{\labelwidth}{\tmplength}
}
\item[\textbf{Declaration}\hfill]
\ifpdf
\begin{flushleft}
\fi
\begin{ttfamily}
public procedure mnuBuildCreatePackageClick(Sender: TObject);\end{ttfamily}

\ifpdf
\end{flushleft}
\fi

\end{list}
\paragraph*{mnuEditAddFilePathClick}\hspace*{\fill}

\label{editor.TfrmEditor-mnuEditAddFilePathClick}
\index{mnuEditAddFilePathClick}
\begin{list}{}{
\settowidth{\tmplength}{\textbf{Description}}
\setlength{\itemindent}{0cm}
\setlength{\listparindent}{0cm}
\setlength{\leftmargin}{\evensidemargin}
\addtolength{\leftmargin}{\tmplength}
\settowidth{\labelsep}{X}
\addtolength{\leftmargin}{\labelsep}
\setlength{\labelwidth}{\tmplength}
}
\item[\textbf{Declaration}\hfill]
\ifpdf
\begin{flushleft}
\fi
\begin{ttfamily}
public procedure mnuEditAddFilePathClick(Sender: TObject);\end{ttfamily}

\ifpdf
\end{flushleft}
\fi

\end{list}
\paragraph*{mnuEditUndoClick}\hspace*{\fill}

\label{editor.TfrmEditor-mnuEditUndoClick}
\index{mnuEditUndoClick}
\begin{list}{}{
\settowidth{\tmplength}{\textbf{Description}}
\setlength{\itemindent}{0cm}
\setlength{\listparindent}{0cm}
\setlength{\leftmargin}{\evensidemargin}
\addtolength{\leftmargin}{\tmplength}
\settowidth{\labelsep}{X}
\addtolength{\leftmargin}{\labelsep}
\setlength{\labelwidth}{\tmplength}
}
\item[\textbf{Declaration}\hfill]
\ifpdf
\begin{flushleft}
\fi
\begin{ttfamily}
public procedure mnuEditUndoClick(Sender: TObject);\end{ttfamily}

\ifpdf
\end{flushleft}
\fi

\end{list}
\paragraph*{mnuFileSaveClick}\hspace*{\fill}

\label{editor.TfrmEditor-mnuFileSaveClick}
\index{mnuFileSaveClick}
\begin{list}{}{
\settowidth{\tmplength}{\textbf{Description}}
\setlength{\itemindent}{0cm}
\setlength{\listparindent}{0cm}
\setlength{\leftmargin}{\evensidemargin}
\addtolength{\leftmargin}{\tmplength}
\settowidth{\labelsep}{X}
\addtolength{\leftmargin}{\labelsep}
\setlength{\labelwidth}{\tmplength}
}
\item[\textbf{Declaration}\hfill]
\ifpdf
\begin{flushleft}
\fi
\begin{ttfamily}
public procedure mnuFileSaveClick(Sender: TObject);\end{ttfamily}

\ifpdf
\end{flushleft}
\fi

\end{list}
\paragraph*{mnuFileNewWizardClick}\hspace*{\fill}

\label{editor.TfrmEditor-mnuFileNewWizardClick}
\index{mnuFileNewWizardClick}
\begin{list}{}{
\settowidth{\tmplength}{\textbf{Description}}
\setlength{\itemindent}{0cm}
\setlength{\listparindent}{0cm}
\setlength{\leftmargin}{\evensidemargin}
\addtolength{\leftmargin}{\tmplength}
\settowidth{\labelsep}{X}
\addtolength{\leftmargin}{\labelsep}
\setlength{\labelwidth}{\tmplength}
}
\item[\textbf{Declaration}\hfill]
\ifpdf
\begin{flushleft}
\fi
\begin{ttfamily}
public procedure mnuFileNewWizardClick(Sender: TObject);\end{ttfamily}

\ifpdf
\end{flushleft}
\fi

\end{list}
\paragraph*{mnuFileNewBlankClick}\hspace*{\fill}

\label{editor.TfrmEditor-mnuFileNewBlankClick}
\index{mnuFileNewBlankClick}
\begin{list}{}{
\settowidth{\tmplength}{\textbf{Description}}
\setlength{\itemindent}{0cm}
\setlength{\listparindent}{0cm}
\setlength{\leftmargin}{\evensidemargin}
\addtolength{\leftmargin}{\tmplength}
\settowidth{\labelsep}{X}
\addtolength{\leftmargin}{\labelsep}
\setlength{\labelwidth}{\tmplength}
}
\item[\textbf{Declaration}\hfill]
\ifpdf
\begin{flushleft}
\fi
\begin{ttfamily}
public procedure mnuFileNewBlankClick(Sender: TObject);\end{ttfamily}

\ifpdf
\end{flushleft}
\fi

\end{list}
\paragraph*{mnuBuildFastClick}\hspace*{\fill}

\label{editor.TfrmEditor-mnuBuildFastClick}
\index{mnuBuildFastClick}
\begin{list}{}{
\settowidth{\tmplength}{\textbf{Description}}
\setlength{\itemindent}{0cm}
\setlength{\listparindent}{0cm}
\setlength{\leftmargin}{\evensidemargin}
\addtolength{\leftmargin}{\tmplength}
\settowidth{\labelsep}{X}
\addtolength{\leftmargin}{\labelsep}
\setlength{\labelwidth}{\tmplength}
}
\item[\textbf{Declaration}\hfill]
\ifpdf
\begin{flushleft}
\fi
\begin{ttfamily}
public procedure mnuBuildFastClick(Sender: TObject);\end{ttfamily}

\ifpdf
\end{flushleft}
\fi

\end{list}
\paragraph*{mnuFileSaveAsClick}\hspace*{\fill}

\label{editor.TfrmEditor-mnuFileSaveAsClick}
\index{mnuFileSaveAsClick}
\begin{list}{}{
\settowidth{\tmplength}{\textbf{Description}}
\setlength{\itemindent}{0cm}
\setlength{\listparindent}{0cm}
\setlength{\leftmargin}{\evensidemargin}
\addtolength{\leftmargin}{\tmplength}
\settowidth{\labelsep}{X}
\addtolength{\leftmargin}{\labelsep}
\setlength{\labelwidth}{\tmplength}
}
\item[\textbf{Declaration}\hfill]
\ifpdf
\begin{flushleft}
\fi
\begin{ttfamily}
public procedure mnuFileSaveAsClick(Sender: TObject);\end{ttfamily}

\ifpdf
\end{flushleft}
\fi

\end{list}
\paragraph*{mnuFileLoadIPSClick}\hspace*{\fill}

\label{editor.TfrmEditor-mnuFileLoadIPSClick}
\index{mnuFileLoadIPSClick}
\begin{list}{}{
\settowidth{\tmplength}{\textbf{Description}}
\setlength{\itemindent}{0cm}
\setlength{\listparindent}{0cm}
\setlength{\leftmargin}{\evensidemargin}
\addtolength{\leftmargin}{\tmplength}
\settowidth{\labelsep}{X}
\addtolength{\leftmargin}{\labelsep}
\setlength{\labelwidth}{\tmplength}
}
\item[\textbf{Declaration}\hfill]
\ifpdf
\begin{flushleft}
\fi
\begin{ttfamily}
public procedure mnuFileLoadIPSClick(Sender: TObject);\end{ttfamily}

\ifpdf
\end{flushleft}
\fi

\end{list}
\paragraph*{mnuFileCloseClick}\hspace*{\fill}

\label{editor.TfrmEditor-mnuFileCloseClick}
\index{mnuFileCloseClick}
\begin{list}{}{
\settowidth{\tmplength}{\textbf{Description}}
\setlength{\itemindent}{0cm}
\setlength{\listparindent}{0cm}
\setlength{\leftmargin}{\evensidemargin}
\addtolength{\leftmargin}{\tmplength}
\settowidth{\labelsep}{X}
\addtolength{\leftmargin}{\labelsep}
\setlength{\labelwidth}{\tmplength}
}
\item[\textbf{Declaration}\hfill]
\ifpdf
\begin{flushleft}
\fi
\begin{ttfamily}
public procedure mnuFileCloseClick(Sender: TObject);\end{ttfamily}

\ifpdf
\end{flushleft}
\fi

\end{list}
\paragraph*{mnuEditFileWizardClick}\hspace*{\fill}

\label{editor.TfrmEditor-mnuEditFileWizardClick}
\index{mnuEditFileWizardClick}
\begin{list}{}{
\settowidth{\tmplength}{\textbf{Description}}
\setlength{\itemindent}{0cm}
\setlength{\listparindent}{0cm}
\setlength{\leftmargin}{\evensidemargin}
\addtolength{\leftmargin}{\tmplength}
\settowidth{\labelsep}{X}
\addtolength{\leftmargin}{\labelsep}
\setlength{\labelwidth}{\tmplength}
}
\item[\textbf{Declaration}\hfill]
\ifpdf
\begin{flushleft}
\fi
\begin{ttfamily}
public procedure mnuEditFileWizardClick(Sender: TObject);\end{ttfamily}

\ifpdf
\end{flushleft}
\fi

\end{list}
\paragraph*{TabSheet1Show}\hspace*{\fill}

\label{editor.TfrmEditor-TabSheet1Show}
\index{TabSheet1Show}
\begin{list}{}{
\settowidth{\tmplength}{\textbf{Description}}
\setlength{\itemindent}{0cm}
\setlength{\listparindent}{0cm}
\setlength{\leftmargin}{\evensidemargin}
\addtolength{\leftmargin}{\tmplength}
\settowidth{\labelsep}{X}
\addtolength{\leftmargin}{\labelsep}
\setlength{\labelwidth}{\tmplength}
}
\item[\textbf{Declaration}\hfill]
\ifpdf
\begin{flushleft}
\fi
\begin{ttfamily}
public procedure TabSheet1Show(Sender: TObject);\end{ttfamily}

\ifpdf
\end{flushleft}
\fi

\end{list}
\paragraph*{TabSheet2Show}\hspace*{\fill}

\label{editor.TfrmEditor-TabSheet2Show}
\index{TabSheet2Show}
\begin{list}{}{
\settowidth{\tmplength}{\textbf{Description}}
\setlength{\itemindent}{0cm}
\setlength{\listparindent}{0cm}
\setlength{\leftmargin}{\evensidemargin}
\addtolength{\leftmargin}{\tmplength}
\settowidth{\labelsep}{X}
\addtolength{\leftmargin}{\labelsep}
\setlength{\labelwidth}{\tmplength}
}
\item[\textbf{Declaration}\hfill]
\ifpdf
\begin{flushleft}
\fi
\begin{ttfamily}
public procedure TabSheet2Show(Sender: TObject);\end{ttfamily}

\ifpdf
\end{flushleft}
\fi

\end{list}
\paragraph*{TabSheet3Show}\hspace*{\fill}

\label{editor.TfrmEditor-TabSheet3Show}
\index{TabSheet3Show}
\begin{list}{}{
\settowidth{\tmplength}{\textbf{Description}}
\setlength{\itemindent}{0cm}
\setlength{\listparindent}{0cm}
\setlength{\leftmargin}{\evensidemargin}
\addtolength{\leftmargin}{\tmplength}
\settowidth{\labelsep}{X}
\addtolength{\leftmargin}{\labelsep}
\setlength{\labelwidth}{\tmplength}
}
\item[\textbf{Declaration}\hfill]
\ifpdf
\begin{flushleft}
\fi
\begin{ttfamily}
public procedure TabSheet3Show(Sender: TObject);\end{ttfamily}

\ifpdf
\end{flushleft}
\fi

\end{list}
\section{Constants}
\ifpdf
\subsection*{\large{\textbf{READ{\_}BYTES}}\normalsize\hspace{1ex}\hrulefill}
\else
\subsection*{READ{\_}BYTES}
\fi
\label{editor-READ_BYTES}
\index{READ{\_}BYTES}
\begin{list}{}{
\settowidth{\tmplength}{\textbf{Description}}
\setlength{\itemindent}{0cm}
\setlength{\listparindent}{0cm}
\setlength{\leftmargin}{\evensidemargin}
\addtolength{\leftmargin}{\tmplength}
\settowidth{\labelsep}{X}
\addtolength{\leftmargin}{\labelsep}
\setlength{\labelwidth}{\tmplength}
}
\item[\textbf{Declaration}\hfill]
\ifpdf
\begin{flushleft}
\fi
\begin{ttfamily}
READ{\_}BYTES = 2048;\end{ttfamily}

\ifpdf
\end{flushleft}
\fi

\par
\item[\textbf{Description}]
Size of the Linux output pipe

\end{list}
\section{Variables}
\ifpdf
\subsection*{\large{\textbf{frmEditor}}\normalsize\hspace{1ex}\hrulefill}
\else
\subsection*{frmEditor}
\fi
\label{editor-frmEditor}
\index{frmEditor}
\begin{list}{}{
\settowidth{\tmplength}{\textbf{Description}}
\setlength{\itemindent}{0cm}
\setlength{\listparindent}{0cm}
\setlength{\leftmargin}{\evensidemargin}
\addtolength{\leftmargin}{\tmplength}
\settowidth{\labelsep}{X}
\addtolength{\leftmargin}{\labelsep}
\setlength{\labelwidth}{\tmplength}
}
\item[\textbf{Declaration}\hfill]
\ifpdf
\begin{flushleft}
\fi
\begin{ttfamily}
frmEditor: TfrmEditor;\end{ttfamily}

\ifpdf
\end{flushleft}
\fi

\end{list}
\ifpdf
\subsection*{\large{\textbf{FName}}\normalsize\hspace{1ex}\hrulefill}
\else
\subsection*{FName}
\fi
\label{editor-FName}
\index{FName}
\begin{list}{}{
\settowidth{\tmplength}{\textbf{Description}}
\setlength{\itemindent}{0cm}
\setlength{\listparindent}{0cm}
\setlength{\leftmargin}{\evensidemargin}
\addtolength{\leftmargin}{\tmplength}
\settowidth{\labelsep}{X}
\addtolength{\leftmargin}{\labelsep}
\setlength{\labelwidth}{\tmplength}
}
\item[\textbf{Declaration}\hfill]
\ifpdf
\begin{flushleft}
\fi
\begin{ttfamily}
FName: String;\end{ttfamily}

\ifpdf
\end{flushleft}
\fi

\par
\item[\textbf{Description}]
Name of the current file

\end{list}
\chapter{Unit ftpsend}
\label{ftpsend}
\index{ftpsend}
\section{Description}
FTP client protocol\hfill\vspace*{1ex}



Used RFC: RFC{-}959, RFC{-}2228, RFC{-}2428
\section{uses}
\begin{itemize}
\item \begin{ttfamily}SysUtils\end{ttfamily}\item \begin{ttfamily}Classes\end{ttfamily}\item \begin{ttfamily}blcksock\end{ttfamily}\item \begin{ttfamily}synautil\end{ttfamily}\item \begin{ttfamily}synaip\end{ttfamily}\item \begin{ttfamily}synsock\end{ttfamily}\end{itemize}
\section{Overview}
\begin{description}
\item[\texttt{\begin{ttfamily}TFTPListRec\end{ttfamily} Class}]Object for holding file information
\item[\texttt{\begin{ttfamily}TFTPList\end{ttfamily} Class}]This is TList of TFTPListRec objects.
\item[\texttt{\begin{ttfamily}TFTPSend\end{ttfamily} Class}]Implementation of FTP protocol.
\end{description}
\begin{description}
\item[\texttt{FtpGetFile}]
\item[\texttt{FtpPutFile}]
\item[\texttt{FtpInterServerTransfer}]
\end{description}
\section{Classes, Interfaces, Objects and Records}
\ifpdf
\subsection*{\large{\textbf{TFTPListRec Class}}\normalsize\hspace{1ex}\hrulefill}
\else
\subsection*{TFTPListRec Class}
\fi
\label{ftpsend.TFTPListRec}
\index{TFTPListRec}
\subsubsection*{\large{\textbf{Hierarchy}}\normalsize\hspace{1ex}\hfill}
TFTPListRec {$>$} TObject
\subsubsection*{\large{\textbf{Description}}\normalsize\hspace{1ex}\hfill}
Object for holding file information\hfill\vspace*{1ex}

 parsed from directory listing of FTP server.\subsubsection*{\large{\textbf{Properties}}\normalsize\hspace{1ex}\hfill}
\begin{list}{}{
\settowidth{\tmplength}{\textbf{OriginalLine}}
\setlength{\itemindent}{0cm}
\setlength{\listparindent}{0cm}
\setlength{\leftmargin}{\evensidemargin}
\addtolength{\leftmargin}{\tmplength}
\settowidth{\labelsep}{X}
\addtolength{\leftmargin}{\labelsep}
\setlength{\labelwidth}{\tmplength}
}
\label{ftpsend.TFTPListRec-FileName}
\index{FileName}
\item[\textbf{FileName}\hfill]
\ifpdf
\begin{flushleft}
\fi
\begin{ttfamily}
published property FileName: string read FFileName write FFileName;\end{ttfamily}

\ifpdf
\end{flushleft}
\fi


\par name of file\label{ftpsend.TFTPListRec-Directory}
\index{Directory}
\item[\textbf{Directory}\hfill]
\ifpdf
\begin{flushleft}
\fi
\begin{ttfamily}
published property Directory: Boolean read FDirectory write FDirectory;\end{ttfamily}

\ifpdf
\end{flushleft}
\fi


\par if name is subdirectory not file.\label{ftpsend.TFTPListRec-Readable}
\index{Readable}
\item[\textbf{Readable}\hfill]
\ifpdf
\begin{flushleft}
\fi
\begin{ttfamily}
published property Readable: Boolean read FReadable write FReadable;\end{ttfamily}

\ifpdf
\end{flushleft}
\fi


\par if you have rights to read\label{ftpsend.TFTPListRec-FileSize}
\index{FileSize}
\item[\textbf{FileSize}\hfill]
\ifpdf
\begin{flushleft}
\fi
\begin{ttfamily}
published property FileSize: Longint read FFileSize write FFileSize;\end{ttfamily}

\ifpdf
\end{flushleft}
\fi


\par size of file in bytes\label{ftpsend.TFTPListRec-FileTime}
\index{FileTime}
\item[\textbf{FileTime}\hfill]
\ifpdf
\begin{flushleft}
\fi
\begin{ttfamily}
published property FileTime: TDateTime read FFileTime write FFileTime;\end{ttfamily}

\ifpdf
\end{flushleft}
\fi


\par date and time of file. Local server timezone is used. Any timezone conversions was not done!\label{ftpsend.TFTPListRec-OriginalLine}
\index{OriginalLine}
\item[\textbf{OriginalLine}\hfill]
\ifpdf
\begin{flushleft}
\fi
\begin{ttfamily}
published property OriginalLine: string read FOriginalLine write FOriginalLine;\end{ttfamily}

\ifpdf
\end{flushleft}
\fi


\par original unparsed line\label{ftpsend.TFTPListRec-Mask}
\index{Mask}
\item[\textbf{Mask}\hfill]
\ifpdf
\begin{flushleft}
\fi
\begin{ttfamily}
published property Mask: string read FMask write FMask;\end{ttfamily}

\ifpdf
\end{flushleft}
\fi


\par mask what was used for parsing\label{ftpsend.TFTPListRec-Permission}
\index{Permission}
\item[\textbf{Permission}\hfill]
\ifpdf
\begin{flushleft}
\fi
\begin{ttfamily}
published property Permission: string read FPermission write FPermission;\end{ttfamily}

\ifpdf
\end{flushleft}
\fi


\par permission string (depending on used mask!)\end{list}
\subsubsection*{\large{\textbf{Methods}}\normalsize\hspace{1ex}\hfill}
\paragraph*{Assign}\hspace*{\fill}

\label{ftpsend.TFTPListRec-Assign}
\index{Assign}
\begin{list}{}{
\settowidth{\tmplength}{\textbf{Description}}
\setlength{\itemindent}{0cm}
\setlength{\listparindent}{0cm}
\setlength{\leftmargin}{\evensidemargin}
\addtolength{\leftmargin}{\tmplength}
\settowidth{\labelsep}{X}
\addtolength{\leftmargin}{\labelsep}
\setlength{\labelwidth}{\tmplength}
}
\item[\textbf{Declaration}\hfill]
\ifpdf
\begin{flushleft}
\fi
\begin{ttfamily}
public procedure Assign(Value: TFTPListRec); virtual;\end{ttfamily}

\ifpdf
\end{flushleft}
\fi

\par
\item[\textbf{Description}]
You can assign another TFTPListRec to this object.

\end{list}
\ifpdf
\subsection*{\large{\textbf{TFTPList Class}}\normalsize\hspace{1ex}\hrulefill}
\else
\subsection*{TFTPList Class}
\fi
\label{ftpsend.TFTPList}
\index{TFTPList}
\subsubsection*{\large{\textbf{Hierarchy}}\normalsize\hspace{1ex}\hfill}
TFTPList {$>$} TObject
\subsubsection*{\large{\textbf{Description}}\normalsize\hspace{1ex}\hfill}
This is TList of TFTPListRec objects.\hfill\vspace*{1ex}

 This object is used for holding lististing of all files information in listed directory on FTP server.\subsubsection*{\large{\textbf{Properties}}\normalsize\hspace{1ex}\hfill}
\begin{list}{}{
\settowidth{\tmplength}{\textbf{UnparsedLines}}
\setlength{\itemindent}{0cm}
\setlength{\listparindent}{0cm}
\setlength{\leftmargin}{\evensidemargin}
\addtolength{\leftmargin}{\tmplength}
\settowidth{\labelsep}{X}
\addtolength{\leftmargin}{\labelsep}
\setlength{\labelwidth}{\tmplength}
}
\label{ftpsend.TFTPList-List}
\index{List}
\item[\textbf{List}\hfill]
\ifpdf
\begin{flushleft}
\fi
\begin{ttfamily}
public property List: TList read FList;\end{ttfamily}

\ifpdf
\end{flushleft}
\fi


\par By this property you have access to list of \begin{ttfamily}TFTPListRec\end{ttfamily}(\ref{ftpsend.TFTPListRec}). This is for compatibility only. Please, use \begin{ttfamily}Items\end{ttfamily}(\ref{ftpsend.TFTPList-Items}) instead.\label{ftpsend.TFTPList-Items}
\index{Items}
\item[\textbf{Items}\hfill]
\ifpdf
\begin{flushleft}
\fi
\begin{ttfamily}
public property Items[Index: Integer]: TFTPListRec read GetListItem;\end{ttfamily}

\ifpdf
\end{flushleft}
\fi


\par By this property you have access to list of \begin{ttfamily}TFTPListRec\end{ttfamily}(\ref{ftpsend.TFTPListRec}).\label{ftpsend.TFTPList-Lines}
\index{Lines}
\item[\textbf{Lines}\hfill]
\ifpdf
\begin{flushleft}
\fi
\begin{ttfamily}
public property Lines: TStringList read FLines;\end{ttfamily}

\ifpdf
\end{flushleft}
\fi


\par Set of lines with RAW directory listing for \begin{ttfamily}parseLines\end{ttfamily}(\ref{ftpsend.TFTPList-ParseLines})\label{ftpsend.TFTPList-Masks}
\index{Masks}
\item[\textbf{Masks}\hfill]
\ifpdf
\begin{flushleft}
\fi
\begin{ttfamily}
public property Masks: TStringList read FMasks;\end{ttfamily}

\ifpdf
\end{flushleft}
\fi


\par Set of masks for directory listing parser. It is predefined by default, however you can modify it as you need. (for example, you can add your own definition mask.) Mask is same as mask used in TotalCommander.\label{ftpsend.TFTPList-UnparsedLines}
\index{UnparsedLines}
\item[\textbf{UnparsedLines}\hfill]
\ifpdf
\begin{flushleft}
\fi
\begin{ttfamily}
public property UnparsedLines: TStringList read FUnparsedLines;\end{ttfamily}

\ifpdf
\end{flushleft}
\fi


\par After \begin{ttfamily}ParseLines\end{ttfamily}(\ref{ftpsend.TFTPList-ParseLines}) it holding lines what was not sucessfully parsed.\end{list}
\subsubsection*{\large{\textbf{Fields}}\normalsize\hspace{1ex}\hfill}
\begin{list}{}{
\settowidth{\tmplength}{\textbf{FUnparsedLines}}
\setlength{\itemindent}{0cm}
\setlength{\listparindent}{0cm}
\setlength{\leftmargin}{\evensidemargin}
\addtolength{\leftmargin}{\tmplength}
\settowidth{\labelsep}{X}
\addtolength{\leftmargin}{\labelsep}
\setlength{\labelwidth}{\tmplength}
}
\label{ftpsend.TFTPList-FList}
\index{FList}
\item[\textbf{FList}\hfill]
\ifpdf
\begin{flushleft}
\fi
\begin{ttfamily}
protected FList: TList;\end{ttfamily}

\ifpdf
\end{flushleft}
\fi


\par  \label{ftpsend.TFTPList-FLines}
\index{FLines}
\item[\textbf{FLines}\hfill]
\ifpdf
\begin{flushleft}
\fi
\begin{ttfamily}
protected FLines: TStringList;\end{ttfamily}

\ifpdf
\end{flushleft}
\fi


\par  \label{ftpsend.TFTPList-FMasks}
\index{FMasks}
\item[\textbf{FMasks}\hfill]
\ifpdf
\begin{flushleft}
\fi
\begin{ttfamily}
protected FMasks: TStringList;\end{ttfamily}

\ifpdf
\end{flushleft}
\fi


\par  \label{ftpsend.TFTPList-FUnparsedLines}
\index{FUnparsedLines}
\item[\textbf{FUnparsedLines}\hfill]
\ifpdf
\begin{flushleft}
\fi
\begin{ttfamily}
protected FUnparsedLines: TStringList;\end{ttfamily}

\ifpdf
\end{flushleft}
\fi


\par  \label{ftpsend.TFTPList-Monthnames}
\index{Monthnames}
\item[\textbf{Monthnames}\hfill]
\ifpdf
\begin{flushleft}
\fi
\begin{ttfamily}
protected Monthnames: string;\end{ttfamily}

\ifpdf
\end{flushleft}
\fi


\par  \label{ftpsend.TFTPList-BlockSize}
\index{BlockSize}
\item[\textbf{BlockSize}\hfill]
\ifpdf
\begin{flushleft}
\fi
\begin{ttfamily}
protected BlockSize: string;\end{ttfamily}

\ifpdf
\end{flushleft}
\fi


\par  \label{ftpsend.TFTPList-DirFlagValue}
\index{DirFlagValue}
\item[\textbf{DirFlagValue}\hfill]
\ifpdf
\begin{flushleft}
\fi
\begin{ttfamily}
protected DirFlagValue: string;\end{ttfamily}

\ifpdf
\end{flushleft}
\fi


\par  \label{ftpsend.TFTPList-FileName}
\index{FileName}
\item[\textbf{FileName}\hfill]
\ifpdf
\begin{flushleft}
\fi
\begin{ttfamily}
protected FileName: string;\end{ttfamily}

\ifpdf
\end{flushleft}
\fi


\par  \label{ftpsend.TFTPList-VMSFileName}
\index{VMSFileName}
\item[\textbf{VMSFileName}\hfill]
\ifpdf
\begin{flushleft}
\fi
\begin{ttfamily}
protected VMSFileName: string;\end{ttfamily}

\ifpdf
\end{flushleft}
\fi


\par  \label{ftpsend.TFTPList-Day}
\index{Day}
\item[\textbf{Day}\hfill]
\ifpdf
\begin{flushleft}
\fi
\begin{ttfamily}
protected Day: string;\end{ttfamily}

\ifpdf
\end{flushleft}
\fi


\par  \label{ftpsend.TFTPList-Month}
\index{Month}
\item[\textbf{Month}\hfill]
\ifpdf
\begin{flushleft}
\fi
\begin{ttfamily}
protected Month: string;\end{ttfamily}

\ifpdf
\end{flushleft}
\fi


\par  \label{ftpsend.TFTPList-ThreeMonth}
\index{ThreeMonth}
\item[\textbf{ThreeMonth}\hfill]
\ifpdf
\begin{flushleft}
\fi
\begin{ttfamily}
protected ThreeMonth: string;\end{ttfamily}

\ifpdf
\end{flushleft}
\fi


\par  \label{ftpsend.TFTPList-YearTime}
\index{YearTime}
\item[\textbf{YearTime}\hfill]
\ifpdf
\begin{flushleft}
\fi
\begin{ttfamily}
protected YearTime: string;\end{ttfamily}

\ifpdf
\end{flushleft}
\fi


\par  \label{ftpsend.TFTPList-Year}
\index{Year}
\item[\textbf{Year}\hfill]
\ifpdf
\begin{flushleft}
\fi
\begin{ttfamily}
protected Year: string;\end{ttfamily}

\ifpdf
\end{flushleft}
\fi


\par  \label{ftpsend.TFTPList-Hours}
\index{Hours}
\item[\textbf{Hours}\hfill]
\ifpdf
\begin{flushleft}
\fi
\begin{ttfamily}
protected Hours: string;\end{ttfamily}

\ifpdf
\end{flushleft}
\fi


\par  \label{ftpsend.TFTPList-HoursModif}
\index{HoursModif}
\item[\textbf{HoursModif}\hfill]
\ifpdf
\begin{flushleft}
\fi
\begin{ttfamily}
protected HoursModif: string;\end{ttfamily}

\ifpdf
\end{flushleft}
\fi


\par  \label{ftpsend.TFTPList-Minutes}
\index{Minutes}
\item[\textbf{Minutes}\hfill]
\ifpdf
\begin{flushleft}
\fi
\begin{ttfamily}
protected Minutes: string;\end{ttfamily}

\ifpdf
\end{flushleft}
\fi


\par  \label{ftpsend.TFTPList-Seconds}
\index{Seconds}
\item[\textbf{Seconds}\hfill]
\ifpdf
\begin{flushleft}
\fi
\begin{ttfamily}
protected Seconds: string;\end{ttfamily}

\ifpdf
\end{flushleft}
\fi


\par  \label{ftpsend.TFTPList-Size}
\index{Size}
\item[\textbf{Size}\hfill]
\ifpdf
\begin{flushleft}
\fi
\begin{ttfamily}
protected Size: string;\end{ttfamily}

\ifpdf
\end{flushleft}
\fi


\par  \label{ftpsend.TFTPList-Permissions}
\index{Permissions}
\item[\textbf{Permissions}\hfill]
\ifpdf
\begin{flushleft}
\fi
\begin{ttfamily}
protected Permissions: string;\end{ttfamily}

\ifpdf
\end{flushleft}
\fi


\par  \label{ftpsend.TFTPList-DirFlag}
\index{DirFlag}
\item[\textbf{DirFlag}\hfill]
\ifpdf
\begin{flushleft}
\fi
\begin{ttfamily}
protected DirFlag: string;\end{ttfamily}

\ifpdf
\end{flushleft}
\fi


\par  \end{list}
\subsubsection*{\large{\textbf{Methods}}\normalsize\hspace{1ex}\hfill}
\paragraph*{GetListItem}\hspace*{\fill}

\label{ftpsend.TFTPList-GetListItem}
\index{GetListItem}
\begin{list}{}{
\settowidth{\tmplength}{\textbf{Description}}
\setlength{\itemindent}{0cm}
\setlength{\listparindent}{0cm}
\setlength{\leftmargin}{\evensidemargin}
\addtolength{\leftmargin}{\tmplength}
\settowidth{\labelsep}{X}
\addtolength{\leftmargin}{\labelsep}
\setlength{\labelwidth}{\tmplength}
}
\item[\textbf{Declaration}\hfill]
\ifpdf
\begin{flushleft}
\fi
\begin{ttfamily}
protected function GetListItem(Index: integer): TFTPListRec; virtual;\end{ttfamily}

\ifpdf
\end{flushleft}
\fi

\end{list}
\paragraph*{ParseEPLF}\hspace*{\fill}

\label{ftpsend.TFTPList-ParseEPLF}
\index{ParseEPLF}
\begin{list}{}{
\settowidth{\tmplength}{\textbf{Description}}
\setlength{\itemindent}{0cm}
\setlength{\listparindent}{0cm}
\setlength{\leftmargin}{\evensidemargin}
\addtolength{\leftmargin}{\tmplength}
\settowidth{\labelsep}{X}
\addtolength{\leftmargin}{\labelsep}
\setlength{\labelwidth}{\tmplength}
}
\item[\textbf{Declaration}\hfill]
\ifpdf
\begin{flushleft}
\fi
\begin{ttfamily}
protected function ParseEPLF(Value: string): Boolean; virtual;\end{ttfamily}

\ifpdf
\end{flushleft}
\fi

\end{list}
\paragraph*{ClearStore}\hspace*{\fill}

\label{ftpsend.TFTPList-ClearStore}
\index{ClearStore}
\begin{list}{}{
\settowidth{\tmplength}{\textbf{Description}}
\setlength{\itemindent}{0cm}
\setlength{\listparindent}{0cm}
\setlength{\leftmargin}{\evensidemargin}
\addtolength{\leftmargin}{\tmplength}
\settowidth{\labelsep}{X}
\addtolength{\leftmargin}{\labelsep}
\setlength{\labelwidth}{\tmplength}
}
\item[\textbf{Declaration}\hfill]
\ifpdf
\begin{flushleft}
\fi
\begin{ttfamily}
protected procedure ClearStore; virtual;\end{ttfamily}

\ifpdf
\end{flushleft}
\fi

\end{list}
\paragraph*{ParseByMask}\hspace*{\fill}

\label{ftpsend.TFTPList-ParseByMask}
\index{ParseByMask}
\begin{list}{}{
\settowidth{\tmplength}{\textbf{Description}}
\setlength{\itemindent}{0cm}
\setlength{\listparindent}{0cm}
\setlength{\leftmargin}{\evensidemargin}
\addtolength{\leftmargin}{\tmplength}
\settowidth{\labelsep}{X}
\addtolength{\leftmargin}{\labelsep}
\setlength{\labelwidth}{\tmplength}
}
\item[\textbf{Declaration}\hfill]
\ifpdf
\begin{flushleft}
\fi
\begin{ttfamily}
protected function ParseByMask(Value, NextValue, Mask: string): Integer; virtual;\end{ttfamily}

\ifpdf
\end{flushleft}
\fi

\end{list}
\paragraph*{CheckValues}\hspace*{\fill}

\label{ftpsend.TFTPList-CheckValues}
\index{CheckValues}
\begin{list}{}{
\settowidth{\tmplength}{\textbf{Description}}
\setlength{\itemindent}{0cm}
\setlength{\listparindent}{0cm}
\setlength{\leftmargin}{\evensidemargin}
\addtolength{\leftmargin}{\tmplength}
\settowidth{\labelsep}{X}
\addtolength{\leftmargin}{\labelsep}
\setlength{\labelwidth}{\tmplength}
}
\item[\textbf{Declaration}\hfill]
\ifpdf
\begin{flushleft}
\fi
\begin{ttfamily}
protected function CheckValues: Boolean; virtual;\end{ttfamily}

\ifpdf
\end{flushleft}
\fi

\end{list}
\paragraph*{FillRecord}\hspace*{\fill}

\label{ftpsend.TFTPList-FillRecord}
\index{FillRecord}
\begin{list}{}{
\settowidth{\tmplength}{\textbf{Description}}
\setlength{\itemindent}{0cm}
\setlength{\listparindent}{0cm}
\setlength{\leftmargin}{\evensidemargin}
\addtolength{\leftmargin}{\tmplength}
\settowidth{\labelsep}{X}
\addtolength{\leftmargin}{\labelsep}
\setlength{\labelwidth}{\tmplength}
}
\item[\textbf{Declaration}\hfill]
\ifpdf
\begin{flushleft}
\fi
\begin{ttfamily}
protected procedure FillRecord(const Value: TFTPListRec); virtual;\end{ttfamily}

\ifpdf
\end{flushleft}
\fi

\end{list}
\paragraph*{Create}\hspace*{\fill}

\label{ftpsend.TFTPList-Create}
\index{Create}
\begin{list}{}{
\settowidth{\tmplength}{\textbf{Description}}
\setlength{\itemindent}{0cm}
\setlength{\listparindent}{0cm}
\setlength{\leftmargin}{\evensidemargin}
\addtolength{\leftmargin}{\tmplength}
\settowidth{\labelsep}{X}
\addtolength{\leftmargin}{\labelsep}
\setlength{\labelwidth}{\tmplength}
}
\item[\textbf{Declaration}\hfill]
\ifpdf
\begin{flushleft}
\fi
\begin{ttfamily}
public constructor Create;\end{ttfamily}

\ifpdf
\end{flushleft}
\fi

\par
\item[\textbf{Description}]
Constructor. You not need create this object, it is created by TFTPSend class as their property.

\end{list}
\paragraph*{Destroy}\hspace*{\fill}

\label{ftpsend.TFTPList-Destroy}
\index{Destroy}
\begin{list}{}{
\settowidth{\tmplength}{\textbf{Description}}
\setlength{\itemindent}{0cm}
\setlength{\listparindent}{0cm}
\setlength{\leftmargin}{\evensidemargin}
\addtolength{\leftmargin}{\tmplength}
\settowidth{\labelsep}{X}
\addtolength{\leftmargin}{\labelsep}
\setlength{\labelwidth}{\tmplength}
}
\item[\textbf{Declaration}\hfill]
\ifpdf
\begin{flushleft}
\fi
\begin{ttfamily}
public destructor Destroy; override;\end{ttfamily}

\ifpdf
\end{flushleft}
\fi

\end{list}
\paragraph*{Clear}\hspace*{\fill}

\label{ftpsend.TFTPList-Clear}
\index{Clear}
\begin{list}{}{
\settowidth{\tmplength}{\textbf{Description}}
\setlength{\itemindent}{0cm}
\setlength{\listparindent}{0cm}
\setlength{\leftmargin}{\evensidemargin}
\addtolength{\leftmargin}{\tmplength}
\settowidth{\labelsep}{X}
\addtolength{\leftmargin}{\labelsep}
\setlength{\labelwidth}{\tmplength}
}
\item[\textbf{Declaration}\hfill]
\ifpdf
\begin{flushleft}
\fi
\begin{ttfamily}
public procedure Clear; virtual;\end{ttfamily}

\ifpdf
\end{flushleft}
\fi

\par
\item[\textbf{Description}]
Clear list.

\end{list}
\paragraph*{Count}\hspace*{\fill}

\label{ftpsend.TFTPList-Count}
\index{Count}
\begin{list}{}{
\settowidth{\tmplength}{\textbf{Description}}
\setlength{\itemindent}{0cm}
\setlength{\listparindent}{0cm}
\setlength{\leftmargin}{\evensidemargin}
\addtolength{\leftmargin}{\tmplength}
\settowidth{\labelsep}{X}
\addtolength{\leftmargin}{\labelsep}
\setlength{\labelwidth}{\tmplength}
}
\item[\textbf{Declaration}\hfill]
\ifpdf
\begin{flushleft}
\fi
\begin{ttfamily}
public function Count: integer; virtual;\end{ttfamily}

\ifpdf
\end{flushleft}
\fi

\par
\item[\textbf{Description}]
count of holded \begin{ttfamily}TFTPListRec\end{ttfamily}(\ref{ftpsend.TFTPListRec}) objects

\end{list}
\paragraph*{Assign}\hspace*{\fill}

\label{ftpsend.TFTPList-Assign}
\index{Assign}
\begin{list}{}{
\settowidth{\tmplength}{\textbf{Description}}
\setlength{\itemindent}{0cm}
\setlength{\listparindent}{0cm}
\setlength{\leftmargin}{\evensidemargin}
\addtolength{\leftmargin}{\tmplength}
\settowidth{\labelsep}{X}
\addtolength{\leftmargin}{\labelsep}
\setlength{\labelwidth}{\tmplength}
}
\item[\textbf{Declaration}\hfill]
\ifpdf
\begin{flushleft}
\fi
\begin{ttfamily}
public procedure Assign(Value: TFTPList); virtual;\end{ttfamily}

\ifpdf
\end{flushleft}
\fi

\par
\item[\textbf{Description}]
Assigns one list to another

\end{list}
\paragraph*{ParseLines}\hspace*{\fill}

\label{ftpsend.TFTPList-ParseLines}
\index{ParseLines}
\begin{list}{}{
\settowidth{\tmplength}{\textbf{Description}}
\setlength{\itemindent}{0cm}
\setlength{\listparindent}{0cm}
\setlength{\leftmargin}{\evensidemargin}
\addtolength{\leftmargin}{\tmplength}
\settowidth{\labelsep}{X}
\addtolength{\leftmargin}{\labelsep}
\setlength{\labelwidth}{\tmplength}
}
\item[\textbf{Declaration}\hfill]
\ifpdf
\begin{flushleft}
\fi
\begin{ttfamily}
public procedure ParseLines; virtual;\end{ttfamily}

\ifpdf
\end{flushleft}
\fi

\par
\item[\textbf{Description}]
try to parse raw directory listing in \begin{ttfamily}lines\end{ttfamily}(\ref{ftpsend.TFTPList-Lines}) to list of \begin{ttfamily}TFTPListRec\end{ttfamily}(\ref{ftpsend.TFTPListRec}).

\end{list}
\ifpdf
\subsection*{\large{\textbf{TFTPSend Class}}\normalsize\hspace{1ex}\hrulefill}
\else
\subsection*{TFTPSend Class}
\fi
\label{ftpsend.TFTPSend}
\index{TFTPSend}
\subsubsection*{\large{\textbf{Hierarchy}}\normalsize\hspace{1ex}\hfill}
TFTPSend {$>$} TSynaClient
\subsubsection*{\large{\textbf{Description}}\normalsize\hspace{1ex}\hfill}
Implementation of FTP protocol.\hfill\vspace*{1ex}

 Note: Are you missing properties for setting Username and Password? Look to parent \begin{ttfamily}TSynaClient\end{ttfamily} object! (Username and Password have default values for "anonymous" FTP login)

Are you missing properties for specify server address and port? Look to parent \begin{ttfamily}TSynaClient\end{ttfamily} too!\subsubsection*{\large{\textbf{Properties}}\normalsize\hspace{1ex}\hfill}
\begin{list}{}{
\settowidth{\tmplength}{\textbf{ForceDefaultPort}}
\setlength{\itemindent}{0cm}
\setlength{\listparindent}{0cm}
\setlength{\leftmargin}{\evensidemargin}
\addtolength{\leftmargin}{\tmplength}
\settowidth{\labelsep}{X}
\addtolength{\leftmargin}{\labelsep}
\setlength{\labelwidth}{\tmplength}
}
\label{ftpsend.TFTPSend-ResultCode}
\index{ResultCode}
\item[\textbf{ResultCode}\hfill]
\ifpdf
\begin{flushleft}
\fi
\begin{ttfamily}
published property ResultCode: Integer read FResultCode;\end{ttfamily}

\ifpdf
\end{flushleft}
\fi


\par After FTP command contains result number of this operation.\label{ftpsend.TFTPSend-ResultString}
\index{ResultString}
\item[\textbf{ResultString}\hfill]
\ifpdf
\begin{flushleft}
\fi
\begin{ttfamily}
published property ResultString: string read FResultString;\end{ttfamily}

\ifpdf
\end{flushleft}
\fi


\par After FTP command contains main line of result.\label{ftpsend.TFTPSend-FullResult}
\index{FullResult}
\item[\textbf{FullResult}\hfill]
\ifpdf
\begin{flushleft}
\fi
\begin{ttfamily}
published property FullResult: TStringList read FFullResult;\end{ttfamily}

\ifpdf
\end{flushleft}
\fi


\par After any FTP command it contains all lines of FTP server reply.\label{ftpsend.TFTPSend-Account}
\index{Account}
\item[\textbf{Account}\hfill]
\ifpdf
\begin{flushleft}
\fi
\begin{ttfamily}
published property Account: string read FAccount Write FAccount;\end{ttfamily}

\ifpdf
\end{flushleft}
\fi


\par Account information used in some cases inside login sequence.\label{ftpsend.TFTPSend-FWHost}
\index{FWHost}
\item[\textbf{FWHost}\hfill]
\ifpdf
\begin{flushleft}
\fi
\begin{ttfamily}
published property FWHost: string read FFWHost Write FFWHost;\end{ttfamily}

\ifpdf
\end{flushleft}
\fi


\par Address of firewall. If empty string (default), firewall not used.\label{ftpsend.TFTPSend-FWPort}
\index{FWPort}
\item[\textbf{FWPort}\hfill]
\ifpdf
\begin{flushleft}
\fi
\begin{ttfamily}
published property FWPort: string read FFWPort Write FFWPort;\end{ttfamily}

\ifpdf
\end{flushleft}
\fi


\par port of firewall. standard value is same port as ftp server used. (21)\label{ftpsend.TFTPSend-FWUsername}
\index{FWUsername}
\item[\textbf{FWUsername}\hfill]
\ifpdf
\begin{flushleft}
\fi
\begin{ttfamily}
published property FWUsername: string read FFWUsername Write FFWUsername;\end{ttfamily}

\ifpdf
\end{flushleft}
\fi


\par Username for login to firewall. (if needed)\label{ftpsend.TFTPSend-FWPassword}
\index{FWPassword}
\item[\textbf{FWPassword}\hfill]
\ifpdf
\begin{flushleft}
\fi
\begin{ttfamily}
published property FWPassword: string read FFWPassword Write FFWPassword;\end{ttfamily}

\ifpdf
\end{flushleft}
\fi


\par password for login to firewall. (if needed)\label{ftpsend.TFTPSend-FWMode}
\index{FWMode}
\item[\textbf{FWMode}\hfill]
\ifpdf
\begin{flushleft}
\fi
\begin{ttfamily}
published property FWMode: integer read FFWMode Write FFWMode;\end{ttfamily}

\ifpdf
\end{flushleft}
\fi


\par Type of Firewall. Used only if you set some firewall address. Supported predefined firewall login sequences are described by comments in source file where you can see pseudocode decribing each sequence.\label{ftpsend.TFTPSend-Sock}
\index{Sock}
\item[\textbf{Sock}\hfill]
\ifpdf
\begin{flushleft}
\fi
\begin{ttfamily}
published property Sock: TTCPBlockSocket read FSock;\end{ttfamily}

\ifpdf
\end{flushleft}
\fi


\par Socket object used for TCP/IP operation on control channel. Good for seting OnStatus hook, etc.\label{ftpsend.TFTPSend-DSock}
\index{DSock}
\item[\textbf{DSock}\hfill]
\ifpdf
\begin{flushleft}
\fi
\begin{ttfamily}
published property DSock: TTCPBlockSocket read FDSock;\end{ttfamily}

\ifpdf
\end{flushleft}
\fi


\par Socket object used for TCP/IP operation on data channel. Good for seting OnStatus hook, etc.\label{ftpsend.TFTPSend-DataStream}
\index{DataStream}
\item[\textbf{DataStream}\hfill]
\ifpdf
\begin{flushleft}
\fi
\begin{ttfamily}
published property DataStream: TMemoryStream read FDataStream;\end{ttfamily}

\ifpdf
\end{flushleft}
\fi


\par If you not use \begin{ttfamily}DirectFile\end{ttfamily}(\ref{ftpsend.TFTPSend-DirectFile}) mode, all data transfers is made to or from this stream.\label{ftpsend.TFTPSend-DataIP}
\index{DataIP}
\item[\textbf{DataIP}\hfill]
\ifpdf
\begin{flushleft}
\fi
\begin{ttfamily}
published property DataIP: string read FDataIP;\end{ttfamily}

\ifpdf
\end{flushleft}
\fi


\par After data connection is established, contains remote side IP of this connection.\label{ftpsend.TFTPSend-DataPort}
\index{DataPort}
\item[\textbf{DataPort}\hfill]
\ifpdf
\begin{flushleft}
\fi
\begin{ttfamily}
published property DataPort: string read FDataPort;\end{ttfamily}

\ifpdf
\end{flushleft}
\fi


\par After data connection is established, contains remote side port of this connection.\label{ftpsend.TFTPSend-DirectFile}
\index{DirectFile}
\item[\textbf{DirectFile}\hfill]
\ifpdf
\begin{flushleft}
\fi
\begin{ttfamily}
published property DirectFile: Boolean read FDirectFile Write FDirectFile;\end{ttfamily}

\ifpdf
\end{flushleft}
\fi


\par Mode of data handling by data connection. If \begin{ttfamily}False\end{ttfamily}, all data operations are made to or from \begin{ttfamily}DataStream\end{ttfamily}(\ref{ftpsend.TFTPSend-DataStream}) TMemoryStream. If \begin{ttfamily}True\end{ttfamily}, data operations is made directly to file in your disk. (filename is specified by \begin{ttfamily}DirectFileName\end{ttfamily}(\ref{ftpsend.TFTPSend-DirectFileName}) property.) Dafault is \begin{ttfamily}False\end{ttfamily}!\label{ftpsend.TFTPSend-DirectFileName}
\index{DirectFileName}
\item[\textbf{DirectFileName}\hfill]
\ifpdf
\begin{flushleft}
\fi
\begin{ttfamily}
published property DirectFileName: string read FDirectFileName Write FDirectFileName;\end{ttfamily}

\ifpdf
\end{flushleft}
\fi


\par Filename for direct disk data operations.\label{ftpsend.TFTPSend-CanResume}
\index{CanResume}
\item[\textbf{CanResume}\hfill]
\ifpdf
\begin{flushleft}
\fi
\begin{ttfamily}
published property CanResume: Boolean read FCanResume;\end{ttfamily}

\ifpdf
\end{flushleft}
\fi


\par Indicate after \begin{ttfamily}Login\end{ttfamily}(\ref{ftpsend.TFTPSend-Login}) if remote server support resume downloads and uploads.\label{ftpsend.TFTPSend-PassiveMode}
\index{PassiveMode}
\item[\textbf{PassiveMode}\hfill]
\ifpdf
\begin{flushleft}
\fi
\begin{ttfamily}
published property PassiveMode: Boolean read FPassiveMode Write FPassiveMode;\end{ttfamily}

\ifpdf
\end{flushleft}
\fi


\par If true (default value), all transfers is made by passive method. It is safer method for various firewalls.\label{ftpsend.TFTPSend-ForceDefaultPort}
\index{ForceDefaultPort}
\item[\textbf{ForceDefaultPort}\hfill]
\ifpdf
\begin{flushleft}
\fi
\begin{ttfamily}
published property ForceDefaultPort: Boolean read FForceDefaultPort Write FForceDefaultPort;\end{ttfamily}

\ifpdf
\end{flushleft}
\fi


\par Force to listen for dataconnection on standard port (20). Default is \begin{ttfamily}False\end{ttfamily}, dataconnections will be made to any non{-}standard port reported by PORT FTP command. This setting is not used, if you use passive mode.\label{ftpsend.TFTPSend-ForceOldPort}
\index{ForceOldPort}
\item[\textbf{ForceOldPort}\hfill]
\ifpdf
\begin{flushleft}
\fi
\begin{ttfamily}
published property ForceOldPort: Boolean read FForceOldPort Write FForceOldPort;\end{ttfamily}

\ifpdf
\end{flushleft}
\fi


\par When is \begin{ttfamily}True\end{ttfamily}, then is disabled EPSV and EPRT support. However without this commands you cannot use IPv6! (Disabling of this commands is needed only when you are behind some crap firewall/NAT.\label{ftpsend.TFTPSend-OnStatus}
\index{OnStatus}
\item[\textbf{OnStatus}\hfill]
\ifpdf
\begin{flushleft}
\fi
\begin{ttfamily}
published property OnStatus: TFTPStatus read FOnStatus write FOnStatus;\end{ttfamily}

\ifpdf
\end{flushleft}
\fi


\par You may set this hook for monitoring FTP commands and replies.\label{ftpsend.TFTPSend-FtpList}
\index{FtpList}
\item[\textbf{FtpList}\hfill]
\ifpdf
\begin{flushleft}
\fi
\begin{ttfamily}
published property FtpList: TFTPList read FFtpList;\end{ttfamily}

\ifpdf
\end{flushleft}
\fi


\par After LIST command is here parsed list of files in given directory.\label{ftpsend.TFTPSend-BinaryMode}
\index{BinaryMode}
\item[\textbf{BinaryMode}\hfill]
\ifpdf
\begin{flushleft}
\fi
\begin{ttfamily}
published property BinaryMode: Boolean read FBinaryMode Write FBinaryMode;\end{ttfamily}

\ifpdf
\end{flushleft}
\fi


\par if \begin{ttfamily}True\end{ttfamily} (default), then data transfers is in binary mode. If this is set to \begin{ttfamily}False\end{ttfamily}, then ASCII mode is used.\label{ftpsend.TFTPSend-AutoTLS}
\index{AutoTLS}
\item[\textbf{AutoTLS}\hfill]
\ifpdf
\begin{flushleft}
\fi
\begin{ttfamily}
published property AutoTLS: Boolean read FAutoTLS Write FAutoTLS;\end{ttfamily}

\ifpdf
\end{flushleft}
\fi


\par if is true, then if server support upgrade to SSL/TLS mode, then use them.\label{ftpsend.TFTPSend-FullSSL}
\index{FullSSL}
\item[\textbf{FullSSL}\hfill]
\ifpdf
\begin{flushleft}
\fi
\begin{ttfamily}
published property FullSSL: Boolean read FFullSSL Write FFullSSL;\end{ttfamily}

\ifpdf
\end{flushleft}
\fi


\par if server listen on SSL/TLS port, then you set this to true.\label{ftpsend.TFTPSend-IsTLS}
\index{IsTLS}
\item[\textbf{IsTLS}\hfill]
\ifpdf
\begin{flushleft}
\fi
\begin{ttfamily}
published property IsTLS: Boolean read FIsTLS;\end{ttfamily}

\ifpdf
\end{flushleft}
\fi


\par Signalise, if control channel is in SSL/TLS mode.\label{ftpsend.TFTPSend-IsDataTLS}
\index{IsDataTLS}
\item[\textbf{IsDataTLS}\hfill]
\ifpdf
\begin{flushleft}
\fi
\begin{ttfamily}
published property IsDataTLS: Boolean read FIsDataTLS;\end{ttfamily}

\ifpdf
\end{flushleft}
\fi


\par Signalise, if data transfers is in SSL/TLS mode.\label{ftpsend.TFTPSend-TLSonData}
\index{TLSonData}
\item[\textbf{TLSonData}\hfill]
\ifpdf
\begin{flushleft}
\fi
\begin{ttfamily}
published property TLSonData: Boolean read FTLSonData write FTLSonData;\end{ttfamily}

\ifpdf
\end{flushleft}
\fi


\par If \begin{ttfamily}True\end{ttfamily} (default), then try to use SSL/TLS on data transfers too. If \begin{ttfamily}False\end{ttfamily}, then SSL/TLS is used only for control connection.\end{list}
\subsubsection*{\large{\textbf{Fields}}\normalsize\hspace{1ex}\hfill}
\begin{list}{}{
\settowidth{\tmplength}{\textbf{FForceDefaultPort}}
\setlength{\itemindent}{0cm}
\setlength{\listparindent}{0cm}
\setlength{\leftmargin}{\evensidemargin}
\addtolength{\leftmargin}{\tmplength}
\settowidth{\labelsep}{X}
\addtolength{\leftmargin}{\labelsep}
\setlength{\labelwidth}{\tmplength}
}
\label{ftpsend.TFTPSend-FOnStatus}
\index{FOnStatus}
\item[\textbf{FOnStatus}\hfill]
\ifpdf
\begin{flushleft}
\fi
\begin{ttfamily}
protected FOnStatus: TFTPStatus;\end{ttfamily}

\ifpdf
\end{flushleft}
\fi


\par  \label{ftpsend.TFTPSend-FSock}
\index{FSock}
\item[\textbf{FSock}\hfill]
\ifpdf
\begin{flushleft}
\fi
\begin{ttfamily}
protected FSock: TTCPBlockSocket;\end{ttfamily}

\ifpdf
\end{flushleft}
\fi


\par  \label{ftpsend.TFTPSend-FDSock}
\index{FDSock}
\item[\textbf{FDSock}\hfill]
\ifpdf
\begin{flushleft}
\fi
\begin{ttfamily}
protected FDSock: TTCPBlockSocket;\end{ttfamily}

\ifpdf
\end{flushleft}
\fi


\par  \label{ftpsend.TFTPSend-FResultCode}
\index{FResultCode}
\item[\textbf{FResultCode}\hfill]
\ifpdf
\begin{flushleft}
\fi
\begin{ttfamily}
protected FResultCode: Integer;\end{ttfamily}

\ifpdf
\end{flushleft}
\fi


\par  \label{ftpsend.TFTPSend-FResultString}
\index{FResultString}
\item[\textbf{FResultString}\hfill]
\ifpdf
\begin{flushleft}
\fi
\begin{ttfamily}
protected FResultString: string;\end{ttfamily}

\ifpdf
\end{flushleft}
\fi


\par  \label{ftpsend.TFTPSend-FFullResult}
\index{FFullResult}
\item[\textbf{FFullResult}\hfill]
\ifpdf
\begin{flushleft}
\fi
\begin{ttfamily}
protected FFullResult: TStringList;\end{ttfamily}

\ifpdf
\end{flushleft}
\fi


\par  \label{ftpsend.TFTPSend-FAccount}
\index{FAccount}
\item[\textbf{FAccount}\hfill]
\ifpdf
\begin{flushleft}
\fi
\begin{ttfamily}
protected FAccount: string;\end{ttfamily}

\ifpdf
\end{flushleft}
\fi


\par  \label{ftpsend.TFTPSend-FFWHost}
\index{FFWHost}
\item[\textbf{FFWHost}\hfill]
\ifpdf
\begin{flushleft}
\fi
\begin{ttfamily}
protected FFWHost: string;\end{ttfamily}

\ifpdf
\end{flushleft}
\fi


\par  \label{ftpsend.TFTPSend-FFWPort}
\index{FFWPort}
\item[\textbf{FFWPort}\hfill]
\ifpdf
\begin{flushleft}
\fi
\begin{ttfamily}
protected FFWPort: string;\end{ttfamily}

\ifpdf
\end{flushleft}
\fi


\par  \label{ftpsend.TFTPSend-FFWUsername}
\index{FFWUsername}
\item[\textbf{FFWUsername}\hfill]
\ifpdf
\begin{flushleft}
\fi
\begin{ttfamily}
protected FFWUsername: string;\end{ttfamily}

\ifpdf
\end{flushleft}
\fi


\par  \label{ftpsend.TFTPSend-FFWPassword}
\index{FFWPassword}
\item[\textbf{FFWPassword}\hfill]
\ifpdf
\begin{flushleft}
\fi
\begin{ttfamily}
protected FFWPassword: string;\end{ttfamily}

\ifpdf
\end{flushleft}
\fi


\par  \label{ftpsend.TFTPSend-FFWMode}
\index{FFWMode}
\item[\textbf{FFWMode}\hfill]
\ifpdf
\begin{flushleft}
\fi
\begin{ttfamily}
protected FFWMode: integer;\end{ttfamily}

\ifpdf
\end{flushleft}
\fi


\par  \label{ftpsend.TFTPSend-FDataStream}
\index{FDataStream}
\item[\textbf{FDataStream}\hfill]
\ifpdf
\begin{flushleft}
\fi
\begin{ttfamily}
protected FDataStream: TMemoryStream;\end{ttfamily}

\ifpdf
\end{flushleft}
\fi


\par  \label{ftpsend.TFTPSend-FDataIP}
\index{FDataIP}
\item[\textbf{FDataIP}\hfill]
\ifpdf
\begin{flushleft}
\fi
\begin{ttfamily}
protected FDataIP: string;\end{ttfamily}

\ifpdf
\end{flushleft}
\fi


\par  \label{ftpsend.TFTPSend-FDataPort}
\index{FDataPort}
\item[\textbf{FDataPort}\hfill]
\ifpdf
\begin{flushleft}
\fi
\begin{ttfamily}
protected FDataPort: string;\end{ttfamily}

\ifpdf
\end{flushleft}
\fi


\par  \label{ftpsend.TFTPSend-FDirectFile}
\index{FDirectFile}
\item[\textbf{FDirectFile}\hfill]
\ifpdf
\begin{flushleft}
\fi
\begin{ttfamily}
protected FDirectFile: Boolean;\end{ttfamily}

\ifpdf
\end{flushleft}
\fi


\par  \label{ftpsend.TFTPSend-FDirectFileName}
\index{FDirectFileName}
\item[\textbf{FDirectFileName}\hfill]
\ifpdf
\begin{flushleft}
\fi
\begin{ttfamily}
protected FDirectFileName: string;\end{ttfamily}

\ifpdf
\end{flushleft}
\fi


\par  \label{ftpsend.TFTPSend-FCanResume}
\index{FCanResume}
\item[\textbf{FCanResume}\hfill]
\ifpdf
\begin{flushleft}
\fi
\begin{ttfamily}
protected FCanResume: Boolean;\end{ttfamily}

\ifpdf
\end{flushleft}
\fi


\par  \label{ftpsend.TFTPSend-FPassiveMode}
\index{FPassiveMode}
\item[\textbf{FPassiveMode}\hfill]
\ifpdf
\begin{flushleft}
\fi
\begin{ttfamily}
protected FPassiveMode: Boolean;\end{ttfamily}

\ifpdf
\end{flushleft}
\fi


\par  \label{ftpsend.TFTPSend-FForceDefaultPort}
\index{FForceDefaultPort}
\item[\textbf{FForceDefaultPort}\hfill]
\ifpdf
\begin{flushleft}
\fi
\begin{ttfamily}
protected FForceDefaultPort: Boolean;\end{ttfamily}

\ifpdf
\end{flushleft}
\fi


\par  \label{ftpsend.TFTPSend-FForceOldPort}
\index{FForceOldPort}
\item[\textbf{FForceOldPort}\hfill]
\ifpdf
\begin{flushleft}
\fi
\begin{ttfamily}
protected FForceOldPort: Boolean;\end{ttfamily}

\ifpdf
\end{flushleft}
\fi


\par  \label{ftpsend.TFTPSend-FFtpList}
\index{FFtpList}
\item[\textbf{FFtpList}\hfill]
\ifpdf
\begin{flushleft}
\fi
\begin{ttfamily}
protected FFtpList: TFTPList;\end{ttfamily}

\ifpdf
\end{flushleft}
\fi


\par  \label{ftpsend.TFTPSend-FBinaryMode}
\index{FBinaryMode}
\item[\textbf{FBinaryMode}\hfill]
\ifpdf
\begin{flushleft}
\fi
\begin{ttfamily}
protected FBinaryMode: Boolean;\end{ttfamily}

\ifpdf
\end{flushleft}
\fi


\par  \label{ftpsend.TFTPSend-FAutoTLS}
\index{FAutoTLS}
\item[\textbf{FAutoTLS}\hfill]
\ifpdf
\begin{flushleft}
\fi
\begin{ttfamily}
protected FAutoTLS: Boolean;\end{ttfamily}

\ifpdf
\end{flushleft}
\fi


\par  \label{ftpsend.TFTPSend-FIsTLS}
\index{FIsTLS}
\item[\textbf{FIsTLS}\hfill]
\ifpdf
\begin{flushleft}
\fi
\begin{ttfamily}
protected FIsTLS: Boolean;\end{ttfamily}

\ifpdf
\end{flushleft}
\fi


\par  \label{ftpsend.TFTPSend-FIsDataTLS}
\index{FIsDataTLS}
\item[\textbf{FIsDataTLS}\hfill]
\ifpdf
\begin{flushleft}
\fi
\begin{ttfamily}
protected FIsDataTLS: Boolean;\end{ttfamily}

\ifpdf
\end{flushleft}
\fi


\par  \label{ftpsend.TFTPSend-FTLSonData}
\index{FTLSonData}
\item[\textbf{FTLSonData}\hfill]
\ifpdf
\begin{flushleft}
\fi
\begin{ttfamily}
protected FTLSonData: Boolean;\end{ttfamily}

\ifpdf
\end{flushleft}
\fi


\par  \label{ftpsend.TFTPSend-FFullSSL}
\index{FFullSSL}
\item[\textbf{FFullSSL}\hfill]
\ifpdf
\begin{flushleft}
\fi
\begin{ttfamily}
protected FFullSSL: Boolean;\end{ttfamily}

\ifpdf
\end{flushleft}
\fi


\par  \label{ftpsend.TFTPSend-CustomLogon}
\index{CustomLogon}
\item[\textbf{CustomLogon}\hfill]
\ifpdf
\begin{flushleft}
\fi
\begin{ttfamily}
public CustomLogon: TLogonActions;\end{ttfamily}

\ifpdf
\end{flushleft}
\fi


\par Custom definition of login sequence. You can use this when you set \begin{ttfamily}FWMode\end{ttfamily}(\ref{ftpsend.TFTPSend-FWMode}) to value {-}1.\end{list}
\subsubsection*{\large{\textbf{Methods}}\normalsize\hspace{1ex}\hfill}
\paragraph*{Auth}\hspace*{\fill}

\label{ftpsend.TFTPSend-Auth}
\index{Auth}
\begin{list}{}{
\settowidth{\tmplength}{\textbf{Description}}
\setlength{\itemindent}{0cm}
\setlength{\listparindent}{0cm}
\setlength{\leftmargin}{\evensidemargin}
\addtolength{\leftmargin}{\tmplength}
\settowidth{\labelsep}{X}
\addtolength{\leftmargin}{\labelsep}
\setlength{\labelwidth}{\tmplength}
}
\item[\textbf{Declaration}\hfill]
\ifpdf
\begin{flushleft}
\fi
\begin{ttfamily}
protected function Auth(Mode: integer): Boolean; virtual;\end{ttfamily}

\ifpdf
\end{flushleft}
\fi

\end{list}
\paragraph*{Connect}\hspace*{\fill}

\label{ftpsend.TFTPSend-Connect}
\index{Connect}
\begin{list}{}{
\settowidth{\tmplength}{\textbf{Description}}
\setlength{\itemindent}{0cm}
\setlength{\listparindent}{0cm}
\setlength{\leftmargin}{\evensidemargin}
\addtolength{\leftmargin}{\tmplength}
\settowidth{\labelsep}{X}
\addtolength{\leftmargin}{\labelsep}
\setlength{\labelwidth}{\tmplength}
}
\item[\textbf{Declaration}\hfill]
\ifpdf
\begin{flushleft}
\fi
\begin{ttfamily}
protected function Connect: Boolean; virtual;\end{ttfamily}

\ifpdf
\end{flushleft}
\fi

\end{list}
\paragraph*{InternalStor}\hspace*{\fill}

\label{ftpsend.TFTPSend-InternalStor}
\index{InternalStor}
\begin{list}{}{
\settowidth{\tmplength}{\textbf{Description}}
\setlength{\itemindent}{0cm}
\setlength{\listparindent}{0cm}
\setlength{\leftmargin}{\evensidemargin}
\addtolength{\leftmargin}{\tmplength}
\settowidth{\labelsep}{X}
\addtolength{\leftmargin}{\labelsep}
\setlength{\labelwidth}{\tmplength}
}
\item[\textbf{Declaration}\hfill]
\ifpdf
\begin{flushleft}
\fi
\begin{ttfamily}
protected function InternalStor(const Command: string; RestoreAt: integer): Boolean; virtual;\end{ttfamily}

\ifpdf
\end{flushleft}
\fi

\end{list}
\paragraph*{DataSocket}\hspace*{\fill}

\label{ftpsend.TFTPSend-DataSocket}
\index{DataSocket}
\begin{list}{}{
\settowidth{\tmplength}{\textbf{Description}}
\setlength{\itemindent}{0cm}
\setlength{\listparindent}{0cm}
\setlength{\leftmargin}{\evensidemargin}
\addtolength{\leftmargin}{\tmplength}
\settowidth{\labelsep}{X}
\addtolength{\leftmargin}{\labelsep}
\setlength{\labelwidth}{\tmplength}
}
\item[\textbf{Declaration}\hfill]
\ifpdf
\begin{flushleft}
\fi
\begin{ttfamily}
protected function DataSocket: Boolean; virtual;\end{ttfamily}

\ifpdf
\end{flushleft}
\fi

\end{list}
\paragraph*{AcceptDataSocket}\hspace*{\fill}

\label{ftpsend.TFTPSend-AcceptDataSocket}
\index{AcceptDataSocket}
\begin{list}{}{
\settowidth{\tmplength}{\textbf{Description}}
\setlength{\itemindent}{0cm}
\setlength{\listparindent}{0cm}
\setlength{\leftmargin}{\evensidemargin}
\addtolength{\leftmargin}{\tmplength}
\settowidth{\labelsep}{X}
\addtolength{\leftmargin}{\labelsep}
\setlength{\labelwidth}{\tmplength}
}
\item[\textbf{Declaration}\hfill]
\ifpdf
\begin{flushleft}
\fi
\begin{ttfamily}
protected function AcceptDataSocket: Boolean; virtual;\end{ttfamily}

\ifpdf
\end{flushleft}
\fi

\end{list}
\paragraph*{DoStatus}\hspace*{\fill}

\label{ftpsend.TFTPSend-DoStatus}
\index{DoStatus}
\begin{list}{}{
\settowidth{\tmplength}{\textbf{Description}}
\setlength{\itemindent}{0cm}
\setlength{\listparindent}{0cm}
\setlength{\leftmargin}{\evensidemargin}
\addtolength{\leftmargin}{\tmplength}
\settowidth{\labelsep}{X}
\addtolength{\leftmargin}{\labelsep}
\setlength{\labelwidth}{\tmplength}
}
\item[\textbf{Declaration}\hfill]
\ifpdf
\begin{flushleft}
\fi
\begin{ttfamily}
protected procedure DoStatus(Response: Boolean; const Value: string); virtual;\end{ttfamily}

\ifpdf
\end{flushleft}
\fi

\end{list}
\paragraph*{Create}\hspace*{\fill}

\label{ftpsend.TFTPSend-Create}
\index{Create}
\begin{list}{}{
\settowidth{\tmplength}{\textbf{Description}}
\setlength{\itemindent}{0cm}
\setlength{\listparindent}{0cm}
\setlength{\leftmargin}{\evensidemargin}
\addtolength{\leftmargin}{\tmplength}
\settowidth{\labelsep}{X}
\addtolength{\leftmargin}{\labelsep}
\setlength{\labelwidth}{\tmplength}
}
\item[\textbf{Declaration}\hfill]
\ifpdf
\begin{flushleft}
\fi
\begin{ttfamily}
public constructor Create;\end{ttfamily}

\ifpdf
\end{flushleft}
\fi

\end{list}
\paragraph*{Destroy}\hspace*{\fill}

\label{ftpsend.TFTPSend-Destroy}
\index{Destroy}
\begin{list}{}{
\settowidth{\tmplength}{\textbf{Description}}
\setlength{\itemindent}{0cm}
\setlength{\listparindent}{0cm}
\setlength{\leftmargin}{\evensidemargin}
\addtolength{\leftmargin}{\tmplength}
\settowidth{\labelsep}{X}
\addtolength{\leftmargin}{\labelsep}
\setlength{\labelwidth}{\tmplength}
}
\item[\textbf{Declaration}\hfill]
\ifpdf
\begin{flushleft}
\fi
\begin{ttfamily}
public destructor Destroy; override;\end{ttfamily}

\ifpdf
\end{flushleft}
\fi

\end{list}
\paragraph*{ReadResult}\hspace*{\fill}

\label{ftpsend.TFTPSend-ReadResult}
\index{ReadResult}
\begin{list}{}{
\settowidth{\tmplength}{\textbf{Description}}
\setlength{\itemindent}{0cm}
\setlength{\listparindent}{0cm}
\setlength{\leftmargin}{\evensidemargin}
\addtolength{\leftmargin}{\tmplength}
\settowidth{\labelsep}{X}
\addtolength{\leftmargin}{\labelsep}
\setlength{\labelwidth}{\tmplength}
}
\item[\textbf{Declaration}\hfill]
\ifpdf
\begin{flushleft}
\fi
\begin{ttfamily}
public function ReadResult: Integer; virtual;\end{ttfamily}

\ifpdf
\end{flushleft}
\fi

\par
\item[\textbf{Description}]
Waits and read FTP server response. You need this only in special cases!

\end{list}
\paragraph*{ParseRemote}\hspace*{\fill}

\label{ftpsend.TFTPSend-ParseRemote}
\index{ParseRemote}
\begin{list}{}{
\settowidth{\tmplength}{\textbf{Description}}
\setlength{\itemindent}{0cm}
\setlength{\listparindent}{0cm}
\setlength{\leftmargin}{\evensidemargin}
\addtolength{\leftmargin}{\tmplength}
\settowidth{\labelsep}{X}
\addtolength{\leftmargin}{\labelsep}
\setlength{\labelwidth}{\tmplength}
}
\item[\textbf{Declaration}\hfill]
\ifpdf
\begin{flushleft}
\fi
\begin{ttfamily}
public procedure ParseRemote(Value: string); virtual;\end{ttfamily}

\ifpdf
\end{flushleft}
\fi

\par
\item[\textbf{Description}]
Parse remote side information of data channel from value string (returned by PASV command). This function you need only in special cases!

\end{list}
\paragraph*{ParseRemoteEPSV}\hspace*{\fill}

\label{ftpsend.TFTPSend-ParseRemoteEPSV}
\index{ParseRemoteEPSV}
\begin{list}{}{
\settowidth{\tmplength}{\textbf{Description}}
\setlength{\itemindent}{0cm}
\setlength{\listparindent}{0cm}
\setlength{\leftmargin}{\evensidemargin}
\addtolength{\leftmargin}{\tmplength}
\settowidth{\labelsep}{X}
\addtolength{\leftmargin}{\labelsep}
\setlength{\labelwidth}{\tmplength}
}
\item[\textbf{Declaration}\hfill]
\ifpdf
\begin{flushleft}
\fi
\begin{ttfamily}
public procedure ParseRemoteEPSV(Value: string); virtual;\end{ttfamily}

\ifpdf
\end{flushleft}
\fi

\par
\item[\textbf{Description}]
Parse remote side information of data channel from value string (returned by EPSV command). This function you need only in special cases!

\end{list}
\paragraph*{FTPCommand}\hspace*{\fill}

\label{ftpsend.TFTPSend-FTPCommand}
\index{FTPCommand}
\begin{list}{}{
\settowidth{\tmplength}{\textbf{Description}}
\setlength{\itemindent}{0cm}
\setlength{\listparindent}{0cm}
\setlength{\leftmargin}{\evensidemargin}
\addtolength{\leftmargin}{\tmplength}
\settowidth{\labelsep}{X}
\addtolength{\leftmargin}{\labelsep}
\setlength{\labelwidth}{\tmplength}
}
\item[\textbf{Declaration}\hfill]
\ifpdf
\begin{flushleft}
\fi
\begin{ttfamily}
public function FTPCommand(const Value: string): integer; virtual;\end{ttfamily}

\ifpdf
\end{flushleft}
\fi

\par
\item[\textbf{Description}]
Send Value as FTP command to FTP server. Returned result code is result of this function. This command is good for sending site specific command, or non{-}standard commands.

\end{list}
\paragraph*{Login}\hspace*{\fill}

\label{ftpsend.TFTPSend-Login}
\index{Login}
\begin{list}{}{
\settowidth{\tmplength}{\textbf{Description}}
\setlength{\itemindent}{0cm}
\setlength{\listparindent}{0cm}
\setlength{\leftmargin}{\evensidemargin}
\addtolength{\leftmargin}{\tmplength}
\settowidth{\labelsep}{X}
\addtolength{\leftmargin}{\labelsep}
\setlength{\labelwidth}{\tmplength}
}
\item[\textbf{Declaration}\hfill]
\ifpdf
\begin{flushleft}
\fi
\begin{ttfamily}
public function Login: Boolean; virtual;\end{ttfamily}

\ifpdf
\end{flushleft}
\fi

\par
\item[\textbf{Description}]
Connect and logon to FTP server. If you specify any FireWall, connect to firewall and throw them connect to FTP server. Login sequence depending on \begin{ttfamily}FWMode\end{ttfamily}(\ref{ftpsend.TFTPSend-FWMode}).

\end{list}
\paragraph*{Logout}\hspace*{\fill}

\label{ftpsend.TFTPSend-Logout}
\index{Logout}
\begin{list}{}{
\settowidth{\tmplength}{\textbf{Description}}
\setlength{\itemindent}{0cm}
\setlength{\listparindent}{0cm}
\setlength{\leftmargin}{\evensidemargin}
\addtolength{\leftmargin}{\tmplength}
\settowidth{\labelsep}{X}
\addtolength{\leftmargin}{\labelsep}
\setlength{\labelwidth}{\tmplength}
}
\item[\textbf{Declaration}\hfill]
\ifpdf
\begin{flushleft}
\fi
\begin{ttfamily}
public function Logout: Boolean; virtual;\end{ttfamily}

\ifpdf
\end{flushleft}
\fi

\par
\item[\textbf{Description}]
Logoff and disconnect from FTP server.

\end{list}
\paragraph*{Abort}\hspace*{\fill}

\label{ftpsend.TFTPSend-Abort}
\index{Abort}
\begin{list}{}{
\settowidth{\tmplength}{\textbf{Description}}
\setlength{\itemindent}{0cm}
\setlength{\listparindent}{0cm}
\setlength{\leftmargin}{\evensidemargin}
\addtolength{\leftmargin}{\tmplength}
\settowidth{\labelsep}{X}
\addtolength{\leftmargin}{\labelsep}
\setlength{\labelwidth}{\tmplength}
}
\item[\textbf{Declaration}\hfill]
\ifpdf
\begin{flushleft}
\fi
\begin{ttfamily}
public procedure Abort; virtual;\end{ttfamily}

\ifpdf
\end{flushleft}
\fi

\par
\item[\textbf{Description}]
Break current transmission of data. (You can call this method from Sock.OnStatus event, or from another thread.)

\end{list}
\paragraph*{TelnetAbort}\hspace*{\fill}

\label{ftpsend.TFTPSend-TelnetAbort}
\index{TelnetAbort}
\begin{list}{}{
\settowidth{\tmplength}{\textbf{Description}}
\setlength{\itemindent}{0cm}
\setlength{\listparindent}{0cm}
\setlength{\leftmargin}{\evensidemargin}
\addtolength{\leftmargin}{\tmplength}
\settowidth{\labelsep}{X}
\addtolength{\leftmargin}{\labelsep}
\setlength{\labelwidth}{\tmplength}
}
\item[\textbf{Declaration}\hfill]
\ifpdf
\begin{flushleft}
\fi
\begin{ttfamily}
public procedure TelnetAbort; virtual;\end{ttfamily}

\ifpdf
\end{flushleft}
\fi

\par
\item[\textbf{Description}]
Break current transmission of data. It is same as Abort, but it send abort telnet commands prior ABOR FTP command. Some servers need it. (You can call this method from Sock.OnStatus event, or from another thread.)

\end{list}
\paragraph*{List}\hspace*{\fill}

\label{ftpsend.TFTPSend-List}
\index{List}
\begin{list}{}{
\settowidth{\tmplength}{\textbf{Description}}
\setlength{\itemindent}{0cm}
\setlength{\listparindent}{0cm}
\setlength{\leftmargin}{\evensidemargin}
\addtolength{\leftmargin}{\tmplength}
\settowidth{\labelsep}{X}
\addtolength{\leftmargin}{\labelsep}
\setlength{\labelwidth}{\tmplength}
}
\item[\textbf{Declaration}\hfill]
\ifpdf
\begin{flushleft}
\fi
\begin{ttfamily}
public function List(Directory: string; NameList: Boolean): Boolean; virtual;\end{ttfamily}

\ifpdf
\end{flushleft}
\fi

\par
\item[\textbf{Description}]
Download directory listing of Directory on FTP server. If Directory is empty string, download listing of current working directory. If NameList is \begin{ttfamily}True\end{ttfamily}, download only names of files in directory. (internally use NLST command instead LIST command) If NameList is \begin{ttfamily}False\end{ttfamily}, returned list is also parsed to \begin{ttfamily}FTPList\end{ttfamily}(\ref{ftpsend.TFTPSend-FtpList}) property.

\end{list}
\paragraph*{RetrieveFile}\hspace*{\fill}

\label{ftpsend.TFTPSend-RetrieveFile}
\index{RetrieveFile}
\begin{list}{}{
\settowidth{\tmplength}{\textbf{Description}}
\setlength{\itemindent}{0cm}
\setlength{\listparindent}{0cm}
\setlength{\leftmargin}{\evensidemargin}
\addtolength{\leftmargin}{\tmplength}
\settowidth{\labelsep}{X}
\addtolength{\leftmargin}{\labelsep}
\setlength{\labelwidth}{\tmplength}
}
\item[\textbf{Declaration}\hfill]
\ifpdf
\begin{flushleft}
\fi
\begin{ttfamily}
public function RetrieveFile(const FileName: string; Restore: Boolean): Boolean; virtual;\end{ttfamily}

\ifpdf
\end{flushleft}
\fi

\par
\item[\textbf{Description}]
Read data from FileName on FTP server. If Restore is \begin{ttfamily}True\end{ttfamily} and server supports resume dowloads, download is resumed. (received is only rest of file)

\end{list}
\paragraph*{StoreFile}\hspace*{\fill}

\label{ftpsend.TFTPSend-StoreFile}
\index{StoreFile}
\begin{list}{}{
\settowidth{\tmplength}{\textbf{Description}}
\setlength{\itemindent}{0cm}
\setlength{\listparindent}{0cm}
\setlength{\leftmargin}{\evensidemargin}
\addtolength{\leftmargin}{\tmplength}
\settowidth{\labelsep}{X}
\addtolength{\leftmargin}{\labelsep}
\setlength{\labelwidth}{\tmplength}
}
\item[\textbf{Declaration}\hfill]
\ifpdf
\begin{flushleft}
\fi
\begin{ttfamily}
public function StoreFile(const FileName: string; Restore: Boolean): Boolean; virtual;\end{ttfamily}

\ifpdf
\end{flushleft}
\fi

\par
\item[\textbf{Description}]
Send data to FileName on FTP server. If Restore is \begin{ttfamily}True\end{ttfamily} and server supports resume upload, upload is resumed. (send only rest of file) In this case if remote file is same length as local file, nothing will be done. If remote file is larger then local, resume is disabled and file is transfered from begin!

\end{list}
\paragraph*{StoreUniqueFile}\hspace*{\fill}

\label{ftpsend.TFTPSend-StoreUniqueFile}
\index{StoreUniqueFile}
\begin{list}{}{
\settowidth{\tmplength}{\textbf{Description}}
\setlength{\itemindent}{0cm}
\setlength{\listparindent}{0cm}
\setlength{\leftmargin}{\evensidemargin}
\addtolength{\leftmargin}{\tmplength}
\settowidth{\labelsep}{X}
\addtolength{\leftmargin}{\labelsep}
\setlength{\labelwidth}{\tmplength}
}
\item[\textbf{Declaration}\hfill]
\ifpdf
\begin{flushleft}
\fi
\begin{ttfamily}
public function StoreUniqueFile: Boolean; virtual;\end{ttfamily}

\ifpdf
\end{flushleft}
\fi

\par
\item[\textbf{Description}]
Send data to FTP server and assing unique name for this file.

\end{list}
\paragraph*{AppendFile}\hspace*{\fill}

\label{ftpsend.TFTPSend-AppendFile}
\index{AppendFile}
\begin{list}{}{
\settowidth{\tmplength}{\textbf{Description}}
\setlength{\itemindent}{0cm}
\setlength{\listparindent}{0cm}
\setlength{\leftmargin}{\evensidemargin}
\addtolength{\leftmargin}{\tmplength}
\settowidth{\labelsep}{X}
\addtolength{\leftmargin}{\labelsep}
\setlength{\labelwidth}{\tmplength}
}
\item[\textbf{Declaration}\hfill]
\ifpdf
\begin{flushleft}
\fi
\begin{ttfamily}
public function AppendFile(const FileName: string): Boolean; virtual;\end{ttfamily}

\ifpdf
\end{flushleft}
\fi

\par
\item[\textbf{Description}]
Append data to FileName on FTP server.

\end{list}
\paragraph*{RenameFile}\hspace*{\fill}

\label{ftpsend.TFTPSend-RenameFile}
\index{RenameFile}
\begin{list}{}{
\settowidth{\tmplength}{\textbf{Description}}
\setlength{\itemindent}{0cm}
\setlength{\listparindent}{0cm}
\setlength{\leftmargin}{\evensidemargin}
\addtolength{\leftmargin}{\tmplength}
\settowidth{\labelsep}{X}
\addtolength{\leftmargin}{\labelsep}
\setlength{\labelwidth}{\tmplength}
}
\item[\textbf{Declaration}\hfill]
\ifpdf
\begin{flushleft}
\fi
\begin{ttfamily}
public function RenameFile(const OldName, NewName: string): Boolean; virtual;\end{ttfamily}

\ifpdf
\end{flushleft}
\fi

\par
\item[\textbf{Description}]
Rename on FTP server file with OldName to NewName.

\end{list}
\paragraph*{DeleteFile}\hspace*{\fill}

\label{ftpsend.TFTPSend-DeleteFile}
\index{DeleteFile}
\begin{list}{}{
\settowidth{\tmplength}{\textbf{Description}}
\setlength{\itemindent}{0cm}
\setlength{\listparindent}{0cm}
\setlength{\leftmargin}{\evensidemargin}
\addtolength{\leftmargin}{\tmplength}
\settowidth{\labelsep}{X}
\addtolength{\leftmargin}{\labelsep}
\setlength{\labelwidth}{\tmplength}
}
\item[\textbf{Declaration}\hfill]
\ifpdf
\begin{flushleft}
\fi
\begin{ttfamily}
public function DeleteFile(const FileName: string): Boolean; virtual;\end{ttfamily}

\ifpdf
\end{flushleft}
\fi

\par
\item[\textbf{Description}]
Delete file FileName on FTP server.

\end{list}
\paragraph*{FileSize}\hspace*{\fill}

\label{ftpsend.TFTPSend-FileSize}
\index{FileSize}
\begin{list}{}{
\settowidth{\tmplength}{\textbf{Description}}
\setlength{\itemindent}{0cm}
\setlength{\listparindent}{0cm}
\setlength{\leftmargin}{\evensidemargin}
\addtolength{\leftmargin}{\tmplength}
\settowidth{\labelsep}{X}
\addtolength{\leftmargin}{\labelsep}
\setlength{\labelwidth}{\tmplength}
}
\item[\textbf{Declaration}\hfill]
\ifpdf
\begin{flushleft}
\fi
\begin{ttfamily}
public function FileSize(const FileName: string): integer; virtual;\end{ttfamily}

\ifpdf
\end{flushleft}
\fi

\par
\item[\textbf{Description}]
Return size of Filename file on FTP server. If command failed (i.e. not implemented), return {-}1.

\end{list}
\paragraph*{NoOp}\hspace*{\fill}

\label{ftpsend.TFTPSend-NoOp}
\index{NoOp}
\begin{list}{}{
\settowidth{\tmplength}{\textbf{Description}}
\setlength{\itemindent}{0cm}
\setlength{\listparindent}{0cm}
\setlength{\leftmargin}{\evensidemargin}
\addtolength{\leftmargin}{\tmplength}
\settowidth{\labelsep}{X}
\addtolength{\leftmargin}{\labelsep}
\setlength{\labelwidth}{\tmplength}
}
\item[\textbf{Declaration}\hfill]
\ifpdf
\begin{flushleft}
\fi
\begin{ttfamily}
public function NoOp: Boolean; virtual;\end{ttfamily}

\ifpdf
\end{flushleft}
\fi

\par
\item[\textbf{Description}]
Send NOOP command to FTP server for preserve of disconnect by inactivity timeout.

\end{list}
\paragraph*{ChangeWorkingDir}\hspace*{\fill}

\label{ftpsend.TFTPSend-ChangeWorkingDir}
\index{ChangeWorkingDir}
\begin{list}{}{
\settowidth{\tmplength}{\textbf{Description}}
\setlength{\itemindent}{0cm}
\setlength{\listparindent}{0cm}
\setlength{\leftmargin}{\evensidemargin}
\addtolength{\leftmargin}{\tmplength}
\settowidth{\labelsep}{X}
\addtolength{\leftmargin}{\labelsep}
\setlength{\labelwidth}{\tmplength}
}
\item[\textbf{Declaration}\hfill]
\ifpdf
\begin{flushleft}
\fi
\begin{ttfamily}
public function ChangeWorkingDir(const Directory: string): Boolean; virtual;\end{ttfamily}

\ifpdf
\end{flushleft}
\fi

\par
\item[\textbf{Description}]
Change currect working directory to Directory on FTP server.

\end{list}
\paragraph*{ChangeToParentDir}\hspace*{\fill}

\label{ftpsend.TFTPSend-ChangeToParentDir}
\index{ChangeToParentDir}
\begin{list}{}{
\settowidth{\tmplength}{\textbf{Description}}
\setlength{\itemindent}{0cm}
\setlength{\listparindent}{0cm}
\setlength{\leftmargin}{\evensidemargin}
\addtolength{\leftmargin}{\tmplength}
\settowidth{\labelsep}{X}
\addtolength{\leftmargin}{\labelsep}
\setlength{\labelwidth}{\tmplength}
}
\item[\textbf{Declaration}\hfill]
\ifpdf
\begin{flushleft}
\fi
\begin{ttfamily}
public function ChangeToParentDir: Boolean; virtual;\end{ttfamily}

\ifpdf
\end{flushleft}
\fi

\par
\item[\textbf{Description}]
walk to upper directory on FTP server.

\end{list}
\paragraph*{ChangeToRootDir}\hspace*{\fill}

\label{ftpsend.TFTPSend-ChangeToRootDir}
\index{ChangeToRootDir}
\begin{list}{}{
\settowidth{\tmplength}{\textbf{Description}}
\setlength{\itemindent}{0cm}
\setlength{\listparindent}{0cm}
\setlength{\leftmargin}{\evensidemargin}
\addtolength{\leftmargin}{\tmplength}
\settowidth{\labelsep}{X}
\addtolength{\leftmargin}{\labelsep}
\setlength{\labelwidth}{\tmplength}
}
\item[\textbf{Declaration}\hfill]
\ifpdf
\begin{flushleft}
\fi
\begin{ttfamily}
public function ChangeToRootDir: Boolean; virtual;\end{ttfamily}

\ifpdf
\end{flushleft}
\fi

\par
\item[\textbf{Description}]
walk to root directory on FTP server. (May not work with all servers properly!)

\end{list}
\paragraph*{DeleteDir}\hspace*{\fill}

\label{ftpsend.TFTPSend-DeleteDir}
\index{DeleteDir}
\begin{list}{}{
\settowidth{\tmplength}{\textbf{Description}}
\setlength{\itemindent}{0cm}
\setlength{\listparindent}{0cm}
\setlength{\leftmargin}{\evensidemargin}
\addtolength{\leftmargin}{\tmplength}
\settowidth{\labelsep}{X}
\addtolength{\leftmargin}{\labelsep}
\setlength{\labelwidth}{\tmplength}
}
\item[\textbf{Declaration}\hfill]
\ifpdf
\begin{flushleft}
\fi
\begin{ttfamily}
public function DeleteDir(const Directory: string): Boolean; virtual;\end{ttfamily}

\ifpdf
\end{flushleft}
\fi

\par
\item[\textbf{Description}]
Delete Directory on FTP server.

\end{list}
\paragraph*{CreateDir}\hspace*{\fill}

\label{ftpsend.TFTPSend-CreateDir}
\index{CreateDir}
\begin{list}{}{
\settowidth{\tmplength}{\textbf{Description}}
\setlength{\itemindent}{0cm}
\setlength{\listparindent}{0cm}
\setlength{\leftmargin}{\evensidemargin}
\addtolength{\leftmargin}{\tmplength}
\settowidth{\labelsep}{X}
\addtolength{\leftmargin}{\labelsep}
\setlength{\labelwidth}{\tmplength}
}
\item[\textbf{Declaration}\hfill]
\ifpdf
\begin{flushleft}
\fi
\begin{ttfamily}
public function CreateDir(const Directory: string): Boolean; virtual;\end{ttfamily}

\ifpdf
\end{flushleft}
\fi

\par
\item[\textbf{Description}]
Create Directory on FTP server.

\end{list}
\paragraph*{GetCurrentDir}\hspace*{\fill}

\label{ftpsend.TFTPSend-GetCurrentDir}
\index{GetCurrentDir}
\begin{list}{}{
\settowidth{\tmplength}{\textbf{Description}}
\setlength{\itemindent}{0cm}
\setlength{\listparindent}{0cm}
\setlength{\leftmargin}{\evensidemargin}
\addtolength{\leftmargin}{\tmplength}
\settowidth{\labelsep}{X}
\addtolength{\leftmargin}{\labelsep}
\setlength{\labelwidth}{\tmplength}
}
\item[\textbf{Declaration}\hfill]
\ifpdf
\begin{flushleft}
\fi
\begin{ttfamily}
public function GetCurrentDir: String; virtual;\end{ttfamily}

\ifpdf
\end{flushleft}
\fi

\par
\item[\textbf{Description}]
Return current working directory on FTP server.

\end{list}
\paragraph*{DataRead}\hspace*{\fill}

\label{ftpsend.TFTPSend-DataRead}
\index{DataRead}
\begin{list}{}{
\settowidth{\tmplength}{\textbf{Description}}
\setlength{\itemindent}{0cm}
\setlength{\listparindent}{0cm}
\setlength{\leftmargin}{\evensidemargin}
\addtolength{\leftmargin}{\tmplength}
\settowidth{\labelsep}{X}
\addtolength{\leftmargin}{\labelsep}
\setlength{\labelwidth}{\tmplength}
}
\item[\textbf{Declaration}\hfill]
\ifpdf
\begin{flushleft}
\fi
\begin{ttfamily}
public function DataRead(const DestStream: TStream): Boolean; virtual;\end{ttfamily}

\ifpdf
\end{flushleft}
\fi

\par
\item[\textbf{Description}]
Establish data channel to FTP server and retrieve data. This function you need only in special cases, i.e. when you need to implement some special unsupported FTP command!

\end{list}
\paragraph*{DataWrite}\hspace*{\fill}

\label{ftpsend.TFTPSend-DataWrite}
\index{DataWrite}
\begin{list}{}{
\settowidth{\tmplength}{\textbf{Description}}
\setlength{\itemindent}{0cm}
\setlength{\listparindent}{0cm}
\setlength{\leftmargin}{\evensidemargin}
\addtolength{\leftmargin}{\tmplength}
\settowidth{\labelsep}{X}
\addtolength{\leftmargin}{\labelsep}
\setlength{\labelwidth}{\tmplength}
}
\item[\textbf{Declaration}\hfill]
\ifpdf
\begin{flushleft}
\fi
\begin{ttfamily}
public function DataWrite(const SourceStream: TStream): Boolean; virtual;\end{ttfamily}

\ifpdf
\end{flushleft}
\fi

\par
\item[\textbf{Description}]
Establish data channel to FTP server and send data. This function you need only in special cases, i.e. when you need to implement some special unsupported FTP command.

\end{list}
\section{Functions and Procedures}
\ifpdf
\subsection*{\large{\textbf{FtpGetFile}}\normalsize\hspace{1ex}\hrulefill}
\else
\subsection*{FtpGetFile}
\fi
\label{ftpsend-FtpGetFile}
\index{FtpGetFile}
\begin{list}{}{
\settowidth{\tmplength}{\textbf{Description}}
\setlength{\itemindent}{0cm}
\setlength{\listparindent}{0cm}
\setlength{\leftmargin}{\evensidemargin}
\addtolength{\leftmargin}{\tmplength}
\settowidth{\labelsep}{X}
\addtolength{\leftmargin}{\labelsep}
\setlength{\labelwidth}{\tmplength}
}
\item[\textbf{Declaration}\hfill]
\ifpdf
\begin{flushleft}
\fi
\begin{ttfamily}
function FtpGetFile(const IP, Port, FileName, LocalFile, User, Pass: string): Boolean;\end{ttfamily}

\ifpdf
\end{flushleft}
\fi

\par
\item[\textbf{Description}]
A very useful function, and example of use can be found in the TFtpSend object. Dowload specified file from FTP server to LocalFile.

\end{list}
\ifpdf
\subsection*{\large{\textbf{FtpPutFile}}\normalsize\hspace{1ex}\hrulefill}
\else
\subsection*{FtpPutFile}
\fi
\label{ftpsend-FtpPutFile}
\index{FtpPutFile}
\begin{list}{}{
\settowidth{\tmplength}{\textbf{Description}}
\setlength{\itemindent}{0cm}
\setlength{\listparindent}{0cm}
\setlength{\leftmargin}{\evensidemargin}
\addtolength{\leftmargin}{\tmplength}
\settowidth{\labelsep}{X}
\addtolength{\leftmargin}{\labelsep}
\setlength{\labelwidth}{\tmplength}
}
\item[\textbf{Declaration}\hfill]
\ifpdf
\begin{flushleft}
\fi
\begin{ttfamily}
function FtpPutFile(const IP, Port, FileName, LocalFile, User, Pass: string): Boolean;\end{ttfamily}

\ifpdf
\end{flushleft}
\fi

\par
\item[\textbf{Description}]
A very useful function, and example of use can be found in the TFtpSend object. Upload specified LocalFile to FTP server.

\end{list}
\ifpdf
\subsection*{\large{\textbf{FtpInterServerTransfer}}\normalsize\hspace{1ex}\hrulefill}
\else
\subsection*{FtpInterServerTransfer}
\fi
\label{ftpsend-FtpInterServerTransfer}
\index{FtpInterServerTransfer}
\begin{list}{}{
\settowidth{\tmplength}{\textbf{Description}}
\setlength{\itemindent}{0cm}
\setlength{\listparindent}{0cm}
\setlength{\leftmargin}{\evensidemargin}
\addtolength{\leftmargin}{\tmplength}
\settowidth{\labelsep}{X}
\addtolength{\leftmargin}{\labelsep}
\setlength{\labelwidth}{\tmplength}
}
\item[\textbf{Declaration}\hfill]
\ifpdf
\begin{flushleft}
\fi
\begin{ttfamily}
function FtpInterServerTransfer( const FromIP, FromPort, FromFile, FromUser, FromPass: string; const ToIP, ToPort, ToFile, ToUser, ToPass: string): Boolean;\end{ttfamily}

\ifpdf
\end{flushleft}
\fi

\par
\item[\textbf{Description}]
A very useful function, and example of use can be found in the TFtpSend object. Initiate transfer of file between two FTP servers.

\end{list}
\section{Types}
\ifpdf
\subsection*{\large{\textbf{TLogonActions}}\normalsize\hspace{1ex}\hrulefill}
\else
\subsection*{TLogonActions}
\fi
\label{ftpsend-TLogonActions}
\index{TLogonActions}
\begin{list}{}{
\settowidth{\tmplength}{\textbf{Description}}
\setlength{\itemindent}{0cm}
\setlength{\listparindent}{0cm}
\setlength{\leftmargin}{\evensidemargin}
\addtolength{\leftmargin}{\tmplength}
\settowidth{\labelsep}{X}
\addtolength{\leftmargin}{\labelsep}
\setlength{\labelwidth}{\tmplength}
}
\item[\textbf{Declaration}\hfill]
\ifpdf
\begin{flushleft}
\fi
\begin{ttfamily}
TLogonActions = array [0..17] of byte;\end{ttfamily}

\ifpdf
\end{flushleft}
\fi

\par
\item[\textbf{Description}]
Array for holding definition of logon sequence.

\end{list}
\ifpdf
\subsection*{\large{\textbf{TFTPStatus}}\normalsize\hspace{1ex}\hrulefill}
\else
\subsection*{TFTPStatus}
\fi
\label{ftpsend-TFTPStatus}
\index{TFTPStatus}
\begin{list}{}{
\settowidth{\tmplength}{\textbf{Description}}
\setlength{\itemindent}{0cm}
\setlength{\listparindent}{0cm}
\setlength{\leftmargin}{\evensidemargin}
\addtolength{\leftmargin}{\tmplength}
\settowidth{\labelsep}{X}
\addtolength{\leftmargin}{\labelsep}
\setlength{\labelwidth}{\tmplength}
}
\item[\textbf{Declaration}\hfill]
\ifpdf
\begin{flushleft}
\fi
\begin{ttfamily}
TFTPStatus = procedure(Sender: TObject; Response: Boolean; const Value: string) of object;\end{ttfamily}

\ifpdf
\end{flushleft}
\fi

\par
\item[\textbf{Description}]
Procedural type for OnStatus event. Sender is calling \begin{ttfamily}TFTPSend\end{ttfamily}(\ref{ftpsend.TFTPSend}) object. Value is FTP command or reply to this comand. (if it is reply, Response is \begin{ttfamily}True\end{ttfamily}).

\end{list}
\section{Constants}
\ifpdf
\subsection*{\large{\textbf{cFtpProtocol}}\normalsize\hspace{1ex}\hrulefill}
\else
\subsection*{cFtpProtocol}
\fi
\label{ftpsend-cFtpProtocol}
\index{cFtpProtocol}
\begin{list}{}{
\settowidth{\tmplength}{\textbf{Description}}
\setlength{\itemindent}{0cm}
\setlength{\listparindent}{0cm}
\setlength{\leftmargin}{\evensidemargin}
\addtolength{\leftmargin}{\tmplength}
\settowidth{\labelsep}{X}
\addtolength{\leftmargin}{\labelsep}
\setlength{\labelwidth}{\tmplength}
}
\item[\textbf{Declaration}\hfill]
\ifpdf
\begin{flushleft}
\fi
\begin{ttfamily}
cFtpProtocol = '21';\end{ttfamily}

\ifpdf
\end{flushleft}
\fi

\end{list}
\ifpdf
\subsection*{\large{\textbf{cFtpDataProtocol}}\normalsize\hspace{1ex}\hrulefill}
\else
\subsection*{cFtpDataProtocol}
\fi
\label{ftpsend-cFtpDataProtocol}
\index{cFtpDataProtocol}
\begin{list}{}{
\settowidth{\tmplength}{\textbf{Description}}
\setlength{\itemindent}{0cm}
\setlength{\listparindent}{0cm}
\setlength{\leftmargin}{\evensidemargin}
\addtolength{\leftmargin}{\tmplength}
\settowidth{\labelsep}{X}
\addtolength{\leftmargin}{\labelsep}
\setlength{\labelwidth}{\tmplength}
}
\item[\textbf{Declaration}\hfill]
\ifpdf
\begin{flushleft}
\fi
\begin{ttfamily}
cFtpDataProtocol = '20';\end{ttfamily}

\ifpdf
\end{flushleft}
\fi

\end{list}
\ifpdf
\subsection*{\large{\textbf{FTP{\_}OK}}\normalsize\hspace{1ex}\hrulefill}
\else
\subsection*{FTP{\_}OK}
\fi
\label{ftpsend-FTP_OK}
\index{FTP{\_}OK}
\begin{list}{}{
\settowidth{\tmplength}{\textbf{Description}}
\setlength{\itemindent}{0cm}
\setlength{\listparindent}{0cm}
\setlength{\leftmargin}{\evensidemargin}
\addtolength{\leftmargin}{\tmplength}
\settowidth{\labelsep}{X}
\addtolength{\leftmargin}{\labelsep}
\setlength{\labelwidth}{\tmplength}
}
\item[\textbf{Declaration}\hfill]
\ifpdf
\begin{flushleft}
\fi
\begin{ttfamily}
FTP{\_}OK = 255;\end{ttfamily}

\ifpdf
\end{flushleft}
\fi

\par
\item[\textbf{Description}]
Terminating value for TLogonActions

\end{list}
\ifpdf
\subsection*{\large{\textbf{FTP{\_}ERR}}\normalsize\hspace{1ex}\hrulefill}
\else
\subsection*{FTP{\_}ERR}
\fi
\label{ftpsend-FTP_ERR}
\index{FTP{\_}ERR}
\begin{list}{}{
\settowidth{\tmplength}{\textbf{Description}}
\setlength{\itemindent}{0cm}
\setlength{\listparindent}{0cm}
\setlength{\leftmargin}{\evensidemargin}
\addtolength{\leftmargin}{\tmplength}
\settowidth{\labelsep}{X}
\addtolength{\leftmargin}{\labelsep}
\setlength{\labelwidth}{\tmplength}
}
\item[\textbf{Declaration}\hfill]
\ifpdf
\begin{flushleft}
\fi
\begin{ttfamily}
FTP{\_}ERR = 254;\end{ttfamily}

\ifpdf
\end{flushleft}
\fi

\par
\item[\textbf{Description}]
Terminating value for TLogonActions

\end{list}
\chapter{Unit httpsend}
\label{httpsend}
\index{httpsend}
\section{Description}
HTTP protocol client\hfill\vspace*{1ex}



Used RFC: RFC{-}1867, RFC{-}1947, RFC{-}2388, RFC{-}2616
\section{uses}
\begin{itemize}
\item \begin{ttfamily}SysUtils\end{ttfamily}\item \begin{ttfamily}Classes\end{ttfamily}\item \begin{ttfamily}blcksock\end{ttfamily}\item \begin{ttfamily}synautil\end{ttfamily}\item \begin{ttfamily}synaip\end{ttfamily}\item \begin{ttfamily}synacode\end{ttfamily}\item \begin{ttfamily}synsock\end{ttfamily}\end{itemize}
\section{Overview}
\begin{description}
\item[\texttt{\begin{ttfamily}THTTPSend\end{ttfamily} Class}]
\end{description}
\begin{description}
\item[\texttt{HttpGetText}]
\item[\texttt{HttpGetBinary}]
\item[\texttt{HttpPostBinary}]
\item[\texttt{HttpPostURL}]
\item[\texttt{HttpPostFile}]
\end{description}
\section{Classes, Interfaces, Objects and Records}
\ifpdf
\subsection*{\large{\textbf{THTTPSend Class}}\normalsize\hspace{1ex}\hrulefill}
\else
\subsection*{THTTPSend Class}
\fi
\label{httpsend.THTTPSend}
\index{THTTPSend}
\subsubsection*{\large{\textbf{Hierarchy}}\normalsize\hspace{1ex}\hfill}
THTTPSend {$>$} TSynaClient
\subsubsection*{\large{\textbf{Description}}\normalsize\hspace{1ex}\hfill}
abstract(Implementation of HTTP protocol.)\subsubsection*{\large{\textbf{Properties}}\normalsize\hspace{1ex}\hfill}
\begin{list}{}{
\settowidth{\tmplength}{\textbf{AddPortNumberToHost}}
\setlength{\itemindent}{0cm}
\setlength{\listparindent}{0cm}
\setlength{\leftmargin}{\evensidemargin}
\addtolength{\leftmargin}{\tmplength}
\settowidth{\labelsep}{X}
\addtolength{\leftmargin}{\labelsep}
\setlength{\labelwidth}{\tmplength}
}
\label{httpsend.THTTPSend-Headers}
\index{Headers}
\item[\textbf{Headers}\hfill]
\ifpdf
\begin{flushleft}
\fi
\begin{ttfamily}
published property Headers: TStringList read FHeaders;\end{ttfamily}

\ifpdf
\end{flushleft}
\fi


\par Before HTTP operation you may define any non{-}standard headers for HTTP request, except of: 'Expect: 100{-}continue', 'Content{-}Length', 'Content{-}Type', 'Connection', 'Authorization', 'Proxy{-}Authorization' and 'Host' headers. After HTTP operation contains full headers of returned document.\label{httpsend.THTTPSend-Cookies}
\index{Cookies}
\item[\textbf{Cookies}\hfill]
\ifpdf
\begin{flushleft}
\fi
\begin{ttfamily}
published property Cookies: TStringList read FCookies;\end{ttfamily}

\ifpdf
\end{flushleft}
\fi


\par This is stringlist with name{-}value stringlist pairs. Each this pair is one cookie. After HTTP request is returned cookies parsed to this stringlist. You can leave this cookies untouched for next HTTP request. You can also save this stringlist for later use.\label{httpsend.THTTPSend-Document}
\index{Document}
\item[\textbf{Document}\hfill]
\ifpdf
\begin{flushleft}
\fi
\begin{ttfamily}
published property Document: TMemoryStream read FDocument;\end{ttfamily}

\ifpdf
\end{flushleft}
\fi


\par Stream with document to send (before request, or with document received from HTTP server (after request).\label{httpsend.THTTPSend-RangeStart}
\index{RangeStart}
\item[\textbf{RangeStart}\hfill]
\ifpdf
\begin{flushleft}
\fi
\begin{ttfamily}
published property RangeStart: integer read FRangeStart Write FRangeStart;\end{ttfamily}

\ifpdf
\end{flushleft}
\fi


\par If you need download only part of requested document, here specify possition of subpart begin. If here 0, then is requested full document.\label{httpsend.THTTPSend-RangeEnd}
\index{RangeEnd}
\item[\textbf{RangeEnd}\hfill]
\ifpdf
\begin{flushleft}
\fi
\begin{ttfamily}
published property RangeEnd: integer read FRangeEnd Write FRangeEnd;\end{ttfamily}

\ifpdf
\end{flushleft}
\fi


\par If you need download only part of requested document, here specify possition of subpart end. If here 0, then is requested document from rangeStart to end of document. (for broken download restoration, for example.)\label{httpsend.THTTPSend-MimeType}
\index{MimeType}
\item[\textbf{MimeType}\hfill]
\ifpdf
\begin{flushleft}
\fi
\begin{ttfamily}
published property MimeType: string read FMimeType Write FMimeType;\end{ttfamily}

\ifpdf
\end{flushleft}
\fi


\par Mime type of sending data. Default is: 'text/html'.\label{httpsend.THTTPSend-Protocol}
\index{Protocol}
\item[\textbf{Protocol}\hfill]
\ifpdf
\begin{flushleft}
\fi
\begin{ttfamily}
published property Protocol: string read FProtocol Write FProtocol;\end{ttfamily}

\ifpdf
\end{flushleft}
\fi


\par Define protocol version. Possible values are: '1.1', '1.0' (default) and '0.9'.\label{httpsend.THTTPSend-KeepAlive}
\index{KeepAlive}
\item[\textbf{KeepAlive}\hfill]
\ifpdf
\begin{flushleft}
\fi
\begin{ttfamily}
published property KeepAlive: Boolean read FKeepAlive Write FKeepAlive;\end{ttfamily}

\ifpdf
\end{flushleft}
\fi


\par If \begin{ttfamily}True\end{ttfamily} (default value), keepalives in HTTP protocol 1.1 is enabled.\label{httpsend.THTTPSend-Status100}
\index{Status100}
\item[\textbf{Status100}\hfill]
\ifpdf
\begin{flushleft}
\fi
\begin{ttfamily}
published property Status100: Boolean read FStatus100 Write FStatus100;\end{ttfamily}

\ifpdf
\end{flushleft}
\fi


\par if \begin{ttfamily}True\end{ttfamily}, then server is requested for 100status capability when uploading data. Default is \begin{ttfamily}False\end{ttfamily} (off).\label{httpsend.THTTPSend-ProxyHost}
\index{ProxyHost}
\item[\textbf{ProxyHost}\hfill]
\ifpdf
\begin{flushleft}
\fi
\begin{ttfamily}
published property ProxyHost: string read FProxyHost Write FProxyHost;\end{ttfamily}

\ifpdf
\end{flushleft}
\fi


\par Address of proxy server (IP address or domain name) where you want to connect in \begin{ttfamily}HTTPMethod\end{ttfamily}(\ref{httpsend.THTTPSend-HTTPMethod}) method.\label{httpsend.THTTPSend-ProxyPort}
\index{ProxyPort}
\item[\textbf{ProxyPort}\hfill]
\ifpdf
\begin{flushleft}
\fi
\begin{ttfamily}
published property ProxyPort: string read FProxyPort Write FProxyPort;\end{ttfamily}

\ifpdf
\end{flushleft}
\fi


\par Port number for proxy connection. Default value is 8080.\label{httpsend.THTTPSend-ProxyUser}
\index{ProxyUser}
\item[\textbf{ProxyUser}\hfill]
\ifpdf
\begin{flushleft}
\fi
\begin{ttfamily}
published property ProxyUser: string read FProxyUser Write FProxyUser;\end{ttfamily}

\ifpdf
\end{flushleft}
\fi


\par Username for connect to proxy server where you want to connect in HTTPMethod method.\label{httpsend.THTTPSend-ProxyPass}
\index{ProxyPass}
\item[\textbf{ProxyPass}\hfill]
\ifpdf
\begin{flushleft}
\fi
\begin{ttfamily}
published property ProxyPass: string read FProxyPass Write FProxyPass;\end{ttfamily}

\ifpdf
\end{flushleft}
\fi


\par Password for connect to proxy server where you want to connect in HTTPMethod method.\label{httpsend.THTTPSend-UserAgent}
\index{UserAgent}
\item[\textbf{UserAgent}\hfill]
\ifpdf
\begin{flushleft}
\fi
\begin{ttfamily}
published property UserAgent: string read FUserAgent Write FUserAgent;\end{ttfamily}

\ifpdf
\end{flushleft}
\fi


\par Here you can specify custom User{-}Agent indentification. By default is used: 'Mozilla/4.0 (compatible; Synapse)'\label{httpsend.THTTPSend-ResultCode}
\index{ResultCode}
\item[\textbf{ResultCode}\hfill]
\ifpdf
\begin{flushleft}
\fi
\begin{ttfamily}
published property ResultCode: Integer read FResultCode;\end{ttfamily}

\ifpdf
\end{flushleft}
\fi


\par After successful \begin{ttfamily}HTTPMethod\end{ttfamily}(\ref{httpsend.THTTPSend-HTTPMethod}) method contains result code of operation.\label{httpsend.THTTPSend-ResultString}
\index{ResultString}
\item[\textbf{ResultString}\hfill]
\ifpdf
\begin{flushleft}
\fi
\begin{ttfamily}
published property ResultString: string read FResultString;\end{ttfamily}

\ifpdf
\end{flushleft}
\fi


\par After successful \begin{ttfamily}HTTPMethod\end{ttfamily}(\ref{httpsend.THTTPSend-HTTPMethod}) method contains string after result code.\label{httpsend.THTTPSend-DownloadSize}
\index{DownloadSize}
\item[\textbf{DownloadSize}\hfill]
\ifpdf
\begin{flushleft}
\fi
\begin{ttfamily}
published property DownloadSize: integer read FDownloadSize;\end{ttfamily}

\ifpdf
\end{flushleft}
\fi


\par if this value is not 0, then data download pending. In this case you have here total sice of downloaded data. It is good for draw download progressbar from OnStatus event.\label{httpsend.THTTPSend-UploadSize}
\index{UploadSize}
\item[\textbf{UploadSize}\hfill]
\ifpdf
\begin{flushleft}
\fi
\begin{ttfamily}
published property UploadSize: integer read FUploadSize;\end{ttfamily}

\ifpdf
\end{flushleft}
\fi


\par if this value is not 0, then data upload pending. In this case you have here total sice of uploaded data. It is good for draw upload progressbar from OnStatus event.\label{httpsend.THTTPSend-Sock}
\index{Sock}
\item[\textbf{Sock}\hfill]
\ifpdf
\begin{flushleft}
\fi
\begin{ttfamily}
published property Sock: TTCPBlockSocket read FSock;\end{ttfamily}

\ifpdf
\end{flushleft}
\fi


\par Socket object used for TCP/IP operation. Good for seting OnStatus hook, etc.\label{httpsend.THTTPSend-AddPortNumberToHost}
\index{AddPortNumberToHost}
\item[\textbf{AddPortNumberToHost}\hfill]
\ifpdf
\begin{flushleft}
\fi
\begin{ttfamily}
published property AddPortNumberToHost: Boolean read FAddPortNumberToHost write FAddPortNumberToHost;\end{ttfamily}

\ifpdf
\end{flushleft}
\fi


\par To have possibility to switch off port number in 'Host:' HTTP header, by default \begin{ttfamily}True\end{ttfamily}. Some buggy servers not like port informations in this header.\end{list}
\subsubsection*{\large{\textbf{Fields}}\normalsize\hspace{1ex}\hfill}
\begin{list}{}{
\settowidth{\tmplength}{\textbf{FAddPortNumberToHost}}
\setlength{\itemindent}{0cm}
\setlength{\listparindent}{0cm}
\setlength{\leftmargin}{\evensidemargin}
\addtolength{\leftmargin}{\tmplength}
\settowidth{\labelsep}{X}
\addtolength{\leftmargin}{\labelsep}
\setlength{\labelwidth}{\tmplength}
}
\label{httpsend.THTTPSend-FSock}
\index{FSock}
\item[\textbf{FSock}\hfill]
\ifpdf
\begin{flushleft}
\fi
\begin{ttfamily}
protected FSock: TTCPBlockSocket;\end{ttfamily}

\ifpdf
\end{flushleft}
\fi


\par  \label{httpsend.THTTPSend-FTransferEncoding}
\index{FTransferEncoding}
\item[\textbf{FTransferEncoding}\hfill]
\ifpdf
\begin{flushleft}
\fi
\begin{ttfamily}
protected FTransferEncoding: TTransferEncoding;\end{ttfamily}

\ifpdf
\end{flushleft}
\fi


\par  \label{httpsend.THTTPSend-FAliveHost}
\index{FAliveHost}
\item[\textbf{FAliveHost}\hfill]
\ifpdf
\begin{flushleft}
\fi
\begin{ttfamily}
protected FAliveHost: string;\end{ttfamily}

\ifpdf
\end{flushleft}
\fi


\par  \label{httpsend.THTTPSend-FAlivePort}
\index{FAlivePort}
\item[\textbf{FAlivePort}\hfill]
\ifpdf
\begin{flushleft}
\fi
\begin{ttfamily}
protected FAlivePort: string;\end{ttfamily}

\ifpdf
\end{flushleft}
\fi


\par  \label{httpsend.THTTPSend-FHeaders}
\index{FHeaders}
\item[\textbf{FHeaders}\hfill]
\ifpdf
\begin{flushleft}
\fi
\begin{ttfamily}
protected FHeaders: TStringList;\end{ttfamily}

\ifpdf
\end{flushleft}
\fi


\par  \label{httpsend.THTTPSend-FDocument}
\index{FDocument}
\item[\textbf{FDocument}\hfill]
\ifpdf
\begin{flushleft}
\fi
\begin{ttfamily}
protected FDocument: TMemoryStream;\end{ttfamily}

\ifpdf
\end{flushleft}
\fi


\par  \label{httpsend.THTTPSend-FMimeType}
\index{FMimeType}
\item[\textbf{FMimeType}\hfill]
\ifpdf
\begin{flushleft}
\fi
\begin{ttfamily}
protected FMimeType: string;\end{ttfamily}

\ifpdf
\end{flushleft}
\fi


\par  \label{httpsend.THTTPSend-FProtocol}
\index{FProtocol}
\item[\textbf{FProtocol}\hfill]
\ifpdf
\begin{flushleft}
\fi
\begin{ttfamily}
protected FProtocol: string;\end{ttfamily}

\ifpdf
\end{flushleft}
\fi


\par  \label{httpsend.THTTPSend-FKeepAlive}
\index{FKeepAlive}
\item[\textbf{FKeepAlive}\hfill]
\ifpdf
\begin{flushleft}
\fi
\begin{ttfamily}
protected FKeepAlive: Boolean;\end{ttfamily}

\ifpdf
\end{flushleft}
\fi


\par  \label{httpsend.THTTPSend-FStatus100}
\index{FStatus100}
\item[\textbf{FStatus100}\hfill]
\ifpdf
\begin{flushleft}
\fi
\begin{ttfamily}
protected FStatus100: Boolean;\end{ttfamily}

\ifpdf
\end{flushleft}
\fi


\par  \label{httpsend.THTTPSend-FProxyHost}
\index{FProxyHost}
\item[\textbf{FProxyHost}\hfill]
\ifpdf
\begin{flushleft}
\fi
\begin{ttfamily}
protected FProxyHost: string;\end{ttfamily}

\ifpdf
\end{flushleft}
\fi


\par  \label{httpsend.THTTPSend-FProxyPort}
\index{FProxyPort}
\item[\textbf{FProxyPort}\hfill]
\ifpdf
\begin{flushleft}
\fi
\begin{ttfamily}
protected FProxyPort: string;\end{ttfamily}

\ifpdf
\end{flushleft}
\fi


\par  \label{httpsend.THTTPSend-FProxyUser}
\index{FProxyUser}
\item[\textbf{FProxyUser}\hfill]
\ifpdf
\begin{flushleft}
\fi
\begin{ttfamily}
protected FProxyUser: string;\end{ttfamily}

\ifpdf
\end{flushleft}
\fi


\par  \label{httpsend.THTTPSend-FProxyPass}
\index{FProxyPass}
\item[\textbf{FProxyPass}\hfill]
\ifpdf
\begin{flushleft}
\fi
\begin{ttfamily}
protected FProxyPass: string;\end{ttfamily}

\ifpdf
\end{flushleft}
\fi


\par  \label{httpsend.THTTPSend-FResultCode}
\index{FResultCode}
\item[\textbf{FResultCode}\hfill]
\ifpdf
\begin{flushleft}
\fi
\begin{ttfamily}
protected FResultCode: Integer;\end{ttfamily}

\ifpdf
\end{flushleft}
\fi


\par  \label{httpsend.THTTPSend-FResultString}
\index{FResultString}
\item[\textbf{FResultString}\hfill]
\ifpdf
\begin{flushleft}
\fi
\begin{ttfamily}
protected FResultString: string;\end{ttfamily}

\ifpdf
\end{flushleft}
\fi


\par  \label{httpsend.THTTPSend-FUserAgent}
\index{FUserAgent}
\item[\textbf{FUserAgent}\hfill]
\ifpdf
\begin{flushleft}
\fi
\begin{ttfamily}
protected FUserAgent: string;\end{ttfamily}

\ifpdf
\end{flushleft}
\fi


\par  \label{httpsend.THTTPSend-FCookies}
\index{FCookies}
\item[\textbf{FCookies}\hfill]
\ifpdf
\begin{flushleft}
\fi
\begin{ttfamily}
protected FCookies: TStringList;\end{ttfamily}

\ifpdf
\end{flushleft}
\fi


\par  \label{httpsend.THTTPSend-FDownloadSize}
\index{FDownloadSize}
\item[\textbf{FDownloadSize}\hfill]
\ifpdf
\begin{flushleft}
\fi
\begin{ttfamily}
protected FDownloadSize: integer;\end{ttfamily}

\ifpdf
\end{flushleft}
\fi


\par  \label{httpsend.THTTPSend-FUploadSize}
\index{FUploadSize}
\item[\textbf{FUploadSize}\hfill]
\ifpdf
\begin{flushleft}
\fi
\begin{ttfamily}
protected FUploadSize: integer;\end{ttfamily}

\ifpdf
\end{flushleft}
\fi


\par  \label{httpsend.THTTPSend-FRangeStart}
\index{FRangeStart}
\item[\textbf{FRangeStart}\hfill]
\ifpdf
\begin{flushleft}
\fi
\begin{ttfamily}
protected FRangeStart: integer;\end{ttfamily}

\ifpdf
\end{flushleft}
\fi


\par  \label{httpsend.THTTPSend-FRangeEnd}
\index{FRangeEnd}
\item[\textbf{FRangeEnd}\hfill]
\ifpdf
\begin{flushleft}
\fi
\begin{ttfamily}
protected FRangeEnd: integer;\end{ttfamily}

\ifpdf
\end{flushleft}
\fi


\par  \label{httpsend.THTTPSend-FAddPortNumberToHost}
\index{FAddPortNumberToHost}
\item[\textbf{FAddPortNumberToHost}\hfill]
\ifpdf
\begin{flushleft}
\fi
\begin{ttfamily}
protected FAddPortNumberToHost: Boolean;\end{ttfamily}

\ifpdf
\end{flushleft}
\fi


\par  \end{list}
\subsubsection*{\large{\textbf{Methods}}\normalsize\hspace{1ex}\hfill}
\paragraph*{ReadUnknown}\hspace*{\fill}

\label{httpsend.THTTPSend-ReadUnknown}
\index{ReadUnknown}
\begin{list}{}{
\settowidth{\tmplength}{\textbf{Description}}
\setlength{\itemindent}{0cm}
\setlength{\listparindent}{0cm}
\setlength{\leftmargin}{\evensidemargin}
\addtolength{\leftmargin}{\tmplength}
\settowidth{\labelsep}{X}
\addtolength{\leftmargin}{\labelsep}
\setlength{\labelwidth}{\tmplength}
}
\item[\textbf{Declaration}\hfill]
\ifpdf
\begin{flushleft}
\fi
\begin{ttfamily}
protected function ReadUnknown: Boolean;\end{ttfamily}

\ifpdf
\end{flushleft}
\fi

\end{list}
\paragraph*{ReadIdentity}\hspace*{\fill}

\label{httpsend.THTTPSend-ReadIdentity}
\index{ReadIdentity}
\begin{list}{}{
\settowidth{\tmplength}{\textbf{Description}}
\setlength{\itemindent}{0cm}
\setlength{\listparindent}{0cm}
\setlength{\leftmargin}{\evensidemargin}
\addtolength{\leftmargin}{\tmplength}
\settowidth{\labelsep}{X}
\addtolength{\leftmargin}{\labelsep}
\setlength{\labelwidth}{\tmplength}
}
\item[\textbf{Declaration}\hfill]
\ifpdf
\begin{flushleft}
\fi
\begin{ttfamily}
protected function ReadIdentity(Size: Integer): Boolean;\end{ttfamily}

\ifpdf
\end{flushleft}
\fi

\end{list}
\paragraph*{ReadChunked}\hspace*{\fill}

\label{httpsend.THTTPSend-ReadChunked}
\index{ReadChunked}
\begin{list}{}{
\settowidth{\tmplength}{\textbf{Description}}
\setlength{\itemindent}{0cm}
\setlength{\listparindent}{0cm}
\setlength{\leftmargin}{\evensidemargin}
\addtolength{\leftmargin}{\tmplength}
\settowidth{\labelsep}{X}
\addtolength{\leftmargin}{\labelsep}
\setlength{\labelwidth}{\tmplength}
}
\item[\textbf{Declaration}\hfill]
\ifpdf
\begin{flushleft}
\fi
\begin{ttfamily}
protected function ReadChunked: Boolean;\end{ttfamily}

\ifpdf
\end{flushleft}
\fi

\end{list}
\paragraph*{ParseCookies}\hspace*{\fill}

\label{httpsend.THTTPSend-ParseCookies}
\index{ParseCookies}
\begin{list}{}{
\settowidth{\tmplength}{\textbf{Description}}
\setlength{\itemindent}{0cm}
\setlength{\listparindent}{0cm}
\setlength{\leftmargin}{\evensidemargin}
\addtolength{\leftmargin}{\tmplength}
\settowidth{\labelsep}{X}
\addtolength{\leftmargin}{\labelsep}
\setlength{\labelwidth}{\tmplength}
}
\item[\textbf{Declaration}\hfill]
\ifpdf
\begin{flushleft}
\fi
\begin{ttfamily}
protected procedure ParseCookies;\end{ttfamily}

\ifpdf
\end{flushleft}
\fi

\end{list}
\paragraph*{PrepareHeaders}\hspace*{\fill}

\label{httpsend.THTTPSend-PrepareHeaders}
\index{PrepareHeaders}
\begin{list}{}{
\settowidth{\tmplength}{\textbf{Description}}
\setlength{\itemindent}{0cm}
\setlength{\listparindent}{0cm}
\setlength{\leftmargin}{\evensidemargin}
\addtolength{\leftmargin}{\tmplength}
\settowidth{\labelsep}{X}
\addtolength{\leftmargin}{\labelsep}
\setlength{\labelwidth}{\tmplength}
}
\item[\textbf{Declaration}\hfill]
\ifpdf
\begin{flushleft}
\fi
\begin{ttfamily}
protected function PrepareHeaders: string;\end{ttfamily}

\ifpdf
\end{flushleft}
\fi

\end{list}
\paragraph*{InternalDoConnect}\hspace*{\fill}

\label{httpsend.THTTPSend-InternalDoConnect}
\index{InternalDoConnect}
\begin{list}{}{
\settowidth{\tmplength}{\textbf{Description}}
\setlength{\itemindent}{0cm}
\setlength{\listparindent}{0cm}
\setlength{\leftmargin}{\evensidemargin}
\addtolength{\leftmargin}{\tmplength}
\settowidth{\labelsep}{X}
\addtolength{\leftmargin}{\labelsep}
\setlength{\labelwidth}{\tmplength}
}
\item[\textbf{Declaration}\hfill]
\ifpdf
\begin{flushleft}
\fi
\begin{ttfamily}
protected function InternalDoConnect(needssl: Boolean): Boolean;\end{ttfamily}

\ifpdf
\end{flushleft}
\fi

\end{list}
\paragraph*{InternalConnect}\hspace*{\fill}

\label{httpsend.THTTPSend-InternalConnect}
\index{InternalConnect}
\begin{list}{}{
\settowidth{\tmplength}{\textbf{Description}}
\setlength{\itemindent}{0cm}
\setlength{\listparindent}{0cm}
\setlength{\leftmargin}{\evensidemargin}
\addtolength{\leftmargin}{\tmplength}
\settowidth{\labelsep}{X}
\addtolength{\leftmargin}{\labelsep}
\setlength{\labelwidth}{\tmplength}
}
\item[\textbf{Declaration}\hfill]
\ifpdf
\begin{flushleft}
\fi
\begin{ttfamily}
protected function InternalConnect(needssl: Boolean): Boolean;\end{ttfamily}

\ifpdf
\end{flushleft}
\fi

\end{list}
\paragraph*{Create}\hspace*{\fill}

\label{httpsend.THTTPSend-Create}
\index{Create}
\begin{list}{}{
\settowidth{\tmplength}{\textbf{Description}}
\setlength{\itemindent}{0cm}
\setlength{\listparindent}{0cm}
\setlength{\leftmargin}{\evensidemargin}
\addtolength{\leftmargin}{\tmplength}
\settowidth{\labelsep}{X}
\addtolength{\leftmargin}{\labelsep}
\setlength{\labelwidth}{\tmplength}
}
\item[\textbf{Declaration}\hfill]
\ifpdf
\begin{flushleft}
\fi
\begin{ttfamily}
public constructor Create;\end{ttfamily}

\ifpdf
\end{flushleft}
\fi

\end{list}
\paragraph*{Destroy}\hspace*{\fill}

\label{httpsend.THTTPSend-Destroy}
\index{Destroy}
\begin{list}{}{
\settowidth{\tmplength}{\textbf{Description}}
\setlength{\itemindent}{0cm}
\setlength{\listparindent}{0cm}
\setlength{\leftmargin}{\evensidemargin}
\addtolength{\leftmargin}{\tmplength}
\settowidth{\labelsep}{X}
\addtolength{\leftmargin}{\labelsep}
\setlength{\labelwidth}{\tmplength}
}
\item[\textbf{Declaration}\hfill]
\ifpdf
\begin{flushleft}
\fi
\begin{ttfamily}
public destructor Destroy; override;\end{ttfamily}

\ifpdf
\end{flushleft}
\fi

\end{list}
\paragraph*{Clear}\hspace*{\fill}

\label{httpsend.THTTPSend-Clear}
\index{Clear}
\begin{list}{}{
\settowidth{\tmplength}{\textbf{Description}}
\setlength{\itemindent}{0cm}
\setlength{\listparindent}{0cm}
\setlength{\leftmargin}{\evensidemargin}
\addtolength{\leftmargin}{\tmplength}
\settowidth{\labelsep}{X}
\addtolength{\leftmargin}{\labelsep}
\setlength{\labelwidth}{\tmplength}
}
\item[\textbf{Declaration}\hfill]
\ifpdf
\begin{flushleft}
\fi
\begin{ttfamily}
public procedure Clear;\end{ttfamily}

\ifpdf
\end{flushleft}
\fi

\par
\item[\textbf{Description}]
Reset headers and document and Mimetype.

\end{list}
\paragraph*{DecodeStatus}\hspace*{\fill}

\label{httpsend.THTTPSend-DecodeStatus}
\index{DecodeStatus}
\begin{list}{}{
\settowidth{\tmplength}{\textbf{Description}}
\setlength{\itemindent}{0cm}
\setlength{\listparindent}{0cm}
\setlength{\leftmargin}{\evensidemargin}
\addtolength{\leftmargin}{\tmplength}
\settowidth{\labelsep}{X}
\addtolength{\leftmargin}{\labelsep}
\setlength{\labelwidth}{\tmplength}
}
\item[\textbf{Declaration}\hfill]
\ifpdf
\begin{flushleft}
\fi
\begin{ttfamily}
public procedure DecodeStatus(const Value: string);\end{ttfamily}

\ifpdf
\end{flushleft}
\fi

\par
\item[\textbf{Description}]
Decode ResultCode and ResultString from Value.

\end{list}
\paragraph*{HTTPMethod}\hspace*{\fill}

\label{httpsend.THTTPSend-HTTPMethod}
\index{HTTPMethod}
\begin{list}{}{
\settowidth{\tmplength}{\textbf{Description}}
\setlength{\itemindent}{0cm}
\setlength{\listparindent}{0cm}
\setlength{\leftmargin}{\evensidemargin}
\addtolength{\leftmargin}{\tmplength}
\settowidth{\labelsep}{X}
\addtolength{\leftmargin}{\labelsep}
\setlength{\labelwidth}{\tmplength}
}
\item[\textbf{Declaration}\hfill]
\ifpdf
\begin{flushleft}
\fi
\begin{ttfamily}
public function HTTPMethod(const Method, URL: string): Boolean;\end{ttfamily}

\ifpdf
\end{flushleft}
\fi

\par
\item[\textbf{Description}]
Connects to host define in URL and access to resource defined in URL by method. If Document is not empty, send it to server as part of HTTP request. Server response is in Document and headers. Connection may be authorised by username and password in URL. If you define proxy properties, connection is made by this proxy. If all OK, result is \begin{ttfamily}True\end{ttfamily}, else result is \begin{ttfamily}False\end{ttfamily}.

If you use in URL 'https:' instead only 'http:', then your request is made by SSL/TLS connection (if you not specify port, then port 443 is used instead standard port 80). If you use SSL/TLS request and you have defined HTTP proxy, then HTTP{-}tunnel mode is automaticly used .

\end{list}
\paragraph*{Abort}\hspace*{\fill}

\label{httpsend.THTTPSend-Abort}
\index{Abort}
\begin{list}{}{
\settowidth{\tmplength}{\textbf{Description}}
\setlength{\itemindent}{0cm}
\setlength{\listparindent}{0cm}
\setlength{\leftmargin}{\evensidemargin}
\addtolength{\leftmargin}{\tmplength}
\settowidth{\labelsep}{X}
\addtolength{\leftmargin}{\labelsep}
\setlength{\labelwidth}{\tmplength}
}
\item[\textbf{Declaration}\hfill]
\ifpdf
\begin{flushleft}
\fi
\begin{ttfamily}
public procedure Abort;\end{ttfamily}

\ifpdf
\end{flushleft}
\fi

\par
\item[\textbf{Description}]
You can call this method from OnStatus event for break current data transfer. (or from another thread.)

\end{list}
\section{Functions and Procedures}
\ifpdf
\subsection*{\large{\textbf{HttpGetText}}\normalsize\hspace{1ex}\hrulefill}
\else
\subsection*{HttpGetText}
\fi
\label{httpsend-HttpGetText}
\index{HttpGetText}
\begin{list}{}{
\settowidth{\tmplength}{\textbf{Description}}
\setlength{\itemindent}{0cm}
\setlength{\listparindent}{0cm}
\setlength{\leftmargin}{\evensidemargin}
\addtolength{\leftmargin}{\tmplength}
\settowidth{\labelsep}{X}
\addtolength{\leftmargin}{\labelsep}
\setlength{\labelwidth}{\tmplength}
}
\item[\textbf{Declaration}\hfill]
\ifpdf
\begin{flushleft}
\fi
\begin{ttfamily}
function HttpGetText(const URL: string; const Response: TStrings): Boolean;\end{ttfamily}

\ifpdf
\end{flushleft}
\fi

\par
\item[\textbf{Description}]
A very usefull function, and example of use can be found in the THTTPSend object. It implements the GET method of the HTTP protocol. This function sends the GET method for URL document to an HTTP server. Returned document is in the "Response" stringlist (without any headers). Returns boolean TRUE if all went well.

\end{list}
\ifpdf
\subsection*{\large{\textbf{HttpGetBinary}}\normalsize\hspace{1ex}\hrulefill}
\else
\subsection*{HttpGetBinary}
\fi
\label{httpsend-HttpGetBinary}
\index{HttpGetBinary}
\begin{list}{}{
\settowidth{\tmplength}{\textbf{Description}}
\setlength{\itemindent}{0cm}
\setlength{\listparindent}{0cm}
\setlength{\leftmargin}{\evensidemargin}
\addtolength{\leftmargin}{\tmplength}
\settowidth{\labelsep}{X}
\addtolength{\leftmargin}{\labelsep}
\setlength{\labelwidth}{\tmplength}
}
\item[\textbf{Declaration}\hfill]
\ifpdf
\begin{flushleft}
\fi
\begin{ttfamily}
function HttpGetBinary(const URL: string; const Response: TStream): Boolean;\end{ttfamily}

\ifpdf
\end{flushleft}
\fi

\par
\item[\textbf{Description}]
A very usefull function, and example of use can be found in the THTTPSend object. It implements the GET method of the HTTP protocol. This function sends the GET method for URL document to an HTTP server. Returned document is in the "Response" stream. Returns boolean TRUE if all went well.

\end{list}
\ifpdf
\subsection*{\large{\textbf{HttpPostBinary}}\normalsize\hspace{1ex}\hrulefill}
\else
\subsection*{HttpPostBinary}
\fi
\label{httpsend-HttpPostBinary}
\index{HttpPostBinary}
\begin{list}{}{
\settowidth{\tmplength}{\textbf{Description}}
\setlength{\itemindent}{0cm}
\setlength{\listparindent}{0cm}
\setlength{\leftmargin}{\evensidemargin}
\addtolength{\leftmargin}{\tmplength}
\settowidth{\labelsep}{X}
\addtolength{\leftmargin}{\labelsep}
\setlength{\labelwidth}{\tmplength}
}
\item[\textbf{Declaration}\hfill]
\ifpdf
\begin{flushleft}
\fi
\begin{ttfamily}
function HttpPostBinary(const URL: string; const Data: TStream): Boolean;\end{ttfamily}

\ifpdf
\end{flushleft}
\fi

\par
\item[\textbf{Description}]
A very useful function, and example of use can be found in the THTTPSend object. It implements the POST method of the HTTP protocol. This function sends the SEND method for a URL document to an HTTP server. The document to be sent is located in "Data" stream. The returned document is in the "Data" stream. Returns boolean TRUE if all went well.

\end{list}
\ifpdf
\subsection*{\large{\textbf{HttpPostURL}}\normalsize\hspace{1ex}\hrulefill}
\else
\subsection*{HttpPostURL}
\fi
\label{httpsend-HttpPostURL}
\index{HttpPostURL}
\begin{list}{}{
\settowidth{\tmplength}{\textbf{Description}}
\setlength{\itemindent}{0cm}
\setlength{\listparindent}{0cm}
\setlength{\leftmargin}{\evensidemargin}
\addtolength{\leftmargin}{\tmplength}
\settowidth{\labelsep}{X}
\addtolength{\leftmargin}{\labelsep}
\setlength{\labelwidth}{\tmplength}
}
\item[\textbf{Declaration}\hfill]
\ifpdf
\begin{flushleft}
\fi
\begin{ttfamily}
function HttpPostURL(const URL, URLData: string; const Data: TStream): Boolean;\end{ttfamily}

\ifpdf
\end{flushleft}
\fi

\par
\item[\textbf{Description}]
A very useful function, and example of use can be found in the THTTPSend object. It implements the POST method of the HTTP protocol. This function is good for POSTing form data. It sends the POST method for a URL document to an HTTP server. You must prepare the form data in the same manner as you would the URL data, and pass this prepared data to "URLdata". The following is a sample of how the data would appear: 'name=Lukas{\&}field1=some{\%}20data'. The information in the field must be encoded by EncodeURLElement function. The returned document is in the "Data" stream. Returns boolean TRUE if all went well.

\end{list}
\ifpdf
\subsection*{\large{\textbf{HttpPostFile}}\normalsize\hspace{1ex}\hrulefill}
\else
\subsection*{HttpPostFile}
\fi
\label{httpsend-HttpPostFile}
\index{HttpPostFile}
\begin{list}{}{
\settowidth{\tmplength}{\textbf{Description}}
\setlength{\itemindent}{0cm}
\setlength{\listparindent}{0cm}
\setlength{\leftmargin}{\evensidemargin}
\addtolength{\leftmargin}{\tmplength}
\settowidth{\labelsep}{X}
\addtolength{\leftmargin}{\labelsep}
\setlength{\labelwidth}{\tmplength}
}
\item[\textbf{Declaration}\hfill]
\ifpdf
\begin{flushleft}
\fi
\begin{ttfamily}
function HttpPostFile(const URL, FieldName, FileName: string; const Data: TStream; const ResultData: TStrings): Boolean;\end{ttfamily}

\ifpdf
\end{flushleft}
\fi

\par
\item[\textbf{Description}]
A very useful function, and example of use can be found in the THTTPSend object. It implements the POST method of the HTTP protocol. This function sends the POST method for a URL document to an HTTP server. This function simulate posting of file by HTML form used method 'multipart/form{-}data'. Posting file is in DATA stream. Its name is Filename string. Fieldname is for name of formular field with file. (simulate HTML INPUT FILE) The returned document is in the ResultData Stringlist. Returns boolean TRUE if all went well.

\end{list}
\section{Types}
\ifpdf
\subsection*{\large{\textbf{TTransferEncoding}}\normalsize\hspace{1ex}\hrulefill}
\else
\subsection*{TTransferEncoding}
\fi
\label{httpsend-TTransferEncoding}
\index{TTransferEncoding}
\begin{list}{}{
\settowidth{\tmplength}{\textbf{Description}}
\setlength{\itemindent}{0cm}
\setlength{\listparindent}{0cm}
\setlength{\leftmargin}{\evensidemargin}
\addtolength{\leftmargin}{\tmplength}
\settowidth{\labelsep}{X}
\addtolength{\leftmargin}{\labelsep}
\setlength{\labelwidth}{\tmplength}
}
\item[\textbf{Declaration}\hfill]
\ifpdf
\begin{flushleft}
\fi
\begin{ttfamily}
TTransferEncoding = (...);\end{ttfamily}

\ifpdf
\end{flushleft}
\fi

\par
\item[\textbf{Description}]
These encoding types are used internally by the THTTPSend object to identify the transfer data types.\item[\textbf{Values}]
\begin{description}
\item[\texttt{TE{\_}UNKNOWN}]  
\item[\texttt{TE{\_}IDENTITY}]  
\item[\texttt{TE{\_}CHUNKED}]  
\end{description}


\end{list}
\section{Constants}
\ifpdf
\subsection*{\large{\textbf{cHttpProtocol}}\normalsize\hspace{1ex}\hrulefill}
\else
\subsection*{cHttpProtocol}
\fi
\label{httpsend-cHttpProtocol}
\index{cHttpProtocol}
\begin{list}{}{
\settowidth{\tmplength}{\textbf{Description}}
\setlength{\itemindent}{0cm}
\setlength{\listparindent}{0cm}
\setlength{\leftmargin}{\evensidemargin}
\addtolength{\leftmargin}{\tmplength}
\settowidth{\labelsep}{X}
\addtolength{\leftmargin}{\labelsep}
\setlength{\labelwidth}{\tmplength}
}
\item[\textbf{Declaration}\hfill]
\ifpdf
\begin{flushleft}
\fi
\begin{ttfamily}
cHttpProtocol = '80';\end{ttfamily}

\ifpdf
\end{flushleft}
\fi

\end{list}
\chapter{Unit ipkbuild}
\label{ipkbuild}
\index{ipkbuild}
\section{Description}
Functions to build IPK packages an sources
\section{uses}
\begin{itemize}
\item \begin{ttfamily}Classes\end{ttfamily}\item \begin{ttfamily}SysUtils\end{ttfamily}\item \begin{ttfamily}FileUtil\end{ttfamily}\item \begin{ttfamily}XMLRead\end{ttfamily}\item \begin{ttfamily}XMLWrite\end{ttfamily}\item \begin{ttfamily}DOM\end{ttfamily}\item \begin{ttfamily}AbBase\end{ttfamily}\item \begin{ttfamily}AbZipper\end{ttfamily}\item \begin{ttfamily}AbArcTyp\end{ttfamily}\item \begin{ttfamily}MD5\end{ttfamily}\item \begin{ttfamily}utilities\end{ttfamily}(\ref{utilities})\item \begin{ttfamily}Process\end{ttfamily}\item \begin{ttfamily}OPBitmapFormats\end{ttfamily}\end{itemize}
\section{Overview}
\begin{description}
\item[\texttt{\begin{ttfamily}TPackInfo\end{ttfamily} record}]
\end{description}
\begin{description}
\item[\texttt{BuildZIP}]
\item[\texttt{FindChildNode}]
\item[\texttt{SetNode}]
\item[\texttt{CreateUpdateSource}]
\item[\texttt{BuildApplication}]
\item[\texttt{BuildPackage}]
\item[\texttt{CreateLiCompButton}]
\end{description}
\section{Classes, Interfaces, Objects and Records}
\ifpdf
\subsection*{\large{\textbf{TPackInfo record}}\normalsize\hspace{1ex}\hrulefill}
\else
\subsection*{TPackInfo record}
\fi
\label{ipkbuild.TPackInfo}
\index{TPackInfo}
%%%%Description
\subsubsection*{\large{\textbf{Fields}}\normalsize\hspace{1ex}\hfill}
\begin{list}{}{
\settowidth{\tmplength}{\textbf{Maintainer}}
\setlength{\itemindent}{0cm}
\setlength{\listparindent}{0cm}
\setlength{\leftmargin}{\evensidemargin}
\addtolength{\leftmargin}{\tmplength}
\settowidth{\labelsep}{X}
\addtolength{\leftmargin}{\labelsep}
\setlength{\labelwidth}{\tmplength}
}
\label{ipkbuild.TPackInfo-desc}
\index{desc}
\item[\textbf{desc}\hfill]
\ifpdf
\begin{flushleft}
\fi
\begin{ttfamily}
desc: TStringList;\end{ttfamily}

\ifpdf
\end{flushleft}
\fi


\par  \label{ipkbuild.TPackInfo-pkName}
\index{pkName}
\item[\textbf{pkName}\hfill]
\ifpdf
\begin{flushleft}
\fi
\begin{ttfamily}
pkName: String;\end{ttfamily}

\ifpdf
\end{flushleft}
\fi


\par  \label{ipkbuild.TPackInfo-Maintainer}
\index{Maintainer}
\item[\textbf{Maintainer}\hfill]
\ifpdf
\begin{flushleft}
\fi
\begin{ttfamily}
Maintainer: String;\end{ttfamily}

\ifpdf
\end{flushleft}
\fi


\par  \label{ipkbuild.TPackInfo-Version}
\index{Version}
\item[\textbf{Version}\hfill]
\ifpdf
\begin{flushleft}
\fi
\begin{ttfamily}
Version: String;\end{ttfamily}

\ifpdf
\end{flushleft}
\fi


\par  \label{ipkbuild.TPackInfo-build}
\index{build}
\item[\textbf{build}\hfill]
\ifpdf
\begin{flushleft}
\fi
\begin{ttfamily}
build: TStringList;\end{ttfamily}

\ifpdf
\end{flushleft}
\fi


\par  \label{ipkbuild.TPackInfo-depDEB}
\index{depDEB}
\item[\textbf{depDEB}\hfill]
\ifpdf
\begin{flushleft}
\fi
\begin{ttfamily}
depDEB: TStringList;\end{ttfamily}

\ifpdf
\end{flushleft}
\fi


\par  \label{ipkbuild.TPackInfo-depRPM}
\index{depRPM}
\item[\textbf{depRPM}\hfill]
\ifpdf
\begin{flushleft}
\fi
\begin{ttfamily}
depRPM: TStringList;\end{ttfamily}

\ifpdf
\end{flushleft}
\fi


\par  \label{ipkbuild.TPackInfo-path}
\index{path}
\item[\textbf{path}\hfill]
\ifpdf
\begin{flushleft}
\fi
\begin{ttfamily}
path: String;\end{ttfamily}

\ifpdf
\end{flushleft}
\fi


\par  \label{ipkbuild.TPackInfo-out}
\index{out}
\item[\textbf{out}\hfill]
\ifpdf
\begin{flushleft}
\fi
\begin{ttfamily}
out: String;\end{ttfamily}

\ifpdf
\end{flushleft}
\fi


\par  \end{list}
\section{Functions and Procedures}
\ifpdf
\subsection*{\large{\textbf{BuildZIP}}\normalsize\hspace{1ex}\hrulefill}
\else
\subsection*{BuildZIP}
\fi
\label{ipkbuild-BuildZIP}
\index{BuildZIP}
\begin{list}{}{
\settowidth{\tmplength}{\textbf{Description}}
\setlength{\itemindent}{0cm}
\setlength{\listparindent}{0cm}
\setlength{\leftmargin}{\evensidemargin}
\addtolength{\leftmargin}{\tmplength}
\settowidth{\labelsep}{X}
\addtolength{\leftmargin}{\labelsep}
\setlength{\labelwidth}{\tmplength}
}
\item[\textbf{Declaration}\hfill]
\ifpdf
\begin{flushleft}
\fi
\begin{ttfamily}
procedure BuildZIP(azipfilename : string; afiles : TStringList);\end{ttfamily}

\ifpdf
\end{flushleft}
\fi

\par
\item[\textbf{Description}]
Builds a ZIP structure  \par
\item[\textbf{Parameters}]
\begin{description}
\item[azipfilename] Name of the ZIP file
\item[afiles] List of files the archive should contain
\end{description}


\end{list}
\ifpdf
\subsection*{\large{\textbf{FindChildNode}}\normalsize\hspace{1ex}\hrulefill}
\else
\subsection*{FindChildNode}
\fi
\label{ipkbuild-FindChildNode}
\index{FindChildNode}
\begin{list}{}{
\settowidth{\tmplength}{\textbf{Description}}
\setlength{\itemindent}{0cm}
\setlength{\listparindent}{0cm}
\setlength{\leftmargin}{\evensidemargin}
\addtolength{\leftmargin}{\tmplength}
\settowidth{\labelsep}{X}
\addtolength{\leftmargin}{\labelsep}
\setlength{\labelwidth}{\tmplength}
}
\item[\textbf{Declaration}\hfill]
\ifpdf
\begin{flushleft}
\fi
\begin{ttfamily}
function FindChildNode(dn: TDOMNode; n: String): TDOMNode;\end{ttfamily}

\ifpdf
\end{flushleft}
\fi

\par
\item[\textbf{Description}]
Helper method: Finds a child{-}node in a XML tree structure \par
\item[\textbf{Returns}]Found node as TDOMNode


\end{list}
\ifpdf
\subsection*{\large{\textbf{SetNode}}\normalsize\hspace{1ex}\hrulefill}
\else
\subsection*{SetNode}
\fi
\label{ipkbuild-SetNode}
\index{SetNode}
\begin{list}{}{
\settowidth{\tmplength}{\textbf{Description}}
\setlength{\itemindent}{0cm}
\setlength{\listparindent}{0cm}
\setlength{\leftmargin}{\evensidemargin}
\addtolength{\leftmargin}{\tmplength}
\settowidth{\labelsep}{X}
\addtolength{\leftmargin}{\labelsep}
\setlength{\labelwidth}{\tmplength}
}
\item[\textbf{Declaration}\hfill]
\ifpdf
\begin{flushleft}
\fi
\begin{ttfamily}
procedure SetNode(dn: TDOMNode; val: String);\end{ttfamily}

\ifpdf
\end{flushleft}
\fi

\par
\item[\textbf{Description}]
Set a node's value (Deprecated)

\end{list}
\ifpdf
\subsection*{\large{\textbf{CreateUpdateSource}}\normalsize\hspace{1ex}\hrulefill}
\else
\subsection*{CreateUpdateSource}
\fi
\label{ipkbuild-CreateUpdateSource}
\index{CreateUpdateSource}
\begin{list}{}{
\settowidth{\tmplength}{\textbf{Description}}
\setlength{\itemindent}{0cm}
\setlength{\listparindent}{0cm}
\setlength{\leftmargin}{\evensidemargin}
\addtolength{\leftmargin}{\tmplength}
\settowidth{\labelsep}{X}
\addtolength{\leftmargin}{\labelsep}
\setlength{\labelwidth}{\tmplength}
}
\item[\textbf{Declaration}\hfill]
\ifpdf
\begin{flushleft}
\fi
\begin{ttfamily}
procedure CreateUpdateSource(FName, path: String);\end{ttfamily}

\ifpdf
\end{flushleft}
\fi

\par
\item[\textbf{Description}]
Creates an IPK{-}Update{-}Source (IPKUS)  \par
\item[\textbf{Parameters}]
\begin{description}
\item[FName] Name of the IPS source file
\item[path] Path to an folder where the source should be created
\end{description}


\end{list}
\ifpdf
\subsection*{\large{\textbf{BuildApplication}}\normalsize\hspace{1ex}\hrulefill}
\else
\subsection*{BuildApplication}
\fi
\label{ipkbuild-BuildApplication}
\index{BuildApplication}
\begin{list}{}{
\settowidth{\tmplength}{\textbf{Description}}
\setlength{\itemindent}{0cm}
\setlength{\listparindent}{0cm}
\setlength{\leftmargin}{\evensidemargin}
\addtolength{\leftmargin}{\tmplength}
\settowidth{\labelsep}{X}
\addtolength{\leftmargin}{\labelsep}
\setlength{\labelwidth}{\tmplength}
}
\item[\textbf{Declaration}\hfill]
\ifpdf
\begin{flushleft}
\fi
\begin{ttfamily}
procedure BuildApplication(pk: TPackInfo);\end{ttfamily}

\ifpdf
\end{flushleft}
\fi

\par
\item[\textbf{Description}]
Builds the application from source using the {$<$}build{$>$} elements \par
\item[\textbf{Parameters}]
\begin{description}
\item[Connection] to an TPackageInfo
\end{description}


\end{list}
\ifpdf
\subsection*{\large{\textbf{BuildPackage}}\normalsize\hspace{1ex}\hrulefill}
\else
\subsection*{BuildPackage}
\fi
\label{ipkbuild-BuildPackage}
\index{BuildPackage}
\begin{list}{}{
\settowidth{\tmplength}{\textbf{Description}}
\setlength{\itemindent}{0cm}
\setlength{\listparindent}{0cm}
\setlength{\leftmargin}{\evensidemargin}
\addtolength{\leftmargin}{\tmplength}
\settowidth{\labelsep}{X}
\addtolength{\leftmargin}{\labelsep}
\setlength{\labelwidth}{\tmplength}
}
\item[\textbf{Declaration}\hfill]
\ifpdf
\begin{flushleft}
\fi
\begin{ttfamily}
procedure BuildPackage(fi: String;o:String;genbutton:Boolean=false);\end{ttfamily}

\ifpdf
\end{flushleft}
\fi

\par
\item[\textbf{Description}]
Builds an IPK package from an IPS source   \par
\item[\textbf{Parameters}]
\begin{description}
\item[fi] IPS source
\item[o] Output filename
\item[genbutton] Generate distro info button
\end{description}


\end{list}
\ifpdf
\subsection*{\large{\textbf{CreateLiCompButton}}\normalsize\hspace{1ex}\hrulefill}
\else
\subsection*{CreateLiCompButton}
\fi
\label{ipkbuild-CreateLiCompButton}
\index{CreateLiCompButton}
\begin{list}{}{
\settowidth{\tmplength}{\textbf{Description}}
\setlength{\itemindent}{0cm}
\setlength{\listparindent}{0cm}
\setlength{\leftmargin}{\evensidemargin}
\addtolength{\leftmargin}{\tmplength}
\settowidth{\labelsep}{X}
\addtolength{\leftmargin}{\labelsep}
\setlength{\labelwidth}{\tmplength}
}
\item[\textbf{Declaration}\hfill]
\ifpdf
\begin{flushleft}
\fi
\begin{ttfamily}
procedure CreateLiCompButton(dlist: TStringList;op: String);\end{ttfamily}

\ifpdf
\end{flushleft}
\fi

\par
\item[\textbf{Description}]
Creates the "Linux distribution compatible" button.  \par
\item[\textbf{Parameters}]
\begin{description}
\item[dlist] Names of the distributions that are compatible
\item[of] Name of the output PNG file
\end{description}


\end{list}
\chapter{Unit ipkhandle}
\label{ipkhandle}
\index{ipkhandle}
\section{Description}
Functions to handle IPK packages
\section{uses}
\begin{itemize}
\item \begin{ttfamily}Classes\end{ttfamily}\item \begin{ttfamily}SysUtils\end{ttfamily}\item \begin{ttfamily}IniFiles\end{ttfamily}\item \begin{ttfamily}utilities\end{ttfamily}(\ref{utilities})\item \begin{ttfamily}Forms\end{ttfamily}\item \begin{ttfamily}Process\end{ttfamily}\item \begin{ttfamily}trstrings\end{ttfamily}(\ref{trstrings})\end{itemize}
\section{Overview}
\begin{description}
\item[\texttt{UninstallIPKApp}]
\end{description}
\section{Functions and Procedures}
\ifpdf
\subsection*{\large{\textbf{UninstallIPKApp}}\normalsize\hspace{1ex}\hrulefill}
\else
\subsection*{UninstallIPKApp}
\fi
\label{ipkhandle-UninstallIPKApp}
\index{UninstallIPKApp}
\begin{list}{}{
\settowidth{\tmplength}{\textbf{Description}}
\setlength{\itemindent}{0cm}
\setlength{\listparindent}{0cm}
\setlength{\leftmargin}{\evensidemargin}
\addtolength{\leftmargin}{\tmplength}
\settowidth{\labelsep}{X}
\addtolength{\leftmargin}{\labelsep}
\setlength{\labelwidth}{\tmplength}
}
\item[\textbf{Declaration}\hfill]
\ifpdf
\begin{flushleft}
\fi
\begin{ttfamily}
function UninstallIPKApp(AppName, AppID: String; var Log: TStrings; fast: Boolean=false):Integer;\end{ttfamily}

\ifpdf
\end{flushleft}
\fi

\par
\item[\textbf{Description}]
Removes an IPK application     \par
\item[\textbf{Parameters}]
\begin{description}
\item[AppName] Name of the application, that should be uninstalled
\item[AppID] ID of the application
\item[Log] TStrings to get the log output
\item[fast] Does a quick uninstallation if is true
\end{description}
\item[\textbf{Returns}]Current progress of the operation


\end{list}
\section{Variables}
\ifpdf
\subsection*{\large{\textbf{pkit}}\normalsize\hspace{1ex}\hrulefill}
\else
\subsection*{pkit}
\fi
\label{ipkhandle-pkit}
\index{pkit}
\begin{list}{}{
\settowidth{\tmplength}{\textbf{Description}}
\setlength{\itemindent}{0cm}
\setlength{\listparindent}{0cm}
\setlength{\leftmargin}{\evensidemargin}
\addtolength{\leftmargin}{\tmplength}
\settowidth{\labelsep}{X}
\addtolength{\leftmargin}{\labelsep}
\setlength{\labelwidth}{\tmplength}
}
\item[\textbf{Declaration}\hfill]
\ifpdf
\begin{flushleft}
\fi
\begin{ttfamily}
pkit: String;\end{ttfamily}

\ifpdf
\end{flushleft}
\fi

\par
\item[\textbf{Description}]
Path to the Python PackageKit wrapper

\end{list}
\ifpdf
\subsection*{\large{\textbf{RegDir}}\normalsize\hspace{1ex}\hrulefill}
\else
\subsection*{RegDir}
\fi
\label{ipkhandle-RegDir}
\index{RegDir}
\begin{list}{}{
\settowidth{\tmplength}{\textbf{Description}}
\setlength{\itemindent}{0cm}
\setlength{\listparindent}{0cm}
\setlength{\leftmargin}{\evensidemargin}
\addtolength{\leftmargin}{\tmplength}
\settowidth{\labelsep}{X}
\addtolength{\leftmargin}{\labelsep}
\setlength{\labelwidth}{\tmplength}
}
\item[\textbf{Declaration}\hfill]
\ifpdf
\begin{flushleft}
\fi
\begin{ttfamily}
RegDir: String='/etc/lipa/app-reg/';\end{ttfamily}

\ifpdf
\end{flushleft}
\fi

\par
\item[\textbf{Description}]
Path to package registration

\end{list}
\chapter{Program lipa}
\label{lipa}
\index{lipa}
\section{Description}
Command{-}line application that handles IPK{-}packages
\section{uses}
\begin{itemize}
\item \begin{ttfamily}Classes\end{ttfamily}\item \begin{ttfamily}SysUtils\end{ttfamily}\item \begin{ttfamily}CustApp\end{ttfamily}\item \begin{ttfamily}Process\end{ttfamily}\item \begin{ttfamily}ipkbuild\end{ttfamily}(\ref{ipkbuild})\end{itemize}
\chapter{Program LiPkgCreator}
\label{LiPkgCreator}
\index{LiPkgCreator}
\section{uses}
\begin{itemize}
\item \begin{ttfamily}Interfaces\end{ttfamily}\item \begin{ttfamily}Forms\end{ttfamily}\item \begin{ttfamily}editor\end{ttfamily}(\ref{editor})\item \begin{ttfamily}fwiz\end{ttfamily}\item \begin{ttfamily}AbZipper\end{ttfamily}\item \begin{ttfamily}AbBase\end{ttfamily}\item \begin{ttfamily}AbArcTyp\end{ttfamily}\item \begin{ttfamily}AbZipKit\end{ttfamily}\item \begin{ttfamily}prjwizard\end{ttfamily}(\ref{prjwizard})\end{itemize}
\chapter{Program listallgo}
\label{listallgo}
\index{listallgo}
\section{Description}
GUI wizard application for IPK package installations
\section{uses}
\begin{itemize}
\item \begin{ttfamily}Interfaces\end{ttfamily}\item \begin{ttfamily}Forms\end{ttfamily}\item \begin{ttfamily}mainunit\end{ttfamily}(\ref{mainunit})\item \begin{ttfamily}dgunit\end{ttfamily}(\ref{dgunit})\item \begin{ttfamily}trstrings\end{ttfamily}(\ref{trstrings})\item \begin{ttfamily}LResources\end{ttfamily}\item \begin{ttfamily}ipkhandle\end{ttfamily}(\ref{ipkhandle})\item \begin{ttfamily}SysUtils\end{ttfamily}\item \begin{ttfamily}distri\end{ttfamily}(\ref{distri})\item \begin{ttfamily}utilities\end{ttfamily}(\ref{utilities})\end{itemize}
\chapter{Program listallmngr}
\label{listallmngr}
\index{listallmngr}
\section{Description}
Application that manages all installed applications
\section{uses}
\begin{itemize}
\item \begin{ttfamily}Interfaces\end{ttfamily}\item \begin{ttfamily}Forms\end{ttfamily}\item \begin{ttfamily}SysUtils\end{ttfamily}\item \begin{ttfamily}manager\end{ttfamily}(\ref{manager})\item \begin{ttfamily}settings\end{ttfamily}(\ref{settings})\item \begin{ttfamily}uninstall\end{ttfamily}(\ref{uninstall})\item \begin{ttfamily}pkgconvertdisp\end{ttfamily}(\ref{pkgconvertdisp})\item \begin{ttfamily}swcatalog\end{ttfamily}(\ref{swcatalog})\item \begin{ttfamily}LResources\end{ttfamily}\item \begin{ttfamily}ipkhandle\end{ttfamily}(\ref{ipkhandle})\end{itemize}
\chapter{Program litheme}
\label{litheme}
\index{litheme}
\section{uses}
\begin{itemize}
\item \begin{ttfamily}Interfaces\end{ttfamily}\item \begin{ttfamily}Forms\end{ttfamily}\item \begin{ttfamily}thinstall\end{ttfamily}(\ref{thinstall})\item \begin{ttfamily}LResources\end{ttfamily}\end{itemize}
\chapter{Program liupdate}
\label{liupdate}
\index{liupdate}
\section{Description}
Application liUpdater updates IPK{-}installed applications
\section{uses}
\begin{itemize}
\item \begin{ttfamily}Interfaces\end{ttfamily}\item \begin{ttfamily}Forms\end{ttfamily}\item \begin{ttfamily}mnupdate\end{ttfamily}(\ref{mnupdate})\item \begin{ttfamily}httpsend\end{ttfamily}(\ref{httpsend})\item \begin{ttfamily}updexec\end{ttfamily}(\ref{updexec})\item \begin{ttfamily}utilities\end{ttfamily}(\ref{utilities})\item \begin{ttfamily}ipkhandle\end{ttfamily}(\ref{ipkhandle})\item \begin{ttfamily}ldunit\end{ttfamily}\item \begin{ttfamily}LResources\end{ttfamily}\end{itemize}
\chapter{Unit mainunit}
\label{mainunit}
\index{mainunit}
\section{Description}
This unit contains the code for the graphical installation of standard IPK{-}packages
\section{uses}
\begin{itemize}
\item \begin{ttfamily}Classes\end{ttfamily}\item \begin{ttfamily}SysUtils\end{ttfamily}\item \begin{ttfamily}LResources\end{ttfamily}\item \begin{ttfamily}Forms\end{ttfamily}\item \begin{ttfamily}Controls\end{ttfamily}\item \begin{ttfamily}Graphics\end{ttfamily}\item \begin{ttfamily}Dialogs\end{ttfamily}\item \begin{ttfamily}ComCtrls\end{ttfamily}\item \begin{ttfamily}AbUnZper\end{ttfamily}\item \begin{ttfamily}AbArcTyp\end{ttfamily}\item \begin{ttfamily}StdCtrls\end{ttfamily}\item \begin{ttfamily}IniFiles\end{ttfamily}\item \begin{ttfamily}FileUtil\end{ttfamily}\item \begin{ttfamily}ExtCtrls\end{ttfamily}\item \begin{ttfamily}process\end{ttfamily}\item \begin{ttfamily}Buttons\end{ttfamily}\item \begin{ttfamily}LCLType\end{ttfamily}\item \begin{ttfamily}MD5\end{ttfamily}\item \begin{ttfamily}LCLIntf\end{ttfamily}\item \begin{ttfamily}distri\end{ttfamily}(\ref{distri})\item \begin{ttfamily}utilities\end{ttfamily}(\ref{utilities})\item \begin{ttfamily}HTTPSend\end{ttfamily}(\ref{httpsend})\item \begin{ttfamily}blcksock\end{ttfamily}\item \begin{ttfamily}ftpsend\end{ttfamily}(\ref{ftpsend})\item \begin{ttfamily}trstrings\end{ttfamily}(\ref{trstrings})\item \begin{ttfamily}translations\end{ttfamily}\item \begin{ttfamily}gettext\end{ttfamily}\item \begin{ttfamily}gtk2\end{ttfamily}\item \begin{ttfamily}gtkint\end{ttfamily}\item \begin{ttfamily}gtkdef\end{ttfamily}\item \begin{ttfamily}gtkproc\end{ttfamily}\item \begin{ttfamily}XMLRead\end{ttfamily}\item \begin{ttfamily}DOM\end{ttfamily}\item \begin{ttfamily}SynEdit\end{ttfamily}\item \begin{ttfamily}xtypefm\end{ttfamily}(\ref{xtypefm})\item \begin{ttfamily}ipkhandle\end{ttfamily}(\ref{ipkhandle})\end{itemize}
\section{Overview}
\begin{description}
\item[\texttt{\begin{ttfamily}TIWizFrm\end{ttfamily} Class}]
\end{description}
\section{Classes, Interfaces, Objects and Records}
\ifpdf
\subsection*{\large{\textbf{TIWizFrm Class}}\normalsize\hspace{1ex}\hrulefill}
\else
\subsection*{TIWizFrm Class}
\fi
\label{mainunit.TIWizFrm}
\index{TIWizFrm}
\subsubsection*{\large{\textbf{Hierarchy}}\normalsize\hspace{1ex}\hfill}
TIWizFrm {$>$} TForm
\subsubsection*{\large{\textbf{Description}}\normalsize\hspace{1ex}\hfill}
The installer wizard window\subsubsection*{\large{\textbf{Fields}}\normalsize\hspace{1ex}\hfill}
\begin{list}{}{
\settowidth{\tmplength}{\textbf{GetOutPutTimer}}
\setlength{\itemindent}{0cm}
\setlength{\listparindent}{0cm}
\setlength{\leftmargin}{\evensidemargin}
\addtolength{\leftmargin}{\tmplength}
\settowidth{\labelsep}{X}
\addtolength{\leftmargin}{\labelsep}
\setlength{\labelwidth}{\tmplength}
}
\label{mainunit.TIWizFrm-AbortBtn1}
\index{AbortBtn1}
\item[\textbf{AbortBtn1}\hfill]
\ifpdf
\begin{flushleft}
\fi
\begin{ttfamily}
public AbortBtn1: TBitBtn;\end{ttfamily}

\ifpdf
\end{flushleft}
\fi


\par  \label{mainunit.TIWizFrm-Button1}
\index{Button1}
\item[\textbf{Button1}\hfill]
\ifpdf
\begin{flushleft}
\fi
\begin{ttfamily}
public Button1: TBitBtn;\end{ttfamily}

\ifpdf
\end{flushleft}
\fi


\par  \label{mainunit.TIWizFrm-btn_sendinput}
\index{btn{\_}sendinput}
\item[\textbf{btn{\_}sendinput}\hfill]
\ifpdf
\begin{flushleft}
\fi
\begin{ttfamily}
public btn{\_}sendinput: TButton;\end{ttfamily}

\ifpdf
\end{flushleft}
\fi


\par  \label{mainunit.TIWizFrm-Button5}
\index{Button5}
\item[\textbf{Button5}\hfill]
\ifpdf
\begin{flushleft}
\fi
\begin{ttfamily}
public Button5: TBitBtn;\end{ttfamily}

\ifpdf
\end{flushleft}
\fi


\par  \label{mainunit.TIWizFrm-CheckBox1}
\index{CheckBox1}
\item[\textbf{CheckBox1}\hfill]
\ifpdf
\begin{flushleft}
\fi
\begin{ttfamily}
public CheckBox1: TCheckBox;\end{ttfamily}

\ifpdf
\end{flushleft}
\fi


\par  \label{mainunit.TIWizFrm-CbExecApp}
\index{CbExecApp}
\item[\textbf{CbExecApp}\hfill]
\ifpdf
\begin{flushleft}
\fi
\begin{ttfamily}
public CbExecApp: TCheckBox;\end{ttfamily}

\ifpdf
\end{flushleft}
\fi


\par  \label{mainunit.TIWizFrm-Edit1}
\index{Edit1}
\item[\textbf{Edit1}\hfill]
\ifpdf
\begin{flushleft}
\fi
\begin{ttfamily}
public Edit1: TEdit;\end{ttfamily}

\ifpdf
\end{flushleft}
\fi


\par  \label{mainunit.TIWizFrm-ExProgress}
\index{ExProgress}
\item[\textbf{ExProgress}\hfill]
\ifpdf
\begin{flushleft}
\fi
\begin{ttfamily}
public ExProgress: TProgressBar;\end{ttfamily}

\ifpdf
\end{flushleft}
\fi


\par  \label{mainunit.TIWizFrm-FinBtn1}
\index{FinBtn1}
\item[\textbf{FinBtn1}\hfill]
\ifpdf
\begin{flushleft}
\fi
\begin{ttfamily}
public FinBtn1: TBitBtn;\end{ttfamily}

\ifpdf
\end{flushleft}
\fi


\par  \label{mainunit.TIWizFrm-GetOutPutTimer}
\index{GetOutPutTimer}
\item[\textbf{GetOutPutTimer}\hfill]
\ifpdf
\begin{flushleft}
\fi
\begin{ttfamily}
public GetOutPutTimer: TIdleTimer;\end{ttfamily}

\ifpdf
\end{flushleft}
\fi


\par  \label{mainunit.TIWizFrm-GroupBox1}
\index{GroupBox1}
\item[\textbf{GroupBox1}\hfill]
\ifpdf
\begin{flushleft}
\fi
\begin{ttfamily}
public GroupBox1: TGroupBox;\end{ttfamily}

\ifpdf
\end{flushleft}
\fi


\par  \label{mainunit.TIWizFrm-Image2}
\index{Image2}
\item[\textbf{Image2}\hfill]
\ifpdf
\begin{flushleft}
\fi
\begin{ttfamily}
public Image2: TImage;\end{ttfamily}

\ifpdf
\end{flushleft}
\fi


\par  \label{mainunit.TIWizFrm-InfoMemo}
\index{InfoMemo}
\item[\textbf{InfoMemo}\hfill]
\ifpdf
\begin{flushleft}
\fi
\begin{ttfamily}
public InfoMemo: TMemo;\end{ttfamily}

\ifpdf
\end{flushleft}
\fi


\par  \label{mainunit.TIWizFrm-InsProgress}
\index{InsProgress}
\item[\textbf{InsProgress}\hfill]
\ifpdf
\begin{flushleft}
\fi
\begin{ttfamily}
public InsProgress: TProgressBar;\end{ttfamily}

\ifpdf
\end{flushleft}
\fi


\par  \label{mainunit.TIWizFrm-Label1}
\index{Label1}
\item[\textbf{Label1}\hfill]
\ifpdf
\begin{flushleft}
\fi
\begin{ttfamily}
public Label1: TLabel;\end{ttfamily}

\ifpdf
\end{flushleft}
\fi


\par  \label{mainunit.TIWizFrm-Label10}
\index{Label10}
\item[\textbf{Label10}\hfill]
\ifpdf
\begin{flushleft}
\fi
\begin{ttfamily}
public Label10: TLabel;\end{ttfamily}

\ifpdf
\end{flushleft}
\fi


\par  \label{mainunit.TIWizFrm-Label11}
\index{Label11}
\item[\textbf{Label11}\hfill]
\ifpdf
\begin{flushleft}
\fi
\begin{ttfamily}
public Label11: TLabel;\end{ttfamily}

\ifpdf
\end{flushleft}
\fi


\par  \label{mainunit.TIWizFrm-Label12}
\index{Label12}
\item[\textbf{Label12}\hfill]
\ifpdf
\begin{flushleft}
\fi
\begin{ttfamily}
public Label12: TLabel;\end{ttfamily}

\ifpdf
\end{flushleft}
\fi


\par  \label{mainunit.TIWizFrm-Label13}
\index{Label13}
\item[\textbf{Label13}\hfill]
\ifpdf
\begin{flushleft}
\fi
\begin{ttfamily}
public Label13: TLabel;\end{ttfamily}

\ifpdf
\end{flushleft}
\fi


\par  \label{mainunit.TIWizFrm-Label14}
\index{Label14}
\item[\textbf{Label14}\hfill]
\ifpdf
\begin{flushleft}
\fi
\begin{ttfamily}
public Label14: TLabel;\end{ttfamily}

\ifpdf
\end{flushleft}
\fi


\par  \label{mainunit.TIWizFrm-LblTestMode}
\index{LblTestMode}
\item[\textbf{LblTestMode}\hfill]
\ifpdf
\begin{flushleft}
\fi
\begin{ttfamily}
public LblTestMode: TLabel;\end{ttfamily}

\ifpdf
\end{flushleft}
\fi


\par  \label{mainunit.TIWizFrm-Label16}
\index{Label16}
\item[\textbf{Label16}\hfill]
\ifpdf
\begin{flushleft}
\fi
\begin{ttfamily}
public Label16: TLabel;\end{ttfamily}

\ifpdf
\end{flushleft}
\fi


\par  \label{mainunit.TIWizFrm-Label2}
\index{Label2}
\item[\textbf{Label2}\hfill]
\ifpdf
\begin{flushleft}
\fi
\begin{ttfamily}
public Label2: TLabel;\end{ttfamily}

\ifpdf
\end{flushleft}
\fi


\par  \label{mainunit.TIWizFrm-Label3}
\index{Label3}
\item[\textbf{Label3}\hfill]
\ifpdf
\begin{flushleft}
\fi
\begin{ttfamily}
public Label3: TLabel;\end{ttfamily}

\ifpdf
\end{flushleft}
\fi


\par  \label{mainunit.TIWizFrm-Label4}
\index{Label4}
\item[\textbf{Label4}\hfill]
\ifpdf
\begin{flushleft}
\fi
\begin{ttfamily}
public Label4: TLabel;\end{ttfamily}

\ifpdf
\end{flushleft}
\fi


\par  \label{mainunit.TIWizFrm-Label5}
\index{Label5}
\item[\textbf{Label5}\hfill]
\ifpdf
\begin{flushleft}
\fi
\begin{ttfamily}
public Label5: TLabel;\end{ttfamily}

\ifpdf
\end{flushleft}
\fi


\par  \label{mainunit.TIWizFrm-Label6}
\index{Label6}
\item[\textbf{Label6}\hfill]
\ifpdf
\begin{flushleft}
\fi
\begin{ttfamily}
public Label6: TLabel;\end{ttfamily}

\ifpdf
\end{flushleft}
\fi


\par  \label{mainunit.TIWizFrm-Label7}
\index{Label7}
\item[\textbf{Label7}\hfill]
\ifpdf
\begin{flushleft}
\fi
\begin{ttfamily}
public Label7: TLabel;\end{ttfamily}

\ifpdf
\end{flushleft}
\fi


\par  \label{mainunit.TIWizFrm-Label8}
\index{Label8}
\item[\textbf{Label8}\hfill]
\ifpdf
\begin{flushleft}
\fi
\begin{ttfamily}
public Label8: TLabel;\end{ttfamily}

\ifpdf
\end{flushleft}
\fi


\par  \label{mainunit.TIWizFrm-Label9}
\index{Label9}
\item[\textbf{Label9}\hfill]
\ifpdf
\begin{flushleft}
\fi
\begin{ttfamily}
public Label9: TLabel;\end{ttfamily}

\ifpdf
\end{flushleft}
\fi


\par  \label{mainunit.TIWizFrm-ListBox1}
\index{ListBox1}
\item[\textbf{ListBox1}\hfill]
\ifpdf
\begin{flushleft}
\fi
\begin{ttfamily}
public ListBox1: TListBox;\end{ttfamily}

\ifpdf
\end{flushleft}
\fi


\par  \label{mainunit.TIWizFrm-Memo1}
\index{Memo1}
\item[\textbf{Memo1}\hfill]
\ifpdf
\begin{flushleft}
\fi
\begin{ttfamily}
public Memo1: TMemo;\end{ttfamily}

\ifpdf
\end{flushleft}
\fi


\par  \label{mainunit.TIWizFrm-LicMemo}
\index{LicMemo}
\item[\textbf{LicMemo}\hfill]
\ifpdf
\begin{flushleft}
\fi
\begin{ttfamily}
public LicMemo: TMemo;\end{ttfamily}

\ifpdf
\end{flushleft}
\fi


\par  \label{mainunit.TIWizFrm-Notebook1}
\index{Notebook1}
\item[\textbf{Notebook1}\hfill]
\ifpdf
\begin{flushleft}
\fi
\begin{ttfamily}
public Notebook1: TNotebook;\end{ttfamily}

\ifpdf
\end{flushleft}
\fi


\par  \label{mainunit.TIWizFrm-OpenDialog1}
\index{OpenDialog1}
\item[\textbf{OpenDialog1}\hfill]
\ifpdf
\begin{flushleft}
\fi
\begin{ttfamily}
public OpenDialog1: TOpenDialog;\end{ttfamily}

\ifpdf
\end{flushleft}
\fi


\par  \label{mainunit.TIWizFrm-IMPage}
\index{IMPage}
\item[\textbf{IMPage}\hfill]
\ifpdf
\begin{flushleft}
\fi
\begin{ttfamily}
public IMPage: TPage;\end{ttfamily}

\ifpdf
\end{flushleft}
\fi


\par  \label{mainunit.TIWizFrm-ModeGroup}
\index{ModeGroup}
\item[\textbf{ModeGroup}\hfill]
\ifpdf
\begin{flushleft}
\fi
\begin{ttfamily}
public ModeGroup: TRadioGroup;\end{ttfamily}

\ifpdf
\end{flushleft}
\fi


\par  \label{mainunit.TIWizFrm-WPage}
\index{WPage}
\item[\textbf{WPage}\hfill]
\ifpdf
\begin{flushleft}
\fi
\begin{ttfamily}
public WPage: TPage;\end{ttfamily}

\ifpdf
\end{flushleft}
\fi


\par  \label{mainunit.TIWizFrm-DPage}
\index{DPage}
\item[\textbf{DPage}\hfill]
\ifpdf
\begin{flushleft}
\fi
\begin{ttfamily}
public DPage: TPage;\end{ttfamily}

\ifpdf
\end{flushleft}
\fi


\par  \label{mainunit.TIWizFrm-LPage}
\index{LPage}
\item[\textbf{LPage}\hfill]
\ifpdf
\begin{flushleft}
\fi
\begin{ttfamily}
public LPage: TPage;\end{ttfamily}

\ifpdf
\end{flushleft}
\fi


\par  \label{mainunit.TIWizFrm-IPage}
\index{IPage}
\item[\textbf{IPage}\hfill]
\ifpdf
\begin{flushleft}
\fi
\begin{ttfamily}
public IPage: TPage;\end{ttfamily}

\ifpdf
\end{flushleft}
\fi


\par  \label{mainunit.TIWizFrm-FinPage}
\index{FinPage}
\item[\textbf{FinPage}\hfill]
\ifpdf
\begin{flushleft}
\fi
\begin{ttfamily}
public FinPage: TPage;\end{ttfamily}

\ifpdf
\end{flushleft}
\fi


\par  \label{mainunit.TIWizFrm-Panel1}
\index{Panel1}
\item[\textbf{Panel1}\hfill]
\ifpdf
\begin{flushleft}
\fi
\begin{ttfamily}
public Panel1: TPanel;\end{ttfamily}

\ifpdf
\end{flushleft}
\fi


\par  \label{mainunit.TIWizFrm-Process1}
\index{Process1}
\item[\textbf{Process1}\hfill]
\ifpdf
\begin{flushleft}
\fi
\begin{ttfamily}
public Process1: TProcess;\end{ttfamily}

\ifpdf
\end{flushleft}
\fi


\par  \label{mainunit.TIWizFrm-RadioButton1}
\index{RadioButton1}
\item[\textbf{RadioButton1}\hfill]
\ifpdf
\begin{flushleft}
\fi
\begin{ttfamily}
public RadioButton1: TRadioButton;\end{ttfamily}

\ifpdf
\end{flushleft}
\fi


\par  \label{mainunit.TIWizFrm-RadioButton2}
\index{RadioButton2}
\item[\textbf{RadioButton2}\hfill]
\ifpdf
\begin{flushleft}
\fi
\begin{ttfamily}
public RadioButton2: TRadioButton;\end{ttfamily}

\ifpdf
\end{flushleft}
\fi


\par  \label{mainunit.TIWizFrm-IAppName}
\index{IAppName}
\item[\textbf{IAppName}\hfill]
\ifpdf
\begin{flushleft}
\fi
\begin{ttfamily}
public IAppName: String;\end{ttfamily}

\ifpdf
\end{flushleft}
\fi


\par Information about the application that should be installed\label{mainunit.TIWizFrm-IAppVersion}
\index{IAppVersion}
\item[\textbf{IAppVersion}\hfill]
\ifpdf
\begin{flushleft}
\fi
\begin{ttfamily}
public IAppVersion: String;\end{ttfamily}

\ifpdf
\end{flushleft}
\fi


\par Information about the application that should be installed\label{mainunit.TIWizFrm-IAppCMD}
\index{IAppCMD}
\item[\textbf{IAppCMD}\hfill]
\ifpdf
\begin{flushleft}
\fi
\begin{ttfamily}
public IAppCMD: String;\end{ttfamily}

\ifpdf
\end{flushleft}
\fi


\par Information about the application that should be installed\label{mainunit.TIWizFrm-IAuthor}
\index{IAuthor}
\item[\textbf{IAuthor}\hfill]
\ifpdf
\begin{flushleft}
\fi
\begin{ttfamily}
public IAuthor: String;\end{ttfamily}

\ifpdf
\end{flushleft}
\fi


\par Information about the application that should be installed\label{mainunit.TIWizFrm-DescFile}
\index{DescFile}
\item[\textbf{DescFile}\hfill]
\ifpdf
\begin{flushleft}
\fi
\begin{ttfamily}
public DescFile: String;\end{ttfamily}

\ifpdf
\end{flushleft}
\fi


\par Information about the application that should be installed\label{mainunit.TIWizFrm-ShDesc}
\index{ShDesc}
\item[\textbf{ShDesc}\hfill]
\ifpdf
\begin{flushleft}
\fi
\begin{ttfamily}
public ShDesc: String;\end{ttfamily}

\ifpdf
\end{flushleft}
\fi


\par Information about the application that should be installed\label{mainunit.TIWizFrm-LicenseFile}
\index{LicenseFile}
\item[\textbf{LicenseFile}\hfill]
\ifpdf
\begin{flushleft}
\fi
\begin{ttfamily}
public LicenseFile: String;\end{ttfamily}

\ifpdf
\end{flushleft}
\fi


\par Information about the application that should be installed\label{mainunit.TIWizFrm-PkgName}
\index{PkgName}
\item[\textbf{PkgName}\hfill]
\ifpdf
\begin{flushleft}
\fi
\begin{ttfamily}
public PkgName: String;\end{ttfamily}

\ifpdf
\end{flushleft}
\fi


\par Information about the current package\label{mainunit.TIWizFrm-pID}
\index{pID}
\item[\textbf{pID}\hfill]
\ifpdf
\begin{flushleft}
\fi
\begin{ttfamily}
public pID: String;\end{ttfamily}

\ifpdf
\end{flushleft}
\fi


\par Information about the current package\label{mainunit.TIWizFrm-idName}
\index{idName}
\item[\textbf{idName}\hfill]
\ifpdf
\begin{flushleft}
\fi
\begin{ttfamily}
public idName: String;\end{ttfamily}

\ifpdf
\end{flushleft}
\fi


\par Information about the current package\label{mainunit.TIWizFrm-AType}
\index{AType}
\item[\textbf{AType}\hfill]
\ifpdf
\begin{flushleft}
\fi
\begin{ttfamily}
public AType: String;\end{ttfamily}

\ifpdf
\end{flushleft}
\fi


\par Information about the current package\label{mainunit.TIWizFrm-Dependencies}
\index{Dependencies}
\item[\textbf{Dependencies}\hfill]
\ifpdf
\begin{flushleft}
\fi
\begin{ttfamily}
public Dependencies: TStringList;\end{ttfamily}

\ifpdf
\end{flushleft}
\fi


\par Dependency list\label{mainunit.TIWizFrm-Profiles}
\index{Profiles}
\item[\textbf{Profiles}\hfill]
\ifpdf
\begin{flushleft}
\fi
\begin{ttfamily}
public Profiles: TStringList;\end{ttfamily}

\ifpdf
\end{flushleft}
\fi


\par Profiles list\label{mainunit.TIWizFrm-FFileInfo}
\index{FFileInfo}
\item[\textbf{FFileInfo}\hfill]
\ifpdf
\begin{flushleft}
\fi
\begin{ttfamily}
public FFileInfo: String;\end{ttfamily}

\ifpdf
\end{flushleft}
\fi


\par File information\label{mainunit.TIWizFrm-MDHash}
\index{MDHash}
\item[\textbf{MDHash}\hfill]
\ifpdf
\begin{flushleft}
\fi
\begin{ttfamily}
public MDHash: String;\end{ttfamily}

\ifpdf
\end{flushleft}
\fi


\par Current MD5 Hash\label{mainunit.TIWizFrm-USource}
\index{USource}
\item[\textbf{USource}\hfill]
\ifpdf
\begin{flushleft}
\fi
\begin{ttfamily}
public USource: String;\end{ttfamily}

\ifpdf
\end{flushleft}
\fi


\par Update source\label{mainunit.TIWizFrm-IconPath}
\index{IconPath}
\item[\textbf{IconPath}\hfill]
\ifpdf
\begin{flushleft}
\fi
\begin{ttfamily}
public IconPath: String;\end{ttfamily}

\ifpdf
\end{flushleft}
\fi


\par Path of the package icon\label{mainunit.TIWizFrm-ExecA}
\index{ExecA}
\item[\textbf{ExecA}\hfill]
\ifpdf
\begin{flushleft}
\fi
\begin{ttfamily}
public ExecA: String;\end{ttfamily}

\ifpdf
\end{flushleft}
\fi


\par Execute external applications that are linked in the IPK{-}file\label{mainunit.TIWizFrm-ExecB}
\index{ExecB}
\item[\textbf{ExecB}\hfill]
\ifpdf
\begin{flushleft}
\fi
\begin{ttfamily}
public ExecB: String;\end{ttfamily}

\ifpdf
\end{flushleft}
\fi


\par Execute external applications that are linked in the IPK{-}file\label{mainunit.TIWizFrm-ExecX}
\index{ExecX}
\item[\textbf{ExecX}\hfill]
\ifpdf
\begin{flushleft}
\fi
\begin{ttfamily}
public ExecX: String;\end{ttfamily}

\ifpdf
\end{flushleft}
\fi


\par Execute external applications that are linked in the IPK{-}file\end{list}
\subsubsection*{\large{\textbf{Methods}}\normalsize\hspace{1ex}\hfill}
\paragraph*{AbortBtn1Click}\hspace*{\fill}

\label{mainunit.TIWizFrm-AbortBtn1Click}
\index{AbortBtn1Click}
\begin{list}{}{
\settowidth{\tmplength}{\textbf{Description}}
\setlength{\itemindent}{0cm}
\setlength{\listparindent}{0cm}
\setlength{\leftmargin}{\evensidemargin}
\addtolength{\leftmargin}{\tmplength}
\settowidth{\labelsep}{X}
\addtolength{\leftmargin}{\labelsep}
\setlength{\labelwidth}{\tmplength}
}
\item[\textbf{Declaration}\hfill]
\ifpdf
\begin{flushleft}
\fi
\begin{ttfamily}
public procedure AbortBtn1Click(Sender: TObject);\end{ttfamily}

\ifpdf
\end{flushleft}
\fi

\end{list}
\paragraph*{Button1Click}\hspace*{\fill}

\label{mainunit.TIWizFrm-Button1Click}
\index{Button1Click}
\begin{list}{}{
\settowidth{\tmplength}{\textbf{Description}}
\setlength{\itemindent}{0cm}
\setlength{\listparindent}{0cm}
\setlength{\leftmargin}{\evensidemargin}
\addtolength{\leftmargin}{\tmplength}
\settowidth{\labelsep}{X}
\addtolength{\leftmargin}{\labelsep}
\setlength{\labelwidth}{\tmplength}
}
\item[\textbf{Declaration}\hfill]
\ifpdf
\begin{flushleft}
\fi
\begin{ttfamily}
public procedure Button1Click(Sender: TObject);\end{ttfamily}

\ifpdf
\end{flushleft}
\fi

\end{list}
\paragraph*{Button5Click}\hspace*{\fill}

\label{mainunit.TIWizFrm-Button5Click}
\index{Button5Click}
\begin{list}{}{
\settowidth{\tmplength}{\textbf{Description}}
\setlength{\itemindent}{0cm}
\setlength{\listparindent}{0cm}
\setlength{\leftmargin}{\evensidemargin}
\addtolength{\leftmargin}{\tmplength}
\settowidth{\labelsep}{X}
\addtolength{\leftmargin}{\labelsep}
\setlength{\labelwidth}{\tmplength}
}
\item[\textbf{Declaration}\hfill]
\ifpdf
\begin{flushleft}
\fi
\begin{ttfamily}
public procedure Button5Click(Sender: TObject);\end{ttfamily}

\ifpdf
\end{flushleft}
\fi

\end{list}
\paragraph*{CheckBox1Change}\hspace*{\fill}

\label{mainunit.TIWizFrm-CheckBox1Change}
\index{CheckBox1Change}
\begin{list}{}{
\settowidth{\tmplength}{\textbf{Description}}
\setlength{\itemindent}{0cm}
\setlength{\listparindent}{0cm}
\setlength{\leftmargin}{\evensidemargin}
\addtolength{\leftmargin}{\tmplength}
\settowidth{\labelsep}{X}
\addtolength{\leftmargin}{\labelsep}
\setlength{\labelwidth}{\tmplength}
}
\item[\textbf{Declaration}\hfill]
\ifpdf
\begin{flushleft}
\fi
\begin{ttfamily}
public procedure CheckBox1Change(Sender: TObject);\end{ttfamily}

\ifpdf
\end{flushleft}
\fi

\end{list}
\paragraph*{FinBtn1Click}\hspace*{\fill}

\label{mainunit.TIWizFrm-FinBtn1Click}
\index{FinBtn1Click}
\begin{list}{}{
\settowidth{\tmplength}{\textbf{Description}}
\setlength{\itemindent}{0cm}
\setlength{\listparindent}{0cm}
\setlength{\leftmargin}{\evensidemargin}
\addtolength{\leftmargin}{\tmplength}
\settowidth{\labelsep}{X}
\addtolength{\leftmargin}{\labelsep}
\setlength{\labelwidth}{\tmplength}
}
\item[\textbf{Declaration}\hfill]
\ifpdf
\begin{flushleft}
\fi
\begin{ttfamily}
public procedure FinBtn1Click(Sender: TObject);\end{ttfamily}

\ifpdf
\end{flushleft}
\fi

\end{list}
\paragraph*{FormCreate}\hspace*{\fill}

\label{mainunit.TIWizFrm-FormCreate}
\index{FormCreate}
\begin{list}{}{
\settowidth{\tmplength}{\textbf{Description}}
\setlength{\itemindent}{0cm}
\setlength{\listparindent}{0cm}
\setlength{\leftmargin}{\evensidemargin}
\addtolength{\leftmargin}{\tmplength}
\settowidth{\labelsep}{X}
\addtolength{\leftmargin}{\labelsep}
\setlength{\labelwidth}{\tmplength}
}
\item[\textbf{Declaration}\hfill]
\ifpdf
\begin{flushleft}
\fi
\begin{ttfamily}
public procedure FormCreate(Sender: TObject);\end{ttfamily}

\ifpdf
\end{flushleft}
\fi

\end{list}
\paragraph*{FormDestroy}\hspace*{\fill}

\label{mainunit.TIWizFrm-FormDestroy}
\index{FormDestroy}
\begin{list}{}{
\settowidth{\tmplength}{\textbf{Description}}
\setlength{\itemindent}{0cm}
\setlength{\listparindent}{0cm}
\setlength{\leftmargin}{\evensidemargin}
\addtolength{\leftmargin}{\tmplength}
\settowidth{\labelsep}{X}
\addtolength{\leftmargin}{\labelsep}
\setlength{\labelwidth}{\tmplength}
}
\item[\textbf{Declaration}\hfill]
\ifpdf
\begin{flushleft}
\fi
\begin{ttfamily}
public procedure FormDestroy(Sender: TObject);\end{ttfamily}

\ifpdf
\end{flushleft}
\fi

\end{list}
\paragraph*{FormShow}\hspace*{\fill}

\label{mainunit.TIWizFrm-FormShow}
\index{FormShow}
\begin{list}{}{
\settowidth{\tmplength}{\textbf{Description}}
\setlength{\itemindent}{0cm}
\setlength{\listparindent}{0cm}
\setlength{\leftmargin}{\evensidemargin}
\addtolength{\leftmargin}{\tmplength}
\settowidth{\labelsep}{X}
\addtolength{\leftmargin}{\labelsep}
\setlength{\labelwidth}{\tmplength}
}
\item[\textbf{Declaration}\hfill]
\ifpdf
\begin{flushleft}
\fi
\begin{ttfamily}
public procedure FormShow(Sender: TObject);\end{ttfamily}

\ifpdf
\end{flushleft}
\fi

\end{list}
\paragraph*{GetOutputTimerTimer}\hspace*{\fill}

\label{mainunit.TIWizFrm-GetOutputTimerTimer}
\index{GetOutputTimerTimer}
\begin{list}{}{
\settowidth{\tmplength}{\textbf{Description}}
\setlength{\itemindent}{0cm}
\setlength{\listparindent}{0cm}
\setlength{\leftmargin}{\evensidemargin}
\addtolength{\leftmargin}{\tmplength}
\settowidth{\labelsep}{X}
\addtolength{\leftmargin}{\labelsep}
\setlength{\labelwidth}{\tmplength}
}
\item[\textbf{Declaration}\hfill]
\ifpdf
\begin{flushleft}
\fi
\begin{ttfamily}
public procedure GetOutputTimerTimer(Sender: TObject);\end{ttfamily}

\ifpdf
\end{flushleft}
\fi

\end{list}
\paragraph*{btn{\_}sendinputClick}\hspace*{\fill}

\label{mainunit.TIWizFrm-btn_sendinputClick}
\index{btn{\_}sendinputClick}
\begin{list}{}{
\settowidth{\tmplength}{\textbf{Description}}
\setlength{\itemindent}{0cm}
\setlength{\listparindent}{0cm}
\setlength{\leftmargin}{\evensidemargin}
\addtolength{\leftmargin}{\tmplength}
\settowidth{\labelsep}{X}
\addtolength{\leftmargin}{\labelsep}
\setlength{\labelwidth}{\tmplength}
}
\item[\textbf{Declaration}\hfill]
\ifpdf
\begin{flushleft}
\fi
\begin{ttfamily}
public procedure btn{\_}sendinputClick(Sender: TObject);\end{ttfamily}

\ifpdf
\end{flushleft}
\fi

\end{list}
\paragraph*{RadioButton1Change}\hspace*{\fill}

\label{mainunit.TIWizFrm-RadioButton1Change}
\index{RadioButton1Change}
\begin{list}{}{
\settowidth{\tmplength}{\textbf{Description}}
\setlength{\itemindent}{0cm}
\setlength{\listparindent}{0cm}
\setlength{\leftmargin}{\evensidemargin}
\addtolength{\leftmargin}{\tmplength}
\settowidth{\labelsep}{X}
\addtolength{\leftmargin}{\labelsep}
\setlength{\labelwidth}{\tmplength}
}
\item[\textbf{Declaration}\hfill]
\ifpdf
\begin{flushleft}
\fi
\begin{ttfamily}
public procedure RadioButton1Change(Sender: TObject);\end{ttfamily}

\ifpdf
\end{flushleft}
\fi

\end{list}
\paragraph*{RadioButton2Change}\hspace*{\fill}

\label{mainunit.TIWizFrm-RadioButton2Change}
\index{RadioButton2Change}
\begin{list}{}{
\settowidth{\tmplength}{\textbf{Description}}
\setlength{\itemindent}{0cm}
\setlength{\listparindent}{0cm}
\setlength{\leftmargin}{\evensidemargin}
\addtolength{\leftmargin}{\tmplength}
\settowidth{\labelsep}{X}
\addtolength{\leftmargin}{\labelsep}
\setlength{\labelwidth}{\tmplength}
}
\item[\textbf{Declaration}\hfill]
\ifpdf
\begin{flushleft}
\fi
\begin{ttfamily}
public procedure RadioButton2Change(Sender: TObject);\end{ttfamily}

\ifpdf
\end{flushleft}
\fi

\end{list}
\section{Constants}
\ifpdf
\subsection*{\large{\textbf{lp}}\normalsize\hspace{1ex}\hrulefill}
\else
\subsection*{lp}
\fi
\label{mainunit-lp}
\index{lp}
\begin{list}{}{
\settowidth{\tmplength}{\textbf{Description}}
\setlength{\itemindent}{0cm}
\setlength{\listparindent}{0cm}
\setlength{\leftmargin}{\evensidemargin}
\addtolength{\leftmargin}{\tmplength}
\settowidth{\labelsep}{X}
\addtolength{\leftmargin}{\labelsep}
\setlength{\labelwidth}{\tmplength}
}
\item[\textbf{Declaration}\hfill]
\ifpdf
\begin{flushleft}
\fi
\begin{ttfamily}
lp='/tmp/';\end{ttfamily}

\ifpdf
\end{flushleft}
\fi

\par
\item[\textbf{Description}]
Working directory of Listaller

\end{list}
\section{Variables}
\ifpdf
\subsection*{\large{\textbf{IWizFrm}}\normalsize\hspace{1ex}\hrulefill}
\else
\subsection*{IWizFrm}
\fi
\label{mainunit-IWizFrm}
\index{IWizFrm}
\begin{list}{}{
\settowidth{\tmplength}{\textbf{Description}}
\setlength{\itemindent}{0cm}
\setlength{\listparindent}{0cm}
\setlength{\leftmargin}{\evensidemargin}
\addtolength{\leftmargin}{\tmplength}
\settowidth{\labelsep}{X}
\addtolength{\leftmargin}{\labelsep}
\setlength{\labelwidth}{\tmplength}
}
\item[\textbf{Declaration}\hfill]
\ifpdf
\begin{flushleft}
\fi
\begin{ttfamily}
IWizFrm: TIWizFrm;\end{ttfamily}

\ifpdf
\end{flushleft}
\fi

\end{list}
\ifpdf
\subsection*{\large{\textbf{FDir}}\normalsize\hspace{1ex}\hrulefill}
\else
\subsection*{FDir}
\fi
\label{mainunit-FDir}
\index{FDir}
\begin{list}{}{
\settowidth{\tmplength}{\textbf{Description}}
\setlength{\itemindent}{0cm}
\setlength{\listparindent}{0cm}
\setlength{\leftmargin}{\evensidemargin}
\addtolength{\leftmargin}{\tmplength}
\settowidth{\labelsep}{X}
\addtolength{\leftmargin}{\labelsep}
\setlength{\labelwidth}{\tmplength}
}
\item[\textbf{Declaration}\hfill]
\ifpdf
\begin{flushleft}
\fi
\begin{ttfamily}
FDir:  String;\end{ttfamily}

\ifpdf
\end{flushleft}
\fi

\end{list}
\ifpdf
\subsection*{\large{\textbf{DInfo}}\normalsize\hspace{1ex}\hrulefill}
\else
\subsection*{DInfo}
\fi
\label{mainunit-DInfo}
\index{DInfo}
\begin{list}{}{
\settowidth{\tmplength}{\textbf{Description}}
\setlength{\itemindent}{0cm}
\setlength{\listparindent}{0cm}
\setlength{\leftmargin}{\evensidemargin}
\addtolength{\leftmargin}{\tmplength}
\settowidth{\labelsep}{X}
\addtolength{\leftmargin}{\labelsep}
\setlength{\labelwidth}{\tmplength}
}
\item[\textbf{Declaration}\hfill]
\ifpdf
\begin{flushleft}
\fi
\begin{ttfamily}
DInfo: TDistroInfo;\end{ttfamily}

\ifpdf
\end{flushleft}
\fi

\par
\item[\textbf{Description}]
Distribution information

\end{list}
\ifpdf
\subsection*{\large{\textbf{ContSFiles}}\normalsize\hspace{1ex}\hrulefill}
\else
\subsection*{ContSFiles}
\fi
\label{mainunit-ContSFiles}
\index{ContSFiles}
\begin{list}{}{
\settowidth{\tmplength}{\textbf{Description}}
\setlength{\itemindent}{0cm}
\setlength{\listparindent}{0cm}
\setlength{\leftmargin}{\evensidemargin}
\addtolength{\leftmargin}{\tmplength}
\settowidth{\labelsep}{X}
\addtolength{\leftmargin}{\labelsep}
\setlength{\labelwidth}{\tmplength}
}
\item[\textbf{Declaration}\hfill]
\ifpdf
\begin{flushleft}
\fi
\begin{ttfamily}
ContSFiles: Boolean=false;\end{ttfamily}

\ifpdf
\end{flushleft}
\fi

\par
\item[\textbf{Description}]
Set if application installs shared files

\end{list}
\chapter{Unit manager}
\label{manager}
\index{manager}
\section{Description}
This unit contains the code to manage installed packages
\section{uses}
\begin{itemize}
\item \begin{ttfamily}Classes\end{ttfamily}\item \begin{ttfamily}SysUtils\end{ttfamily}\item \begin{ttfamily}LResources\end{ttfamily}\item \begin{ttfamily}Forms\end{ttfamily}\item \begin{ttfamily}Controls\end{ttfamily}\item \begin{ttfamily}Graphics\end{ttfamily}\item \begin{ttfamily}Dialogs\end{ttfamily}\item \begin{ttfamily}ComCtrls\end{ttfamily}\item \begin{ttfamily}Inifiles\end{ttfamily}\item \begin{ttfamily}StdCtrls\end{ttfamily}\item \begin{ttfamily}process\end{ttfamily}\item \begin{ttfamily}LCLType\end{ttfamily}\item \begin{ttfamily}Buttons\end{ttfamily}\item \begin{ttfamily}FileCtrl\end{ttfamily}\item \begin{ttfamily}EditBtn\end{ttfamily}\item \begin{ttfamily}ExtCtrls\end{ttfamily}\item \begin{ttfamily}distri\end{ttfamily}(\ref{distri})\item \begin{ttfamily}utilities\end{ttfamily}(\ref{utilities})\item \begin{ttfamily}uninstall\end{ttfamily}(\ref{uninstall})\item \begin{ttfamily}translations\end{ttfamily}\item \begin{ttfamily}trstrings\end{ttfamily}(\ref{trstrings})\item \begin{ttfamily}gettext\end{ttfamily}\item \begin{ttfamily}FileUtil\end{ttfamily}\item \begin{ttfamily}XMLRead\end{ttfamily}\item \begin{ttfamily}DOM\end{ttfamily}\item \begin{ttfamily}xtypefm\end{ttfamily}(\ref{xtypefm})\item \begin{ttfamily}ipkhandle\end{ttfamily}(\ref{ipkhandle})\end{itemize}
\section{Overview}
\begin{description}
\item[\texttt{\begin{ttfamily}TmnFrm\end{ttfamily} Class}]
\end{description}
\section{Classes, Interfaces, Objects and Records}
\ifpdf
\subsection*{\large{\textbf{TmnFrm Class}}\normalsize\hspace{1ex}\hrulefill}
\else
\subsection*{TmnFrm Class}
\fi
\label{manager.TmnFrm}
\index{TmnFrm}
\subsubsection*{\large{\textbf{Hierarchy}}\normalsize\hspace{1ex}\hfill}
TmnFrm {$>$} TForm
%%%%Description
\subsubsection*{\large{\textbf{Fields}}\normalsize\hspace{1ex}\hfill}
\begin{list}{}{
\settowidth{\tmplength}{\textbf{ProgressBar1}}
\setlength{\itemindent}{0cm}
\setlength{\listparindent}{0cm}
\setlength{\leftmargin}{\evensidemargin}
\addtolength{\leftmargin}{\tmplength}
\settowidth{\labelsep}{X}
\addtolength{\leftmargin}{\labelsep}
\setlength{\labelwidth}{\tmplength}
}
\label{manager.TmnFrm-btnInstall}
\index{btnInstall}
\item[\textbf{btnInstall}\hfill]
\ifpdf
\begin{flushleft}
\fi
\begin{ttfamily}
public btnInstall: TBitBtn;\end{ttfamily}

\ifpdf
\end{flushleft}
\fi


\par  \label{manager.TmnFrm-btnSettings}
\index{btnSettings}
\item[\textbf{btnSettings}\hfill]
\ifpdf
\begin{flushleft}
\fi
\begin{ttfamily}
public btnSettings: TBitBtn;\end{ttfamily}

\ifpdf
\end{flushleft}
\fi


\par  \label{manager.TmnFrm-btnCat}
\index{btnCat}
\item[\textbf{btnCat}\hfill]
\ifpdf
\begin{flushleft}
\fi
\begin{ttfamily}
public btnCat: TBitBtn;\end{ttfamily}

\ifpdf
\end{flushleft}
\fi


\par  \label{manager.TmnFrm-CBox}
\index{CBox}
\item[\textbf{CBox}\hfill]
\ifpdf
\begin{flushleft}
\fi
\begin{ttfamily}
public CBox: TComboBox;\end{ttfamily}

\ifpdf
\end{flushleft}
\fi


\par  \label{manager.TmnFrm-edtFilter}
\index{edtFilter}
\item[\textbf{edtFilter}\hfill]
\ifpdf
\begin{flushleft}
\fi
\begin{ttfamily}
public edtFilter: TEdit;\end{ttfamily}

\ifpdf
\end{flushleft}
\fi


\par  \label{manager.TmnFrm-Image1}
\index{Image1}
\item[\textbf{Image1}\hfill]
\ifpdf
\begin{flushleft}
\fi
\begin{ttfamily}
public Image1: TImage;\end{ttfamily}

\ifpdf
\end{flushleft}
\fi


\par  \label{manager.TmnFrm-Label1}
\index{Label1}
\item[\textbf{Label1}\hfill]
\ifpdf
\begin{flushleft}
\fi
\begin{ttfamily}
public Label1: TLabel;\end{ttfamily}

\ifpdf
\end{flushleft}
\fi


\par  \label{manager.TmnFrm-Label2}
\index{Label2}
\item[\textbf{Label2}\hfill]
\ifpdf
\begin{flushleft}
\fi
\begin{ttfamily}
public Label2: TLabel;\end{ttfamily}

\ifpdf
\end{flushleft}
\fi


\par  \label{manager.TmnFrm-Label3}
\index{Label3}
\item[\textbf{Label3}\hfill]
\ifpdf
\begin{flushleft}
\fi
\begin{ttfamily}
public Label3: TLabel;\end{ttfamily}

\ifpdf
\end{flushleft}
\fi


\par  \label{manager.TmnFrm-OpenDialog1}
\index{OpenDialog1}
\item[\textbf{OpenDialog1}\hfill]
\ifpdf
\begin{flushleft}
\fi
\begin{ttfamily}
public OpenDialog1: TOpenDialog;\end{ttfamily}

\ifpdf
\end{flushleft}
\fi


\par  \label{manager.TmnFrm-Process1}
\index{Process1}
\item[\textbf{Process1}\hfill]
\ifpdf
\begin{flushleft}
\fi
\begin{ttfamily}
public Process1: TProcess;\end{ttfamily}

\ifpdf
\end{flushleft}
\fi


\par  \label{manager.TmnFrm-ProgressBar1}
\index{ProgressBar1}
\item[\textbf{ProgressBar1}\hfill]
\ifpdf
\begin{flushleft}
\fi
\begin{ttfamily}
public ProgressBar1: TProgressBar;\end{ttfamily}

\ifpdf
\end{flushleft}
\fi


\par  \label{manager.TmnFrm-StatusBar1}
\index{StatusBar1}
\item[\textbf{StatusBar1}\hfill]
\ifpdf
\begin{flushleft}
\fi
\begin{ttfamily}
public StatusBar1: TStatusBar;\end{ttfamily}

\ifpdf
\end{flushleft}
\fi


\par  \label{manager.TmnFrm-SWBox}
\index{SWBox}
\item[\textbf{SWBox}\hfill]
\ifpdf
\begin{flushleft}
\fi
\begin{ttfamily}
public SWBox: TScrollBox;\end{ttfamily}

\ifpdf
\end{flushleft}
\fi


\par  \label{manager.TmnFrm-DInfo}
\index{DInfo}
\item[\textbf{DInfo}\hfill]
\ifpdf
\begin{flushleft}
\fi
\begin{ttfamily}
public DInfo: TDistroInfo;\end{ttfamily}

\ifpdf
\end{flushleft}
\fi


\par  \label{manager.TmnFrm-IdList}
\index{IdList}
\item[\textbf{IdList}\hfill]
\ifpdf
\begin{flushleft}
\fi
\begin{ttfamily}
public IdList: TStringList;\end{ttfamily}

\ifpdf
\end{flushleft}
\fi


\par List of package id's\label{manager.TmnFrm-AList}
\index{AList}
\item[\textbf{AList}\hfill]
\ifpdf
\begin{flushleft}
\fi
\begin{ttfamily}
public AList: Array of TListEntry;\end{ttfamily}

\ifpdf
\end{flushleft}
\fi


\par Visual package list\label{manager.TmnFrm-ListLength}
\index{ListLength}
\item[\textbf{ListLength}\hfill]
\ifpdf
\begin{flushleft}
\fi
\begin{ttfamily}
public ListLength: Integer;\end{ttfamily}

\ifpdf
\end{flushleft}
\fi


\par Length of AList\label{manager.TmnFrm-uID}
\index{uID}
\item[\textbf{uID}\hfill]
\ifpdf
\begin{flushleft}
\fi
\begin{ttfamily}
public uID: Integer;\end{ttfamily}

\ifpdf
\end{flushleft}
\fi


\par Current id of package that should be uninstalled\end{list}
\subsubsection*{\large{\textbf{Methods}}\normalsize\hspace{1ex}\hfill}
\paragraph*{btnInstallClick}\hspace*{\fill}

\label{manager.TmnFrm-btnInstallClick}
\index{btnInstallClick}
\begin{list}{}{
\settowidth{\tmplength}{\textbf{Description}}
\setlength{\itemindent}{0cm}
\setlength{\listparindent}{0cm}
\setlength{\leftmargin}{\evensidemargin}
\addtolength{\leftmargin}{\tmplength}
\settowidth{\labelsep}{X}
\addtolength{\leftmargin}{\labelsep}
\setlength{\labelwidth}{\tmplength}
}
\item[\textbf{Declaration}\hfill]
\ifpdf
\begin{flushleft}
\fi
\begin{ttfamily}
public procedure btnInstallClick(Sender: TObject);\end{ttfamily}

\ifpdf
\end{flushleft}
\fi

\end{list}
\paragraph*{BitBtn2Click}\hspace*{\fill}

\label{manager.TmnFrm-BitBtn2Click}
\index{BitBtn2Click}
\begin{list}{}{
\settowidth{\tmplength}{\textbf{Description}}
\setlength{\itemindent}{0cm}
\setlength{\listparindent}{0cm}
\setlength{\leftmargin}{\evensidemargin}
\addtolength{\leftmargin}{\tmplength}
\settowidth{\labelsep}{X}
\addtolength{\leftmargin}{\labelsep}
\setlength{\labelwidth}{\tmplength}
}
\item[\textbf{Declaration}\hfill]
\ifpdf
\begin{flushleft}
\fi
\begin{ttfamily}
public procedure BitBtn2Click(Sender: TObject);\end{ttfamily}

\ifpdf
\end{flushleft}
\fi

\end{list}
\paragraph*{btnSettingsClick}\hspace*{\fill}

\label{manager.TmnFrm-btnSettingsClick}
\index{btnSettingsClick}
\begin{list}{}{
\settowidth{\tmplength}{\textbf{Description}}
\setlength{\itemindent}{0cm}
\setlength{\listparindent}{0cm}
\setlength{\leftmargin}{\evensidemargin}
\addtolength{\leftmargin}{\tmplength}
\settowidth{\labelsep}{X}
\addtolength{\leftmargin}{\labelsep}
\setlength{\labelwidth}{\tmplength}
}
\item[\textbf{Declaration}\hfill]
\ifpdf
\begin{flushleft}
\fi
\begin{ttfamily}
public procedure btnSettingsClick(Sender: TObject);\end{ttfamily}

\ifpdf
\end{flushleft}
\fi

\end{list}
\paragraph*{btnCatClick}\hspace*{\fill}

\label{manager.TmnFrm-btnCatClick}
\index{btnCatClick}
\begin{list}{}{
\settowidth{\tmplength}{\textbf{Description}}
\setlength{\itemindent}{0cm}
\setlength{\listparindent}{0cm}
\setlength{\leftmargin}{\evensidemargin}
\addtolength{\leftmargin}{\tmplength}
\settowidth{\labelsep}{X}
\addtolength{\leftmargin}{\labelsep}
\setlength{\labelwidth}{\tmplength}
}
\item[\textbf{Declaration}\hfill]
\ifpdf
\begin{flushleft}
\fi
\begin{ttfamily}
public procedure btnCatClick(Sender: TObject);\end{ttfamily}

\ifpdf
\end{flushleft}
\fi

\end{list}
\paragraph*{CBoxChange}\hspace*{\fill}

\label{manager.TmnFrm-CBoxChange}
\index{CBoxChange}
\begin{list}{}{
\settowidth{\tmplength}{\textbf{Description}}
\setlength{\itemindent}{0cm}
\setlength{\listparindent}{0cm}
\setlength{\leftmargin}{\evensidemargin}
\addtolength{\leftmargin}{\tmplength}
\settowidth{\labelsep}{X}
\addtolength{\leftmargin}{\labelsep}
\setlength{\labelwidth}{\tmplength}
}
\item[\textbf{Declaration}\hfill]
\ifpdf
\begin{flushleft}
\fi
\begin{ttfamily}
public procedure CBoxChange(Sender: TObject);\end{ttfamily}

\ifpdf
\end{flushleft}
\fi

\end{list}
\paragraph*{edtFilterChange}\hspace*{\fill}

\label{manager.TmnFrm-edtFilterChange}
\index{edtFilterChange}
\begin{list}{}{
\settowidth{\tmplength}{\textbf{Description}}
\setlength{\itemindent}{0cm}
\setlength{\listparindent}{0cm}
\setlength{\leftmargin}{\evensidemargin}
\addtolength{\leftmargin}{\tmplength}
\settowidth{\labelsep}{X}
\addtolength{\leftmargin}{\labelsep}
\setlength{\labelwidth}{\tmplength}
}
\item[\textbf{Declaration}\hfill]
\ifpdf
\begin{flushleft}
\fi
\begin{ttfamily}
public procedure edtFilterChange(Sender: TObject);\end{ttfamily}

\ifpdf
\end{flushleft}
\fi

\end{list}
\paragraph*{FormActivate}\hspace*{\fill}

\label{manager.TmnFrm-FormActivate}
\index{FormActivate}
\begin{list}{}{
\settowidth{\tmplength}{\textbf{Description}}
\setlength{\itemindent}{0cm}
\setlength{\listparindent}{0cm}
\setlength{\leftmargin}{\evensidemargin}
\addtolength{\leftmargin}{\tmplength}
\settowidth{\labelsep}{X}
\addtolength{\leftmargin}{\labelsep}
\setlength{\labelwidth}{\tmplength}
}
\item[\textbf{Declaration}\hfill]
\ifpdf
\begin{flushleft}
\fi
\begin{ttfamily}
public procedure FormActivate(Sender: TObject);\end{ttfamily}

\ifpdf
\end{flushleft}
\fi

\end{list}
\paragraph*{FormCreate}\hspace*{\fill}

\label{manager.TmnFrm-FormCreate}
\index{FormCreate}
\begin{list}{}{
\settowidth{\tmplength}{\textbf{Description}}
\setlength{\itemindent}{0cm}
\setlength{\listparindent}{0cm}
\setlength{\leftmargin}{\evensidemargin}
\addtolength{\leftmargin}{\tmplength}
\settowidth{\labelsep}{X}
\addtolength{\leftmargin}{\labelsep}
\setlength{\labelwidth}{\tmplength}
}
\item[\textbf{Declaration}\hfill]
\ifpdf
\begin{flushleft}
\fi
\begin{ttfamily}
public procedure FormCreate(Sender: TObject);\end{ttfamily}

\ifpdf
\end{flushleft}
\fi

\end{list}
\paragraph*{FormDestroy}\hspace*{\fill}

\label{manager.TmnFrm-FormDestroy}
\index{FormDestroy}
\begin{list}{}{
\settowidth{\tmplength}{\textbf{Description}}
\setlength{\itemindent}{0cm}
\setlength{\listparindent}{0cm}
\setlength{\leftmargin}{\evensidemargin}
\addtolength{\leftmargin}{\tmplength}
\settowidth{\labelsep}{X}
\addtolength{\leftmargin}{\labelsep}
\setlength{\labelwidth}{\tmplength}
}
\item[\textbf{Declaration}\hfill]
\ifpdf
\begin{flushleft}
\fi
\begin{ttfamily}
public procedure FormDestroy(Sender: TObject);\end{ttfamily}

\ifpdf
\end{flushleft}
\fi

\end{list}
\paragraph*{FormResize}\hspace*{\fill}

\label{manager.TmnFrm-FormResize}
\index{FormResize}
\begin{list}{}{
\settowidth{\tmplength}{\textbf{Description}}
\setlength{\itemindent}{0cm}
\setlength{\listparindent}{0cm}
\setlength{\leftmargin}{\evensidemargin}
\addtolength{\leftmargin}{\tmplength}
\settowidth{\labelsep}{X}
\addtolength{\leftmargin}{\labelsep}
\setlength{\labelwidth}{\tmplength}
}
\item[\textbf{Declaration}\hfill]
\ifpdf
\begin{flushleft}
\fi
\begin{ttfamily}
public procedure FormResize(Sender: TObject);\end{ttfamily}

\ifpdf
\end{flushleft}
\fi

\end{list}
\paragraph*{FormShow}\hspace*{\fill}

\label{manager.TmnFrm-FormShow}
\index{FormShow}
\begin{list}{}{
\settowidth{\tmplength}{\textbf{Description}}
\setlength{\itemindent}{0cm}
\setlength{\listparindent}{0cm}
\setlength{\leftmargin}{\evensidemargin}
\addtolength{\leftmargin}{\tmplength}
\settowidth{\labelsep}{X}
\addtolength{\leftmargin}{\labelsep}
\setlength{\labelwidth}{\tmplength}
}
\item[\textbf{Declaration}\hfill]
\ifpdf
\begin{flushleft}
\fi
\begin{ttfamily}
public procedure FormShow(Sender: TObject);\end{ttfamily}

\ifpdf
\end{flushleft}
\fi

\end{list}
\paragraph*{ProgressBar1ContextPopup}\hspace*{\fill}

\label{manager.TmnFrm-ProgressBar1ContextPopup}
\index{ProgressBar1ContextPopup}
\begin{list}{}{
\settowidth{\tmplength}{\textbf{Description}}
\setlength{\itemindent}{0cm}
\setlength{\listparindent}{0cm}
\setlength{\leftmargin}{\evensidemargin}
\addtolength{\leftmargin}{\tmplength}
\settowidth{\labelsep}{X}
\addtolength{\leftmargin}{\labelsep}
\setlength{\labelwidth}{\tmplength}
}
\item[\textbf{Declaration}\hfill]
\ifpdf
\begin{flushleft}
\fi
\begin{ttfamily}
public procedure ProgressBar1ContextPopup(Sender: TObject; MousePos: TPoint; var Handled: Boolean);\end{ttfamily}

\ifpdf
\end{flushleft}
\fi

\end{list}
\paragraph*{LoadEntries}\hspace*{\fill}

\label{manager.TmnFrm-LoadEntries}
\index{LoadEntries}
\begin{list}{}{
\settowidth{\tmplength}{\textbf{Description}}
\setlength{\itemindent}{0cm}
\setlength{\listparindent}{0cm}
\setlength{\leftmargin}{\evensidemargin}
\addtolength{\leftmargin}{\tmplength}
\settowidth{\labelsep}{X}
\addtolength{\leftmargin}{\labelsep}
\setlength{\labelwidth}{\tmplength}
}
\item[\textbf{Declaration}\hfill]
\ifpdf
\begin{flushleft}
\fi
\begin{ttfamily}
public procedure LoadEntries;\end{ttfamily}

\ifpdf
\end{flushleft}
\fi

\par
\item[\textbf{Description}]
Load software list entries

\end{list}
\section{Variables}
\ifpdf
\subsection*{\large{\textbf{mnFrm}}\normalsize\hspace{1ex}\hrulefill}
\else
\subsection*{mnFrm}
\fi
\label{manager-mnFrm}
\index{mnFrm}
\begin{list}{}{
\settowidth{\tmplength}{\textbf{Description}}
\setlength{\itemindent}{0cm}
\setlength{\listparindent}{0cm}
\setlength{\leftmargin}{\evensidemargin}
\addtolength{\leftmargin}{\tmplength}
\settowidth{\labelsep}{X}
\addtolength{\leftmargin}{\labelsep}
\setlength{\labelwidth}{\tmplength}
}
\item[\textbf{Declaration}\hfill]
\ifpdf
\begin{flushleft}
\fi
\begin{ttfamily}
mnFrm:   TmnFrm;\end{ttfamily}

\ifpdf
\end{flushleft}
\fi

\end{list}
\ifpdf
\subsection*{\large{\textbf{instLst}}\normalsize\hspace{1ex}\hrulefill}
\else
\subsection*{instLst}
\fi
\label{manager-instLst}
\index{instLst}
\begin{list}{}{
\settowidth{\tmplength}{\textbf{Description}}
\setlength{\itemindent}{0cm}
\setlength{\listparindent}{0cm}
\setlength{\leftmargin}{\evensidemargin}
\addtolength{\leftmargin}{\tmplength}
\settowidth{\labelsep}{X}
\addtolength{\leftmargin}{\labelsep}
\setlength{\labelwidth}{\tmplength}
}
\item[\textbf{Declaration}\hfill]
\ifpdf
\begin{flushleft}
\fi
\begin{ttfamily}
instLst: TStringList;\end{ttfamily}

\ifpdf
\end{flushleft}
\fi

\end{list}
\ifpdf
\subsection*{\large{\textbf{RegDir}}\normalsize\hspace{1ex}\hrulefill}
\else
\subsection*{RegDir}
\fi
\label{manager-RegDir}
\index{RegDir}
\begin{list}{}{
\settowidth{\tmplength}{\textbf{Description}}
\setlength{\itemindent}{0cm}
\setlength{\listparindent}{0cm}
\setlength{\leftmargin}{\evensidemargin}
\addtolength{\leftmargin}{\tmplength}
\settowidth{\labelsep}{X}
\addtolength{\leftmargin}{\labelsep}
\setlength{\labelwidth}{\tmplength}
}
\item[\textbf{Declaration}\hfill]
\ifpdf
\begin{flushleft}
\fi
\begin{ttfamily}
RegDir: String;\end{ttfamily}

\ifpdf
\end{flushleft}
\fi

\par
\item[\textbf{Description}]
In this directory package information will be stored

\end{list}
\chapter{Unit mnupdate}
\label{mnupdate}
\index{mnupdate}
\section{Description}
Main unit of the updater application
\section{uses}
\begin{itemize}
\item \begin{ttfamily}Classes\end{ttfamily}\item \begin{ttfamily}SysUtils\end{ttfamily}\item \begin{ttfamily}LResources\end{ttfamily}\item \begin{ttfamily}Forms\end{ttfamily}\item \begin{ttfamily}Controls\end{ttfamily}\item \begin{ttfamily}Graphics\end{ttfamily}\item \begin{ttfamily}Dialogs\end{ttfamily}\item \begin{ttfamily}ExtCtrls\end{ttfamily}\item \begin{ttfamily}StdCtrls\end{ttfamily}\item \begin{ttfamily}Buttons\end{ttfamily}\item \begin{ttfamily}CheckLst\end{ttfamily}\item \begin{ttfamily}HTTPSend\end{ttfamily}(\ref{httpsend})\item \begin{ttfamily}IniFiles\end{ttfamily}\item \begin{ttfamily}MD5\end{ttfamily}\item \begin{ttfamily}utilities\end{ttfamily}(\ref{utilities})\item \begin{ttfamily}updexec\end{ttfamily}(\ref{updexec})\item \begin{ttfamily}LCLType\end{ttfamily}\item \begin{ttfamily}Process\end{ttfamily}\item \begin{ttfamily}Menus\end{ttfamily}\item \begin{ttfamily}trstrings\end{ttfamily}(\ref{trstrings})\item \begin{ttfamily}GetText\end{ttfamily}\item \begin{ttfamily}Translations\end{ttfamily}\item \begin{ttfamily}gtk2\end{ttfamily}\item \begin{ttfamily}gtkint\end{ttfamily}\item \begin{ttfamily}gtkdef\end{ttfamily}\item \begin{ttfamily}XMLRead\end{ttfamily}\item \begin{ttfamily}DOM\end{ttfamily}\item \begin{ttfamily}ldunit\end{ttfamily}\item \begin{ttfamily}ipkhandle\end{ttfamily}(\ref{ipkhandle})\end{itemize}
\section{Overview}
\begin{description}
\item[\texttt{\begin{ttfamily}TAppNotes\end{ttfamily} record}]
\item[\texttt{\begin{ttfamily}TForm1\end{ttfamily} Class}]
\end{description}
\section{Classes, Interfaces, Objects and Records}
\ifpdf
\subsection*{\large{\textbf{TAppNotes record}}\normalsize\hspace{1ex}\hrulefill}
\else
\subsection*{TAppNotes record}
\fi
\label{mnupdate.TAppNotes}
\index{TAppNotes}
%%%%Description
\subsubsection*{\large{\textbf{Fields}}\normalsize\hspace{1ex}\hfill}
\begin{list}{}{
\settowidth{\tmplength}{\textbf{NVersion}}
\setlength{\itemindent}{0cm}
\setlength{\listparindent}{0cm}
\setlength{\leftmargin}{\evensidemargin}
\addtolength{\leftmargin}{\tmplength}
\settowidth{\labelsep}{X}
\addtolength{\leftmargin}{\labelsep}
\setlength{\labelwidth}{\tmplength}
}
\label{mnupdate.TAppNotes-AppName}
\index{AppName}
\item[\textbf{AppName}\hfill]
\ifpdf
\begin{flushleft}
\fi
\begin{ttfamily}
AppName: String;\end{ttfamily}

\ifpdf
\end{flushleft}
\fi


\par  \label{mnupdate.TAppNotes-NVersion}
\index{NVersion}
\item[\textbf{NVersion}\hfill]
\ifpdf
\begin{flushleft}
\fi
\begin{ttfamily}
NVersion: String;\end{ttfamily}

\ifpdf
\end{flushleft}
\fi


\par  \label{mnupdate.TAppNotes-OVersion}
\index{OVersion}
\item[\textbf{OVersion}\hfill]
\ifpdf
\begin{flushleft}
\fi
\begin{ttfamily}
OVersion: String;\end{ttfamily}

\ifpdf
\end{flushleft}
\fi


\par  \label{mnupdate.TAppNotes-ID}
\index{ID}
\item[\textbf{ID}\hfill]
\ifpdf
\begin{flushleft}
\fi
\begin{ttfamily}
ID: String;\end{ttfamily}

\ifpdf
\end{flushleft}
\fi


\par  \end{list}
\ifpdf
\subsection*{\large{\textbf{TForm1 Class}}\normalsize\hspace{1ex}\hrulefill}
\else
\subsection*{TForm1 Class}
\fi
\label{mnupdate.TForm1}
\index{TForm1}
\subsubsection*{\large{\textbf{Hierarchy}}\normalsize\hspace{1ex}\hfill}
TForm1 {$>$} TForm
%%%%Description
\subsubsection*{\large{\textbf{Fields}}\normalsize\hspace{1ex}\hfill}
\begin{list}{}{
\settowidth{\tmplength}{\textbf{CheckListBox1}}
\setlength{\itemindent}{0cm}
\setlength{\listparindent}{0cm}
\setlength{\leftmargin}{\evensidemargin}
\addtolength{\leftmargin}{\tmplength}
\settowidth{\labelsep}{X}
\addtolength{\leftmargin}{\labelsep}
\setlength{\labelwidth}{\tmplength}
}
\label{mnupdate.TForm1-BitBtn1}
\index{BitBtn1}
\item[\textbf{BitBtn1}\hfill]
\ifpdf
\begin{flushleft}
\fi
\begin{ttfamily}
public BitBtn1: TBitBtn;\end{ttfamily}

\ifpdf
\end{flushleft}
\fi


\par  \label{mnupdate.TForm1-BitBtn2}
\index{BitBtn2}
\item[\textbf{BitBtn2}\hfill]
\ifpdf
\begin{flushleft}
\fi
\begin{ttfamily}
public BitBtn2: TBitBtn;\end{ttfamily}

\ifpdf
\end{flushleft}
\fi


\par  \label{mnupdate.TForm1-CheckListBox1}
\index{CheckListBox1}
\item[\textbf{CheckListBox1}\hfill]
\ifpdf
\begin{flushleft}
\fi
\begin{ttfamily}
public CheckListBox1: TCheckListBox;\end{ttfamily}

\ifpdf
\end{flushleft}
\fi


\par  \label{mnupdate.TForm1-InfoMemo}
\index{InfoMemo}
\item[\textbf{InfoMemo}\hfill]
\ifpdf
\begin{flushleft}
\fi
\begin{ttfamily}
public InfoMemo: TMemo;\end{ttfamily}

\ifpdf
\end{flushleft}
\fi


\par  \label{mnupdate.TForm1-MenuItem1}
\index{MenuItem1}
\item[\textbf{MenuItem1}\hfill]
\ifpdf
\begin{flushleft}
\fi
\begin{ttfamily}
public MenuItem1: TMenuItem;\end{ttfamily}

\ifpdf
\end{flushleft}
\fi


\par  \label{mnupdate.TForm1-MenuItem2}
\index{MenuItem2}
\item[\textbf{MenuItem2}\hfill]
\ifpdf
\begin{flushleft}
\fi
\begin{ttfamily}
public MenuItem2: TMenuItem;\end{ttfamily}

\ifpdf
\end{flushleft}
\fi


\par  \label{mnupdate.TForm1-PopupMenu1}
\index{PopupMenu1}
\item[\textbf{PopupMenu1}\hfill]
\ifpdf
\begin{flushleft}
\fi
\begin{ttfamily}
public PopupMenu1: TPopupMenu;\end{ttfamily}

\ifpdf
\end{flushleft}
\fi


\par  \label{mnupdate.TForm1-TrayIcon1}
\index{TrayIcon1}
\item[\textbf{TrayIcon1}\hfill]
\ifpdf
\begin{flushleft}
\fi
\begin{ttfamily}
public TrayIcon1: TTrayIcon;\end{ttfamily}

\ifpdf
\end{flushleft}
\fi


\par  \label{mnupdate.TForm1-ulist}
\index{ulist}
\item[\textbf{ulist}\hfill]
\ifpdf
\begin{flushleft}
\fi
\begin{ttfamily}
public ulist: Array of TStringList;\end{ttfamily}

\ifpdf
\end{flushleft}
\fi


\par  \label{mnupdate.TForm1-ANotes}
\index{ANotes}
\item[\textbf{ANotes}\hfill]
\ifpdf
\begin{flushleft}
\fi
\begin{ttfamily}
public ANotes: Array of TAppNotes;\end{ttfamily}

\ifpdf
\end{flushleft}
\fi


\par  \end{list}
\subsubsection*{\large{\textbf{Methods}}\normalsize\hspace{1ex}\hfill}
\paragraph*{BitBtn1Click}\hspace*{\fill}

\label{mnupdate.TForm1-BitBtn1Click}
\index{BitBtn1Click}
\begin{list}{}{
\settowidth{\tmplength}{\textbf{Description}}
\setlength{\itemindent}{0cm}
\setlength{\listparindent}{0cm}
\setlength{\leftmargin}{\evensidemargin}
\addtolength{\leftmargin}{\tmplength}
\settowidth{\labelsep}{X}
\addtolength{\leftmargin}{\labelsep}
\setlength{\labelwidth}{\tmplength}
}
\item[\textbf{Declaration}\hfill]
\ifpdf
\begin{flushleft}
\fi
\begin{ttfamily}
public procedure BitBtn1Click(Sender: TObject);\end{ttfamily}

\ifpdf
\end{flushleft}
\fi

\end{list}
\paragraph*{BitBtn2Click}\hspace*{\fill}

\label{mnupdate.TForm1-BitBtn2Click}
\index{BitBtn2Click}
\begin{list}{}{
\settowidth{\tmplength}{\textbf{Description}}
\setlength{\itemindent}{0cm}
\setlength{\listparindent}{0cm}
\setlength{\leftmargin}{\evensidemargin}
\addtolength{\leftmargin}{\tmplength}
\settowidth{\labelsep}{X}
\addtolength{\leftmargin}{\labelsep}
\setlength{\labelwidth}{\tmplength}
}
\item[\textbf{Declaration}\hfill]
\ifpdf
\begin{flushleft}
\fi
\begin{ttfamily}
public procedure BitBtn2Click(Sender: TObject);\end{ttfamily}

\ifpdf
\end{flushleft}
\fi

\end{list}
\paragraph*{CheckListBox1Click}\hspace*{\fill}

\label{mnupdate.TForm1-CheckListBox1Click}
\index{CheckListBox1Click}
\begin{list}{}{
\settowidth{\tmplength}{\textbf{Description}}
\setlength{\itemindent}{0cm}
\setlength{\listparindent}{0cm}
\setlength{\leftmargin}{\evensidemargin}
\addtolength{\leftmargin}{\tmplength}
\settowidth{\labelsep}{X}
\addtolength{\leftmargin}{\labelsep}
\setlength{\labelwidth}{\tmplength}
}
\item[\textbf{Declaration}\hfill]
\ifpdf
\begin{flushleft}
\fi
\begin{ttfamily}
public procedure CheckListBox1Click(Sender: TObject);\end{ttfamily}

\ifpdf
\end{flushleft}
\fi

\end{list}
\paragraph*{FormCloseQuery}\hspace*{\fill}

\label{mnupdate.TForm1-FormCloseQuery}
\index{FormCloseQuery}
\begin{list}{}{
\settowidth{\tmplength}{\textbf{Description}}
\setlength{\itemindent}{0cm}
\setlength{\listparindent}{0cm}
\setlength{\leftmargin}{\evensidemargin}
\addtolength{\leftmargin}{\tmplength}
\settowidth{\labelsep}{X}
\addtolength{\leftmargin}{\labelsep}
\setlength{\labelwidth}{\tmplength}
}
\item[\textbf{Declaration}\hfill]
\ifpdf
\begin{flushleft}
\fi
\begin{ttfamily}
public procedure FormCloseQuery(Sender: TObject; var CanClose: boolean);\end{ttfamily}

\ifpdf
\end{flushleft}
\fi

\end{list}
\paragraph*{FormCreate}\hspace*{\fill}

\label{mnupdate.TForm1-FormCreate}
\index{FormCreate}
\begin{list}{}{
\settowidth{\tmplength}{\textbf{Description}}
\setlength{\itemindent}{0cm}
\setlength{\listparindent}{0cm}
\setlength{\leftmargin}{\evensidemargin}
\addtolength{\leftmargin}{\tmplength}
\settowidth{\labelsep}{X}
\addtolength{\leftmargin}{\labelsep}
\setlength{\labelwidth}{\tmplength}
}
\item[\textbf{Declaration}\hfill]
\ifpdf
\begin{flushleft}
\fi
\begin{ttfamily}
public procedure FormCreate(Sender: TObject);\end{ttfamily}

\ifpdf
\end{flushleft}
\fi

\end{list}
\paragraph*{FormDestroy}\hspace*{\fill}

\label{mnupdate.TForm1-FormDestroy}
\index{FormDestroy}
\begin{list}{}{
\settowidth{\tmplength}{\textbf{Description}}
\setlength{\itemindent}{0cm}
\setlength{\listparindent}{0cm}
\setlength{\leftmargin}{\evensidemargin}
\addtolength{\leftmargin}{\tmplength}
\settowidth{\labelsep}{X}
\addtolength{\leftmargin}{\labelsep}
\setlength{\labelwidth}{\tmplength}
}
\item[\textbf{Declaration}\hfill]
\ifpdf
\begin{flushleft}
\fi
\begin{ttfamily}
public procedure FormDestroy(Sender: TObject);\end{ttfamily}

\ifpdf
\end{flushleft}
\fi

\end{list}
\paragraph*{FormHide}\hspace*{\fill}

\label{mnupdate.TForm1-FormHide}
\index{FormHide}
\begin{list}{}{
\settowidth{\tmplength}{\textbf{Description}}
\setlength{\itemindent}{0cm}
\setlength{\listparindent}{0cm}
\setlength{\leftmargin}{\evensidemargin}
\addtolength{\leftmargin}{\tmplength}
\settowidth{\labelsep}{X}
\addtolength{\leftmargin}{\labelsep}
\setlength{\labelwidth}{\tmplength}
}
\item[\textbf{Declaration}\hfill]
\ifpdf
\begin{flushleft}
\fi
\begin{ttfamily}
public procedure FormHide(Sender: TObject);\end{ttfamily}

\ifpdf
\end{flushleft}
\fi

\end{list}
\paragraph*{FormShow}\hspace*{\fill}

\label{mnupdate.TForm1-FormShow}
\index{FormShow}
\begin{list}{}{
\settowidth{\tmplength}{\textbf{Description}}
\setlength{\itemindent}{0cm}
\setlength{\listparindent}{0cm}
\setlength{\leftmargin}{\evensidemargin}
\addtolength{\leftmargin}{\tmplength}
\settowidth{\labelsep}{X}
\addtolength{\leftmargin}{\labelsep}
\setlength{\labelwidth}{\tmplength}
}
\item[\textbf{Declaration}\hfill]
\ifpdf
\begin{flushleft}
\fi
\begin{ttfamily}
public procedure FormShow(Sender: TObject);\end{ttfamily}

\ifpdf
\end{flushleft}
\fi

\end{list}
\paragraph*{MenuItem1Click}\hspace*{\fill}

\label{mnupdate.TForm1-MenuItem1Click}
\index{MenuItem1Click}
\begin{list}{}{
\settowidth{\tmplength}{\textbf{Description}}
\setlength{\itemindent}{0cm}
\setlength{\listparindent}{0cm}
\setlength{\leftmargin}{\evensidemargin}
\addtolength{\leftmargin}{\tmplength}
\settowidth{\labelsep}{X}
\addtolength{\leftmargin}{\labelsep}
\setlength{\labelwidth}{\tmplength}
}
\item[\textbf{Declaration}\hfill]
\ifpdf
\begin{flushleft}
\fi
\begin{ttfamily}
public procedure MenuItem1Click(Sender: TObject);\end{ttfamily}

\ifpdf
\end{flushleft}
\fi

\end{list}
\paragraph*{MenuItem2Click}\hspace*{\fill}

\label{mnupdate.TForm1-MenuItem2Click}
\index{MenuItem2Click}
\begin{list}{}{
\settowidth{\tmplength}{\textbf{Description}}
\setlength{\itemindent}{0cm}
\setlength{\listparindent}{0cm}
\setlength{\leftmargin}{\evensidemargin}
\addtolength{\leftmargin}{\tmplength}
\settowidth{\labelsep}{X}
\addtolength{\leftmargin}{\labelsep}
\setlength{\labelwidth}{\tmplength}
}
\item[\textbf{Declaration}\hfill]
\ifpdf
\begin{flushleft}
\fi
\begin{ttfamily}
public procedure MenuItem2Click(Sender: TObject);\end{ttfamily}

\ifpdf
\end{flushleft}
\fi

\end{list}
\paragraph*{TrayIcon1Click}\hspace*{\fill}

\label{mnupdate.TForm1-TrayIcon1Click}
\index{TrayIcon1Click}
\begin{list}{}{
\settowidth{\tmplength}{\textbf{Description}}
\setlength{\itemindent}{0cm}
\setlength{\listparindent}{0cm}
\setlength{\leftmargin}{\evensidemargin}
\addtolength{\leftmargin}{\tmplength}
\settowidth{\labelsep}{X}
\addtolength{\leftmargin}{\labelsep}
\setlength{\labelwidth}{\tmplength}
}
\item[\textbf{Declaration}\hfill]
\ifpdf
\begin{flushleft}
\fi
\begin{ttfamily}
public procedure TrayIcon1Click(Sender: TObject);\end{ttfamily}

\ifpdf
\end{flushleft}
\fi

\end{list}
\paragraph*{TrayIcon1DblClick}\hspace*{\fill}

\label{mnupdate.TForm1-TrayIcon1DblClick}
\index{TrayIcon1DblClick}
\begin{list}{}{
\settowidth{\tmplength}{\textbf{Description}}
\setlength{\itemindent}{0cm}
\setlength{\listparindent}{0cm}
\setlength{\leftmargin}{\evensidemargin}
\addtolength{\leftmargin}{\tmplength}
\settowidth{\labelsep}{X}
\addtolength{\leftmargin}{\labelsep}
\setlength{\labelwidth}{\tmplength}
}
\item[\textbf{Declaration}\hfill]
\ifpdf
\begin{flushleft}
\fi
\begin{ttfamily}
public procedure TrayIcon1DblClick(Sender: TObject);\end{ttfamily}

\ifpdf
\end{flushleft}
\fi

\end{list}
\paragraph*{CheckForUpdates}\hspace*{\fill}

\label{mnupdate.TForm1-CheckForUpdates}
\index{CheckForUpdates}
\begin{list}{}{
\settowidth{\tmplength}{\textbf{Description}}
\setlength{\itemindent}{0cm}
\setlength{\listparindent}{0cm}
\setlength{\leftmargin}{\evensidemargin}
\addtolength{\leftmargin}{\tmplength}
\settowidth{\labelsep}{X}
\addtolength{\leftmargin}{\labelsep}
\setlength{\labelwidth}{\tmplength}
}
\item[\textbf{Declaration}\hfill]
\ifpdf
\begin{flushleft}
\fi
\begin{ttfamily}
public procedure CheckForUpdates;\end{ttfamily}

\ifpdf
\end{flushleft}
\fi

\end{list}
\section{Constants}
\ifpdf
\subsection*{\large{\textbf{lp}}\normalsize\hspace{1ex}\hrulefill}
\else
\subsection*{lp}
\fi
\label{mnupdate-lp}
\index{lp}
\begin{list}{}{
\settowidth{\tmplength}{\textbf{Description}}
\setlength{\itemindent}{0cm}
\setlength{\listparindent}{0cm}
\setlength{\leftmargin}{\evensidemargin}
\addtolength{\leftmargin}{\tmplength}
\settowidth{\labelsep}{X}
\addtolength{\leftmargin}{\labelsep}
\setlength{\labelwidth}{\tmplength}
}
\item[\textbf{Declaration}\hfill]
\ifpdf
\begin{flushleft}
\fi
\begin{ttfamily}
lp='/tmp/';\end{ttfamily}

\ifpdf
\end{flushleft}
\fi

\end{list}
\section{Variables}
\ifpdf
\subsection*{\large{\textbf{Form1}}\normalsize\hspace{1ex}\hrulefill}
\else
\subsection*{Form1}
\fi
\label{mnupdate-Form1}
\index{Form1}
\begin{list}{}{
\settowidth{\tmplength}{\textbf{Description}}
\setlength{\itemindent}{0cm}
\setlength{\listparindent}{0cm}
\setlength{\leftmargin}{\evensidemargin}
\addtolength{\leftmargin}{\tmplength}
\settowidth{\labelsep}{X}
\addtolength{\leftmargin}{\labelsep}
\setlength{\labelwidth}{\tmplength}
}
\item[\textbf{Declaration}\hfill]
\ifpdf
\begin{flushleft}
\fi
\begin{ttfamily}
Form1: TForm1;\end{ttfamily}

\ifpdf
\end{flushleft}
\fi

\end{list}
\chapter{Unit pkgconvertdisp}
\label{pkgconvertdisp}
\index{pkgconvertdisp}
\section{uses}
\begin{itemize}
\item \begin{ttfamily}Classes\end{ttfamily}\item \begin{ttfamily}SysUtils\end{ttfamily}\item \begin{ttfamily}LResources\end{ttfamily}\item \begin{ttfamily}Forms\end{ttfamily}\item \begin{ttfamily}Controls\end{ttfamily}\item \begin{ttfamily}Graphics\end{ttfamily}\item \begin{ttfamily}Dialogs\end{ttfamily}\item \begin{ttfamily}StdCtrls\end{ttfamily}\item \begin{ttfamily}process\end{ttfamily}\item \begin{ttfamily}ExtCtrls\end{ttfamily}\item \begin{ttfamily}utilities\end{ttfamily}(\ref{utilities})\item \begin{ttfamily}LCLType\end{ttfamily}\item \begin{ttfamily}manager\end{ttfamily}(\ref{manager})\end{itemize}
\section{Overview}
\begin{description}
\item[\texttt{\begin{ttfamily}TConvDisp\end{ttfamily} Class}]
\end{description}
\section{Classes, Interfaces, Objects and Records}
\ifpdf
\subsection*{\large{\textbf{TConvDisp Class}}\normalsize\hspace{1ex}\hrulefill}
\else
\subsection*{TConvDisp Class}
\fi
\label{pkgconvertdisp.TConvDisp}
\index{TConvDisp}
\subsubsection*{\large{\textbf{Hierarchy}}\normalsize\hspace{1ex}\hfill}
TConvDisp {$>$} TForm
%%%%Description
\subsubsection*{\large{\textbf{Fields}}\normalsize\hspace{1ex}\hfill}
\begin{list}{}{
\settowidth{\tmplength}{\textbf{GetOutPutTimer}}
\setlength{\itemindent}{0cm}
\setlength{\listparindent}{0cm}
\setlength{\leftmargin}{\evensidemargin}
\addtolength{\leftmargin}{\tmplength}
\settowidth{\labelsep}{X}
\addtolength{\leftmargin}{\labelsep}
\setlength{\labelwidth}{\tmplength}
}
\label{pkgconvertdisp.TConvDisp-GetOutPutTimer}
\index{GetOutPutTimer}
\item[\textbf{GetOutPutTimer}\hfill]
\ifpdf
\begin{flushleft}
\fi
\begin{ttfamily}
public GetOutPutTimer: TIdleTimer;\end{ttfamily}

\ifpdf
\end{flushleft}
\fi


\par  \label{pkgconvertdisp.TConvDisp-Label1}
\index{Label1}
\item[\textbf{Label1}\hfill]
\ifpdf
\begin{flushleft}
\fi
\begin{ttfamily}
public Label1: TLabel;\end{ttfamily}

\ifpdf
\end{flushleft}
\fi


\par  \label{pkgconvertdisp.TConvDisp-Memo1}
\index{Memo1}
\item[\textbf{Memo1}\hfill]
\ifpdf
\begin{flushleft}
\fi
\begin{ttfamily}
public Memo1: TMemo;\end{ttfamily}

\ifpdf
\end{flushleft}
\fi


\par  \label{pkgconvertdisp.TConvDisp-Process1}
\index{Process1}
\item[\textbf{Process1}\hfill]
\ifpdf
\begin{flushleft}
\fi
\begin{ttfamily}
public Process1: TProcess;\end{ttfamily}

\ifpdf
\end{flushleft}
\fi


\par  \end{list}
\subsubsection*{\large{\textbf{Methods}}\normalsize\hspace{1ex}\hfill}
\paragraph*{FormCreate}\hspace*{\fill}

\label{pkgconvertdisp.TConvDisp-FormCreate}
\index{FormCreate}
\begin{list}{}{
\settowidth{\tmplength}{\textbf{Description}}
\setlength{\itemindent}{0cm}
\setlength{\listparindent}{0cm}
\setlength{\leftmargin}{\evensidemargin}
\addtolength{\leftmargin}{\tmplength}
\settowidth{\labelsep}{X}
\addtolength{\leftmargin}{\labelsep}
\setlength{\labelwidth}{\tmplength}
}
\item[\textbf{Declaration}\hfill]
\ifpdf
\begin{flushleft}
\fi
\begin{ttfamily}
public procedure FormCreate(Sender: TObject);\end{ttfamily}

\ifpdf
\end{flushleft}
\fi

\end{list}
\paragraph*{GetOutPutTimerTimer}\hspace*{\fill}

\label{pkgconvertdisp.TConvDisp-GetOutPutTimerTimer}
\index{GetOutPutTimerTimer}
\begin{list}{}{
\settowidth{\tmplength}{\textbf{Description}}
\setlength{\itemindent}{0cm}
\setlength{\listparindent}{0cm}
\setlength{\leftmargin}{\evensidemargin}
\addtolength{\leftmargin}{\tmplength}
\settowidth{\labelsep}{X}
\addtolength{\leftmargin}{\labelsep}
\setlength{\labelwidth}{\tmplength}
}
\item[\textbf{Declaration}\hfill]
\ifpdf
\begin{flushleft}
\fi
\begin{ttfamily}
public procedure GetOutPutTimerTimer(Sender: TObject);\end{ttfamily}

\ifpdf
\end{flushleft}
\fi

\end{list}
\section{Variables}
\ifpdf
\subsection*{\large{\textbf{ConvDisp}}\normalsize\hspace{1ex}\hrulefill}
\else
\subsection*{ConvDisp}
\fi
\label{pkgconvertdisp-ConvDisp}
\index{ConvDisp}
\begin{list}{}{
\settowidth{\tmplength}{\textbf{Description}}
\setlength{\itemindent}{0cm}
\setlength{\listparindent}{0cm}
\setlength{\leftmargin}{\evensidemargin}
\addtolength{\leftmargin}{\tmplength}
\settowidth{\labelsep}{X}
\addtolength{\leftmargin}{\labelsep}
\setlength{\labelwidth}{\tmplength}
}
\item[\textbf{Declaration}\hfill]
\ifpdf
\begin{flushleft}
\fi
\begin{ttfamily}
ConvDisp: TConvDisp;\end{ttfamily}

\ifpdf
\end{flushleft}
\fi

\end{list}
\chapter{Unit prjwizard}
\label{prjwizard}
\index{prjwizard}
\section{Description}
Graphical wizard for IPS creation
\section{uses}
\begin{itemize}
\item \begin{ttfamily}Classes\end{ttfamily}\item \begin{ttfamily}SysUtils\end{ttfamily}\item \begin{ttfamily}LResources\end{ttfamily}\item \begin{ttfamily}Forms\end{ttfamily}\item \begin{ttfamily}Controls\end{ttfamily}\item \begin{ttfamily}Graphics\end{ttfamily}\item \begin{ttfamily}Dialogs\end{ttfamily}\item \begin{ttfamily}StdCtrls\end{ttfamily}\item \begin{ttfamily}Buttons\end{ttfamily}\item \begin{ttfamily}ExtCtrls\end{ttfamily}\item \begin{ttfamily}ComCtrls\end{ttfamily}\item \begin{ttfamily}EditBtn\end{ttfamily}\item \begin{ttfamily}Grids\end{ttfamily}\item \begin{ttfamily}popupnotifier\end{ttfamily}\item \begin{ttfamily}FileCtrl\end{ttfamily}\item \begin{ttfamily}FileUtil\end{ttfamily}\item \begin{ttfamily}MD5\end{ttfamily}\item \begin{ttfamily}Menus\end{ttfamily}\item \begin{ttfamily}XMLRead\end{ttfamily}\item \begin{ttfamily}XMLWrite\end{ttfamily}\item \begin{ttfamily}DOM\end{ttfamily}\item \begin{ttfamily}editor\end{ttfamily}(\ref{editor})\item \begin{ttfamily}LCLType\end{ttfamily}\item \begin{ttfamily}GTK2\end{ttfamily}\item \begin{ttfamily}utilities\end{ttfamily}(\ref{utilities})\end{itemize}
\section{Overview}
\begin{description}
\item[\texttt{\begin{ttfamily}TfrmProjectWizard\end{ttfamily} Class}]
\end{description}
\section{Classes, Interfaces, Objects and Records}
\ifpdf
\subsection*{\large{\textbf{TfrmProjectWizard Class}}\normalsize\hspace{1ex}\hrulefill}
\else
\subsection*{TfrmProjectWizard Class}
\fi
\label{prjwizard.TfrmProjectWizard}
\index{TfrmProjectWizard}
\subsubsection*{\large{\textbf{Hierarchy}}\normalsize\hspace{1ex}\hfill}
TfrmProjectWizard {$>$} TForm
%%%%Description
\subsubsection*{\large{\textbf{Fields}}\normalsize\hspace{1ex}\hfill}
\begin{list}{}{
\settowidth{\tmplength}{\textbf{btnAssignShortDescription}}
\setlength{\itemindent}{0cm}
\setlength{\listparindent}{0cm}
\setlength{\leftmargin}{\evensidemargin}
\addtolength{\leftmargin}{\tmplength}
\settowidth{\labelsep}{X}
\addtolength{\leftmargin}{\labelsep}
\setlength{\labelwidth}{\tmplength}
}
\label{prjwizard.TfrmProjectWizard-BitBtn1}
\index{BitBtn1}
\item[\textbf{BitBtn1}\hfill]
\ifpdf
\begin{flushleft}
\fi
\begin{ttfamily}
public BitBtn1: TBitBtn;\end{ttfamily}

\ifpdf
\end{flushleft}
\fi


\par  \label{prjwizard.TfrmProjectWizard-BitBtn2}
\index{BitBtn2}
\item[\textbf{BitBtn2}\hfill]
\ifpdf
\begin{flushleft}
\fi
\begin{ttfamily}
public BitBtn2: TBitBtn;\end{ttfamily}

\ifpdf
\end{flushleft}
\fi


\par  \label{prjwizard.TfrmProjectWizard-BitBtn3}
\index{BitBtn3}
\item[\textbf{BitBtn3}\hfill]
\ifpdf
\begin{flushleft}
\fi
\begin{ttfamily}
public BitBtn3: TBitBtn;\end{ttfamily}

\ifpdf
\end{flushleft}
\fi


\par  \label{prjwizard.TfrmProjectWizard-btnAssignShortDescription}
\index{btnAssignShortDescription}
\item[\textbf{btnAssignShortDescription}\hfill]
\ifpdf
\begin{flushleft}
\fi
\begin{ttfamily}
public btnAssignShortDescription: TButton;\end{ttfamily}

\ifpdf
\end{flushleft}
\fi


\par  \label{prjwizard.TfrmProjectWizard-Button2}
\index{Button2}
\item[\textbf{Button2}\hfill]
\ifpdf
\begin{flushleft}
\fi
\begin{ttfamily}
public Button2: TButton;\end{ttfamily}

\ifpdf
\end{flushleft}
\fi


\par  \label{prjwizard.TfrmProjectWizard-Button3}
\index{Button3}
\item[\textbf{Button3}\hfill]
\ifpdf
\begin{flushleft}
\fi
\begin{ttfamily}
public Button3: TButton;\end{ttfamily}

\ifpdf
\end{flushleft}
\fi


\par  \label{prjwizard.TfrmProjectWizard-Button4}
\index{Button4}
\item[\textbf{Button4}\hfill]
\ifpdf
\begin{flushleft}
\fi
\begin{ttfamily}
public Button4: TButton;\end{ttfamily}

\ifpdf
\end{flushleft}
\fi


\par  \label{prjwizard.TfrmProjectWizard-btnAddLangCode}
\index{btnAddLangCode}
\item[\textbf{btnAddLangCode}\hfill]
\ifpdf
\begin{flushleft}
\fi
\begin{ttfamily}
public btnAddLangCode: TButton;\end{ttfamily}

\ifpdf
\end{flushleft}
\fi


\par  \label{prjwizard.TfrmProjectWizard-Button6}
\index{Button6}
\item[\textbf{Button6}\hfill]
\ifpdf
\begin{flushleft}
\fi
\begin{ttfamily}
public Button6: TButton;\end{ttfamily}

\ifpdf
\end{flushleft}
\fi


\par  \label{prjwizard.TfrmProjectWizard-chkShowInTerminal}
\index{chkShowInTerminal}
\item[\textbf{chkShowInTerminal}\hfill]
\ifpdf
\begin{flushleft}
\fi
\begin{ttfamily}
public chkShowInTerminal: TCheckBox;\end{ttfamily}

\ifpdf
\end{flushleft}
\fi


\par  \label{prjwizard.TfrmProjectWizard-ComboBox1}
\index{ComboBox1}
\item[\textbf{ComboBox1}\hfill]
\ifpdf
\begin{flushleft}
\fi
\begin{ttfamily}
public ComboBox1: TComboBox;\end{ttfamily}

\ifpdf
\end{flushleft}
\fi


\par  \label{prjwizard.TfrmProjectWizard-DirectoryEdit1}
\index{DirectoryEdit1}
\item[\textbf{DirectoryEdit1}\hfill]
\ifpdf
\begin{flushleft}
\fi
\begin{ttfamily}
public DirectoryEdit1: TDirectoryEdit;\end{ttfamily}

\ifpdf
\end{flushleft}
\fi


\par  \label{prjwizard.TfrmProjectWizard-Edit1}
\index{Edit1}
\item[\textbf{Edit1}\hfill]
\ifpdf
\begin{flushleft}
\fi
\begin{ttfamily}
public Edit1: TEdit;\end{ttfamily}

\ifpdf
\end{flushleft}
\fi


\par  \label{prjwizard.TfrmProjectWizard-Edit10}
\index{Edit10}
\item[\textbf{Edit10}\hfill]
\ifpdf
\begin{flushleft}
\fi
\begin{ttfamily}
public Edit10: TEdit;\end{ttfamily}

\ifpdf
\end{flushleft}
\fi


\par  \label{prjwizard.TfrmProjectWizard-Edit11}
\index{Edit11}
\item[\textbf{Edit11}\hfill]
\ifpdf
\begin{flushleft}
\fi
\begin{ttfamily}
public Edit11: TEdit;\end{ttfamily}

\ifpdf
\end{flushleft}
\fi


\par  \label{prjwizard.TfrmProjectWizard-Edit12}
\index{Edit12}
\item[\textbf{Edit12}\hfill]
\ifpdf
\begin{flushleft}
\fi
\begin{ttfamily}
public Edit12: TEdit;\end{ttfamily}

\ifpdf
\end{flushleft}
\fi


\par  \label{prjwizard.TfrmProjectWizard-Edit2}
\index{Edit2}
\item[\textbf{Edit2}\hfill]
\ifpdf
\begin{flushleft}
\fi
\begin{ttfamily}
public Edit2: TEdit;\end{ttfamily}

\ifpdf
\end{flushleft}
\fi


\par  \label{prjwizard.TfrmProjectWizard-Edit3}
\index{Edit3}
\item[\textbf{Edit3}\hfill]
\ifpdf
\begin{flushleft}
\fi
\begin{ttfamily}
public Edit3: TEdit;\end{ttfamily}

\ifpdf
\end{flushleft}
\fi


\par  \label{prjwizard.TfrmProjectWizard-edtShortDescription}
\index{edtShortDescription}
\item[\textbf{edtShortDescription}\hfill]
\ifpdf
\begin{flushleft}
\fi
\begin{ttfamily}
public edtShortDescription: TEdit;\end{ttfamily}

\ifpdf
\end{flushleft}
\fi


\par  \label{prjwizard.TfrmProjectWizard-Edit5}
\index{Edit5}
\item[\textbf{Edit5}\hfill]
\ifpdf
\begin{flushleft}
\fi
\begin{ttfamily}
public Edit5: TEdit;\end{ttfamily}

\ifpdf
\end{flushleft}
\fi


\par  \label{prjwizard.TfrmProjectWizard-Edit6}
\index{Edit6}
\item[\textbf{Edit6}\hfill]
\ifpdf
\begin{flushleft}
\fi
\begin{ttfamily}
public Edit6: TEdit;\end{ttfamily}

\ifpdf
\end{flushleft}
\fi


\par  \label{prjwizard.TfrmProjectWizard-Edit7}
\index{Edit7}
\item[\textbf{Edit7}\hfill]
\ifpdf
\begin{flushleft}
\fi
\begin{ttfamily}
public Edit7: TEdit;\end{ttfamily}

\ifpdf
\end{flushleft}
\fi


\par  \label{prjwizard.TfrmProjectWizard-Edit8}
\index{Edit8}
\item[\textbf{Edit8}\hfill]
\ifpdf
\begin{flushleft}
\fi
\begin{ttfamily}
public Edit8: TEdit;\end{ttfamily}

\ifpdf
\end{flushleft}
\fi


\par  \label{prjwizard.TfrmProjectWizard-edtLangCode}
\index{edtLangCode}
\item[\textbf{edtLangCode}\hfill]
\ifpdf
\begin{flushleft}
\fi
\begin{ttfamily}
public edtLangCode: TEdit;\end{ttfamily}

\ifpdf
\end{flushleft}
\fi


\par  \label{prjwizard.TfrmProjectWizard-FileNameEdit1}
\index{FileNameEdit1}
\item[\textbf{FileNameEdit1}\hfill]
\ifpdf
\begin{flushleft}
\fi
\begin{ttfamily}
public FileNameEdit1: TFileNameEdit;\end{ttfamily}

\ifpdf
\end{flushleft}
\fi


\par  \label{prjwizard.TfrmProjectWizard-FileNameEdit2}
\index{FileNameEdit2}
\item[\textbf{FileNameEdit2}\hfill]
\ifpdf
\begin{flushleft}
\fi
\begin{ttfamily}
public FileNameEdit2: TFileNameEdit;\end{ttfamily}

\ifpdf
\end{flushleft}
\fi


\par  \label{prjwizard.TfrmProjectWizard-FileNameEdit3}
\index{FileNameEdit3}
\item[\textbf{FileNameEdit3}\hfill]
\ifpdf
\begin{flushleft}
\fi
\begin{ttfamily}
public FileNameEdit3: TFileNameEdit;\end{ttfamily}

\ifpdf
\end{flushleft}
\fi


\par  \label{prjwizard.TfrmProjectWizard-FileNameEdit4}
\index{FileNameEdit4}
\item[\textbf{FileNameEdit4}\hfill]
\ifpdf
\begin{flushleft}
\fi
\begin{ttfamily}
public FileNameEdit4: TFileNameEdit;\end{ttfamily}

\ifpdf
\end{flushleft}
\fi


\par  \label{prjwizard.TfrmProjectWizard-FileNameEdit5}
\index{FileNameEdit5}
\item[\textbf{FileNameEdit5}\hfill]
\ifpdf
\begin{flushleft}
\fi
\begin{ttfamily}
public FileNameEdit5: TFileNameEdit;\end{ttfamily}

\ifpdf
\end{flushleft}
\fi


\par  \label{prjwizard.TfrmProjectWizard-GroupBox1}
\index{GroupBox1}
\item[\textbf{GroupBox1}\hfill]
\ifpdf
\begin{flushleft}
\fi
\begin{ttfamily}
public GroupBox1: TGroupBox;\end{ttfamily}

\ifpdf
\end{flushleft}
\fi


\par  \label{prjwizard.TfrmProjectWizard-GroupBox10}
\index{GroupBox10}
\item[\textbf{GroupBox10}\hfill]
\ifpdf
\begin{flushleft}
\fi
\begin{ttfamily}
public GroupBox10: TGroupBox;\end{ttfamily}

\ifpdf
\end{flushleft}
\fi


\par  \label{prjwizard.TfrmProjectWizard-GroupBox11}
\index{GroupBox11}
\item[\textbf{GroupBox11}\hfill]
\ifpdf
\begin{flushleft}
\fi
\begin{ttfamily}
public GroupBox11: TGroupBox;\end{ttfamily}

\ifpdf
\end{flushleft}
\fi


\par  \label{prjwizard.TfrmProjectWizard-GroupBox12}
\index{GroupBox12}
\item[\textbf{GroupBox12}\hfill]
\ifpdf
\begin{flushleft}
\fi
\begin{ttfamily}
public GroupBox12: TGroupBox;\end{ttfamily}

\ifpdf
\end{flushleft}
\fi


\par  \label{prjwizard.TfrmProjectWizard-GroupBox2}
\index{GroupBox2}
\item[\textbf{GroupBox2}\hfill]
\ifpdf
\begin{flushleft}
\fi
\begin{ttfamily}
public GroupBox2: TGroupBox;\end{ttfamily}

\ifpdf
\end{flushleft}
\fi


\par  \label{prjwizard.TfrmProjectWizard-GroupBox3}
\index{GroupBox3}
\item[\textbf{GroupBox3}\hfill]
\ifpdf
\begin{flushleft}
\fi
\begin{ttfamily}
public GroupBox3: TGroupBox;\end{ttfamily}

\ifpdf
\end{flushleft}
\fi


\par  \label{prjwizard.TfrmProjectWizard-GroupBox4}
\index{GroupBox4}
\item[\textbf{GroupBox4}\hfill]
\ifpdf
\begin{flushleft}
\fi
\begin{ttfamily}
public GroupBox4: TGroupBox;\end{ttfamily}

\ifpdf
\end{flushleft}
\fi


\par  \label{prjwizard.TfrmProjectWizard-GroupBox5}
\index{GroupBox5}
\item[\textbf{GroupBox5}\hfill]
\ifpdf
\begin{flushleft}
\fi
\begin{ttfamily}
public GroupBox5: TGroupBox;\end{ttfamily}

\ifpdf
\end{flushleft}
\fi


\par  \label{prjwizard.TfrmProjectWizard-GroupBox6}
\index{GroupBox6}
\item[\textbf{GroupBox6}\hfill]
\ifpdf
\begin{flushleft}
\fi
\begin{ttfamily}
public GroupBox6: TGroupBox;\end{ttfamily}

\ifpdf
\end{flushleft}
\fi


\par  \label{prjwizard.TfrmProjectWizard-GroupBox7}
\index{GroupBox7}
\item[\textbf{GroupBox7}\hfill]
\ifpdf
\begin{flushleft}
\fi
\begin{ttfamily}
public GroupBox7: TGroupBox;\end{ttfamily}

\ifpdf
\end{flushleft}
\fi


\par  \label{prjwizard.TfrmProjectWizard-GroupBox8}
\index{GroupBox8}
\item[\textbf{GroupBox8}\hfill]
\ifpdf
\begin{flushleft}
\fi
\begin{ttfamily}
public GroupBox8: TGroupBox;\end{ttfamily}

\ifpdf
\end{flushleft}
\fi


\par  \label{prjwizard.TfrmProjectWizard-GroupBox9}
\index{GroupBox9}
\item[\textbf{GroupBox9}\hfill]
\ifpdf
\begin{flushleft}
\fi
\begin{ttfamily}
public GroupBox9: TGroupBox;\end{ttfamily}

\ifpdf
\end{flushleft}
\fi


\par  \label{prjwizard.TfrmProjectWizard-Label1}
\index{Label1}
\item[\textbf{Label1}\hfill]
\ifpdf
\begin{flushleft}
\fi
\begin{ttfamily}
public Label1: TLabel;\end{ttfamily}

\ifpdf
\end{flushleft}
\fi


\par  \label{prjwizard.TfrmProjectWizard-Label10}
\index{Label10}
\item[\textbf{Label10}\hfill]
\ifpdf
\begin{flushleft}
\fi
\begin{ttfamily}
public Label10: TLabel;\end{ttfamily}

\ifpdf
\end{flushleft}
\fi


\par  \label{prjwizard.TfrmProjectWizard-Label11}
\index{Label11}
\item[\textbf{Label11}\hfill]
\ifpdf
\begin{flushleft}
\fi
\begin{ttfamily}
public Label11: TLabel;\end{ttfamily}

\ifpdf
\end{flushleft}
\fi


\par  \label{prjwizard.TfrmProjectWizard-Label12}
\index{Label12}
\item[\textbf{Label12}\hfill]
\ifpdf
\begin{flushleft}
\fi
\begin{ttfamily}
public Label12: TLabel;\end{ttfamily}

\ifpdf
\end{flushleft}
\fi


\par  \label{prjwizard.TfrmProjectWizard-Label13}
\index{Label13}
\item[\textbf{Label13}\hfill]
\ifpdf
\begin{flushleft}
\fi
\begin{ttfamily}
public Label13: TLabel;\end{ttfamily}

\ifpdf
\end{flushleft}
\fi


\par  \label{prjwizard.TfrmProjectWizard-Label14}
\index{Label14}
\item[\textbf{Label14}\hfill]
\ifpdf
\begin{flushleft}
\fi
\begin{ttfamily}
public Label14: TLabel;\end{ttfamily}

\ifpdf
\end{flushleft}
\fi


\par  \label{prjwizard.TfrmProjectWizard-Label15}
\index{Label15}
\item[\textbf{Label15}\hfill]
\ifpdf
\begin{flushleft}
\fi
\begin{ttfamily}
public Label15: TLabel;\end{ttfamily}

\ifpdf
\end{flushleft}
\fi


\par  \label{prjwizard.TfrmProjectWizard-Label16}
\index{Label16}
\item[\textbf{Label16}\hfill]
\ifpdf
\begin{flushleft}
\fi
\begin{ttfamily}
public Label16: TLabel;\end{ttfamily}

\ifpdf
\end{flushleft}
\fi


\par  \label{prjwizard.TfrmProjectWizard-Label17}
\index{Label17}
\item[\textbf{Label17}\hfill]
\ifpdf
\begin{flushleft}
\fi
\begin{ttfamily}
public Label17: TLabel;\end{ttfamily}

\ifpdf
\end{flushleft}
\fi


\par  \label{prjwizard.TfrmProjectWizard-Label18}
\index{Label18}
\item[\textbf{Label18}\hfill]
\ifpdf
\begin{flushleft}
\fi
\begin{ttfamily}
public Label18: TLabel;\end{ttfamily}

\ifpdf
\end{flushleft}
\fi


\par  \label{prjwizard.TfrmProjectWizard-Label19}
\index{Label19}
\item[\textbf{Label19}\hfill]
\ifpdf
\begin{flushleft}
\fi
\begin{ttfamily}
public Label19: TLabel;\end{ttfamily}

\ifpdf
\end{flushleft}
\fi


\par  \label{prjwizard.TfrmProjectWizard-Label2}
\index{Label2}
\item[\textbf{Label2}\hfill]
\ifpdf
\begin{flushleft}
\fi
\begin{ttfamily}
public Label2: TLabel;\end{ttfamily}

\ifpdf
\end{flushleft}
\fi


\par  \label{prjwizard.TfrmProjectWizard-Label20}
\index{Label20}
\item[\textbf{Label20}\hfill]
\ifpdf
\begin{flushleft}
\fi
\begin{ttfamily}
public Label20: TLabel;\end{ttfamily}

\ifpdf
\end{flushleft}
\fi


\par  \label{prjwizard.TfrmProjectWizard-Label21}
\index{Label21}
\item[\textbf{Label21}\hfill]
\ifpdf
\begin{flushleft}
\fi
\begin{ttfamily}
public Label21: TLabel;\end{ttfamily}

\ifpdf
\end{flushleft}
\fi


\par  \label{prjwizard.TfrmProjectWizard-Label22}
\index{Label22}
\item[\textbf{Label22}\hfill]
\ifpdf
\begin{flushleft}
\fi
\begin{ttfamily}
public Label22: TLabel;\end{ttfamily}

\ifpdf
\end{flushleft}
\fi


\par  \label{prjwizard.TfrmProjectWizard-Label23}
\index{Label23}
\item[\textbf{Label23}\hfill]
\ifpdf
\begin{flushleft}
\fi
\begin{ttfamily}
public Label23: TLabel;\end{ttfamily}

\ifpdf
\end{flushleft}
\fi


\par  \label{prjwizard.TfrmProjectWizard-Label24}
\index{Label24}
\item[\textbf{Label24}\hfill]
\ifpdf
\begin{flushleft}
\fi
\begin{ttfamily}
public Label24: TLabel;\end{ttfamily}

\ifpdf
\end{flushleft}
\fi


\par  \label{prjwizard.TfrmProjectWizard-Label25}
\index{Label25}
\item[\textbf{Label25}\hfill]
\ifpdf
\begin{flushleft}
\fi
\begin{ttfamily}
public Label25: TLabel;\end{ttfamily}

\ifpdf
\end{flushleft}
\fi


\par  \label{prjwizard.TfrmProjectWizard-Label26}
\index{Label26}
\item[\textbf{Label26}\hfill]
\ifpdf
\begin{flushleft}
\fi
\begin{ttfamily}
public Label26: TLabel;\end{ttfamily}

\ifpdf
\end{flushleft}
\fi


\par  \label{prjwizard.TfrmProjectWizard-Label27}
\index{Label27}
\item[\textbf{Label27}\hfill]
\ifpdf
\begin{flushleft}
\fi
\begin{ttfamily}
public Label27: TLabel;\end{ttfamily}

\ifpdf
\end{flushleft}
\fi


\par  \label{prjwizard.TfrmProjectWizard-Label28}
\index{Label28}
\item[\textbf{Label28}\hfill]
\ifpdf
\begin{flushleft}
\fi
\begin{ttfamily}
public Label28: TLabel;\end{ttfamily}

\ifpdf
\end{flushleft}
\fi


\par  \label{prjwizard.TfrmProjectWizard-Label29}
\index{Label29}
\item[\textbf{Label29}\hfill]
\ifpdf
\begin{flushleft}
\fi
\begin{ttfamily}
public Label29: TLabel;\end{ttfamily}

\ifpdf
\end{flushleft}
\fi


\par  \label{prjwizard.TfrmProjectWizard-Label3}
\index{Label3}
\item[\textbf{Label3}\hfill]
\ifpdf
\begin{flushleft}
\fi
\begin{ttfamily}
public Label3: TLabel;\end{ttfamily}

\ifpdf
\end{flushleft}
\fi


\par  \label{prjwizard.TfrmProjectWizard-Label30}
\index{Label30}
\item[\textbf{Label30}\hfill]
\ifpdf
\begin{flushleft}
\fi
\begin{ttfamily}
public Label30: TLabel;\end{ttfamily}

\ifpdf
\end{flushleft}
\fi


\par  \label{prjwizard.TfrmProjectWizard-Label4}
\index{Label4}
\item[\textbf{Label4}\hfill]
\ifpdf
\begin{flushleft}
\fi
\begin{ttfamily}
public Label4: TLabel;\end{ttfamily}

\ifpdf
\end{flushleft}
\fi


\par  \label{prjwizard.TfrmProjectWizard-Label5}
\index{Label5}
\item[\textbf{Label5}\hfill]
\ifpdf
\begin{flushleft}
\fi
\begin{ttfamily}
public Label5: TLabel;\end{ttfamily}

\ifpdf
\end{flushleft}
\fi


\par  \label{prjwizard.TfrmProjectWizard-Label6}
\index{Label6}
\item[\textbf{Label6}\hfill]
\ifpdf
\begin{flushleft}
\fi
\begin{ttfamily}
public Label6: TLabel;\end{ttfamily}

\ifpdf
\end{flushleft}
\fi


\par  \label{prjwizard.TfrmProjectWizard-Label7}
\index{Label7}
\item[\textbf{Label7}\hfill]
\ifpdf
\begin{flushleft}
\fi
\begin{ttfamily}
public Label7: TLabel;\end{ttfamily}

\ifpdf
\end{flushleft}
\fi


\par  \label{prjwizard.TfrmProjectWizard-Label8}
\index{Label8}
\item[\textbf{Label8}\hfill]
\ifpdf
\begin{flushleft}
\fi
\begin{ttfamily}
public Label8: TLabel;\end{ttfamily}

\ifpdf
\end{flushleft}
\fi


\par  \label{prjwizard.TfrmProjectWizard-Label9}
\index{Label9}
\item[\textbf{Label9}\hfill]
\ifpdf
\begin{flushleft}
\fi
\begin{ttfamily}
public Label9: TLabel;\end{ttfamily}

\ifpdf
\end{flushleft}
\fi


\par  \label{prjwizard.TfrmProjectWizard-ListBox1}
\index{ListBox1}
\item[\textbf{ListBox1}\hfill]
\ifpdf
\begin{flushleft}
\fi
\begin{ttfamily}
public ListBox1: TListBox;\end{ttfamily}

\ifpdf
\end{flushleft}
\fi


\par  \label{prjwizard.TfrmProjectWizard-lvPackageFiles}
\index{lvPackageFiles}
\item[\textbf{lvPackageFiles}\hfill]
\ifpdf
\begin{flushleft}
\fi
\begin{ttfamily}
public lvPackageFiles: TListView;\end{ttfamily}

\ifpdf
\end{flushleft}
\fi


\par  \label{prjwizard.TfrmProjectWizard-Memo1}
\index{Memo1}
\item[\textbf{Memo1}\hfill]
\ifpdf
\begin{flushleft}
\fi
\begin{ttfamily}
public Memo1: TMemo;\end{ttfamily}

\ifpdf
\end{flushleft}
\fi


\par  \label{prjwizard.TfrmProjectWizard-MenuItem1}
\index{MenuItem1}
\item[\textbf{MenuItem1}\hfill]
\ifpdf
\begin{flushleft}
\fi
\begin{ttfamily}
public MenuItem1: TMenuItem;\end{ttfamily}

\ifpdf
\end{flushleft}
\fi


\par  \label{prjwizard.TfrmProjectWizard-Notebook1}
\index{Notebook1}
\item[\textbf{Notebook1}\hfill]
\ifpdf
\begin{flushleft}
\fi
\begin{ttfamily}
public Notebook1: TNotebook;\end{ttfamily}

\ifpdf
\end{flushleft}
\fi


\par  \label{prjwizard.TfrmProjectWizard-Page1}
\index{Page1}
\item[\textbf{Page1}\hfill]
\ifpdf
\begin{flushleft}
\fi
\begin{ttfamily}
public Page1: TPage;\end{ttfamily}

\ifpdf
\end{flushleft}
\fi


\par  \label{prjwizard.TfrmProjectWizard-Page2}
\index{Page2}
\item[\textbf{Page2}\hfill]
\ifpdf
\begin{flushleft}
\fi
\begin{ttfamily}
public Page2: TPage;\end{ttfamily}

\ifpdf
\end{flushleft}
\fi


\par  \label{prjwizard.TfrmProjectWizard-Page3}
\index{Page3}
\item[\textbf{Page3}\hfill]
\ifpdf
\begin{flushleft}
\fi
\begin{ttfamily}
public Page3: TPage;\end{ttfamily}

\ifpdf
\end{flushleft}
\fi


\par  \label{prjwizard.TfrmProjectWizard-Page4}
\index{Page4}
\item[\textbf{Page4}\hfill]
\ifpdf
\begin{flushleft}
\fi
\begin{ttfamily}
public Page4: TPage;\end{ttfamily}

\ifpdf
\end{flushleft}
\fi


\par  \label{prjwizard.TfrmProjectWizard-Page5}
\index{Page5}
\item[\textbf{Page5}\hfill]
\ifpdf
\begin{flushleft}
\fi
\begin{ttfamily}
public Page5: TPage;\end{ttfamily}

\ifpdf
\end{flushleft}
\fi


\par  \label{prjwizard.TfrmProjectWizard-Page6}
\index{Page6}
\item[\textbf{Page6}\hfill]
\ifpdf
\begin{flushleft}
\fi
\begin{ttfamily}
public Page6: TPage;\end{ttfamily}

\ifpdf
\end{flushleft}
\fi


\par  \label{prjwizard.TfrmProjectWizard-Panel1}
\index{Panel1}
\item[\textbf{Panel1}\hfill]
\ifpdf
\begin{flushleft}
\fi
\begin{ttfamily}
public Panel1: TPanel;\end{ttfamily}

\ifpdf
\end{flushleft}
\fi


\par  \label{prjwizard.TfrmProjectWizard-PopupMenu1}
\index{PopupMenu1}
\item[\textbf{PopupMenu1}\hfill]
\ifpdf
\begin{flushleft}
\fi
\begin{ttfamily}
public PopupMenu1: TPopupMenu;\end{ttfamily}

\ifpdf
\end{flushleft}
\fi


\par  \label{prjwizard.TfrmProjectWizard-RadioButton1}
\index{RadioButton1}
\item[\textbf{RadioButton1}\hfill]
\ifpdf
\begin{flushleft}
\fi
\begin{ttfamily}
public RadioButton1: TRadioButton;\end{ttfamily}

\ifpdf
\end{flushleft}
\fi


\par  \label{prjwizard.TfrmProjectWizard-RadioButton2}
\index{RadioButton2}
\item[\textbf{RadioButton2}\hfill]
\ifpdf
\begin{flushleft}
\fi
\begin{ttfamily}
public RadioButton2: TRadioButton;\end{ttfamily}

\ifpdf
\end{flushleft}
\fi


\par  \label{prjwizard.TfrmProjectWizard-SpeedButton1}
\index{SpeedButton1}
\item[\textbf{SpeedButton1}\hfill]
\ifpdf
\begin{flushleft}
\fi
\begin{ttfamily}
public SpeedButton1: TSpeedButton;\end{ttfamily}

\ifpdf
\end{flushleft}
\fi


\par  \label{prjwizard.TfrmProjectWizard-SpeedButton2}
\index{SpeedButton2}
\item[\textbf{SpeedButton2}\hfill]
\ifpdf
\begin{flushleft}
\fi
\begin{ttfamily}
public SpeedButton2: TSpeedButton;\end{ttfamily}

\ifpdf
\end{flushleft}
\fi


\par  \label{prjwizard.TfrmProjectWizard-SpeedButton3}
\index{SpeedButton3}
\item[\textbf{SpeedButton3}\hfill]
\ifpdf
\begin{flushleft}
\fi
\begin{ttfamily}
public SpeedButton3: TSpeedButton;\end{ttfamily}

\ifpdf
\end{flushleft}
\fi


\par  \label{prjwizard.TfrmProjectWizard-SpeedButton4}
\index{SpeedButton4}
\item[\textbf{SpeedButton4}\hfill]
\ifpdf
\begin{flushleft}
\fi
\begin{ttfamily}
public SpeedButton4: TSpeedButton;\end{ttfamily}

\ifpdf
\end{flushleft}
\fi


\par  \label{prjwizard.TfrmProjectWizard-tvShortDescriptions}
\index{tvShortDescriptions}
\item[\textbf{tvShortDescriptions}\hfill]
\ifpdf
\begin{flushleft}
\fi
\begin{ttfamily}
public tvShortDescriptions: TTreeView;\end{ttfamily}

\ifpdf
\end{flushleft}
\fi


\par  \label{prjwizard.TfrmProjectWizard-tvDependencies}
\index{tvDependencies}
\item[\textbf{tvDependencies}\hfill]
\ifpdf
\begin{flushleft}
\fi
\begin{ttfamily}
public tvDependencies: TTreeView;\end{ttfamily}

\ifpdf
\end{flushleft}
\fi


\par  \end{list}
\subsubsection*{\large{\textbf{Methods}}\normalsize\hspace{1ex}\hfill}
\paragraph*{BitBtn1Click}\hspace*{\fill}

\label{prjwizard.TfrmProjectWizard-BitBtn1Click}
\index{BitBtn1Click}
\begin{list}{}{
\settowidth{\tmplength}{\textbf{Description}}
\setlength{\itemindent}{0cm}
\setlength{\listparindent}{0cm}
\setlength{\leftmargin}{\evensidemargin}
\addtolength{\leftmargin}{\tmplength}
\settowidth{\labelsep}{X}
\addtolength{\leftmargin}{\labelsep}
\setlength{\labelwidth}{\tmplength}
}
\item[\textbf{Declaration}\hfill]
\ifpdf
\begin{flushleft}
\fi
\begin{ttfamily}
public procedure BitBtn1Click(Sender: TObject);\end{ttfamily}

\ifpdf
\end{flushleft}
\fi

\end{list}
\paragraph*{BitBtn2Click}\hspace*{\fill}

\label{prjwizard.TfrmProjectWizard-BitBtn2Click}
\index{BitBtn2Click}
\begin{list}{}{
\settowidth{\tmplength}{\textbf{Description}}
\setlength{\itemindent}{0cm}
\setlength{\listparindent}{0cm}
\setlength{\leftmargin}{\evensidemargin}
\addtolength{\leftmargin}{\tmplength}
\settowidth{\labelsep}{X}
\addtolength{\leftmargin}{\labelsep}
\setlength{\labelwidth}{\tmplength}
}
\item[\textbf{Declaration}\hfill]
\ifpdf
\begin{flushleft}
\fi
\begin{ttfamily}
public procedure BitBtn2Click(Sender: TObject);\end{ttfamily}

\ifpdf
\end{flushleft}
\fi

\end{list}
\paragraph*{BitBtn3Click}\hspace*{\fill}

\label{prjwizard.TfrmProjectWizard-BitBtn3Click}
\index{BitBtn3Click}
\begin{list}{}{
\settowidth{\tmplength}{\textbf{Description}}
\setlength{\itemindent}{0cm}
\setlength{\listparindent}{0cm}
\setlength{\leftmargin}{\evensidemargin}
\addtolength{\leftmargin}{\tmplength}
\settowidth{\labelsep}{X}
\addtolength{\leftmargin}{\labelsep}
\setlength{\labelwidth}{\tmplength}
}
\item[\textbf{Declaration}\hfill]
\ifpdf
\begin{flushleft}
\fi
\begin{ttfamily}
public procedure BitBtn3Click(Sender: TObject);\end{ttfamily}

\ifpdf
\end{flushleft}
\fi

\end{list}
\paragraph*{btnAssignShortDescriptionClick}\hspace*{\fill}

\label{prjwizard.TfrmProjectWizard-btnAssignShortDescriptionClick}
\index{btnAssignShortDescriptionClick}
\begin{list}{}{
\settowidth{\tmplength}{\textbf{Description}}
\setlength{\itemindent}{0cm}
\setlength{\listparindent}{0cm}
\setlength{\leftmargin}{\evensidemargin}
\addtolength{\leftmargin}{\tmplength}
\settowidth{\labelsep}{X}
\addtolength{\leftmargin}{\labelsep}
\setlength{\labelwidth}{\tmplength}
}
\item[\textbf{Declaration}\hfill]
\ifpdf
\begin{flushleft}
\fi
\begin{ttfamily}
public procedure btnAssignShortDescriptionClick(Sender: TObject);\end{ttfamily}

\ifpdf
\end{flushleft}
\fi

\end{list}
\paragraph*{Button2Click}\hspace*{\fill}

\label{prjwizard.TfrmProjectWizard-Button2Click}
\index{Button2Click}
\begin{list}{}{
\settowidth{\tmplength}{\textbf{Description}}
\setlength{\itemindent}{0cm}
\setlength{\listparindent}{0cm}
\setlength{\leftmargin}{\evensidemargin}
\addtolength{\leftmargin}{\tmplength}
\settowidth{\labelsep}{X}
\addtolength{\leftmargin}{\labelsep}
\setlength{\labelwidth}{\tmplength}
}
\item[\textbf{Declaration}\hfill]
\ifpdf
\begin{flushleft}
\fi
\begin{ttfamily}
public procedure Button2Click(Sender: TObject);\end{ttfamily}

\ifpdf
\end{flushleft}
\fi

\end{list}
\paragraph*{Button3Click}\hspace*{\fill}

\label{prjwizard.TfrmProjectWizard-Button3Click}
\index{Button3Click}
\begin{list}{}{
\settowidth{\tmplength}{\textbf{Description}}
\setlength{\itemindent}{0cm}
\setlength{\listparindent}{0cm}
\setlength{\leftmargin}{\evensidemargin}
\addtolength{\leftmargin}{\tmplength}
\settowidth{\labelsep}{X}
\addtolength{\leftmargin}{\labelsep}
\setlength{\labelwidth}{\tmplength}
}
\item[\textbf{Declaration}\hfill]
\ifpdf
\begin{flushleft}
\fi
\begin{ttfamily}
public procedure Button3Click(Sender: TObject);\end{ttfamily}

\ifpdf
\end{flushleft}
\fi

\end{list}
\paragraph*{Button4Click}\hspace*{\fill}

\label{prjwizard.TfrmProjectWizard-Button4Click}
\index{Button4Click}
\begin{list}{}{
\settowidth{\tmplength}{\textbf{Description}}
\setlength{\itemindent}{0cm}
\setlength{\listparindent}{0cm}
\setlength{\leftmargin}{\evensidemargin}
\addtolength{\leftmargin}{\tmplength}
\settowidth{\labelsep}{X}
\addtolength{\leftmargin}{\labelsep}
\setlength{\labelwidth}{\tmplength}
}
\item[\textbf{Declaration}\hfill]
\ifpdf
\begin{flushleft}
\fi
\begin{ttfamily}
public procedure Button4Click(Sender: TObject);\end{ttfamily}

\ifpdf
\end{flushleft}
\fi

\end{list}
\paragraph*{btnAddLangCodeClick}\hspace*{\fill}

\label{prjwizard.TfrmProjectWizard-btnAddLangCodeClick}
\index{btnAddLangCodeClick}
\begin{list}{}{
\settowidth{\tmplength}{\textbf{Description}}
\setlength{\itemindent}{0cm}
\setlength{\listparindent}{0cm}
\setlength{\leftmargin}{\evensidemargin}
\addtolength{\leftmargin}{\tmplength}
\settowidth{\labelsep}{X}
\addtolength{\leftmargin}{\labelsep}
\setlength{\labelwidth}{\tmplength}
}
\item[\textbf{Declaration}\hfill]
\ifpdf
\begin{flushleft}
\fi
\begin{ttfamily}
public procedure btnAddLangCodeClick(Sender: TObject);\end{ttfamily}

\ifpdf
\end{flushleft}
\fi

\end{list}
\paragraph*{Button6Click}\hspace*{\fill}

\label{prjwizard.TfrmProjectWizard-Button6Click}
\index{Button6Click}
\begin{list}{}{
\settowidth{\tmplength}{\textbf{Description}}
\setlength{\itemindent}{0cm}
\setlength{\listparindent}{0cm}
\setlength{\leftmargin}{\evensidemargin}
\addtolength{\leftmargin}{\tmplength}
\settowidth{\labelsep}{X}
\addtolength{\leftmargin}{\labelsep}
\setlength{\labelwidth}{\tmplength}
}
\item[\textbf{Declaration}\hfill]
\ifpdf
\begin{flushleft}
\fi
\begin{ttfamily}
public procedure Button6Click(Sender: TObject);\end{ttfamily}

\ifpdf
\end{flushleft}
\fi

\end{list}
\paragraph*{Edit1Change}\hspace*{\fill}

\label{prjwizard.TfrmProjectWizard-Edit1Change}
\index{Edit1Change}
\begin{list}{}{
\settowidth{\tmplength}{\textbf{Description}}
\setlength{\itemindent}{0cm}
\setlength{\listparindent}{0cm}
\setlength{\leftmargin}{\evensidemargin}
\addtolength{\leftmargin}{\tmplength}
\settowidth{\labelsep}{X}
\addtolength{\leftmargin}{\labelsep}
\setlength{\labelwidth}{\tmplength}
}
\item[\textbf{Declaration}\hfill]
\ifpdf
\begin{flushleft}
\fi
\begin{ttfamily}
public procedure Edit1Change(Sender: TObject);\end{ttfamily}

\ifpdf
\end{flushleft}
\fi

\end{list}
\paragraph*{Edit2Change}\hspace*{\fill}

\label{prjwizard.TfrmProjectWizard-Edit2Change}
\index{Edit2Change}
\begin{list}{}{
\settowidth{\tmplength}{\textbf{Description}}
\setlength{\itemindent}{0cm}
\setlength{\listparindent}{0cm}
\setlength{\leftmargin}{\evensidemargin}
\addtolength{\leftmargin}{\tmplength}
\settowidth{\labelsep}{X}
\addtolength{\leftmargin}{\labelsep}
\setlength{\labelwidth}{\tmplength}
}
\item[\textbf{Declaration}\hfill]
\ifpdf
\begin{flushleft}
\fi
\begin{ttfamily}
public procedure Edit2Change(Sender: TObject);\end{ttfamily}

\ifpdf
\end{flushleft}
\fi

\end{list}
\paragraph*{FormCreate}\hspace*{\fill}

\label{prjwizard.TfrmProjectWizard-FormCreate}
\index{FormCreate}
\begin{list}{}{
\settowidth{\tmplength}{\textbf{Description}}
\setlength{\itemindent}{0cm}
\setlength{\listparindent}{0cm}
\setlength{\leftmargin}{\evensidemargin}
\addtolength{\leftmargin}{\tmplength}
\settowidth{\labelsep}{X}
\addtolength{\leftmargin}{\labelsep}
\setlength{\labelwidth}{\tmplength}
}
\item[\textbf{Declaration}\hfill]
\ifpdf
\begin{flushleft}
\fi
\begin{ttfamily}
public procedure FormCreate(Sender: TObject);\end{ttfamily}

\ifpdf
\end{flushleft}
\fi

\end{list}
\paragraph*{FormShow}\hspace*{\fill}

\label{prjwizard.TfrmProjectWizard-FormShow}
\index{FormShow}
\begin{list}{}{
\settowidth{\tmplength}{\textbf{Description}}
\setlength{\itemindent}{0cm}
\setlength{\listparindent}{0cm}
\setlength{\leftmargin}{\evensidemargin}
\addtolength{\leftmargin}{\tmplength}
\settowidth{\labelsep}{X}
\addtolength{\leftmargin}{\labelsep}
\setlength{\labelwidth}{\tmplength}
}
\item[\textbf{Declaration}\hfill]
\ifpdf
\begin{flushleft}
\fi
\begin{ttfamily}
public procedure FormShow(Sender: TObject);\end{ttfamily}

\ifpdf
\end{flushleft}
\fi

\end{list}
\paragraph*{Label28Click}\hspace*{\fill}

\label{prjwizard.TfrmProjectWizard-Label28Click}
\index{Label28Click}
\begin{list}{}{
\settowidth{\tmplength}{\textbf{Description}}
\setlength{\itemindent}{0cm}
\setlength{\listparindent}{0cm}
\setlength{\leftmargin}{\evensidemargin}
\addtolength{\leftmargin}{\tmplength}
\settowidth{\labelsep}{X}
\addtolength{\leftmargin}{\labelsep}
\setlength{\labelwidth}{\tmplength}
}
\item[\textbf{Declaration}\hfill]
\ifpdf
\begin{flushleft}
\fi
\begin{ttfamily}
public procedure Label28Click(Sender: TObject);\end{ttfamily}

\ifpdf
\end{flushleft}
\fi

\end{list}
\paragraph*{lvPackageFilesDblClick}\hspace*{\fill}

\label{prjwizard.TfrmProjectWizard-lvPackageFilesDblClick}
\index{lvPackageFilesDblClick}
\begin{list}{}{
\settowidth{\tmplength}{\textbf{Description}}
\setlength{\itemindent}{0cm}
\setlength{\listparindent}{0cm}
\setlength{\leftmargin}{\evensidemargin}
\addtolength{\leftmargin}{\tmplength}
\settowidth{\labelsep}{X}
\addtolength{\leftmargin}{\labelsep}
\setlength{\labelwidth}{\tmplength}
}
\item[\textbf{Declaration}\hfill]
\ifpdf
\begin{flushleft}
\fi
\begin{ttfamily}
public procedure lvPackageFilesDblClick(Sender: TObject);\end{ttfamily}

\ifpdf
\end{flushleft}
\fi

\end{list}
\paragraph*{lvPackageFilesKeyDown}\hspace*{\fill}

\label{prjwizard.TfrmProjectWizard-lvPackageFilesKeyDown}
\index{lvPackageFilesKeyDown}
\begin{list}{}{
\settowidth{\tmplength}{\textbf{Description}}
\setlength{\itemindent}{0cm}
\setlength{\listparindent}{0cm}
\setlength{\leftmargin}{\evensidemargin}
\addtolength{\leftmargin}{\tmplength}
\settowidth{\labelsep}{X}
\addtolength{\leftmargin}{\labelsep}
\setlength{\labelwidth}{\tmplength}
}
\item[\textbf{Declaration}\hfill]
\ifpdf
\begin{flushleft}
\fi
\begin{ttfamily}
public procedure lvPackageFilesKeyDown(Sender: TObject; var Key: Word; Shift: TShiftState);\end{ttfamily}

\ifpdf
\end{flushleft}
\fi

\end{list}
\paragraph*{MenuItem1Click}\hspace*{\fill}

\label{prjwizard.TfrmProjectWizard-MenuItem1Click}
\index{MenuItem1Click}
\begin{list}{}{
\settowidth{\tmplength}{\textbf{Description}}
\setlength{\itemindent}{0cm}
\setlength{\listparindent}{0cm}
\setlength{\leftmargin}{\evensidemargin}
\addtolength{\leftmargin}{\tmplength}
\settowidth{\labelsep}{X}
\addtolength{\leftmargin}{\labelsep}
\setlength{\labelwidth}{\tmplength}
}
\item[\textbf{Declaration}\hfill]
\ifpdf
\begin{flushleft}
\fi
\begin{ttfamily}
public procedure MenuItem1Click(Sender: TObject);\end{ttfamily}

\ifpdf
\end{flushleft}
\fi

\end{list}
\paragraph*{RadioButton1Change}\hspace*{\fill}

\label{prjwizard.TfrmProjectWizard-RadioButton1Change}
\index{RadioButton1Change}
\begin{list}{}{
\settowidth{\tmplength}{\textbf{Description}}
\setlength{\itemindent}{0cm}
\setlength{\listparindent}{0cm}
\setlength{\leftmargin}{\evensidemargin}
\addtolength{\leftmargin}{\tmplength}
\settowidth{\labelsep}{X}
\addtolength{\leftmargin}{\labelsep}
\setlength{\labelwidth}{\tmplength}
}
\item[\textbf{Declaration}\hfill]
\ifpdf
\begin{flushleft}
\fi
\begin{ttfamily}
public procedure RadioButton1Change(Sender: TObject);\end{ttfamily}

\ifpdf
\end{flushleft}
\fi

\end{list}
\paragraph*{SpeedButton1Click}\hspace*{\fill}

\label{prjwizard.TfrmProjectWizard-SpeedButton1Click}
\index{SpeedButton1Click}
\begin{list}{}{
\settowidth{\tmplength}{\textbf{Description}}
\setlength{\itemindent}{0cm}
\setlength{\listparindent}{0cm}
\setlength{\leftmargin}{\evensidemargin}
\addtolength{\leftmargin}{\tmplength}
\settowidth{\labelsep}{X}
\addtolength{\leftmargin}{\labelsep}
\setlength{\labelwidth}{\tmplength}
}
\item[\textbf{Declaration}\hfill]
\ifpdf
\begin{flushleft}
\fi
\begin{ttfamily}
public procedure SpeedButton1Click(Sender: TObject);\end{ttfamily}

\ifpdf
\end{flushleft}
\fi

\end{list}
\paragraph*{SpeedButton2Click}\hspace*{\fill}

\label{prjwizard.TfrmProjectWizard-SpeedButton2Click}
\index{SpeedButton2Click}
\begin{list}{}{
\settowidth{\tmplength}{\textbf{Description}}
\setlength{\itemindent}{0cm}
\setlength{\listparindent}{0cm}
\setlength{\leftmargin}{\evensidemargin}
\addtolength{\leftmargin}{\tmplength}
\settowidth{\labelsep}{X}
\addtolength{\leftmargin}{\labelsep}
\setlength{\labelwidth}{\tmplength}
}
\item[\textbf{Declaration}\hfill]
\ifpdf
\begin{flushleft}
\fi
\begin{ttfamily}
public procedure SpeedButton2Click(Sender: TObject);\end{ttfamily}

\ifpdf
\end{flushleft}
\fi

\end{list}
\paragraph*{SpeedButton3Click}\hspace*{\fill}

\label{prjwizard.TfrmProjectWizard-SpeedButton3Click}
\index{SpeedButton3Click}
\begin{list}{}{
\settowidth{\tmplength}{\textbf{Description}}
\setlength{\itemindent}{0cm}
\setlength{\listparindent}{0cm}
\setlength{\leftmargin}{\evensidemargin}
\addtolength{\leftmargin}{\tmplength}
\settowidth{\labelsep}{X}
\addtolength{\leftmargin}{\labelsep}
\setlength{\labelwidth}{\tmplength}
}
\item[\textbf{Declaration}\hfill]
\ifpdf
\begin{flushleft}
\fi
\begin{ttfamily}
public procedure SpeedButton3Click(Sender: TObject);\end{ttfamily}

\ifpdf
\end{flushleft}
\fi

\end{list}
\paragraph*{SpeedButton4Click}\hspace*{\fill}

\label{prjwizard.TfrmProjectWizard-SpeedButton4Click}
\index{SpeedButton4Click}
\begin{list}{}{
\settowidth{\tmplength}{\textbf{Description}}
\setlength{\itemindent}{0cm}
\setlength{\listparindent}{0cm}
\setlength{\leftmargin}{\evensidemargin}
\addtolength{\leftmargin}{\tmplength}
\settowidth{\labelsep}{X}
\addtolength{\leftmargin}{\labelsep}
\setlength{\labelwidth}{\tmplength}
}
\item[\textbf{Declaration}\hfill]
\ifpdf
\begin{flushleft}
\fi
\begin{ttfamily}
public procedure SpeedButton4Click(Sender: TObject);\end{ttfamily}

\ifpdf
\end{flushleft}
\fi

\end{list}
\paragraph*{tvDependenciesClick}\hspace*{\fill}

\label{prjwizard.TfrmProjectWizard-tvDependenciesClick}
\index{tvDependenciesClick}
\begin{list}{}{
\settowidth{\tmplength}{\textbf{Description}}
\setlength{\itemindent}{0cm}
\setlength{\listparindent}{0cm}
\setlength{\leftmargin}{\evensidemargin}
\addtolength{\leftmargin}{\tmplength}
\settowidth{\labelsep}{X}
\addtolength{\leftmargin}{\labelsep}
\setlength{\labelwidth}{\tmplength}
}
\item[\textbf{Declaration}\hfill]
\ifpdf
\begin{flushleft}
\fi
\begin{ttfamily}
public procedure tvDependenciesClick(Sender: TObject);\end{ttfamily}

\ifpdf
\end{flushleft}
\fi

\end{list}
\paragraph*{tvDependenciesKeyDown}\hspace*{\fill}

\label{prjwizard.TfrmProjectWizard-tvDependenciesKeyDown}
\index{tvDependenciesKeyDown}
\begin{list}{}{
\settowidth{\tmplength}{\textbf{Description}}
\setlength{\itemindent}{0cm}
\setlength{\listparindent}{0cm}
\setlength{\leftmargin}{\evensidemargin}
\addtolength{\leftmargin}{\tmplength}
\settowidth{\labelsep}{X}
\addtolength{\leftmargin}{\labelsep}
\setlength{\labelwidth}{\tmplength}
}
\item[\textbf{Declaration}\hfill]
\ifpdf
\begin{flushleft}
\fi
\begin{ttfamily}
public procedure tvDependenciesKeyDown(Sender: TObject; var Key: Word; Shift: TShiftState);\end{ttfamily}

\ifpdf
\end{flushleft}
\fi

\end{list}
\paragraph*{tvShortDescriptionsClick}\hspace*{\fill}

\label{prjwizard.TfrmProjectWizard-tvShortDescriptionsClick}
\index{tvShortDescriptionsClick}
\begin{list}{}{
\settowidth{\tmplength}{\textbf{Description}}
\setlength{\itemindent}{0cm}
\setlength{\listparindent}{0cm}
\setlength{\leftmargin}{\evensidemargin}
\addtolength{\leftmargin}{\tmplength}
\settowidth{\labelsep}{X}
\addtolength{\leftmargin}{\labelsep}
\setlength{\labelwidth}{\tmplength}
}
\item[\textbf{Declaration}\hfill]
\ifpdf
\begin{flushleft}
\fi
\begin{ttfamily}
public procedure tvShortDescriptionsClick(Sender: TObject);\end{ttfamily}

\ifpdf
\end{flushleft}
\fi

\end{list}
\paragraph*{tvShortDescriptionsKeyDown}\hspace*{\fill}

\label{prjwizard.TfrmProjectWizard-tvShortDescriptionsKeyDown}
\index{tvShortDescriptionsKeyDown}
\begin{list}{}{
\settowidth{\tmplength}{\textbf{Description}}
\setlength{\itemindent}{0cm}
\setlength{\listparindent}{0cm}
\setlength{\leftmargin}{\evensidemargin}
\addtolength{\leftmargin}{\tmplength}
\settowidth{\labelsep}{X}
\addtolength{\leftmargin}{\labelsep}
\setlength{\labelwidth}{\tmplength}
}
\item[\textbf{Declaration}\hfill]
\ifpdf
\begin{flushleft}
\fi
\begin{ttfamily}
public procedure tvShortDescriptionsKeyDown(Sender: TObject; var Key: Word; Shift: TShiftState);\end{ttfamily}

\ifpdf
\end{flushleft}
\fi

\end{list}
\section{Types}
\ifpdf
\subsection*{\large{\textbf{TListallerPackageType}}\normalsize\hspace{1ex}\hrulefill}
\else
\subsection*{TListallerPackageType}
\fi
\label{prjwizard-TListallerPackageType}
\index{TListallerPackageType}
\begin{list}{}{
\settowidth{\tmplength}{\textbf{Description}}
\setlength{\itemindent}{0cm}
\setlength{\listparindent}{0cm}
\setlength{\leftmargin}{\evensidemargin}
\addtolength{\leftmargin}{\tmplength}
\settowidth{\labelsep}{X}
\addtolength{\leftmargin}{\labelsep}
\setlength{\labelwidth}{\tmplength}
}
\item[\textbf{Declaration}\hfill]
\ifpdf
\begin{flushleft}
\fi
\begin{ttfamily}
TListallerPackageType = (...);\end{ttfamily}

\ifpdf
\end{flushleft}
\fi

\par
\item[\textbf{Description}]
 \item[\textbf{Values}]
\begin{description}
\item[\texttt{lptLinstall}]  
\item[\texttt{lptDLink}]  
\item[\texttt{lptContainer}]  
\end{description}


\end{list}
\section{Variables}
\ifpdf
\subsection*{\large{\textbf{frmProjectWizard}}\normalsize\hspace{1ex}\hrulefill}
\else
\subsection*{frmProjectWizard}
\fi
\label{prjwizard-frmProjectWizard}
\index{frmProjectWizard}
\begin{list}{}{
\settowidth{\tmplength}{\textbf{Description}}
\setlength{\itemindent}{0cm}
\setlength{\listparindent}{0cm}
\setlength{\leftmargin}{\evensidemargin}
\addtolength{\leftmargin}{\tmplength}
\settowidth{\labelsep}{X}
\addtolength{\leftmargin}{\labelsep}
\setlength{\labelwidth}{\tmplength}
}
\item[\textbf{Declaration}\hfill]
\ifpdf
\begin{flushleft}
\fi
\begin{ttfamily}
frmProjectWizard: TfrmProjectWizard;\end{ttfamily}

\ifpdf
\end{flushleft}
\fi

\end{list}
\ifpdf
\subsection*{\large{\textbf{CreaType}}\normalsize\hspace{1ex}\hrulefill}
\else
\subsection*{CreaType}
\fi
\label{prjwizard-CreaType}
\index{CreaType}
\begin{list}{}{
\settowidth{\tmplength}{\textbf{Description}}
\setlength{\itemindent}{0cm}
\setlength{\listparindent}{0cm}
\setlength{\leftmargin}{\evensidemargin}
\addtolength{\leftmargin}{\tmplength}
\settowidth{\labelsep}{X}
\addtolength{\leftmargin}{\labelsep}
\setlength{\labelwidth}{\tmplength}
}
\item[\textbf{Declaration}\hfill]
\ifpdf
\begin{flushleft}
\fi
\begin{ttfamily}
CreaType: TListallerPackageType;\end{ttfamily}

\ifpdf
\end{flushleft}
\fi

\par
\item[\textbf{Description}]
IPK/IPS type that should be created

\end{list}
\chapter{Unit RegExpr}
\label{RegExpr}
\index{RegExpr}
\section{uses}
\begin{itemize}
\item \begin{ttfamily}Classes\end{ttfamily}\item \begin{ttfamily}SysUtils\end{ttfamily}\end{itemize}
\section{Overview}
\begin{description}
\item[\texttt{\begin{ttfamily}TRegExpr\end{ttfamily} Class}]
\item[\texttt{\begin{ttfamily}ERegExpr\end{ttfamily} Class}]
\end{description}
\begin{description}
\item[\texttt{ExecRegExpr}]
\item[\texttt{SplitRegExpr}]
\item[\texttt{ReplaceRegExpr}]
\item[\texttt{QuoteRegExprMetaChars}]
\item[\texttt{RegExprSubExpressions}]
\end{description}
\section{Classes, Interfaces, Objects and Records}
\ifpdf
\subsection*{\large{\textbf{TRegExpr Class}}\normalsize\hspace{1ex}\hrulefill}
\else
\subsection*{TRegExpr Class}
\fi
\label{RegExpr.TRegExpr}
\index{TRegExpr}
\subsubsection*{\large{\textbf{Hierarchy}}\normalsize\hspace{1ex}\hfill}
TRegExpr {$>$} TObject
%%%%Description
\subsubsection*{\large{\textbf{Properties}}\normalsize\hspace{1ex}\hfill}
\begin{list}{}{
\settowidth{\tmplength}{\textbf{LinePairedSeparator}}
\setlength{\itemindent}{0cm}
\setlength{\listparindent}{0cm}
\setlength{\leftmargin}{\evensidemargin}
\addtolength{\leftmargin}{\tmplength}
\settowidth{\labelsep}{X}
\addtolength{\leftmargin}{\labelsep}
\setlength{\labelwidth}{\tmplength}
}
\label{RegExpr.TRegExpr-Expression}
\index{Expression}
\item[\textbf{Expression}\hfill]
\ifpdf
\begin{flushleft}
\fi
\begin{ttfamily}
public property Expression : RegExprString read GetExpression write SetExpression;\end{ttfamily}

\ifpdf
\end{flushleft}
\fi


\par  \label{RegExpr.TRegExpr-ModifierStr}
\index{ModifierStr}
\item[\textbf{ModifierStr}\hfill]
\ifpdf
\begin{flushleft}
\fi
\begin{ttfamily}
public property ModifierStr : RegExprString read GetModifierStr write SetModifierStr;\end{ttfamily}

\ifpdf
\end{flushleft}
\fi


\par  \label{RegExpr.TRegExpr-ModifierI}
\index{ModifierI}
\item[\textbf{ModifierI}\hfill]
\ifpdf
\begin{flushleft}
\fi
\begin{ttfamily}
public property ModifierI : boolean index 1 read GetModifier write SetModifier;\end{ttfamily}

\ifpdf
\end{flushleft}
\fi


\par  \label{RegExpr.TRegExpr-ModifierR}
\index{ModifierR}
\item[\textbf{ModifierR}\hfill]
\ifpdf
\begin{flushleft}
\fi
\begin{ttfamily}
public property ModifierR : boolean index 2 read GetModifier write SetModifier;\end{ttfamily}

\ifpdf
\end{flushleft}
\fi


\par  \label{RegExpr.TRegExpr-ModifierS}
\index{ModifierS}
\item[\textbf{ModifierS}\hfill]
\ifpdf
\begin{flushleft}
\fi
\begin{ttfamily}
public property ModifierS : boolean index 3 read GetModifier write SetModifier;\end{ttfamily}

\ifpdf
\end{flushleft}
\fi


\par  \label{RegExpr.TRegExpr-ModifierG}
\index{ModifierG}
\item[\textbf{ModifierG}\hfill]
\ifpdf
\begin{flushleft}
\fi
\begin{ttfamily}
public property ModifierG : boolean index 4 read GetModifier write SetModifier;\end{ttfamily}

\ifpdf
\end{flushleft}
\fi


\par  \label{RegExpr.TRegExpr-ModifierM}
\index{ModifierM}
\item[\textbf{ModifierM}\hfill]
\ifpdf
\begin{flushleft}
\fi
\begin{ttfamily}
public property ModifierM : boolean index 5 read GetModifier write SetModifier;\end{ttfamily}

\ifpdf
\end{flushleft}
\fi


\par  \label{RegExpr.TRegExpr-ModifierX}
\index{ModifierX}
\item[\textbf{ModifierX}\hfill]
\ifpdf
\begin{flushleft}
\fi
\begin{ttfamily}
public property ModifierX : boolean index 6 read GetModifier write SetModifier;\end{ttfamily}

\ifpdf
\end{flushleft}
\fi


\par  \label{RegExpr.TRegExpr-InputString}
\index{InputString}
\item[\textbf{InputString}\hfill]
\ifpdf
\begin{flushleft}
\fi
\begin{ttfamily}
public property InputString : RegExprString read GetInputString write SetInputString;\end{ttfamily}

\ifpdf
\end{flushleft}
\fi


\par  \label{RegExpr.TRegExpr-SubExprMatchCount}
\index{SubExprMatchCount}
\item[\textbf{SubExprMatchCount}\hfill]
\ifpdf
\begin{flushleft}
\fi
\begin{ttfamily}
public property SubExprMatchCount : integer read GetSubExprMatchCount;\end{ttfamily}

\ifpdf
\end{flushleft}
\fi


\par  \label{RegExpr.TRegExpr-MatchPos}
\index{MatchPos}
\item[\textbf{MatchPos}\hfill]
\ifpdf
\begin{flushleft}
\fi
\begin{ttfamily}
public property MatchPos [Idx : integer]: integer read GetMatchPos;\end{ttfamily}

\ifpdf
\end{flushleft}
\fi


\par  \label{RegExpr.TRegExpr-MatchLen}
\index{MatchLen}
\item[\textbf{MatchLen}\hfill]
\ifpdf
\begin{flushleft}
\fi
\begin{ttfamily}
public property MatchLen [Idx : integer]: integer read GetMatchLen;\end{ttfamily}

\ifpdf
\end{flushleft}
\fi


\par  \label{RegExpr.TRegExpr-Match}
\index{Match}
\item[\textbf{Match}\hfill]
\ifpdf
\begin{flushleft}
\fi
\begin{ttfamily}
public property Match [Idx : integer]: RegExprString read GetMatch;\end{ttfamily}

\ifpdf
\end{flushleft}
\fi


\par  \label{RegExpr.TRegExpr-CompilerErrorPos}
\index{CompilerErrorPos}
\item[\textbf{CompilerErrorPos}\hfill]
\ifpdf
\begin{flushleft}
\fi
\begin{ttfamily}
public property CompilerErrorPos : integer read GetCompilerErrorPos;\end{ttfamily}

\ifpdf
\end{flushleft}
\fi


\par  \label{RegExpr.TRegExpr-SpaceChars}
\index{SpaceChars}
\item[\textbf{SpaceChars}\hfill]
\ifpdf
\begin{flushleft}
\fi
\begin{ttfamily}
public property SpaceChars : RegExprString read fSpaceChars write fSpaceChars;\end{ttfamily}

\ifpdf
\end{flushleft}
\fi


\par  \label{RegExpr.TRegExpr-WordChars}
\index{WordChars}
\item[\textbf{WordChars}\hfill]
\ifpdf
\begin{flushleft}
\fi
\begin{ttfamily}
public property WordChars : RegExprString read fWordChars write fWordChars;\end{ttfamily}

\ifpdf
\end{flushleft}
\fi


\par  \label{RegExpr.TRegExpr-LineSeparators}
\index{LineSeparators}
\item[\textbf{LineSeparators}\hfill]
\ifpdf
\begin{flushleft}
\fi
\begin{ttfamily}
public property LineSeparators : RegExprString read fLineSeparators write SetLineSeparators;\end{ttfamily}

\ifpdf
\end{flushleft}
\fi


\par  \label{RegExpr.TRegExpr-LinePairedSeparator}
\index{LinePairedSeparator}
\item[\textbf{LinePairedSeparator}\hfill]
\ifpdf
\begin{flushleft}
\fi
\begin{ttfamily}
public property LinePairedSeparator : RegExprString read GetLinePairedSeparator write SetLinePairedSeparator;\end{ttfamily}

\ifpdf
\end{flushleft}
\fi


\par  \label{RegExpr.TRegExpr-InvertCase}
\index{InvertCase}
\item[\textbf{InvertCase}\hfill]
\ifpdf
\begin{flushleft}
\fi
\begin{ttfamily}
public property InvertCase : TRegExprInvertCaseFunction read fInvertCase write fInvertCase;\end{ttfamily}

\ifpdf
\end{flushleft}
\fi


\par  \end{list}
\subsubsection*{\large{\textbf{Methods}}\normalsize\hspace{1ex}\hfill}
\paragraph*{Create}\hspace*{\fill}

\label{RegExpr.TRegExpr-Create}
\index{Create}
\begin{list}{}{
\settowidth{\tmplength}{\textbf{Description}}
\setlength{\itemindent}{0cm}
\setlength{\listparindent}{0cm}
\setlength{\leftmargin}{\evensidemargin}
\addtolength{\leftmargin}{\tmplength}
\settowidth{\labelsep}{X}
\addtolength{\leftmargin}{\labelsep}
\setlength{\labelwidth}{\tmplength}
}
\item[\textbf{Declaration}\hfill]
\ifpdf
\begin{flushleft}
\fi
\begin{ttfamily}
public constructor Create;\end{ttfamily}

\ifpdf
\end{flushleft}
\fi

\end{list}
\paragraph*{Destroy}\hspace*{\fill}

\label{RegExpr.TRegExpr-Destroy}
\index{Destroy}
\begin{list}{}{
\settowidth{\tmplength}{\textbf{Description}}
\setlength{\itemindent}{0cm}
\setlength{\listparindent}{0cm}
\setlength{\leftmargin}{\evensidemargin}
\addtolength{\leftmargin}{\tmplength}
\settowidth{\labelsep}{X}
\addtolength{\leftmargin}{\labelsep}
\setlength{\labelwidth}{\tmplength}
}
\item[\textbf{Declaration}\hfill]
\ifpdf
\begin{flushleft}
\fi
\begin{ttfamily}
public destructor Destroy; override;\end{ttfamily}

\ifpdf
\end{flushleft}
\fi

\end{list}
\paragraph*{VersionMajor}\hspace*{\fill}

\label{RegExpr.TRegExpr-VersionMajor}
\index{VersionMajor}
\begin{list}{}{
\settowidth{\tmplength}{\textbf{Description}}
\setlength{\itemindent}{0cm}
\setlength{\listparindent}{0cm}
\setlength{\leftmargin}{\evensidemargin}
\addtolength{\leftmargin}{\tmplength}
\settowidth{\labelsep}{X}
\addtolength{\leftmargin}{\labelsep}
\setlength{\labelwidth}{\tmplength}
}
\item[\textbf{Declaration}\hfill]
\ifpdf
\begin{flushleft}
\fi
\begin{ttfamily}
public class function VersionMajor : integer;\end{ttfamily}

\ifpdf
\end{flushleft}
\fi

\end{list}
\paragraph*{VersionMinor}\hspace*{\fill}

\label{RegExpr.TRegExpr-VersionMinor}
\index{VersionMinor}
\begin{list}{}{
\settowidth{\tmplength}{\textbf{Description}}
\setlength{\itemindent}{0cm}
\setlength{\listparindent}{0cm}
\setlength{\leftmargin}{\evensidemargin}
\addtolength{\leftmargin}{\tmplength}
\settowidth{\labelsep}{X}
\addtolength{\leftmargin}{\labelsep}
\setlength{\labelwidth}{\tmplength}
}
\item[\textbf{Declaration}\hfill]
\ifpdf
\begin{flushleft}
\fi
\begin{ttfamily}
public class function VersionMinor : integer;\end{ttfamily}

\ifpdf
\end{flushleft}
\fi

\end{list}
\paragraph*{Exec}\hspace*{\fill}

\label{RegExpr.TRegExpr-Exec}
\index{Exec}
\begin{list}{}{
\settowidth{\tmplength}{\textbf{Description}}
\setlength{\itemindent}{0cm}
\setlength{\listparindent}{0cm}
\setlength{\leftmargin}{\evensidemargin}
\addtolength{\leftmargin}{\tmplength}
\settowidth{\labelsep}{X}
\addtolength{\leftmargin}{\labelsep}
\setlength{\labelwidth}{\tmplength}
}
\item[\textbf{Declaration}\hfill]
\ifpdf
\begin{flushleft}
\fi
\begin{ttfamily}
public function Exec (const AInputString : RegExprString) : boolean;\end{ttfamily}

\ifpdf
\end{flushleft}
\fi

\end{list}
\paragraph*{ExecNext}\hspace*{\fill}

\label{RegExpr.TRegExpr-ExecNext}
\index{ExecNext}
\begin{list}{}{
\settowidth{\tmplength}{\textbf{Description}}
\setlength{\itemindent}{0cm}
\setlength{\listparindent}{0cm}
\setlength{\leftmargin}{\evensidemargin}
\addtolength{\leftmargin}{\tmplength}
\settowidth{\labelsep}{X}
\addtolength{\leftmargin}{\labelsep}
\setlength{\labelwidth}{\tmplength}
}
\item[\textbf{Declaration}\hfill]
\ifpdf
\begin{flushleft}
\fi
\begin{ttfamily}
public function ExecNext : boolean;\end{ttfamily}

\ifpdf
\end{flushleft}
\fi

\end{list}
\paragraph*{ExecPos}\hspace*{\fill}

\label{RegExpr.TRegExpr-ExecPos}
\index{ExecPos}
\begin{list}{}{
\settowidth{\tmplength}{\textbf{Description}}
\setlength{\itemindent}{0cm}
\setlength{\listparindent}{0cm}
\setlength{\leftmargin}{\evensidemargin}
\addtolength{\leftmargin}{\tmplength}
\settowidth{\labelsep}{X}
\addtolength{\leftmargin}{\labelsep}
\setlength{\labelwidth}{\tmplength}
}
\item[\textbf{Declaration}\hfill]
\ifpdf
\begin{flushleft}
\fi
\begin{ttfamily}
public function ExecPos (AOffset: integer ) : boolean;\end{ttfamily}

\ifpdf
\end{flushleft}
\fi

\end{list}
\paragraph*{Substitute}\hspace*{\fill}

\label{RegExpr.TRegExpr-Substitute}
\index{Substitute}
\begin{list}{}{
\settowidth{\tmplength}{\textbf{Description}}
\setlength{\itemindent}{0cm}
\setlength{\listparindent}{0cm}
\setlength{\leftmargin}{\evensidemargin}
\addtolength{\leftmargin}{\tmplength}
\settowidth{\labelsep}{X}
\addtolength{\leftmargin}{\labelsep}
\setlength{\labelwidth}{\tmplength}
}
\item[\textbf{Declaration}\hfill]
\ifpdf
\begin{flushleft}
\fi
\begin{ttfamily}
public function Substitute (const ATemplate : RegExprString) : RegExprString;\end{ttfamily}

\ifpdf
\end{flushleft}
\fi

\end{list}
\paragraph*{Split}\hspace*{\fill}

\label{RegExpr.TRegExpr-Split}
\index{Split}
\begin{list}{}{
\settowidth{\tmplength}{\textbf{Description}}
\setlength{\itemindent}{0cm}
\setlength{\listparindent}{0cm}
\setlength{\leftmargin}{\evensidemargin}
\addtolength{\leftmargin}{\tmplength}
\settowidth{\labelsep}{X}
\addtolength{\leftmargin}{\labelsep}
\setlength{\labelwidth}{\tmplength}
}
\item[\textbf{Declaration}\hfill]
\ifpdf
\begin{flushleft}
\fi
\begin{ttfamily}
public procedure Split (AInputStr : RegExprString; APieces : TStrings);\end{ttfamily}

\ifpdf
\end{flushleft}
\fi

\end{list}
\paragraph*{Replace}\hspace*{\fill}

\label{RegExpr.TRegExpr-Replace}
\index{Replace}
\begin{list}{}{
\settowidth{\tmplength}{\textbf{Description}}
\setlength{\itemindent}{0cm}
\setlength{\listparindent}{0cm}
\setlength{\leftmargin}{\evensidemargin}
\addtolength{\leftmargin}{\tmplength}
\settowidth{\labelsep}{X}
\addtolength{\leftmargin}{\labelsep}
\setlength{\labelwidth}{\tmplength}
}
\item[\textbf{Declaration}\hfill]
\ifpdf
\begin{flushleft}
\fi
\begin{ttfamily}
public function Replace (AInputStr : RegExprString; const AReplaceStr : RegExprString; AUseSubstitution : boolean) : RegExprString;\end{ttfamily}

\ifpdf
\end{flushleft}
\fi

\end{list}
\paragraph*{ReplaceEx}\hspace*{\fill}

\label{RegExpr.TRegExpr-ReplaceEx}
\index{ReplaceEx}
\begin{list}{}{
\settowidth{\tmplength}{\textbf{Description}}
\setlength{\itemindent}{0cm}
\setlength{\listparindent}{0cm}
\setlength{\leftmargin}{\evensidemargin}
\addtolength{\leftmargin}{\tmplength}
\settowidth{\labelsep}{X}
\addtolength{\leftmargin}{\labelsep}
\setlength{\labelwidth}{\tmplength}
}
\item[\textbf{Declaration}\hfill]
\ifpdf
\begin{flushleft}
\fi
\begin{ttfamily}
public function ReplaceEx (AInputStr : RegExprString; AReplaceFunc : TRegExprReplaceFunction) : RegExprString;\end{ttfamily}

\ifpdf
\end{flushleft}
\fi

\end{list}
\paragraph*{LastError}\hspace*{\fill}

\label{RegExpr.TRegExpr-LastError}
\index{LastError}
\begin{list}{}{
\settowidth{\tmplength}{\textbf{Description}}
\setlength{\itemindent}{0cm}
\setlength{\listparindent}{0cm}
\setlength{\leftmargin}{\evensidemargin}
\addtolength{\leftmargin}{\tmplength}
\settowidth{\labelsep}{X}
\addtolength{\leftmargin}{\labelsep}
\setlength{\labelwidth}{\tmplength}
}
\item[\textbf{Declaration}\hfill]
\ifpdf
\begin{flushleft}
\fi
\begin{ttfamily}
public function LastError : integer;\end{ttfamily}

\ifpdf
\end{flushleft}
\fi

\end{list}
\paragraph*{ErrorMsg}\hspace*{\fill}

\label{RegExpr.TRegExpr-ErrorMsg}
\index{ErrorMsg}
\begin{list}{}{
\settowidth{\tmplength}{\textbf{Description}}
\setlength{\itemindent}{0cm}
\setlength{\listparindent}{0cm}
\setlength{\leftmargin}{\evensidemargin}
\addtolength{\leftmargin}{\tmplength}
\settowidth{\labelsep}{X}
\addtolength{\leftmargin}{\labelsep}
\setlength{\labelwidth}{\tmplength}
}
\item[\textbf{Declaration}\hfill]
\ifpdf
\begin{flushleft}
\fi
\begin{ttfamily}
public function ErrorMsg (AErrorID : integer) : RegExprString; virtual;\end{ttfamily}

\ifpdf
\end{flushleft}
\fi

\end{list}
\paragraph*{InvertCaseFunction}\hspace*{\fill}

\label{RegExpr.TRegExpr-InvertCaseFunction}
\index{InvertCaseFunction}
\begin{list}{}{
\settowidth{\tmplength}{\textbf{Description}}
\setlength{\itemindent}{0cm}
\setlength{\listparindent}{0cm}
\setlength{\leftmargin}{\evensidemargin}
\addtolength{\leftmargin}{\tmplength}
\settowidth{\labelsep}{X}
\addtolength{\leftmargin}{\labelsep}
\setlength{\labelwidth}{\tmplength}
}
\item[\textbf{Declaration}\hfill]
\ifpdf
\begin{flushleft}
\fi
\begin{ttfamily}
public class function InvertCaseFunction (const Ch : REChar) : REChar;\end{ttfamily}

\ifpdf
\end{flushleft}
\fi

\end{list}
\paragraph*{Compile}\hspace*{\fill}

\label{RegExpr.TRegExpr-Compile}
\index{Compile}
\begin{list}{}{
\settowidth{\tmplength}{\textbf{Description}}
\setlength{\itemindent}{0cm}
\setlength{\listparindent}{0cm}
\setlength{\leftmargin}{\evensidemargin}
\addtolength{\leftmargin}{\tmplength}
\settowidth{\labelsep}{X}
\addtolength{\leftmargin}{\labelsep}
\setlength{\labelwidth}{\tmplength}
}
\item[\textbf{Declaration}\hfill]
\ifpdf
\begin{flushleft}
\fi
\begin{ttfamily}
public procedure Compile;\end{ttfamily}

\ifpdf
\end{flushleft}
\fi

\end{list}
\paragraph*{Dump}\hspace*{\fill}

\label{RegExpr.TRegExpr-Dump}
\index{Dump}
\begin{list}{}{
\settowidth{\tmplength}{\textbf{Description}}
\setlength{\itemindent}{0cm}
\setlength{\listparindent}{0cm}
\setlength{\leftmargin}{\evensidemargin}
\addtolength{\leftmargin}{\tmplength}
\settowidth{\labelsep}{X}
\addtolength{\leftmargin}{\labelsep}
\setlength{\labelwidth}{\tmplength}
}
\item[\textbf{Declaration}\hfill]
\ifpdf
\begin{flushleft}
\fi
\begin{ttfamily}
public function Dump : RegExprString;\end{ttfamily}

\ifpdf
\end{flushleft}
\fi

\end{list}
\ifpdf
\subsection*{\large{\textbf{ERegExpr Class}}\normalsize\hspace{1ex}\hrulefill}
\else
\subsection*{ERegExpr Class}
\fi
\label{RegExpr.ERegExpr}
\index{ERegExpr}
\subsubsection*{\large{\textbf{Hierarchy}}\normalsize\hspace{1ex}\hfill}
ERegExpr {$>$} Exception
%%%%Description
\subsubsection*{\large{\textbf{Fields}}\normalsize\hspace{1ex}\hfill}
\begin{list}{}{
\settowidth{\tmplength}{\textbf{CompilerErrorPos}}
\setlength{\itemindent}{0cm}
\setlength{\listparindent}{0cm}
\setlength{\leftmargin}{\evensidemargin}
\addtolength{\leftmargin}{\tmplength}
\settowidth{\labelsep}{X}
\addtolength{\leftmargin}{\labelsep}
\setlength{\labelwidth}{\tmplength}
}
\label{RegExpr.ERegExpr-ErrorCode}
\index{ErrorCode}
\item[\textbf{ErrorCode}\hfill]
\ifpdf
\begin{flushleft}
\fi
\begin{ttfamily}
public ErrorCode: integer;\end{ttfamily}

\ifpdf
\end{flushleft}
\fi


\par  \label{RegExpr.ERegExpr-CompilerErrorPos}
\index{CompilerErrorPos}
\item[\textbf{CompilerErrorPos}\hfill]
\ifpdf
\begin{flushleft}
\fi
\begin{ttfamily}
public CompilerErrorPos: integer;\end{ttfamily}

\ifpdf
\end{flushleft}
\fi


\par  \end{list}
\section{Functions and Procedures}
\ifpdf
\subsection*{\large{\textbf{ExecRegExpr}}\normalsize\hspace{1ex}\hrulefill}
\else
\subsection*{ExecRegExpr}
\fi
\label{RegExpr-ExecRegExpr}
\index{ExecRegExpr}
\begin{list}{}{
\settowidth{\tmplength}{\textbf{Description}}
\setlength{\itemindent}{0cm}
\setlength{\listparindent}{0cm}
\setlength{\leftmargin}{\evensidemargin}
\addtolength{\leftmargin}{\tmplength}
\settowidth{\labelsep}{X}
\addtolength{\leftmargin}{\labelsep}
\setlength{\labelwidth}{\tmplength}
}
\item[\textbf{Declaration}\hfill]
\ifpdf
\begin{flushleft}
\fi
\begin{ttfamily}
function ExecRegExpr (const ARegExpr, AInputStr : RegExprString) : boolean;\end{ttfamily}

\ifpdf
\end{flushleft}
\fi

\end{list}
\ifpdf
\subsection*{\large{\textbf{SplitRegExpr}}\normalsize\hspace{1ex}\hrulefill}
\else
\subsection*{SplitRegExpr}
\fi
\label{RegExpr-SplitRegExpr}
\index{SplitRegExpr}
\begin{list}{}{
\settowidth{\tmplength}{\textbf{Description}}
\setlength{\itemindent}{0cm}
\setlength{\listparindent}{0cm}
\setlength{\leftmargin}{\evensidemargin}
\addtolength{\leftmargin}{\tmplength}
\settowidth{\labelsep}{X}
\addtolength{\leftmargin}{\labelsep}
\setlength{\labelwidth}{\tmplength}
}
\item[\textbf{Declaration}\hfill]
\ifpdf
\begin{flushleft}
\fi
\begin{ttfamily}
procedure SplitRegExpr (const ARegExpr, AInputStr : RegExprString; APieces : TStrings);\end{ttfamily}

\ifpdf
\end{flushleft}
\fi

\end{list}
\ifpdf
\subsection*{\large{\textbf{ReplaceRegExpr}}\normalsize\hspace{1ex}\hrulefill}
\else
\subsection*{ReplaceRegExpr}
\fi
\label{RegExpr-ReplaceRegExpr}
\index{ReplaceRegExpr}
\begin{list}{}{
\settowidth{\tmplength}{\textbf{Description}}
\setlength{\itemindent}{0cm}
\setlength{\listparindent}{0cm}
\setlength{\leftmargin}{\evensidemargin}
\addtolength{\leftmargin}{\tmplength}
\settowidth{\labelsep}{X}
\addtolength{\leftmargin}{\labelsep}
\setlength{\labelwidth}{\tmplength}
}
\item[\textbf{Declaration}\hfill]
\ifpdf
\begin{flushleft}
\fi
\begin{ttfamily}
function ReplaceRegExpr (const ARegExpr, AInputStr, AReplaceStr : RegExprString; AUseSubstitution : boolean) : RegExprString;\end{ttfamily}

\ifpdf
\end{flushleft}
\fi

\end{list}
\ifpdf
\subsection*{\large{\textbf{QuoteRegExprMetaChars}}\normalsize\hspace{1ex}\hrulefill}
\else
\subsection*{QuoteRegExprMetaChars}
\fi
\label{RegExpr-QuoteRegExprMetaChars}
\index{QuoteRegExprMetaChars}
\begin{list}{}{
\settowidth{\tmplength}{\textbf{Description}}
\setlength{\itemindent}{0cm}
\setlength{\listparindent}{0cm}
\setlength{\leftmargin}{\evensidemargin}
\addtolength{\leftmargin}{\tmplength}
\settowidth{\labelsep}{X}
\addtolength{\leftmargin}{\labelsep}
\setlength{\labelwidth}{\tmplength}
}
\item[\textbf{Declaration}\hfill]
\ifpdf
\begin{flushleft}
\fi
\begin{ttfamily}
function QuoteRegExprMetaChars (const AStr : RegExprString) : RegExprString;\end{ttfamily}

\ifpdf
\end{flushleft}
\fi

\end{list}
\ifpdf
\subsection*{\large{\textbf{RegExprSubExpressions}}\normalsize\hspace{1ex}\hrulefill}
\else
\subsection*{RegExprSubExpressions}
\fi
\label{RegExpr-RegExprSubExpressions}
\index{RegExprSubExpressions}
\begin{list}{}{
\settowidth{\tmplength}{\textbf{Description}}
\setlength{\itemindent}{0cm}
\setlength{\listparindent}{0cm}
\setlength{\leftmargin}{\evensidemargin}
\addtolength{\leftmargin}{\tmplength}
\settowidth{\labelsep}{X}
\addtolength{\leftmargin}{\labelsep}
\setlength{\labelwidth}{\tmplength}
}
\item[\textbf{Declaration}\hfill]
\ifpdf
\begin{flushleft}
\fi
\begin{ttfamily}
function RegExprSubExpressions (const ARegExpr : string; ASubExprs : TStrings; AExtendedSyntax : boolean) : integer;\end{ttfamily}

\ifpdf
\end{flushleft}
\fi

\end{list}
\section{Types}
\ifpdf
\subsection*{\large{\textbf{PRegExprChar}}\normalsize\hspace{1ex}\hrulefill}
\else
\subsection*{PRegExprChar}
\fi
\label{RegExpr-PRegExprChar}
\index{PRegExprChar}
\begin{list}{}{
\settowidth{\tmplength}{\textbf{Description}}
\setlength{\itemindent}{0cm}
\setlength{\listparindent}{0cm}
\setlength{\leftmargin}{\evensidemargin}
\addtolength{\leftmargin}{\tmplength}
\settowidth{\labelsep}{X}
\addtolength{\leftmargin}{\labelsep}
\setlength{\labelwidth}{\tmplength}
}
\item[\textbf{Declaration}\hfill]
\ifpdf
\begin{flushleft}
\fi
\begin{ttfamily}
PRegExprChar = PChar;\end{ttfamily}

\ifpdf
\end{flushleft}
\fi

\end{list}
\ifpdf
\subsection*{\large{\textbf{RegExprString}}\normalsize\hspace{1ex}\hrulefill}
\else
\subsection*{RegExprString}
\fi
\label{RegExpr-RegExprString}
\index{RegExprString}
\begin{list}{}{
\settowidth{\tmplength}{\textbf{Description}}
\setlength{\itemindent}{0cm}
\setlength{\listparindent}{0cm}
\setlength{\leftmargin}{\evensidemargin}
\addtolength{\leftmargin}{\tmplength}
\settowidth{\labelsep}{X}
\addtolength{\leftmargin}{\labelsep}
\setlength{\labelwidth}{\tmplength}
}
\item[\textbf{Declaration}\hfill]
\ifpdf
\begin{flushleft}
\fi
\begin{ttfamily}
RegExprString = AnsiString;\end{ttfamily}

\ifpdf
\end{flushleft}
\fi

\end{list}
\ifpdf
\subsection*{\large{\textbf{REChar}}\normalsize\hspace{1ex}\hrulefill}
\else
\subsection*{REChar}
\fi
\label{RegExpr-REChar}
\index{REChar}
\begin{list}{}{
\settowidth{\tmplength}{\textbf{Description}}
\setlength{\itemindent}{0cm}
\setlength{\listparindent}{0cm}
\setlength{\leftmargin}{\evensidemargin}
\addtolength{\leftmargin}{\tmplength}
\settowidth{\labelsep}{X}
\addtolength{\leftmargin}{\labelsep}
\setlength{\labelwidth}{\tmplength}
}
\item[\textbf{Declaration}\hfill]
\ifpdf
\begin{flushleft}
\fi
\begin{ttfamily}
REChar = Char;\end{ttfamily}

\ifpdf
\end{flushleft}
\fi

\end{list}
\ifpdf
\subsection*{\large{\textbf{TREOp}}\normalsize\hspace{1ex}\hrulefill}
\else
\subsection*{TREOp}
\fi
\label{RegExpr-TREOp}
\index{TREOp}
\begin{list}{}{
\settowidth{\tmplength}{\textbf{Description}}
\setlength{\itemindent}{0cm}
\setlength{\listparindent}{0cm}
\setlength{\leftmargin}{\evensidemargin}
\addtolength{\leftmargin}{\tmplength}
\settowidth{\labelsep}{X}
\addtolength{\leftmargin}{\labelsep}
\setlength{\labelwidth}{\tmplength}
}
\item[\textbf{Declaration}\hfill]
\ifpdf
\begin{flushleft}
\fi
\begin{ttfamily}
TREOp = REChar;\end{ttfamily}

\ifpdf
\end{flushleft}
\fi

\end{list}
\ifpdf
\subsection*{\large{\textbf{PREOp}}\normalsize\hspace{1ex}\hrulefill}
\else
\subsection*{PREOp}
\fi
\label{RegExpr-PREOp}
\index{PREOp}
\begin{list}{}{
\settowidth{\tmplength}{\textbf{Description}}
\setlength{\itemindent}{0cm}
\setlength{\listparindent}{0cm}
\setlength{\leftmargin}{\evensidemargin}
\addtolength{\leftmargin}{\tmplength}
\settowidth{\labelsep}{X}
\addtolength{\leftmargin}{\labelsep}
\setlength{\labelwidth}{\tmplength}
}
\item[\textbf{Declaration}\hfill]
\ifpdf
\begin{flushleft}
\fi
\begin{ttfamily}
PREOp = {\^{}}TREOp;\end{ttfamily}

\ifpdf
\end{flushleft}
\fi

\end{list}
\ifpdf
\subsection*{\large{\textbf{TRENextOff}}\normalsize\hspace{1ex}\hrulefill}
\else
\subsection*{TRENextOff}
\fi
\label{RegExpr-TRENextOff}
\index{TRENextOff}
\begin{list}{}{
\settowidth{\tmplength}{\textbf{Description}}
\setlength{\itemindent}{0cm}
\setlength{\listparindent}{0cm}
\setlength{\leftmargin}{\evensidemargin}
\addtolength{\leftmargin}{\tmplength}
\settowidth{\labelsep}{X}
\addtolength{\leftmargin}{\labelsep}
\setlength{\labelwidth}{\tmplength}
}
\item[\textbf{Declaration}\hfill]
\ifpdf
\begin{flushleft}
\fi
\begin{ttfamily}
TRENextOff = integer;\end{ttfamily}

\ifpdf
\end{flushleft}
\fi

\end{list}
\ifpdf
\subsection*{\large{\textbf{PRENextOff}}\normalsize\hspace{1ex}\hrulefill}
\else
\subsection*{PRENextOff}
\fi
\label{RegExpr-PRENextOff}
\index{PRENextOff}
\begin{list}{}{
\settowidth{\tmplength}{\textbf{Description}}
\setlength{\itemindent}{0cm}
\setlength{\listparindent}{0cm}
\setlength{\leftmargin}{\evensidemargin}
\addtolength{\leftmargin}{\tmplength}
\settowidth{\labelsep}{X}
\addtolength{\leftmargin}{\labelsep}
\setlength{\labelwidth}{\tmplength}
}
\item[\textbf{Declaration}\hfill]
\ifpdf
\begin{flushleft}
\fi
\begin{ttfamily}
PRENextOff = {\^{}}TRENextOff;\end{ttfamily}

\ifpdf
\end{flushleft}
\fi

\end{list}
\ifpdf
\subsection*{\large{\textbf{TREBracesArg}}\normalsize\hspace{1ex}\hrulefill}
\else
\subsection*{TREBracesArg}
\fi
\label{RegExpr-TREBracesArg}
\index{TREBracesArg}
\begin{list}{}{
\settowidth{\tmplength}{\textbf{Description}}
\setlength{\itemindent}{0cm}
\setlength{\listparindent}{0cm}
\setlength{\leftmargin}{\evensidemargin}
\addtolength{\leftmargin}{\tmplength}
\settowidth{\labelsep}{X}
\addtolength{\leftmargin}{\labelsep}
\setlength{\labelwidth}{\tmplength}
}
\item[\textbf{Declaration}\hfill]
\ifpdf
\begin{flushleft}
\fi
\begin{ttfamily}
TREBracesArg = integer;\end{ttfamily}

\ifpdf
\end{flushleft}
\fi

\end{list}
\ifpdf
\subsection*{\large{\textbf{PREBracesArg}}\normalsize\hspace{1ex}\hrulefill}
\else
\subsection*{PREBracesArg}
\fi
\label{RegExpr-PREBracesArg}
\index{PREBracesArg}
\begin{list}{}{
\settowidth{\tmplength}{\textbf{Description}}
\setlength{\itemindent}{0cm}
\setlength{\listparindent}{0cm}
\setlength{\leftmargin}{\evensidemargin}
\addtolength{\leftmargin}{\tmplength}
\settowidth{\labelsep}{X}
\addtolength{\leftmargin}{\labelsep}
\setlength{\labelwidth}{\tmplength}
}
\item[\textbf{Declaration}\hfill]
\ifpdf
\begin{flushleft}
\fi
\begin{ttfamily}
PREBracesArg = {\^{}}TREBracesArg;\end{ttfamily}

\ifpdf
\end{flushleft}
\fi

\end{list}
\ifpdf
\subsection*{\large{\textbf{TRegExprInvertCaseFunction}}\normalsize\hspace{1ex}\hrulefill}
\else
\subsection*{TRegExprInvertCaseFunction}
\fi
\label{RegExpr-TRegExprInvertCaseFunction}
\index{TRegExprInvertCaseFunction}
\begin{list}{}{
\settowidth{\tmplength}{\textbf{Description}}
\setlength{\itemindent}{0cm}
\setlength{\listparindent}{0cm}
\setlength{\leftmargin}{\evensidemargin}
\addtolength{\leftmargin}{\tmplength}
\settowidth{\labelsep}{X}
\addtolength{\leftmargin}{\labelsep}
\setlength{\labelwidth}{\tmplength}
}
\item[\textbf{Declaration}\hfill]
\ifpdf
\begin{flushleft}
\fi
\begin{ttfamily}
TRegExprInvertCaseFunction = function (const Ch : REChar) : REChar of object;\end{ttfamily}

\ifpdf
\end{flushleft}
\fi

\end{list}
\ifpdf
\subsection*{\large{\textbf{PSetOfREChar}}\normalsize\hspace{1ex}\hrulefill}
\else
\subsection*{PSetOfREChar}
\fi
\label{RegExpr-PSetOfREChar}
\index{PSetOfREChar}
\begin{list}{}{
\settowidth{\tmplength}{\textbf{Description}}
\setlength{\itemindent}{0cm}
\setlength{\listparindent}{0cm}
\setlength{\leftmargin}{\evensidemargin}
\addtolength{\leftmargin}{\tmplength}
\settowidth{\labelsep}{X}
\addtolength{\leftmargin}{\labelsep}
\setlength{\labelwidth}{\tmplength}
}
\item[\textbf{Declaration}\hfill]
\ifpdf
\begin{flushleft}
\fi
\begin{ttfamily}
PSetOfREChar = {\^{}}TSetOfREChar;\end{ttfamily}

\ifpdf
\end{flushleft}
\fi

\end{list}
\ifpdf
\subsection*{\large{\textbf{TSetOfREChar}}\normalsize\hspace{1ex}\hrulefill}
\else
\subsection*{TSetOfREChar}
\fi
\label{RegExpr-TSetOfREChar}
\index{TSetOfREChar}
\begin{list}{}{
\settowidth{\tmplength}{\textbf{Description}}
\setlength{\itemindent}{0cm}
\setlength{\listparindent}{0cm}
\setlength{\leftmargin}{\evensidemargin}
\addtolength{\leftmargin}{\tmplength}
\settowidth{\labelsep}{X}
\addtolength{\leftmargin}{\labelsep}
\setlength{\labelwidth}{\tmplength}
}
\item[\textbf{Declaration}\hfill]
\ifpdf
\begin{flushleft}
\fi
\begin{ttfamily}
TSetOfREChar = set of REChar;\end{ttfamily}

\ifpdf
\end{flushleft}
\fi

\end{list}
\ifpdf
\subsection*{\large{\textbf{TRegExprReplaceFunction}}\normalsize\hspace{1ex}\hrulefill}
\else
\subsection*{TRegExprReplaceFunction}
\fi
\label{RegExpr-TRegExprReplaceFunction}
\index{TRegExprReplaceFunction}
\begin{list}{}{
\settowidth{\tmplength}{\textbf{Description}}
\setlength{\itemindent}{0cm}
\setlength{\listparindent}{0cm}
\setlength{\leftmargin}{\evensidemargin}
\addtolength{\leftmargin}{\tmplength}
\settowidth{\labelsep}{X}
\addtolength{\leftmargin}{\labelsep}
\setlength{\labelwidth}{\tmplength}
}
\item[\textbf{Declaration}\hfill]
\ifpdf
\begin{flushleft}
\fi
\begin{ttfamily}
TRegExprReplaceFunction = function (ARegExpr : TRegExpr): string of object;\end{ttfamily}

\ifpdf
\end{flushleft}
\fi

\end{list}
\section{Constants}
\ifpdf
\subsection*{\large{\textbf{REOpSz}}\normalsize\hspace{1ex}\hrulefill}
\else
\subsection*{REOpSz}
\fi
\label{RegExpr-REOpSz}
\index{REOpSz}
\begin{list}{}{
\settowidth{\tmplength}{\textbf{Description}}
\setlength{\itemindent}{0cm}
\setlength{\listparindent}{0cm}
\setlength{\leftmargin}{\evensidemargin}
\addtolength{\leftmargin}{\tmplength}
\settowidth{\labelsep}{X}
\addtolength{\leftmargin}{\labelsep}
\setlength{\labelwidth}{\tmplength}
}
\item[\textbf{Declaration}\hfill]
\ifpdf
\begin{flushleft}
\fi
\begin{ttfamily}
REOpSz = SizeOf (TREOp) div SizeOf (REChar);\end{ttfamily}

\ifpdf
\end{flushleft}
\fi

\end{list}
\ifpdf
\subsection*{\large{\textbf{RENextOffSz}}\normalsize\hspace{1ex}\hrulefill}
\else
\subsection*{RENextOffSz}
\fi
\label{RegExpr-RENextOffSz}
\index{RENextOffSz}
\begin{list}{}{
\settowidth{\tmplength}{\textbf{Description}}
\setlength{\itemindent}{0cm}
\setlength{\listparindent}{0cm}
\setlength{\leftmargin}{\evensidemargin}
\addtolength{\leftmargin}{\tmplength}
\settowidth{\labelsep}{X}
\addtolength{\leftmargin}{\labelsep}
\setlength{\labelwidth}{\tmplength}
}
\item[\textbf{Declaration}\hfill]
\ifpdf
\begin{flushleft}
\fi
\begin{ttfamily}
RENextOffSz = SizeOf (TRENextOff) div SizeOf (REChar);\end{ttfamily}

\ifpdf
\end{flushleft}
\fi

\end{list}
\ifpdf
\subsection*{\large{\textbf{REBracesArgSz}}\normalsize\hspace{1ex}\hrulefill}
\else
\subsection*{REBracesArgSz}
\fi
\label{RegExpr-REBracesArgSz}
\index{REBracesArgSz}
\begin{list}{}{
\settowidth{\tmplength}{\textbf{Description}}
\setlength{\itemindent}{0cm}
\setlength{\listparindent}{0cm}
\setlength{\leftmargin}{\evensidemargin}
\addtolength{\leftmargin}{\tmplength}
\settowidth{\labelsep}{X}
\addtolength{\leftmargin}{\labelsep}
\setlength{\labelwidth}{\tmplength}
}
\item[\textbf{Declaration}\hfill]
\ifpdf
\begin{flushleft}
\fi
\begin{ttfamily}
REBracesArgSz = SizeOf (TREBracesArg) div SizeOf (REChar);\end{ttfamily}

\ifpdf
\end{flushleft}
\fi

\end{list}
\ifpdf
\subsection*{\large{\textbf{EscChar}}\normalsize\hspace{1ex}\hrulefill}
\else
\subsection*{EscChar}
\fi
\label{RegExpr-EscChar}
\index{EscChar}
\begin{list}{}{
\settowidth{\tmplength}{\textbf{Description}}
\setlength{\itemindent}{0cm}
\setlength{\listparindent}{0cm}
\setlength{\leftmargin}{\evensidemargin}
\addtolength{\leftmargin}{\tmplength}
\settowidth{\labelsep}{X}
\addtolength{\leftmargin}{\labelsep}
\setlength{\labelwidth}{\tmplength}
}
\item[\textbf{Declaration}\hfill]
\ifpdf
\begin{flushleft}
\fi
\begin{ttfamily}
EscChar = '{\textbackslash}';\end{ttfamily}

\ifpdf
\end{flushleft}
\fi

\end{list}
\ifpdf
\subsection*{\large{\textbf{RegExprModifierI}}\normalsize\hspace{1ex}\hrulefill}
\else
\subsection*{RegExprModifierI}
\fi
\label{RegExpr-RegExprModifierI}
\index{RegExprModifierI}
\begin{list}{}{
\settowidth{\tmplength}{\textbf{Description}}
\setlength{\itemindent}{0cm}
\setlength{\listparindent}{0cm}
\setlength{\leftmargin}{\evensidemargin}
\addtolength{\leftmargin}{\tmplength}
\settowidth{\labelsep}{X}
\addtolength{\leftmargin}{\labelsep}
\setlength{\labelwidth}{\tmplength}
}
\item[\textbf{Declaration}\hfill]
\ifpdf
\begin{flushleft}
\fi
\begin{ttfamily}
RegExprModifierI : boolean = False;\end{ttfamily}

\ifpdf
\end{flushleft}
\fi

\end{list}
\ifpdf
\subsection*{\large{\textbf{RegExprModifierR}}\normalsize\hspace{1ex}\hrulefill}
\else
\subsection*{RegExprModifierR}
\fi
\label{RegExpr-RegExprModifierR}
\index{RegExprModifierR}
\begin{list}{}{
\settowidth{\tmplength}{\textbf{Description}}
\setlength{\itemindent}{0cm}
\setlength{\listparindent}{0cm}
\setlength{\leftmargin}{\evensidemargin}
\addtolength{\leftmargin}{\tmplength}
\settowidth{\labelsep}{X}
\addtolength{\leftmargin}{\labelsep}
\setlength{\labelwidth}{\tmplength}
}
\item[\textbf{Declaration}\hfill]
\ifpdf
\begin{flushleft}
\fi
\begin{ttfamily}
RegExprModifierR : boolean = True;\end{ttfamily}

\ifpdf
\end{flushleft}
\fi

\end{list}
\ifpdf
\subsection*{\large{\textbf{RegExprModifierS}}\normalsize\hspace{1ex}\hrulefill}
\else
\subsection*{RegExprModifierS}
\fi
\label{RegExpr-RegExprModifierS}
\index{RegExprModifierS}
\begin{list}{}{
\settowidth{\tmplength}{\textbf{Description}}
\setlength{\itemindent}{0cm}
\setlength{\listparindent}{0cm}
\setlength{\leftmargin}{\evensidemargin}
\addtolength{\leftmargin}{\tmplength}
\settowidth{\labelsep}{X}
\addtolength{\leftmargin}{\labelsep}
\setlength{\labelwidth}{\tmplength}
}
\item[\textbf{Declaration}\hfill]
\ifpdf
\begin{flushleft}
\fi
\begin{ttfamily}
RegExprModifierS : boolean = True;\end{ttfamily}

\ifpdf
\end{flushleft}
\fi

\end{list}
\ifpdf
\subsection*{\large{\textbf{RegExprModifierG}}\normalsize\hspace{1ex}\hrulefill}
\else
\subsection*{RegExprModifierG}
\fi
\label{RegExpr-RegExprModifierG}
\index{RegExprModifierG}
\begin{list}{}{
\settowidth{\tmplength}{\textbf{Description}}
\setlength{\itemindent}{0cm}
\setlength{\listparindent}{0cm}
\setlength{\leftmargin}{\evensidemargin}
\addtolength{\leftmargin}{\tmplength}
\settowidth{\labelsep}{X}
\addtolength{\leftmargin}{\labelsep}
\setlength{\labelwidth}{\tmplength}
}
\item[\textbf{Declaration}\hfill]
\ifpdf
\begin{flushleft}
\fi
\begin{ttfamily}
RegExprModifierG : boolean = True;\end{ttfamily}

\ifpdf
\end{flushleft}
\fi

\end{list}
\ifpdf
\subsection*{\large{\textbf{RegExprModifierM}}\normalsize\hspace{1ex}\hrulefill}
\else
\subsection*{RegExprModifierM}
\fi
\label{RegExpr-RegExprModifierM}
\index{RegExprModifierM}
\begin{list}{}{
\settowidth{\tmplength}{\textbf{Description}}
\setlength{\itemindent}{0cm}
\setlength{\listparindent}{0cm}
\setlength{\leftmargin}{\evensidemargin}
\addtolength{\leftmargin}{\tmplength}
\settowidth{\labelsep}{X}
\addtolength{\leftmargin}{\labelsep}
\setlength{\labelwidth}{\tmplength}
}
\item[\textbf{Declaration}\hfill]
\ifpdf
\begin{flushleft}
\fi
\begin{ttfamily}
RegExprModifierM : boolean = False;\end{ttfamily}

\ifpdf
\end{flushleft}
\fi

\end{list}
\ifpdf
\subsection*{\large{\textbf{RegExprModifierX}}\normalsize\hspace{1ex}\hrulefill}
\else
\subsection*{RegExprModifierX}
\fi
\label{RegExpr-RegExprModifierX}
\index{RegExprModifierX}
\begin{list}{}{
\settowidth{\tmplength}{\textbf{Description}}
\setlength{\itemindent}{0cm}
\setlength{\listparindent}{0cm}
\setlength{\leftmargin}{\evensidemargin}
\addtolength{\leftmargin}{\tmplength}
\settowidth{\labelsep}{X}
\addtolength{\leftmargin}{\labelsep}
\setlength{\labelwidth}{\tmplength}
}
\item[\textbf{Declaration}\hfill]
\ifpdf
\begin{flushleft}
\fi
\begin{ttfamily}
RegExprModifierX : boolean = False;\end{ttfamily}

\ifpdf
\end{flushleft}
\fi

\end{list}
\ifpdf
\subsection*{\large{\textbf{RegExprSpaceChars}}\normalsize\hspace{1ex}\hrulefill}
\else
\subsection*{RegExprSpaceChars}
\fi
\label{RegExpr-RegExprSpaceChars}
\index{RegExprSpaceChars}
\begin{list}{}{
\settowidth{\tmplength}{\textbf{Description}}
\setlength{\itemindent}{0cm}
\setlength{\listparindent}{0cm}
\setlength{\leftmargin}{\evensidemargin}
\addtolength{\leftmargin}{\tmplength}
\settowidth{\labelsep}{X}
\addtolength{\leftmargin}{\labelsep}
\setlength{\labelwidth}{\tmplength}
}
\item[\textbf{Declaration}\hfill]
\ifpdf
\begin{flushleft}
\fi
\begin{ttfamily}
RegExprSpaceChars : RegExprString =      ' '{\#}{\$}9{\#}{\$}A{\#}{\$}D{\#}{\$}C;\end{ttfamily}

\ifpdf
\end{flushleft}
\fi

\end{list}
\ifpdf
\subsection*{\large{\textbf{RegExprWordChars}}\normalsize\hspace{1ex}\hrulefill}
\else
\subsection*{RegExprWordChars}
\fi
\label{RegExpr-RegExprWordChars}
\index{RegExprWordChars}
\begin{list}{}{
\settowidth{\tmplength}{\textbf{Description}}
\setlength{\itemindent}{0cm}
\setlength{\listparindent}{0cm}
\setlength{\leftmargin}{\evensidemargin}
\addtolength{\leftmargin}{\tmplength}
\settowidth{\labelsep}{X}
\addtolength{\leftmargin}{\labelsep}
\setlength{\labelwidth}{\tmplength}
}
\item[\textbf{Declaration}\hfill]
\ifpdf
\begin{flushleft}
\fi
\begin{ttfamily}
RegExprWordChars : RegExprString =         '0123456789'   + 'abcdefghijklmnopqrstuvwxyz'
  + 'ABCDEFGHIJKLMNOPQRSTUVWXYZ{\_}';\end{ttfamily}

\ifpdf
\end{flushleft}
\fi

\end{list}
\ifpdf
\subsection*{\large{\textbf{RegExprLineSeparators}}\normalsize\hspace{1ex}\hrulefill}
\else
\subsection*{RegExprLineSeparators}
\fi
\label{RegExpr-RegExprLineSeparators}
\index{RegExprLineSeparators}
\begin{list}{}{
\settowidth{\tmplength}{\textbf{Description}}
\setlength{\itemindent}{0cm}
\setlength{\listparindent}{0cm}
\setlength{\leftmargin}{\evensidemargin}
\addtolength{\leftmargin}{\tmplength}
\settowidth{\labelsep}{X}
\addtolength{\leftmargin}{\labelsep}
\setlength{\labelwidth}{\tmplength}
}
\item[\textbf{Declaration}\hfill]
\ifpdf
\begin{flushleft}
\fi
\begin{ttfamily}
RegExprLineSeparators : RegExprString =   {\#}{\$}d{\#}{\$}a;\end{ttfamily}

\ifpdf
\end{flushleft}
\fi

\end{list}
\ifpdf
\subsection*{\large{\textbf{RegExprLinePairedSeparator}}\normalsize\hspace{1ex}\hrulefill}
\else
\subsection*{RegExprLinePairedSeparator}
\fi
\label{RegExpr-RegExprLinePairedSeparator}
\index{RegExprLinePairedSeparator}
\begin{list}{}{
\settowidth{\tmplength}{\textbf{Description}}
\setlength{\itemindent}{0cm}
\setlength{\listparindent}{0cm}
\setlength{\leftmargin}{\evensidemargin}
\addtolength{\leftmargin}{\tmplength}
\settowidth{\labelsep}{X}
\addtolength{\leftmargin}{\labelsep}
\setlength{\labelwidth}{\tmplength}
}
\item[\textbf{Declaration}\hfill]
\ifpdf
\begin{flushleft}
\fi
\begin{ttfamily}
RegExprLinePairedSeparator : RegExprString =   {\#}{\$}d{\#}{\$}a;\end{ttfamily}

\ifpdf
\end{flushleft}
\fi

\end{list}
\ifpdf
\subsection*{\large{\textbf{NSUBEXP}}\normalsize\hspace{1ex}\hrulefill}
\else
\subsection*{NSUBEXP}
\fi
\label{RegExpr-NSUBEXP}
\index{NSUBEXP}
\begin{list}{}{
\settowidth{\tmplength}{\textbf{Description}}
\setlength{\itemindent}{0cm}
\setlength{\listparindent}{0cm}
\setlength{\leftmargin}{\evensidemargin}
\addtolength{\leftmargin}{\tmplength}
\settowidth{\labelsep}{X}
\addtolength{\leftmargin}{\labelsep}
\setlength{\labelwidth}{\tmplength}
}
\item[\textbf{Declaration}\hfill]
\ifpdf
\begin{flushleft}
\fi
\begin{ttfamily}
NSUBEXP = 15;\end{ttfamily}

\ifpdf
\end{flushleft}
\fi

\end{list}
\ifpdf
\subsection*{\large{\textbf{NSUBEXPMAX}}\normalsize\hspace{1ex}\hrulefill}
\else
\subsection*{NSUBEXPMAX}
\fi
\label{RegExpr-NSUBEXPMAX}
\index{NSUBEXPMAX}
\begin{list}{}{
\settowidth{\tmplength}{\textbf{Description}}
\setlength{\itemindent}{0cm}
\setlength{\listparindent}{0cm}
\setlength{\leftmargin}{\evensidemargin}
\addtolength{\leftmargin}{\tmplength}
\settowidth{\labelsep}{X}
\addtolength{\leftmargin}{\labelsep}
\setlength{\labelwidth}{\tmplength}
}
\item[\textbf{Declaration}\hfill]
\ifpdf
\begin{flushleft}
\fi
\begin{ttfamily}
NSUBEXPMAX = 255;\end{ttfamily}

\ifpdf
\end{flushleft}
\fi

\end{list}
\ifpdf
\subsection*{\large{\textbf{MaxBracesArg}}\normalsize\hspace{1ex}\hrulefill}
\else
\subsection*{MaxBracesArg}
\fi
\label{RegExpr-MaxBracesArg}
\index{MaxBracesArg}
\begin{list}{}{
\settowidth{\tmplength}{\textbf{Description}}
\setlength{\itemindent}{0cm}
\setlength{\listparindent}{0cm}
\setlength{\leftmargin}{\evensidemargin}
\addtolength{\leftmargin}{\tmplength}
\settowidth{\labelsep}{X}
\addtolength{\leftmargin}{\labelsep}
\setlength{\labelwidth}{\tmplength}
}
\item[\textbf{Declaration}\hfill]
\ifpdf
\begin{flushleft}
\fi
\begin{ttfamily}
MaxBracesArg = {\$}7FFFFFFF - 1;\end{ttfamily}

\ifpdf
\end{flushleft}
\fi

\end{list}
\ifpdf
\subsection*{\large{\textbf{LoopStackMax}}\normalsize\hspace{1ex}\hrulefill}
\else
\subsection*{LoopStackMax}
\fi
\label{RegExpr-LoopStackMax}
\index{LoopStackMax}
\begin{list}{}{
\settowidth{\tmplength}{\textbf{Description}}
\setlength{\itemindent}{0cm}
\setlength{\listparindent}{0cm}
\setlength{\leftmargin}{\evensidemargin}
\addtolength{\leftmargin}{\tmplength}
\settowidth{\labelsep}{X}
\addtolength{\leftmargin}{\labelsep}
\setlength{\labelwidth}{\tmplength}
}
\item[\textbf{Declaration}\hfill]
\ifpdf
\begin{flushleft}
\fi
\begin{ttfamily}
LoopStackMax = 10;\end{ttfamily}

\ifpdf
\end{flushleft}
\fi

\end{list}
\ifpdf
\subsection*{\large{\textbf{TinySetLen}}\normalsize\hspace{1ex}\hrulefill}
\else
\subsection*{TinySetLen}
\fi
\label{RegExpr-TinySetLen}
\index{TinySetLen}
\begin{list}{}{
\settowidth{\tmplength}{\textbf{Description}}
\setlength{\itemindent}{0cm}
\setlength{\listparindent}{0cm}
\setlength{\leftmargin}{\evensidemargin}
\addtolength{\leftmargin}{\tmplength}
\settowidth{\labelsep}{X}
\addtolength{\leftmargin}{\labelsep}
\setlength{\labelwidth}{\tmplength}
}
\item[\textbf{Declaration}\hfill]
\ifpdf
\begin{flushleft}
\fi
\begin{ttfamily}
TinySetLen = 3;\end{ttfamily}

\ifpdf
\end{flushleft}
\fi

\end{list}
\ifpdf
\subsection*{\large{\textbf{RegExprInvertCaseFunction}}\normalsize\hspace{1ex}\hrulefill}
\else
\subsection*{RegExprInvertCaseFunction}
\fi
\label{RegExpr-RegExprInvertCaseFunction}
\index{RegExprInvertCaseFunction}
\begin{list}{}{
\settowidth{\tmplength}{\textbf{Description}}
\setlength{\itemindent}{0cm}
\setlength{\listparindent}{0cm}
\setlength{\leftmargin}{\evensidemargin}
\addtolength{\leftmargin}{\tmplength}
\settowidth{\labelsep}{X}
\addtolength{\leftmargin}{\labelsep}
\setlength{\labelwidth}{\tmplength}
}
\item[\textbf{Declaration}\hfill]
\ifpdf
\begin{flushleft}
\fi
\begin{ttfamily}
RegExprInvertCaseFunction : TRegExprInvertCaseFunction =  TRegExpr.InvertCaseFunction;\end{ttfamily}

\ifpdf
\end{flushleft}
\fi

\end{list}
\chapter{Unit settings}
\label{settings}
\index{settings}
\section{Description}
Change Listaller's settings
\section{uses}
\begin{itemize}
\item \begin{ttfamily}Classes\end{ttfamily}\item \begin{ttfamily}SysUtils\end{ttfamily}\item \begin{ttfamily}LResources\end{ttfamily}\item \begin{ttfamily}Forms\end{ttfamily}\item \begin{ttfamily}Controls\end{ttfamily}\item \begin{ttfamily}Graphics\end{ttfamily}\item \begin{ttfamily}Dialogs\end{ttfamily}\item \begin{ttfamily}ComCtrls\end{ttfamily}\item \begin{ttfamily}CheckLst\end{ttfamily}\item \begin{ttfamily}StdCtrls\end{ttfamily}\item \begin{ttfamily}Buttons\end{ttfamily}\item \begin{ttfamily}ExtCtrls\end{ttfamily}\item \begin{ttfamily}LCLType\end{ttfamily}\item \begin{ttfamily}manager\end{ttfamily}(\ref{manager})\item \begin{ttfamily}IniFiles\end{ttfamily}\item \begin{ttfamily}Spin\end{ttfamily}\item \begin{ttfamily}Process\end{ttfamily}\item \begin{ttfamily}utilities\end{ttfamily}(\ref{utilities})\item \begin{ttfamily}trstrings\end{ttfamily}(\ref{trstrings})\item \begin{ttfamily}translations\end{ttfamily}\item \begin{ttfamily}gettext\end{ttfamily}\item \begin{ttfamily}Menus\end{ttfamily}\item \begin{ttfamily}aboutbox\end{ttfamily}\end{itemize}
\section{Overview}
\begin{description}
\item[\texttt{\begin{ttfamily}TForm2\end{ttfamily} Class}]
\end{description}
\section{Classes, Interfaces, Objects and Records}
\ifpdf
\subsection*{\large{\textbf{TForm2 Class}}\normalsize\hspace{1ex}\hrulefill}
\else
\subsection*{TForm2 Class}
\fi
\label{settings.TForm2}
\index{TForm2}
\subsubsection*{\large{\textbf{Hierarchy}}\normalsize\hspace{1ex}\hfill}
TForm2 {$>$} TForm
%%%%Description
\subsubsection*{\large{\textbf{Fields}}\normalsize\hspace{1ex}\hfill}
\begin{list}{}{
\settowidth{\tmplength}{\textbf{PageControl1}}
\setlength{\itemindent}{0cm}
\setlength{\listparindent}{0cm}
\setlength{\leftmargin}{\evensidemargin}
\addtolength{\leftmargin}{\tmplength}
\settowidth{\labelsep}{X}
\addtolength{\leftmargin}{\labelsep}
\setlength{\labelwidth}{\tmplength}
}
\label{settings.TForm2-BitBtn1}
\index{BitBtn1}
\item[\textbf{BitBtn1}\hfill]
\ifpdf
\begin{flushleft}
\fi
\begin{ttfamily}
public BitBtn1: TBitBtn;\end{ttfamily}

\ifpdf
\end{flushleft}
\fi


\par  \label{settings.TForm2-BitBtn2}
\index{BitBtn2}
\item[\textbf{BitBtn2}\hfill]
\ifpdf
\begin{flushleft}
\fi
\begin{ttfamily}
public BitBtn2: TBitBtn;\end{ttfamily}

\ifpdf
\end{flushleft}
\fi


\par  \label{settings.TForm2-BitBtn3}
\index{BitBtn3}
\item[\textbf{BitBtn3}\hfill]
\ifpdf
\begin{flushleft}
\fi
\begin{ttfamily}
public BitBtn3: TBitBtn;\end{ttfamily}

\ifpdf
\end{flushleft}
\fi


\par  \label{settings.TForm2-Button1}
\index{Button1}
\item[\textbf{Button1}\hfill]
\ifpdf
\begin{flushleft}
\fi
\begin{ttfamily}
public Button1: TButton;\end{ttfamily}

\ifpdf
\end{flushleft}
\fi


\par  \label{settings.TForm2-CheckBox1}
\index{CheckBox1}
\item[\textbf{CheckBox1}\hfill]
\ifpdf
\begin{flushleft}
\fi
\begin{ttfamily}
public CheckBox1: TCheckBox;\end{ttfamily}

\ifpdf
\end{flushleft}
\fi


\par  \label{settings.TForm2-CheckBox2}
\index{CheckBox2}
\item[\textbf{CheckBox2}\hfill]
\ifpdf
\begin{flushleft}
\fi
\begin{ttfamily}
public CheckBox2: TCheckBox;\end{ttfamily}

\ifpdf
\end{flushleft}
\fi


\par  \label{settings.TForm2-CheckBox4}
\index{CheckBox4}
\item[\textbf{CheckBox4}\hfill]
\ifpdf
\begin{flushleft}
\fi
\begin{ttfamily}
public CheckBox4: TCheckBox;\end{ttfamily}

\ifpdf
\end{flushleft}
\fi


\par  \label{settings.TForm2-Edit1}
\index{Edit1}
\item[\textbf{Edit1}\hfill]
\ifpdf
\begin{flushleft}
\fi
\begin{ttfamily}
public Edit1: TEdit;\end{ttfamily}

\ifpdf
\end{flushleft}
\fi


\par  \label{settings.TForm2-GroupBox1}
\index{GroupBox1}
\item[\textbf{GroupBox1}\hfill]
\ifpdf
\begin{flushleft}
\fi
\begin{ttfamily}
public GroupBox1: TGroupBox;\end{ttfamily}

\ifpdf
\end{flushleft}
\fi


\par  \label{settings.TForm2-Label3}
\index{Label3}
\item[\textbf{Label3}\hfill]
\ifpdf
\begin{flushleft}
\fi
\begin{ttfamily}
public Label3: TLabel;\end{ttfamily}

\ifpdf
\end{flushleft}
\fi


\par  \label{settings.TForm2-Label4}
\index{Label4}
\item[\textbf{Label4}\hfill]
\ifpdf
\begin{flushleft}
\fi
\begin{ttfamily}
public Label4: TLabel;\end{ttfamily}

\ifpdf
\end{flushleft}
\fi


\par  \label{settings.TForm2-edtUsername}
\index{edtUsername}
\item[\textbf{edtUsername}\hfill]
\ifpdf
\begin{flushleft}
\fi
\begin{ttfamily}
public edtUsername: TLabeledEdit;\end{ttfamily}

\ifpdf
\end{flushleft}
\fi


\par  \label{settings.TForm2-edtPasswd}
\index{edtPasswd}
\item[\textbf{edtPasswd}\hfill]
\ifpdf
\begin{flushleft}
\fi
\begin{ttfamily}
public edtPasswd: TLabeledEdit;\end{ttfamily}

\ifpdf
\end{flushleft}
\fi


\par  \label{settings.TForm2-Label5}
\index{Label5}
\item[\textbf{Label5}\hfill]
\ifpdf
\begin{flushleft}
\fi
\begin{ttfamily}
public Label5: TLabel;\end{ttfamily}

\ifpdf
\end{flushleft}
\fi


\par  \label{settings.TForm2-edtFTPProxy}
\index{edtFTPProxy}
\item[\textbf{edtFTPProxy}\hfill]
\ifpdf
\begin{flushleft}
\fi
\begin{ttfamily}
public edtFTPProxy: TLabeledEdit;\end{ttfamily}

\ifpdf
\end{flushleft}
\fi


\par  \label{settings.TForm2-SpinEdit1}
\index{SpinEdit1}
\item[\textbf{SpinEdit1}\hfill]
\ifpdf
\begin{flushleft}
\fi
\begin{ttfamily}
public SpinEdit1: TSpinEdit;\end{ttfamily}

\ifpdf
\end{flushleft}
\fi


\par  \label{settings.TForm2-SpinEdit2}
\index{SpinEdit2}
\item[\textbf{SpinEdit2}\hfill]
\ifpdf
\begin{flushleft}
\fi
\begin{ttfamily}
public SpinEdit2: TSpinEdit;\end{ttfamily}

\ifpdf
\end{flushleft}
\fi


\par  \label{settings.TForm2-UListBox1}
\index{UListBox1}
\item[\textbf{UListBox1}\hfill]
\ifpdf
\begin{flushleft}
\fi
\begin{ttfamily}
public UListBox1: TCheckListBox;\end{ttfamily}

\ifpdf
\end{flushleft}
\fi


\par  \label{settings.TForm2-Label1}
\index{Label1}
\item[\textbf{Label1}\hfill]
\ifpdf
\begin{flushleft}
\fi
\begin{ttfamily}
public Label1: TLabel;\end{ttfamily}

\ifpdf
\end{flushleft}
\fi


\par  \label{settings.TForm2-Label2}
\index{Label2}
\item[\textbf{Label2}\hfill]
\ifpdf
\begin{flushleft}
\fi
\begin{ttfamily}
public Label2: TLabel;\end{ttfamily}

\ifpdf
\end{flushleft}
\fi


\par  \label{settings.TForm2-PageControl1}
\index{PageControl1}
\item[\textbf{PageControl1}\hfill]
\ifpdf
\begin{flushleft}
\fi
\begin{ttfamily}
public PageControl1: TPageControl;\end{ttfamily}

\ifpdf
\end{flushleft}
\fi


\par  \label{settings.TForm2-MainPage}
\index{MainPage}
\item[\textbf{MainPage}\hfill]
\ifpdf
\begin{flushleft}
\fi
\begin{ttfamily}
public MainPage: TTabSheet;\end{ttfamily}

\ifpdf
\end{flushleft}
\fi


\par  \label{settings.TForm2-Panel1}
\index{Panel1}
\item[\textbf{Panel1}\hfill]
\ifpdf
\begin{flushleft}
\fi
\begin{ttfamily}
public Panel1: TPanel;\end{ttfamily}

\ifpdf
\end{flushleft}
\fi


\par  \label{settings.TForm2-UpdPage}
\index{UpdPage}
\item[\textbf{UpdPage}\hfill]
\ifpdf
\begin{flushleft}
\fi
\begin{ttfamily}
public UpdPage: TTabSheet;\end{ttfamily}

\ifpdf
\end{flushleft}
\fi


\par  \end{list}
\subsubsection*{\large{\textbf{Methods}}\normalsize\hspace{1ex}\hfill}
\paragraph*{BitBtn1Click}\hspace*{\fill}

\label{settings.TForm2-BitBtn1Click}
\index{BitBtn1Click}
\begin{list}{}{
\settowidth{\tmplength}{\textbf{Description}}
\setlength{\itemindent}{0cm}
\setlength{\listparindent}{0cm}
\setlength{\leftmargin}{\evensidemargin}
\addtolength{\leftmargin}{\tmplength}
\settowidth{\labelsep}{X}
\addtolength{\leftmargin}{\labelsep}
\setlength{\labelwidth}{\tmplength}
}
\item[\textbf{Declaration}\hfill]
\ifpdf
\begin{flushleft}
\fi
\begin{ttfamily}
public procedure BitBtn1Click(Sender: TObject);\end{ttfamily}

\ifpdf
\end{flushleft}
\fi

\end{list}
\paragraph*{BitBtn2Click}\hspace*{\fill}

\label{settings.TForm2-BitBtn2Click}
\index{BitBtn2Click}
\begin{list}{}{
\settowidth{\tmplength}{\textbf{Description}}
\setlength{\itemindent}{0cm}
\setlength{\listparindent}{0cm}
\setlength{\leftmargin}{\evensidemargin}
\addtolength{\leftmargin}{\tmplength}
\settowidth{\labelsep}{X}
\addtolength{\leftmargin}{\labelsep}
\setlength{\labelwidth}{\tmplength}
}
\item[\textbf{Declaration}\hfill]
\ifpdf
\begin{flushleft}
\fi
\begin{ttfamily}
public procedure BitBtn2Click(Sender: TObject);\end{ttfamily}

\ifpdf
\end{flushleft}
\fi

\end{list}
\paragraph*{BitBtn3Click}\hspace*{\fill}

\label{settings.TForm2-BitBtn3Click}
\index{BitBtn3Click}
\begin{list}{}{
\settowidth{\tmplength}{\textbf{Description}}
\setlength{\itemindent}{0cm}
\setlength{\listparindent}{0cm}
\setlength{\leftmargin}{\evensidemargin}
\addtolength{\leftmargin}{\tmplength}
\settowidth{\labelsep}{X}
\addtolength{\leftmargin}{\labelsep}
\setlength{\labelwidth}{\tmplength}
}
\item[\textbf{Declaration}\hfill]
\ifpdf
\begin{flushleft}
\fi
\begin{ttfamily}
public procedure BitBtn3Click(Sender: TObject);\end{ttfamily}

\ifpdf
\end{flushleft}
\fi

\end{list}
\paragraph*{Button1Click}\hspace*{\fill}

\label{settings.TForm2-Button1Click}
\index{Button1Click}
\begin{list}{}{
\settowidth{\tmplength}{\textbf{Description}}
\setlength{\itemindent}{0cm}
\setlength{\listparindent}{0cm}
\setlength{\leftmargin}{\evensidemargin}
\addtolength{\leftmargin}{\tmplength}
\settowidth{\labelsep}{X}
\addtolength{\leftmargin}{\labelsep}
\setlength{\labelwidth}{\tmplength}
}
\item[\textbf{Declaration}\hfill]
\ifpdf
\begin{flushleft}
\fi
\begin{ttfamily}
public procedure Button1Click(Sender: TObject);\end{ttfamily}

\ifpdf
\end{flushleft}
\fi

\end{list}
\paragraph*{cbLoadUnspAppsChange}\hspace*{\fill}

\label{settings.TForm2-cbLoadUnspAppsChange}
\index{cbLoadUnspAppsChange}
\begin{list}{}{
\settowidth{\tmplength}{\textbf{Description}}
\setlength{\itemindent}{0cm}
\setlength{\listparindent}{0cm}
\setlength{\leftmargin}{\evensidemargin}
\addtolength{\leftmargin}{\tmplength}
\settowidth{\labelsep}{X}
\addtolength{\leftmargin}{\labelsep}
\setlength{\labelwidth}{\tmplength}
}
\item[\textbf{Declaration}\hfill]
\ifpdf
\begin{flushleft}
\fi
\begin{ttfamily}
public procedure cbLoadUnspAppsChange(Sender: TObject);\end{ttfamily}

\ifpdf
\end{flushleft}
\fi

\end{list}
\paragraph*{CheckBox1Change}\hspace*{\fill}

\label{settings.TForm2-CheckBox1Change}
\index{CheckBox1Change}
\begin{list}{}{
\settowidth{\tmplength}{\textbf{Description}}
\setlength{\itemindent}{0cm}
\setlength{\listparindent}{0cm}
\setlength{\leftmargin}{\evensidemargin}
\addtolength{\leftmargin}{\tmplength}
\settowidth{\labelsep}{X}
\addtolength{\leftmargin}{\labelsep}
\setlength{\labelwidth}{\tmplength}
}
\item[\textbf{Declaration}\hfill]
\ifpdf
\begin{flushleft}
\fi
\begin{ttfamily}
public procedure CheckBox1Change(Sender: TObject);\end{ttfamily}

\ifpdf
\end{flushleft}
\fi

\end{list}
\paragraph*{CheckBox2Change}\hspace*{\fill}

\label{settings.TForm2-CheckBox2Change}
\index{CheckBox2Change}
\begin{list}{}{
\settowidth{\tmplength}{\textbf{Description}}
\setlength{\itemindent}{0cm}
\setlength{\listparindent}{0cm}
\setlength{\leftmargin}{\evensidemargin}
\addtolength{\leftmargin}{\tmplength}
\settowidth{\labelsep}{X}
\addtolength{\leftmargin}{\labelsep}
\setlength{\labelwidth}{\tmplength}
}
\item[\textbf{Declaration}\hfill]
\ifpdf
\begin{flushleft}
\fi
\begin{ttfamily}
public procedure CheckBox2Change(Sender: TObject);\end{ttfamily}

\ifpdf
\end{flushleft}
\fi

\end{list}
\paragraph*{CheckBox3Change}\hspace*{\fill}

\label{settings.TForm2-CheckBox3Change}
\index{CheckBox3Change}
\begin{list}{}{
\settowidth{\tmplength}{\textbf{Description}}
\setlength{\itemindent}{0cm}
\setlength{\listparindent}{0cm}
\setlength{\leftmargin}{\evensidemargin}
\addtolength{\leftmargin}{\tmplength}
\settowidth{\labelsep}{X}
\addtolength{\leftmargin}{\labelsep}
\setlength{\labelwidth}{\tmplength}
}
\item[\textbf{Declaration}\hfill]
\ifpdf
\begin{flushleft}
\fi
\begin{ttfamily}
public procedure CheckBox3Change(Sender: TObject);\end{ttfamily}

\ifpdf
\end{flushleft}
\fi

\end{list}
\paragraph*{CheckBox4Change}\hspace*{\fill}

\label{settings.TForm2-CheckBox4Change}
\index{CheckBox4Change}
\begin{list}{}{
\settowidth{\tmplength}{\textbf{Description}}
\setlength{\itemindent}{0cm}
\setlength{\listparindent}{0cm}
\setlength{\leftmargin}{\evensidemargin}
\addtolength{\leftmargin}{\tmplength}
\settowidth{\labelsep}{X}
\addtolength{\leftmargin}{\labelsep}
\setlength{\labelwidth}{\tmplength}
}
\item[\textbf{Declaration}\hfill]
\ifpdf
\begin{flushleft}
\fi
\begin{ttfamily}
public procedure CheckBox4Change(Sender: TObject);\end{ttfamily}

\ifpdf
\end{flushleft}
\fi

\end{list}
\paragraph*{FormCreate}\hspace*{\fill}

\label{settings.TForm2-FormCreate}
\index{FormCreate}
\begin{list}{}{
\settowidth{\tmplength}{\textbf{Description}}
\setlength{\itemindent}{0cm}
\setlength{\listparindent}{0cm}
\setlength{\leftmargin}{\evensidemargin}
\addtolength{\leftmargin}{\tmplength}
\settowidth{\labelsep}{X}
\addtolength{\leftmargin}{\labelsep}
\setlength{\labelwidth}{\tmplength}
}
\item[\textbf{Declaration}\hfill]
\ifpdf
\begin{flushleft}
\fi
\begin{ttfamily}
public procedure FormCreate(Sender: TObject);\end{ttfamily}

\ifpdf
\end{flushleft}
\fi

\end{list}
\paragraph*{FormShow}\hspace*{\fill}

\label{settings.TForm2-FormShow}
\index{FormShow}
\begin{list}{}{
\settowidth{\tmplength}{\textbf{Description}}
\setlength{\itemindent}{0cm}
\setlength{\listparindent}{0cm}
\setlength{\leftmargin}{\evensidemargin}
\addtolength{\leftmargin}{\tmplength}
\settowidth{\labelsep}{X}
\addtolength{\leftmargin}{\labelsep}
\setlength{\labelwidth}{\tmplength}
}
\item[\textbf{Declaration}\hfill]
\ifpdf
\begin{flushleft}
\fi
\begin{ttfamily}
public procedure FormShow(Sender: TObject);\end{ttfamily}

\ifpdf
\end{flushleft}
\fi

\end{list}
\paragraph*{MainPageShow}\hspace*{\fill}

\label{settings.TForm2-MainPageShow}
\index{MainPageShow}
\begin{list}{}{
\settowidth{\tmplength}{\textbf{Description}}
\setlength{\itemindent}{0cm}
\setlength{\listparindent}{0cm}
\setlength{\leftmargin}{\evensidemargin}
\addtolength{\leftmargin}{\tmplength}
\settowidth{\labelsep}{X}
\addtolength{\leftmargin}{\labelsep}
\setlength{\labelwidth}{\tmplength}
}
\item[\textbf{Declaration}\hfill]
\ifpdf
\begin{flushleft}
\fi
\begin{ttfamily}
public procedure MainPageShow(Sender: TObject);\end{ttfamily}

\ifpdf
\end{flushleft}
\fi

\end{list}
\paragraph*{PageControl1Change}\hspace*{\fill}

\label{settings.TForm2-PageControl1Change}
\index{PageControl1Change}
\begin{list}{}{
\settowidth{\tmplength}{\textbf{Description}}
\setlength{\itemindent}{0cm}
\setlength{\listparindent}{0cm}
\setlength{\leftmargin}{\evensidemargin}
\addtolength{\leftmargin}{\tmplength}
\settowidth{\labelsep}{X}
\addtolength{\leftmargin}{\labelsep}
\setlength{\labelwidth}{\tmplength}
}
\item[\textbf{Declaration}\hfill]
\ifpdf
\begin{flushleft}
\fi
\begin{ttfamily}
public procedure PageControl1Change(Sender: TObject);\end{ttfamily}

\ifpdf
\end{flushleft}
\fi

\end{list}
\paragraph*{UListBox1Click}\hspace*{\fill}

\label{settings.TForm2-UListBox1Click}
\index{UListBox1Click}
\begin{list}{}{
\settowidth{\tmplength}{\textbf{Description}}
\setlength{\itemindent}{0cm}
\setlength{\listparindent}{0cm}
\setlength{\leftmargin}{\evensidemargin}
\addtolength{\leftmargin}{\tmplength}
\settowidth{\labelsep}{X}
\addtolength{\leftmargin}{\labelsep}
\setlength{\labelwidth}{\tmplength}
}
\item[\textbf{Declaration}\hfill]
\ifpdf
\begin{flushleft}
\fi
\begin{ttfamily}
public procedure UListBox1Click(Sender: TObject);\end{ttfamily}

\ifpdf
\end{flushleft}
\fi

\end{list}
\section{Variables}
\ifpdf
\subsection*{\large{\textbf{Form2}}\normalsize\hspace{1ex}\hrulefill}
\else
\subsection*{Form2}
\fi
\label{settings-Form2}
\index{Form2}
\begin{list}{}{
\settowidth{\tmplength}{\textbf{Description}}
\setlength{\itemindent}{0cm}
\setlength{\listparindent}{0cm}
\setlength{\leftmargin}{\evensidemargin}
\addtolength{\leftmargin}{\tmplength}
\settowidth{\labelsep}{X}
\addtolength{\leftmargin}{\labelsep}
\setlength{\labelwidth}{\tmplength}
}
\item[\textbf{Declaration}\hfill]
\ifpdf
\begin{flushleft}
\fi
\begin{ttfamily}
Form2: TForm2;\end{ttfamily}

\ifpdf
\end{flushleft}
\fi

\end{list}
\chapter{Unit swcatalog}
\label{swcatalog}
\index{swcatalog}
\section{Description}
This unit contains code that is used by the catalogue{-}GUI
\section{uses}
\begin{itemize}
\item \begin{ttfamily}Classes\end{ttfamily}\item \begin{ttfamily}SysUtils\end{ttfamily}\item \begin{ttfamily}LResources\end{ttfamily}\item \begin{ttfamily}Forms\end{ttfamily}\item \begin{ttfamily}Controls\end{ttfamily}\item \begin{ttfamily}Graphics\end{ttfamily}\item \begin{ttfamily}Dialogs\end{ttfamily}\item \begin{ttfamily}StdCtrls\end{ttfamily}\item \begin{ttfamily}ComCtrls\end{ttfamily}\item \begin{ttfamily}Buttons\end{ttfamily}\item \begin{ttfamily}manager\end{ttfamily}(\ref{manager})\item \begin{ttfamily}HTTPSend\end{ttfamily}(\ref{httpsend})\item \begin{ttfamily}XMLRead\end{ttfamily}\item \begin{ttfamily}DOM\end{ttfamily}\item \begin{ttfamily}IniFiles\end{ttfamily}\item \begin{ttfamily}utilities\end{ttfamily}(\ref{utilities})\item \begin{ttfamily}gtk2\end{ttfamily}\item \begin{ttfamily}trstrings\end{ttfamily}(\ref{trstrings})\item \begin{ttfamily}blcksock\end{ttfamily}\item \begin{ttfamily}Process\end{ttfamily}\item \begin{ttfamily}ExtCtrls\end{ttfamily}\item \begin{ttfamily}LCLType\end{ttfamily}\end{itemize}
\section{Overview}
\begin{description}
\item[\texttt{\begin{ttfamily}TSCForm\end{ttfamily} Class}]
\item[\texttt{\begin{ttfamily}TCTLEntry\end{ttfamily} Class}]
\end{description}
\section{Classes, Interfaces, Objects and Records}
\ifpdf
\subsection*{\large{\textbf{TSCForm Class}}\normalsize\hspace{1ex}\hrulefill}
\else
\subsection*{TSCForm Class}
\fi
\label{swcatalog.TSCForm}
\index{TSCForm}
\subsubsection*{\large{\textbf{Hierarchy}}\normalsize\hspace{1ex}\hfill}
TSCForm {$>$} TForm
%%%%Description
\subsubsection*{\large{\textbf{Fields}}\normalsize\hspace{1ex}\hfill}
\begin{list}{}{
\settowidth{\tmplength}{\textbf{ImageList1}}
\setlength{\itemindent}{0cm}
\setlength{\listparindent}{0cm}
\setlength{\leftmargin}{\evensidemargin}
\addtolength{\leftmargin}{\tmplength}
\settowidth{\labelsep}{X}
\addtolength{\leftmargin}{\labelsep}
\setlength{\labelwidth}{\tmplength}
}
\label{swcatalog.TSCForm-BitBtn1}
\index{BitBtn1}
\item[\textbf{BitBtn1}\hfill]
\ifpdf
\begin{flushleft}
\fi
\begin{ttfamily}
public BitBtn1: TBitBtn;\end{ttfamily}

\ifpdf
\end{flushleft}
\fi


\par  \label{swcatalog.TSCForm-BitBtn2}
\index{BitBtn2}
\item[\textbf{BitBtn2}\hfill]
\ifpdf
\begin{flushleft}
\fi
\begin{ttfamily}
public BitBtn2: TBitBtn;\end{ttfamily}

\ifpdf
\end{flushleft}
\fi


\par  \label{swcatalog.TSCForm-ComboBox1}
\index{ComboBox1}
\item[\textbf{ComboBox1}\hfill]
\ifpdf
\begin{flushleft}
\fi
\begin{ttfamily}
public ComboBox1: TComboBox;\end{ttfamily}

\ifpdf
\end{flushleft}
\fi


\par  \label{swcatalog.TSCForm-ImageList1}
\index{ImageList1}
\item[\textbf{ImageList1}\hfill]
\ifpdf
\begin{flushleft}
\fi
\begin{ttfamily}
public ImageList1: TImageList;\end{ttfamily}

\ifpdf
\end{flushleft}
\fi


\par  \label{swcatalog.TSCForm-Label1}
\index{Label1}
\item[\textbf{Label1}\hfill]
\ifpdf
\begin{flushleft}
\fi
\begin{ttfamily}
public Label1: TLabel;\end{ttfamily}

\ifpdf
\end{flushleft}
\fi


\par  \label{swcatalog.TSCForm-Label2}
\index{Label2}
\item[\textbf{Label2}\hfill]
\ifpdf
\begin{flushleft}
\fi
\begin{ttfamily}
public Label2: TLabel;\end{ttfamily}

\ifpdf
\end{flushleft}
\fi


\par  \label{swcatalog.TSCForm-Label3}
\index{Label3}
\item[\textbf{Label3}\hfill]
\ifpdf
\begin{flushleft}
\fi
\begin{ttfamily}
public Label3: TLabel;\end{ttfamily}

\ifpdf
\end{flushleft}
\fi


\par  \label{swcatalog.TSCForm-CView}
\index{CView}
\item[\textbf{CView}\hfill]
\ifpdf
\begin{flushleft}
\fi
\begin{ttfamily}
public CView: TListView;\end{ttfamily}

\ifpdf
\end{flushleft}
\fi


\par  \label{swcatalog.TSCForm-Label4}
\index{Label4}
\item[\textbf{Label4}\hfill]
\ifpdf
\begin{flushleft}
\fi
\begin{ttfamily}
public Label4: TLabel;\end{ttfamily}

\ifpdf
\end{flushleft}
\fi


\par  \label{swcatalog.TSCForm-Memo1}
\index{Memo1}
\item[\textbf{Memo1}\hfill]
\ifpdf
\begin{flushleft}
\fi
\begin{ttfamily}
public Memo1: TMemo;\end{ttfamily}

\ifpdf
\end{flushleft}
\fi


\par  \label{swcatalog.TSCForm-MnProgress}
\index{MnProgress}
\item[\textbf{MnProgress}\hfill]
\ifpdf
\begin{flushleft}
\fi
\begin{ttfamily}
public MnProgress: TProgressBar;\end{ttfamily}

\ifpdf
\end{flushleft}
\fi


\par  \label{swcatalog.TSCForm-DLProgress}
\index{DLProgress}
\item[\textbf{DLProgress}\hfill]
\ifpdf
\begin{flushleft}
\fi
\begin{ttfamily}
public DLProgress: TProgressBar;\end{ttfamily}

\ifpdf
\end{flushleft}
\fi


\par  \label{swcatalog.TSCForm-Panel1}
\index{Panel1}
\item[\textbf{Panel1}\hfill]
\ifpdf
\begin{flushleft}
\fi
\begin{ttfamily}
public Panel1: TPanel;\end{ttfamily}

\ifpdf
\end{flushleft}
\fi


\par  \label{swcatalog.TSCForm-ScrollBox1}
\index{ScrollBox1}
\item[\textbf{ScrollBox1}\hfill]
\ifpdf
\begin{flushleft}
\fi
\begin{ttfamily}
public ScrollBox1: TScrollBox;\end{ttfamily}

\ifpdf
\end{flushleft}
\fi


\par  \label{swcatalog.TSCForm-HQBox}
\index{HQBox}
\item[\textbf{HQBox}\hfill]
\ifpdf
\begin{flushleft}
\fi
\begin{ttfamily}
public HQBox: TToggleBox;\end{ttfamily}

\ifpdf
\end{flushleft}
\fi


\par  \label{swcatalog.TSCForm-Splitter1}
\index{Splitter1}
\item[\textbf{Splitter1}\hfill]
\ifpdf
\begin{flushleft}
\fi
\begin{ttfamily}
public Splitter1: TSplitter;\end{ttfamily}

\ifpdf
\end{flushleft}
\fi


\par  \label{swcatalog.TSCForm-HTTP}
\index{HTTP}
\item[\textbf{HTTP}\hfill]
\ifpdf
\begin{flushleft}
\fi
\begin{ttfamily}
public HTTP: THTTPSend;\end{ttfamily}

\ifpdf
\end{flushleft}
\fi


\par  \label{swcatalog.TSCForm-PB1}
\index{PB1}
\item[\textbf{PB1}\hfill]
\ifpdf
\begin{flushleft}
\fi
\begin{ttfamily}
public PB1: Boolean;\end{ttfamily}

\ifpdf
\end{flushleft}
\fi


\par  \end{list}
\subsubsection*{\large{\textbf{Methods}}\normalsize\hspace{1ex}\hfill}
\paragraph*{BitBtn1Click}\hspace*{\fill}

\label{swcatalog.TSCForm-BitBtn1Click}
\index{BitBtn1Click}
\begin{list}{}{
\settowidth{\tmplength}{\textbf{Description}}
\setlength{\itemindent}{0cm}
\setlength{\listparindent}{0cm}
\setlength{\leftmargin}{\evensidemargin}
\addtolength{\leftmargin}{\tmplength}
\settowidth{\labelsep}{X}
\addtolength{\leftmargin}{\labelsep}
\setlength{\labelwidth}{\tmplength}
}
\item[\textbf{Declaration}\hfill]
\ifpdf
\begin{flushleft}
\fi
\begin{ttfamily}
public procedure BitBtn1Click(Sender: TObject);\end{ttfamily}

\ifpdf
\end{flushleft}
\fi

\end{list}
\paragraph*{BitBtn2Click}\hspace*{\fill}

\label{swcatalog.TSCForm-BitBtn2Click}
\index{BitBtn2Click}
\begin{list}{}{
\settowidth{\tmplength}{\textbf{Description}}
\setlength{\itemindent}{0cm}
\setlength{\listparindent}{0cm}
\setlength{\leftmargin}{\evensidemargin}
\addtolength{\leftmargin}{\tmplength}
\settowidth{\labelsep}{X}
\addtolength{\leftmargin}{\labelsep}
\setlength{\labelwidth}{\tmplength}
}
\item[\textbf{Declaration}\hfill]
\ifpdf
\begin{flushleft}
\fi
\begin{ttfamily}
public procedure BitBtn2Click(Sender: TObject);\end{ttfamily}

\ifpdf
\end{flushleft}
\fi

\end{list}
\paragraph*{CViewClick}\hspace*{\fill}

\label{swcatalog.TSCForm-CViewClick}
\index{CViewClick}
\begin{list}{}{
\settowidth{\tmplength}{\textbf{Description}}
\setlength{\itemindent}{0cm}
\setlength{\listparindent}{0cm}
\setlength{\leftmargin}{\evensidemargin}
\addtolength{\leftmargin}{\tmplength}
\settowidth{\labelsep}{X}
\addtolength{\leftmargin}{\labelsep}
\setlength{\labelwidth}{\tmplength}
}
\item[\textbf{Declaration}\hfill]
\ifpdf
\begin{flushleft}
\fi
\begin{ttfamily}
public procedure CViewClick(Sender: TObject);\end{ttfamily}

\ifpdf
\end{flushleft}
\fi

\end{list}
\paragraph*{CViewKeyDown}\hspace*{\fill}

\label{swcatalog.TSCForm-CViewKeyDown}
\index{CViewKeyDown}
\begin{list}{}{
\settowidth{\tmplength}{\textbf{Description}}
\setlength{\itemindent}{0cm}
\setlength{\listparindent}{0cm}
\setlength{\leftmargin}{\evensidemargin}
\addtolength{\leftmargin}{\tmplength}
\settowidth{\labelsep}{X}
\addtolength{\leftmargin}{\labelsep}
\setlength{\labelwidth}{\tmplength}
}
\item[\textbf{Declaration}\hfill]
\ifpdf
\begin{flushleft}
\fi
\begin{ttfamily}
public procedure CViewKeyDown(Sender: TObject; var Key: Word; Shift: TShiftState);\end{ttfamily}

\ifpdf
\end{flushleft}
\fi

\end{list}
\paragraph*{FormActivate}\hspace*{\fill}

\label{swcatalog.TSCForm-FormActivate}
\index{FormActivate}
\begin{list}{}{
\settowidth{\tmplength}{\textbf{Description}}
\setlength{\itemindent}{0cm}
\setlength{\listparindent}{0cm}
\setlength{\leftmargin}{\evensidemargin}
\addtolength{\leftmargin}{\tmplength}
\settowidth{\labelsep}{X}
\addtolength{\leftmargin}{\labelsep}
\setlength{\labelwidth}{\tmplength}
}
\item[\textbf{Declaration}\hfill]
\ifpdf
\begin{flushleft}
\fi
\begin{ttfamily}
public procedure FormActivate(Sender: TObject);\end{ttfamily}

\ifpdf
\end{flushleft}
\fi

\end{list}
\paragraph*{FormClose}\hspace*{\fill}

\label{swcatalog.TSCForm-FormClose}
\index{FormClose}
\begin{list}{}{
\settowidth{\tmplength}{\textbf{Description}}
\setlength{\itemindent}{0cm}
\setlength{\listparindent}{0cm}
\setlength{\leftmargin}{\evensidemargin}
\addtolength{\leftmargin}{\tmplength}
\settowidth{\labelsep}{X}
\addtolength{\leftmargin}{\labelsep}
\setlength{\labelwidth}{\tmplength}
}
\item[\textbf{Declaration}\hfill]
\ifpdf
\begin{flushleft}
\fi
\begin{ttfamily}
public procedure FormClose(Sender: TObject; var CloseAction: TCloseAction);\end{ttfamily}

\ifpdf
\end{flushleft}
\fi

\end{list}
\paragraph*{FormCreate}\hspace*{\fill}

\label{swcatalog.TSCForm-FormCreate}
\index{FormCreate}
\begin{list}{}{
\settowidth{\tmplength}{\textbf{Description}}
\setlength{\itemindent}{0cm}
\setlength{\listparindent}{0cm}
\setlength{\leftmargin}{\evensidemargin}
\addtolength{\leftmargin}{\tmplength}
\settowidth{\labelsep}{X}
\addtolength{\leftmargin}{\labelsep}
\setlength{\labelwidth}{\tmplength}
}
\item[\textbf{Declaration}\hfill]
\ifpdf
\begin{flushleft}
\fi
\begin{ttfamily}
public procedure FormCreate(Sender: TObject);\end{ttfamily}

\ifpdf
\end{flushleft}
\fi

\end{list}
\paragraph*{FormDestroy}\hspace*{\fill}

\label{swcatalog.TSCForm-FormDestroy}
\index{FormDestroy}
\begin{list}{}{
\settowidth{\tmplength}{\textbf{Description}}
\setlength{\itemindent}{0cm}
\setlength{\listparindent}{0cm}
\setlength{\leftmargin}{\evensidemargin}
\addtolength{\leftmargin}{\tmplength}
\settowidth{\labelsep}{X}
\addtolength{\leftmargin}{\labelsep}
\setlength{\labelwidth}{\tmplength}
}
\item[\textbf{Declaration}\hfill]
\ifpdf
\begin{flushleft}
\fi
\begin{ttfamily}
public procedure FormDestroy(Sender: TObject);\end{ttfamily}

\ifpdf
\end{flushleft}
\fi

\end{list}
\paragraph*{FormResize}\hspace*{\fill}

\label{swcatalog.TSCForm-FormResize}
\index{FormResize}
\begin{list}{}{
\settowidth{\tmplength}{\textbf{Description}}
\setlength{\itemindent}{0cm}
\setlength{\listparindent}{0cm}
\setlength{\leftmargin}{\evensidemargin}
\addtolength{\leftmargin}{\tmplength}
\settowidth{\labelsep}{X}
\addtolength{\leftmargin}{\labelsep}
\setlength{\labelwidth}{\tmplength}
}
\item[\textbf{Declaration}\hfill]
\ifpdf
\begin{flushleft}
\fi
\begin{ttfamily}
public procedure FormResize(Sender: TObject);\end{ttfamily}

\ifpdf
\end{flushleft}
\fi

\end{list}
\paragraph*{HQBoxChange}\hspace*{\fill}

\label{swcatalog.TSCForm-HQBoxChange}
\index{HQBoxChange}
\begin{list}{}{
\settowidth{\tmplength}{\textbf{Description}}
\setlength{\itemindent}{0cm}
\setlength{\listparindent}{0cm}
\setlength{\leftmargin}{\evensidemargin}
\addtolength{\leftmargin}{\tmplength}
\settowidth{\labelsep}{X}
\addtolength{\leftmargin}{\labelsep}
\setlength{\labelwidth}{\tmplength}
}
\item[\textbf{Declaration}\hfill]
\ifpdf
\begin{flushleft}
\fi
\begin{ttfamily}
public procedure HQBoxChange(Sender: TObject);\end{ttfamily}

\ifpdf
\end{flushleft}
\fi

\end{list}
\ifpdf
\subsection*{\large{\textbf{TCTLEntry Class}}\normalsize\hspace{1ex}\hrulefill}
\else
\subsection*{TCTLEntry Class}
\fi
\label{swcatalog.TCTLEntry}
\index{TCTLEntry}
\subsubsection*{\large{\textbf{Hierarchy}}\normalsize\hspace{1ex}\hfill}
TCTLEntry {$>$} \begin{ttfamily}TListEntry\end{ttfamily}(\ref{utilities.TListEntry}) {$>$} 
TPanel
\subsubsection*{\large{\textbf{Description}}\normalsize\hspace{1ex}\hfill}
no description available, TListEntry description followsOne entry of Listaller's visual software lists\subsubsection*{\large{\textbf{Fields}}\normalsize\hspace{1ex}\hfill}
\begin{list}{}{
\settowidth{\tmplength}{\textbf{enr}}
\setlength{\itemindent}{0cm}
\setlength{\listparindent}{0cm}
\setlength{\leftmargin}{\evensidemargin}
\addtolength{\leftmargin}{\tmplength}
\settowidth{\labelsep}{X}
\addtolength{\leftmargin}{\labelsep}
\setlength{\labelwidth}{\tmplength}
}
\label{swcatalog.TCTLEntry-enr}
\index{enr}
\item[\textbf{enr}\hfill]
\ifpdf
\begin{flushleft}
\fi
\begin{ttfamily}
public enr: Integer;\end{ttfamily}

\ifpdf
\end{flushleft}
\fi


\par  \end{list}
\subsubsection*{\large{\textbf{Methods}}\normalsize\hspace{1ex}\hfill}
\paragraph*{Create}\hspace*{\fill}

\label{swcatalog.TCTLEntry-Create}
\index{Create}
\begin{list}{}{
\settowidth{\tmplength}{\textbf{Description}}
\setlength{\itemindent}{0cm}
\setlength{\listparindent}{0cm}
\setlength{\leftmargin}{\evensidemargin}
\addtolength{\leftmargin}{\tmplength}
\settowidth{\labelsep}{X}
\addtolength{\leftmargin}{\labelsep}
\setlength{\labelwidth}{\tmplength}
}
\item[\textbf{Declaration}\hfill]
\ifpdf
\begin{flushleft}
\fi
\begin{ttfamily}
public constructor Create(AOwner: TComponent); override;\end{ttfamily}

\ifpdf
\end{flushleft}
\fi

\end{list}
\paragraph*{Destroy}\hspace*{\fill}

\label{swcatalog.TCTLEntry-Destroy}
\index{Destroy}
\begin{list}{}{
\settowidth{\tmplength}{\textbf{Description}}
\setlength{\itemindent}{0cm}
\setlength{\listparindent}{0cm}
\setlength{\leftmargin}{\evensidemargin}
\addtolength{\leftmargin}{\tmplength}
\settowidth{\labelsep}{X}
\addtolength{\leftmargin}{\labelsep}
\setlength{\labelwidth}{\tmplength}
}
\item[\textbf{Declaration}\hfill]
\ifpdf
\begin{flushleft}
\fi
\begin{ttfamily}
public destructor Destroy; override;\end{ttfamily}

\ifpdf
\end{flushleft}
\fi

\end{list}
\paragraph*{SetPositions}\hspace*{\fill}

\label{swcatalog.TCTLEntry-SetPositions}
\index{SetPositions}
\begin{list}{}{
\settowidth{\tmplength}{\textbf{Description}}
\setlength{\itemindent}{0cm}
\setlength{\listparindent}{0cm}
\setlength{\leftmargin}{\evensidemargin}
\addtolength{\leftmargin}{\tmplength}
\settowidth{\labelsep}{X}
\addtolength{\leftmargin}{\labelsep}
\setlength{\labelwidth}{\tmplength}
}
\item[\textbf{Declaration}\hfill]
\ifpdf
\begin{flushleft}
\fi
\begin{ttfamily}
public procedure SetPositions;\end{ttfamily}

\ifpdf
\end{flushleft}
\fi

\end{list}
\section{Constants}
\ifpdf
\subsection*{\large{\textbf{CatalogPath}}\normalsize\hspace{1ex}\hrulefill}
\else
\subsection*{CatalogPath}
\fi
\label{swcatalog-CatalogPath}
\index{CatalogPath}
\begin{list}{}{
\settowidth{\tmplength}{\textbf{Description}}
\setlength{\itemindent}{0cm}
\setlength{\listparindent}{0cm}
\setlength{\leftmargin}{\evensidemargin}
\addtolength{\leftmargin}{\tmplength}
\settowidth{\labelsep}{X}
\addtolength{\leftmargin}{\labelsep}
\setlength{\labelwidth}{\tmplength}
}
\item[\textbf{Declaration}\hfill]
\ifpdf
\begin{flushleft}
\fi
\begin{ttfamily}
CatalogPath='http://listaller.nlinux.org/repo/catalogue/';\end{ttfamily}

\ifpdf
\end{flushleft}
\fi

\end{list}
\section{Variables}
\ifpdf
\subsection*{\large{\textbf{SCForm}}\normalsize\hspace{1ex}\hrulefill}
\else
\subsection*{SCForm}
\fi
\label{swcatalog-SCForm}
\index{SCForm}
\begin{list}{}{
\settowidth{\tmplength}{\textbf{Description}}
\setlength{\itemindent}{0cm}
\setlength{\listparindent}{0cm}
\setlength{\leftmargin}{\evensidemargin}
\addtolength{\leftmargin}{\tmplength}
\settowidth{\labelsep}{X}
\addtolength{\leftmargin}{\labelsep}
\setlength{\labelwidth}{\tmplength}
}
\item[\textbf{Declaration}\hfill]
\ifpdf
\begin{flushleft}
\fi
\begin{ttfamily}
SCForm: TSCForm;\end{ttfamily}

\ifpdf
\end{flushleft}
\fi

\end{list}
\ifpdf
\subsection*{\large{\textbf{SWList}}\normalsize\hspace{1ex}\hrulefill}
\else
\subsection*{SWList}
\fi
\label{swcatalog-SWList}
\index{SWList}
\begin{list}{}{
\settowidth{\tmplength}{\textbf{Description}}
\setlength{\itemindent}{0cm}
\setlength{\listparindent}{0cm}
\setlength{\leftmargin}{\evensidemargin}
\addtolength{\leftmargin}{\tmplength}
\settowidth{\labelsep}{X}
\addtolength{\leftmargin}{\labelsep}
\setlength{\labelwidth}{\tmplength}
}
\item[\textbf{Declaration}\hfill]
\ifpdf
\begin{flushleft}
\fi
\begin{ttfamily}
SWList: Array of TCTLEntry;\end{ttfamily}

\ifpdf
\end{flushleft}
\fi

\end{list}
\ifpdf
\subsection*{\large{\textbf{fActiv}}\normalsize\hspace{1ex}\hrulefill}
\else
\subsection*{fActiv}
\fi
\label{swcatalog-fActiv}
\index{fActiv}
\begin{list}{}{
\settowidth{\tmplength}{\textbf{Description}}
\setlength{\itemindent}{0cm}
\setlength{\listparindent}{0cm}
\setlength{\leftmargin}{\evensidemargin}
\addtolength{\leftmargin}{\tmplength}
\settowidth{\labelsep}{X}
\addtolength{\leftmargin}{\labelsep}
\setlength{\labelwidth}{\tmplength}
}
\item[\textbf{Declaration}\hfill]
\ifpdf
\begin{flushleft}
\fi
\begin{ttfamily}
fActiv: Boolean=true;\end{ttfamily}

\ifpdf
\end{flushleft}
\fi

\end{list}
\chapter{Unit thinstall}
\label{thinstall}
\index{thinstall}
\section{uses}
\begin{itemize}
\item \begin{ttfamily}Classes\end{ttfamily}\item \begin{ttfamily}SysUtils\end{ttfamily}\item \begin{ttfamily}FileUtil\end{ttfamily}\item \begin{ttfamily}LResources\end{ttfamily}\item \begin{ttfamily}Forms\end{ttfamily}\item \begin{ttfamily}Controls\end{ttfamily}\item \begin{ttfamily}Graphics\end{ttfamily}\item \begin{ttfamily}Dialogs\end{ttfamily}\item \begin{ttfamily}utilities\end{ttfamily}(\ref{utilities})\item \begin{ttfamily}Menus\end{ttfamily}\item \begin{ttfamily}ComCtrls\end{ttfamily}\item \begin{ttfamily}Buttons\end{ttfamily}\item \begin{ttfamily}distri\end{ttfamily}(\ref{distri})\item \begin{ttfamily}ExtCtrls\end{ttfamily}\item \begin{ttfamily}IniFiles\end{ttfamily}\item \begin{ttfamily}xtypefm\end{ttfamily}(\ref{xtypefm})\item \begin{ttfamily}lctheme\end{ttfamily}\item \begin{ttfamily}StdCtrls\end{ttfamily}\item \begin{ttfamily}LCLType\end{ttfamily}\item \begin{ttfamily}Process\end{ttfamily}\end{itemize}
\section{Overview}
\begin{description}
\item[\texttt{\begin{ttfamily}TisFrm\end{ttfamily} Class}]
\end{description}
\section{Classes, Interfaces, Objects and Records}
\ifpdf
\subsection*{\large{\textbf{TisFrm Class}}\normalsize\hspace{1ex}\hrulefill}
\else
\subsection*{TisFrm Class}
\fi
\label{thinstall.TisFrm}
\index{TisFrm}
\subsubsection*{\large{\textbf{Hierarchy}}\normalsize\hspace{1ex}\hfill}
TisFrm {$>$} TForm
%%%%Description
\subsubsection*{\large{\textbf{Fields}}\normalsize\hspace{1ex}\hfill}
\begin{list}{}{
\settowidth{\tmplength}{\textbf{PageControl1}}
\setlength{\itemindent}{0cm}
\setlength{\listparindent}{0cm}
\setlength{\leftmargin}{\evensidemargin}
\addtolength{\leftmargin}{\tmplength}
\settowidth{\labelsep}{X}
\addtolength{\leftmargin}{\labelsep}
\setlength{\labelwidth}{\tmplength}
}
\label{thinstall.TisFrm-headImage}
\index{headImage}
\item[\textbf{headImage}\hfill]
\ifpdf
\begin{flushleft}
\fi
\begin{ttfamily}
public headImage: TImage;\end{ttfamily}

\ifpdf
\end{flushleft}
\fi


\par  \label{thinstall.TisFrm-ImageList1}
\index{ImageList1}
\item[\textbf{ImageList1}\hfill]
\ifpdf
\begin{flushleft}
\fi
\begin{ttfamily}
public ImageList1: TImageList;\end{ttfamily}

\ifpdf
\end{flushleft}
\fi


\par  \label{thinstall.TisFrm-Label1}
\index{Label1}
\item[\textbf{Label1}\hfill]
\ifpdf
\begin{flushleft}
\fi
\begin{ttfamily}
public Label1: TLabel;\end{ttfamily}

\ifpdf
\end{flushleft}
\fi


\par  \label{thinstall.TisFrm-Label2}
\index{Label2}
\item[\textbf{Label2}\hfill]
\ifpdf
\begin{flushleft}
\fi
\begin{ttfamily}
public Label2: TLabel;\end{ttfamily}

\ifpdf
\end{flushleft}
\fi


\par  \label{thinstall.TisFrm-Label3}
\index{Label3}
\item[\textbf{Label3}\hfill]
\ifpdf
\begin{flushleft}
\fi
\begin{ttfamily}
public Label3: TLabel;\end{ttfamily}

\ifpdf
\end{flushleft}
\fi


\par  \label{thinstall.TisFrm-Label4}
\index{Label4}
\item[\textbf{Label4}\hfill]
\ifpdf
\begin{flushleft}
\fi
\begin{ttfamily}
public Label4: TLabel;\end{ttfamily}

\ifpdf
\end{flushleft}
\fi


\par  \label{thinstall.TisFrm-MainMenu1}
\index{MainMenu1}
\item[\textbf{MainMenu1}\hfill]
\ifpdf
\begin{flushleft}
\fi
\begin{ttfamily}
public MainMenu1: TMainMenu;\end{ttfamily}

\ifpdf
\end{flushleft}
\fi


\par  \label{thinstall.TisFrm-MenuItem1}
\index{MenuItem1}
\item[\textbf{MenuItem1}\hfill]
\ifpdf
\begin{flushleft}
\fi
\begin{ttfamily}
public MenuItem1: TMenuItem;\end{ttfamily}

\ifpdf
\end{flushleft}
\fi


\par  \label{thinstall.TisFrm-MenuItem2}
\index{MenuItem2}
\item[\textbf{MenuItem2}\hfill]
\ifpdf
\begin{flushleft}
\fi
\begin{ttfamily}
public MenuItem2: TMenuItem;\end{ttfamily}

\ifpdf
\end{flushleft}
\fi


\par  \label{thinstall.TisFrm-HelpMItem}
\index{HelpMItem}
\item[\textbf{HelpMItem}\hfill]
\ifpdf
\begin{flushleft}
\fi
\begin{ttfamily}
public HelpMItem: TMenuItem;\end{ttfamily}

\ifpdf
\end{flushleft}
\fi


\par  \label{thinstall.TisFrm-PageControl1}
\index{PageControl1}
\item[\textbf{PageControl1}\hfill]
\ifpdf
\begin{flushleft}
\fi
\begin{ttfamily}
public PageControl1: TPageControl;\end{ttfamily}

\ifpdf
\end{flushleft}
\fi


\par  \label{thinstall.TisFrm-Designs}
\index{Designs}
\item[\textbf{Designs}\hfill]
\ifpdf
\begin{flushleft}
\fi
\begin{ttfamily}
public Designs: TTabSheet;\end{ttfamily}

\ifpdf
\end{flushleft}
\fi


\par  \label{thinstall.TisFrm-Iconthemes}
\index{Iconthemes}
\item[\textbf{Iconthemes}\hfill]
\ifpdf
\begin{flushleft}
\fi
\begin{ttfamily}
public Iconthemes: TTabSheet;\end{ttfamily}

\ifpdf
\end{flushleft}
\fi


\par  \label{thinstall.TisFrm-ScrollBox1}
\index{ScrollBox1}
\item[\textbf{ScrollBox1}\hfill]
\ifpdf
\begin{flushleft}
\fi
\begin{ttfamily}
public ScrollBox1: TScrollBox;\end{ttfamily}

\ifpdf
\end{flushleft}
\fi


\par  \label{thinstall.TisFrm-ScrollBox2}
\index{ScrollBox2}
\item[\textbf{ScrollBox2}\hfill]
\ifpdf
\begin{flushleft}
\fi
\begin{ttfamily}
public ScrollBox2: TScrollBox;\end{ttfamily}

\ifpdf
\end{flushleft}
\fi


\par  \label{thinstall.TisFrm-ScrollBox3}
\index{ScrollBox3}
\item[\textbf{ScrollBox3}\hfill]
\ifpdf
\begin{flushleft}
\fi
\begin{ttfamily}
public ScrollBox3: TScrollBox;\end{ttfamily}

\ifpdf
\end{flushleft}
\fi


\par  \label{thinstall.TisFrm-ScrollBox4}
\index{ScrollBox4}
\item[\textbf{ScrollBox4}\hfill]
\ifpdf
\begin{flushleft}
\fi
\begin{ttfamily}
public ScrollBox4: TScrollBox;\end{ttfamily}

\ifpdf
\end{flushleft}
\fi


\par  \label{thinstall.TisFrm-Screensavers}
\index{Screensavers}
\item[\textbf{Screensavers}\hfill]
\ifpdf
\begin{flushleft}
\fi
\begin{ttfamily}
public Screensavers: TTabSheet;\end{ttfamily}

\ifpdf
\end{flushleft}
\fi


\par  \label{thinstall.TisFrm-Wallpapers}
\index{Wallpapers}
\item[\textbf{Wallpapers}\hfill]
\ifpdf
\begin{flushleft}
\fi
\begin{ttfamily}
public Wallpapers: TTabSheet;\end{ttfamily}

\ifpdf
\end{flushleft}
\fi


\par  \end{list}
\subsubsection*{\large{\textbf{Methods}}\normalsize\hspace{1ex}\hfill}
\paragraph*{FormActivate}\hspace*{\fill}

\label{thinstall.TisFrm-FormActivate}
\index{FormActivate}
\begin{list}{}{
\settowidth{\tmplength}{\textbf{Description}}
\setlength{\itemindent}{0cm}
\setlength{\listparindent}{0cm}
\setlength{\leftmargin}{\evensidemargin}
\addtolength{\leftmargin}{\tmplength}
\settowidth{\labelsep}{X}
\addtolength{\leftmargin}{\labelsep}
\setlength{\labelwidth}{\tmplength}
}
\item[\textbf{Declaration}\hfill]
\ifpdf
\begin{flushleft}
\fi
\begin{ttfamily}
public procedure FormActivate(Sender: TObject);\end{ttfamily}

\ifpdf
\end{flushleft}
\fi

\end{list}
\paragraph*{FormCreate}\hspace*{\fill}

\label{thinstall.TisFrm-FormCreate}
\index{FormCreate}
\begin{list}{}{
\settowidth{\tmplength}{\textbf{Description}}
\setlength{\itemindent}{0cm}
\setlength{\listparindent}{0cm}
\setlength{\leftmargin}{\evensidemargin}
\addtolength{\leftmargin}{\tmplength}
\settowidth{\labelsep}{X}
\addtolength{\leftmargin}{\labelsep}
\setlength{\labelwidth}{\tmplength}
}
\item[\textbf{Declaration}\hfill]
\ifpdf
\begin{flushleft}
\fi
\begin{ttfamily}
public procedure FormCreate(Sender: TObject);\end{ttfamily}

\ifpdf
\end{flushleft}
\fi

\end{list}
\paragraph*{FormDestroy}\hspace*{\fill}

\label{thinstall.TisFrm-FormDestroy}
\index{FormDestroy}
\begin{list}{}{
\settowidth{\tmplength}{\textbf{Description}}
\setlength{\itemindent}{0cm}
\setlength{\listparindent}{0cm}
\setlength{\leftmargin}{\evensidemargin}
\addtolength{\leftmargin}{\tmplength}
\settowidth{\labelsep}{X}
\addtolength{\leftmargin}{\labelsep}
\setlength{\labelwidth}{\tmplength}
}
\item[\textbf{Declaration}\hfill]
\ifpdf
\begin{flushleft}
\fi
\begin{ttfamily}
public procedure FormDestroy(Sender: TObject);\end{ttfamily}

\ifpdf
\end{flushleft}
\fi

\end{list}
\paragraph*{FormResize}\hspace*{\fill}

\label{thinstall.TisFrm-FormResize}
\index{FormResize}
\begin{list}{}{
\settowidth{\tmplength}{\textbf{Description}}
\setlength{\itemindent}{0cm}
\setlength{\listparindent}{0cm}
\setlength{\leftmargin}{\evensidemargin}
\addtolength{\leftmargin}{\tmplength}
\settowidth{\labelsep}{X}
\addtolength{\leftmargin}{\labelsep}
\setlength{\labelwidth}{\tmplength}
}
\item[\textbf{Declaration}\hfill]
\ifpdf
\begin{flushleft}
\fi
\begin{ttfamily}
public procedure FormResize(Sender: TObject);\end{ttfamily}

\ifpdf
\end{flushleft}
\fi

\end{list}
\paragraph*{IconthemesShow}\hspace*{\fill}

\label{thinstall.TisFrm-IconthemesShow}
\index{IconthemesShow}
\begin{list}{}{
\settowidth{\tmplength}{\textbf{Description}}
\setlength{\itemindent}{0cm}
\setlength{\listparindent}{0cm}
\setlength{\leftmargin}{\evensidemargin}
\addtolength{\leftmargin}{\tmplength}
\settowidth{\labelsep}{X}
\addtolength{\leftmargin}{\labelsep}
\setlength{\labelwidth}{\tmplength}
}
\item[\textbf{Declaration}\hfill]
\ifpdf
\begin{flushleft}
\fi
\begin{ttfamily}
public procedure IconthemesShow(Sender: TObject);\end{ttfamily}

\ifpdf
\end{flushleft}
\fi

\end{list}
\paragraph*{PageControl1Change}\hspace*{\fill}

\label{thinstall.TisFrm-PageControl1Change}
\index{PageControl1Change}
\begin{list}{}{
\settowidth{\tmplength}{\textbf{Description}}
\setlength{\itemindent}{0cm}
\setlength{\listparindent}{0cm}
\setlength{\leftmargin}{\evensidemargin}
\addtolength{\leftmargin}{\tmplength}
\settowidth{\labelsep}{X}
\addtolength{\leftmargin}{\labelsep}
\setlength{\labelwidth}{\tmplength}
}
\item[\textbf{Declaration}\hfill]
\ifpdf
\begin{flushleft}
\fi
\begin{ttfamily}
public procedure PageControl1Change(Sender: TObject);\end{ttfamily}

\ifpdf
\end{flushleft}
\fi

\end{list}
\paragraph*{ScreensaversShow}\hspace*{\fill}

\label{thinstall.TisFrm-ScreensaversShow}
\index{ScreensaversShow}
\begin{list}{}{
\settowidth{\tmplength}{\textbf{Description}}
\setlength{\itemindent}{0cm}
\setlength{\listparindent}{0cm}
\setlength{\leftmargin}{\evensidemargin}
\addtolength{\leftmargin}{\tmplength}
\settowidth{\labelsep}{X}
\addtolength{\leftmargin}{\labelsep}
\setlength{\labelwidth}{\tmplength}
}
\item[\textbf{Declaration}\hfill]
\ifpdf
\begin{flushleft}
\fi
\begin{ttfamily}
public procedure ScreensaversShow(Sender: TObject);\end{ttfamily}

\ifpdf
\end{flushleft}
\fi

\end{list}
\paragraph*{WallpapersShow}\hspace*{\fill}

\label{thinstall.TisFrm-WallpapersShow}
\index{WallpapersShow}
\begin{list}{}{
\settowidth{\tmplength}{\textbf{Description}}
\setlength{\itemindent}{0cm}
\setlength{\listparindent}{0cm}
\setlength{\leftmargin}{\evensidemargin}
\addtolength{\leftmargin}{\tmplength}
\settowidth{\labelsep}{X}
\addtolength{\leftmargin}{\labelsep}
\setlength{\labelwidth}{\tmplength}
}
\item[\textbf{Declaration}\hfill]
\ifpdf
\begin{flushleft}
\fi
\begin{ttfamily}
public procedure WallpapersShow(Sender: TObject);\end{ttfamily}

\ifpdf
\end{flushleft}
\fi

\end{list}
\section{Variables}
\ifpdf
\subsection*{\large{\textbf{isFrm}}\normalsize\hspace{1ex}\hrulefill}
\else
\subsection*{isFrm}
\fi
\label{thinstall-isFrm}
\index{isFrm}
\begin{list}{}{
\settowidth{\tmplength}{\textbf{Description}}
\setlength{\itemindent}{0cm}
\setlength{\listparindent}{0cm}
\setlength{\leftmargin}{\evensidemargin}
\addtolength{\leftmargin}{\tmplength}
\settowidth{\labelsep}{X}
\addtolength{\leftmargin}{\labelsep}
\setlength{\labelwidth}{\tmplength}
}
\item[\textbf{Declaration}\hfill]
\ifpdf
\begin{flushleft}
\fi
\begin{ttfamily}
isFrm: TisFrm;\end{ttfamily}

\ifpdf
\end{flushleft}
\fi

\end{list}
\chapter{Unit trstrings}
\label{trstrings}
\index{trstrings}
\section{Description}
This unit contains the default strings for translation
\section{Constants}
\ifpdf
\subsection*{\large{\textbf{InstAborted}}\normalsize\hspace{1ex}\hrulefill}
\else
\subsection*{InstAborted}
\fi
\label{trstrings-InstAborted}
\index{InstAborted}
\begin{list}{}{
\settowidth{\tmplength}{\textbf{Description}}
\setlength{\itemindent}{0cm}
\setlength{\listparindent}{0cm}
\setlength{\leftmargin}{\evensidemargin}
\addtolength{\leftmargin}{\tmplength}
\settowidth{\labelsep}{X}
\addtolength{\leftmargin}{\labelsep}
\setlength{\labelwidth}{\tmplength}
}
\item[\textbf{Declaration}\hfill]
\ifpdf
\begin{flushleft}
\fi
\begin{ttfamily}
InstAborted='Installation aborted.';\end{ttfamily}

\ifpdf
\end{flushleft}
\fi

\end{list}
\ifpdf
\subsection*{\large{\textbf{strAppNInstall}}\normalsize\hspace{1ex}\hrulefill}
\else
\subsection*{strAppNInstall}
\fi
\label{trstrings-strAppNInstall}
\index{strAppNInstall}
\begin{list}{}{
\settowidth{\tmplength}{\textbf{Description}}
\setlength{\itemindent}{0cm}
\setlength{\listparindent}{0cm}
\setlength{\leftmargin}{\evensidemargin}
\addtolength{\leftmargin}{\tmplength}
\settowidth{\labelsep}{X}
\addtolength{\leftmargin}{\labelsep}
\setlength{\labelwidth}{\tmplength}
}
\item[\textbf{Declaration}\hfill]
\ifpdf
\begin{flushleft}
\fi
\begin{ttfamily}
strAppNInstall='The application {\%}a was not installed.';\end{ttfamily}

\ifpdf
\end{flushleft}
\fi

\end{list}
\ifpdf
\subsection*{\large{\textbf{strWasInstalled}}\normalsize\hspace{1ex}\hrulefill}
\else
\subsection*{strWasInstalled}
\fi
\label{trstrings-strWasInstalled}
\index{strWasInstalled}
\begin{list}{}{
\settowidth{\tmplength}{\textbf{Description}}
\setlength{\itemindent}{0cm}
\setlength{\listparindent}{0cm}
\setlength{\leftmargin}{\evensidemargin}
\addtolength{\leftmargin}{\tmplength}
\settowidth{\labelsep}{X}
\addtolength{\leftmargin}{\labelsep}
\setlength{\labelwidth}{\tmplength}
}
\item[\textbf{Declaration}\hfill]
\ifpdf
\begin{flushleft}
\fi
\begin{ttfamily}
strWasInstalled='The application {\%}a was installed successfully!';\end{ttfamily}

\ifpdf
\end{flushleft}
\fi

\end{list}
\ifpdf
\subsection*{\large{\textbf{strInstallNow}}\normalsize\hspace{1ex}\hrulefill}
\else
\subsection*{strInstallNow}
\fi
\label{trstrings-strInstallNow}
\index{strInstallNow}
\begin{list}{}{
\settowidth{\tmplength}{\textbf{Description}}
\setlength{\itemindent}{0cm}
\setlength{\listparindent}{0cm}
\setlength{\leftmargin}{\evensidemargin}
\addtolength{\leftmargin}{\tmplength}
\settowidth{\labelsep}{X}
\addtolength{\leftmargin}{\labelsep}
\setlength{\labelwidth}{\tmplength}
}
\item[\textbf{Declaration}\hfill]
\ifpdf
\begin{flushleft}
\fi
\begin{ttfamily}
strInstallNow='Install now';\end{ttfamily}

\ifpdf
\end{flushleft}
\fi

\end{list}
\ifpdf
\subsection*{\large{\textbf{RunParam}}\normalsize\hspace{1ex}\hrulefill}
\else
\subsection*{RunParam}
\fi
\label{trstrings-RunParam}
\index{RunParam}
\begin{list}{}{
\settowidth{\tmplength}{\textbf{Description}}
\setlength{\itemindent}{0cm}
\setlength{\listparindent}{0cm}
\setlength{\leftmargin}{\evensidemargin}
\addtolength{\leftmargin}{\tmplength}
\settowidth{\labelsep}{X}
\addtolength{\leftmargin}{\labelsep}
\setlength{\labelwidth}{\tmplength}
}
\item[\textbf{Declaration}\hfill]
\ifpdf
\begin{flushleft}
\fi
\begin{ttfamily}
RunParam='Please run listallgo with path to install-package as first parameter!';\end{ttfamily}

\ifpdf
\end{flushleft}
\fi

\end{list}
\ifpdf
\subsection*{\large{\textbf{strWelcome}}\normalsize\hspace{1ex}\hrulefill}
\else
\subsection*{strWelcome}
\fi
\label{trstrings-strWelcome}
\index{strWelcome}
\begin{list}{}{
\settowidth{\tmplength}{\textbf{Description}}
\setlength{\itemindent}{0cm}
\setlength{\listparindent}{0cm}
\setlength{\leftmargin}{\evensidemargin}
\addtolength{\leftmargin}{\tmplength}
\settowidth{\labelsep}{X}
\addtolength{\leftmargin}{\labelsep}
\setlength{\labelwidth}{\tmplength}
}
\item[\textbf{Declaration}\hfill]
\ifpdf
\begin{flushleft}
\fi
\begin{ttfamily}
strWelcome='Welcome!';\end{ttfamily}

\ifpdf
\end{flushleft}
\fi

\end{list}
\ifpdf
\subsection*{\large{\textbf{nToStart}}\normalsize\hspace{1ex}\hrulefill}
\else
\subsection*{nToStart}
\fi
\label{trstrings-nToStart}
\index{nToStart}
\begin{list}{}{
\settowidth{\tmplength}{\textbf{Description}}
\setlength{\itemindent}{0cm}
\setlength{\listparindent}{0cm}
\setlength{\leftmargin}{\evensidemargin}
\addtolength{\leftmargin}{\tmplength}
\settowidth{\labelsep}{X}
\addtolength{\leftmargin}{\labelsep}
\setlength{\labelwidth}{\tmplength}
}
\item[\textbf{Declaration}\hfill]
\ifpdf
\begin{flushleft}
\fi
\begin{ttfamily}
nToStart='Press "Next" to start the installation!';\end{ttfamily}

\ifpdf
\end{flushleft}
\fi

\end{list}
\ifpdf
\subsection*{\large{\textbf{progDesc}}\normalsize\hspace{1ex}\hrulefill}
\else
\subsection*{progDesc}
\fi
\label{trstrings-progDesc}
\index{progDesc}
\begin{list}{}{
\settowidth{\tmplength}{\textbf{Description}}
\setlength{\itemindent}{0cm}
\setlength{\listparindent}{0cm}
\setlength{\leftmargin}{\evensidemargin}
\addtolength{\leftmargin}{\tmplength}
\settowidth{\labelsep}{X}
\addtolength{\leftmargin}{\labelsep}
\setlength{\labelwidth}{\tmplength}
}
\item[\textbf{Declaration}\hfill]
\ifpdf
\begin{flushleft}
\fi
\begin{ttfamily}
progDesc='Program description:';\end{ttfamily}

\ifpdf
\end{flushleft}
\fi

\end{list}
\ifpdf
\subsection*{\large{\textbf{License}}\normalsize\hspace{1ex}\hrulefill}
\else
\subsection*{License}
\fi
\label{trstrings-License}
\index{License}
\begin{list}{}{
\settowidth{\tmplength}{\textbf{Description}}
\setlength{\itemindent}{0cm}
\setlength{\listparindent}{0cm}
\setlength{\leftmargin}{\evensidemargin}
\addtolength{\leftmargin}{\tmplength}
\settowidth{\labelsep}{X}
\addtolength{\leftmargin}{\labelsep}
\setlength{\labelwidth}{\tmplength}
}
\item[\textbf{Declaration}\hfill]
\ifpdf
\begin{flushleft}
\fi
\begin{ttfamily}
License='Software license';\end{ttfamily}

\ifpdf
\end{flushleft}
\fi

\end{list}
\ifpdf
\subsection*{\large{\textbf{pleaseRead}}\normalsize\hspace{1ex}\hrulefill}
\else
\subsection*{pleaseRead}
\fi
\label{trstrings-pleaseRead}
\index{pleaseRead}
\begin{list}{}{
\settowidth{\tmplength}{\textbf{Description}}
\setlength{\itemindent}{0cm}
\setlength{\listparindent}{0cm}
\setlength{\leftmargin}{\evensidemargin}
\addtolength{\leftmargin}{\tmplength}
\settowidth{\labelsep}{X}
\addtolength{\leftmargin}{\labelsep}
\setlength{\labelwidth}{\tmplength}
}
\item[\textbf{Declaration}\hfill]
\ifpdf
\begin{flushleft}
\fi
\begin{ttfamily}
pleaseRead='Please read the following information carefully:';\end{ttfamily}

\ifpdf
\end{flushleft}
\fi

\end{list}
\ifpdf
\subsection*{\large{\textbf{running}}\normalsize\hspace{1ex}\hrulefill}
\else
\subsection*{running}
\fi
\label{trstrings-running}
\index{running}
\begin{list}{}{
\settowidth{\tmplength}{\textbf{Description}}
\setlength{\itemindent}{0cm}
\setlength{\listparindent}{0cm}
\setlength{\leftmargin}{\evensidemargin}
\addtolength{\leftmargin}{\tmplength}
\settowidth{\labelsep}{X}
\addtolength{\leftmargin}{\labelsep}
\setlength{\labelwidth}{\tmplength}
}
\item[\textbf{Declaration}\hfill]
\ifpdf
\begin{flushleft}
\fi
\begin{ttfamily}
running='Running installation...';\end{ttfamily}

\ifpdf
\end{flushleft}
\fi

\end{list}
\ifpdf
\subsection*{\large{\textbf{plWait}}\normalsize\hspace{1ex}\hrulefill}
\else
\subsection*{plWait}
\fi
\label{trstrings-plWait}
\index{plWait}
\begin{list}{}{
\settowidth{\tmplength}{\textbf{Description}}
\setlength{\itemindent}{0cm}
\setlength{\listparindent}{0cm}
\setlength{\leftmargin}{\evensidemargin}
\addtolength{\leftmargin}{\tmplength}
\settowidth{\labelsep}{X}
\addtolength{\leftmargin}{\labelsep}
\setlength{\labelwidth}{\tmplength}
}
\item[\textbf{Declaration}\hfill]
\ifpdf
\begin{flushleft}
\fi
\begin{ttfamily}
plWait='Please wait.';\end{ttfamily}

\ifpdf
\end{flushleft}
\fi

\end{list}
\ifpdf
\subsection*{\large{\textbf{strSWarning}}\normalsize\hspace{1ex}\hrulefill}
\else
\subsection*{strSWarning}
\fi
\label{trstrings-strSWarning}
\index{strSWarning}
\begin{list}{}{
\settowidth{\tmplength}{\textbf{Description}}
\setlength{\itemindent}{0cm}
\setlength{\listparindent}{0cm}
\setlength{\leftmargin}{\evensidemargin}
\addtolength{\leftmargin}{\tmplength}
\settowidth{\labelsep}{X}
\addtolength{\leftmargin}{\labelsep}
\setlength{\labelwidth}{\tmplength}
}
\item[\textbf{Declaration}\hfill]
\ifpdf
\begin{flushleft}
\fi
\begin{ttfamily}
strSWarning='Make sure that you can trust this package publisher!';\end{ttfamily}

\ifpdf
\end{flushleft}
\fi

\end{list}
\ifpdf
\subsection*{\large{\textbf{complete}}\normalsize\hspace{1ex}\hrulefill}
\else
\subsection*{complete}
\fi
\label{trstrings-complete}
\index{complete}
\begin{list}{}{
\settowidth{\tmplength}{\textbf{Description}}
\setlength{\itemindent}{0cm}
\setlength{\listparindent}{0cm}
\setlength{\leftmargin}{\evensidemargin}
\addtolength{\leftmargin}{\tmplength}
\settowidth{\labelsep}{X}
\addtolength{\leftmargin}{\labelsep}
\setlength{\labelwidth}{\tmplength}
}
\item[\textbf{Declaration}\hfill]
\ifpdf
\begin{flushleft}
\fi
\begin{ttfamily}
complete='Installation completed!';\end{ttfamily}

\ifpdf
\end{flushleft}
\fi

\end{list}
\ifpdf
\subsection*{\large{\textbf{prFinish}}\normalsize\hspace{1ex}\hrulefill}
\else
\subsection*{prFinish}
\fi
\label{trstrings-prFinish}
\index{prFinish}
\begin{list}{}{
\settowidth{\tmplength}{\textbf{Description}}
\setlength{\itemindent}{0cm}
\setlength{\listparindent}{0cm}
\setlength{\leftmargin}{\evensidemargin}
\addtolength{\leftmargin}{\tmplength}
\settowidth{\labelsep}{X}
\addtolength{\leftmargin}{\labelsep}
\setlength{\labelwidth}{\tmplength}
}
\item[\textbf{Declaration}\hfill]
\ifpdf
\begin{flushleft}
\fi
\begin{ttfamily}
prFinish='Press "Finish" to close.';\end{ttfamily}

\ifpdf
\end{flushleft}
\fi

\end{list}
\ifpdf
\subsection*{\large{\textbf{Finish}}\normalsize\hspace{1ex}\hrulefill}
\else
\subsection*{Finish}
\fi
\label{trstrings-Finish}
\index{Finish}
\begin{list}{}{
\settowidth{\tmplength}{\textbf{Description}}
\setlength{\itemindent}{0cm}
\setlength{\listparindent}{0cm}
\setlength{\leftmargin}{\evensidemargin}
\addtolength{\leftmargin}{\tmplength}
\settowidth{\labelsep}{X}
\addtolength{\leftmargin}{\labelsep}
\setlength{\labelwidth}{\tmplength}
}
\item[\textbf{Declaration}\hfill]
\ifpdf
\begin{flushleft}
\fi
\begin{ttfamily}
Finish='Finish';\end{ttfamily}

\ifpdf
\end{flushleft}
\fi

\end{list}
\ifpdf
\subsection*{\large{\textbf{strAbort}}\normalsize\hspace{1ex}\hrulefill}
\else
\subsection*{strAbort}
\fi
\label{trstrings-strAbort}
\index{strAbort}
\begin{list}{}{
\settowidth{\tmplength}{\textbf{Description}}
\setlength{\itemindent}{0cm}
\setlength{\listparindent}{0cm}
\setlength{\leftmargin}{\evensidemargin}
\addtolength{\leftmargin}{\tmplength}
\settowidth{\labelsep}{X}
\addtolength{\leftmargin}{\labelsep}
\setlength{\labelwidth}{\tmplength}
}
\item[\textbf{Declaration}\hfill]
\ifpdf
\begin{flushleft}
\fi
\begin{ttfamily}
strAbort='Abort';\end{ttfamily}

\ifpdf
\end{flushleft}
\fi

\end{list}
\ifpdf
\subsection*{\large{\textbf{strBack}}\normalsize\hspace{1ex}\hrulefill}
\else
\subsection*{strBack}
\fi
\label{trstrings-strBack}
\index{strBack}
\begin{list}{}{
\settowidth{\tmplength}{\textbf{Description}}
\setlength{\itemindent}{0cm}
\setlength{\listparindent}{0cm}
\setlength{\leftmargin}{\evensidemargin}
\addtolength{\leftmargin}{\tmplength}
\settowidth{\labelsep}{X}
\addtolength{\leftmargin}{\labelsep}
\setlength{\labelwidth}{\tmplength}
}
\item[\textbf{Declaration}\hfill]
\ifpdf
\begin{flushleft}
\fi
\begin{ttfamily}
strBack='Back';\end{ttfamily}

\ifpdf
\end{flushleft}
\fi

\end{list}
\ifpdf
\subsection*{\large{\textbf{strNext}}\normalsize\hspace{1ex}\hrulefill}
\else
\subsection*{strNext}
\fi
\label{trstrings-strNext}
\index{strNext}
\begin{list}{}{
\settowidth{\tmplength}{\textbf{Description}}
\setlength{\itemindent}{0cm}
\setlength{\listparindent}{0cm}
\setlength{\leftmargin}{\evensidemargin}
\addtolength{\leftmargin}{\tmplength}
\settowidth{\labelsep}{X}
\addtolength{\leftmargin}{\labelsep}
\setlength{\labelwidth}{\tmplength}
}
\item[\textbf{Declaration}\hfill]
\ifpdf
\begin{flushleft}
\fi
\begin{ttfamily}
strNext='Next';\end{ttfamily}

\ifpdf
\end{flushleft}
\fi

\end{list}
\ifpdf
\subsection*{\large{\textbf{strDispLog}}\normalsize\hspace{1ex}\hrulefill}
\else
\subsection*{strDispLog}
\fi
\label{trstrings-strDispLog}
\index{strDispLog}
\begin{list}{}{
\settowidth{\tmplength}{\textbf{Description}}
\setlength{\itemindent}{0cm}
\setlength{\listparindent}{0cm}
\setlength{\leftmargin}{\evensidemargin}
\addtolength{\leftmargin}{\tmplength}
\settowidth{\labelsep}{X}
\addtolength{\leftmargin}{\labelsep}
\setlength{\labelwidth}{\tmplength}
}
\item[\textbf{Declaration}\hfill]
\ifpdf
\begin{flushleft}
\fi
\begin{ttfamily}
strDispLog='Display installation log';\end{ttfamily}

\ifpdf
\end{flushleft}
\fi

\end{list}
\ifpdf
\subsection*{\large{\textbf{strIagree}}\normalsize\hspace{1ex}\hrulefill}
\else
\subsection*{strIagree}
\fi
\label{trstrings-strIagree}
\index{strIagree}
\begin{list}{}{
\settowidth{\tmplength}{\textbf{Description}}
\setlength{\itemindent}{0cm}
\setlength{\listparindent}{0cm}
\setlength{\leftmargin}{\evensidemargin}
\addtolength{\leftmargin}{\tmplength}
\settowidth{\labelsep}{X}
\addtolength{\leftmargin}{\labelsep}
\setlength{\labelwidth}{\tmplength}
}
\item[\textbf{Declaration}\hfill]
\ifpdf
\begin{flushleft}
\fi
\begin{ttfamily}
strIagree='I agree with the above terms and conditions';\end{ttfamily}

\ifpdf
\end{flushleft}
\fi

\end{list}
\ifpdf
\subsection*{\large{\textbf{strInagree}}\normalsize\hspace{1ex}\hrulefill}
\else
\subsection*{strInagree}
\fi
\label{trstrings-strInagree}
\index{strInagree}
\begin{list}{}{
\settowidth{\tmplength}{\textbf{Description}}
\setlength{\itemindent}{0cm}
\setlength{\listparindent}{0cm}
\setlength{\leftmargin}{\evensidemargin}
\addtolength{\leftmargin}{\tmplength}
\settowidth{\labelsep}{X}
\addtolength{\leftmargin}{\labelsep}
\setlength{\labelwidth}{\tmplength}
}
\item[\textbf{Declaration}\hfill]
\ifpdf
\begin{flushleft}
\fi
\begin{ttfamily}
strInagree='I don''t agree';\end{ttfamily}

\ifpdf
\end{flushleft}
\fi

\end{list}
\ifpdf
\subsection*{\large{\textbf{strLDnSupported}}\normalsize\hspace{1ex}\hrulefill}
\else
\subsection*{strLDnSupported}
\fi
\label{trstrings-strLDnSupported}
\index{strLDnSupported}
\begin{list}{}{
\settowidth{\tmplength}{\textbf{Description}}
\setlength{\itemindent}{0cm}
\setlength{\listparindent}{0cm}
\setlength{\leftmargin}{\evensidemargin}
\addtolength{\leftmargin}{\tmplength}
\settowidth{\labelsep}{X}
\addtolength{\leftmargin}{\labelsep}
\setlength{\labelwidth}{\tmplength}
}
\item[\textbf{Declaration}\hfill]
\ifpdf
\begin{flushleft}
\fi
\begin{ttfamily}
strLDnSupported='Your Linux distribution is not supported by Listaller yet!';\end{ttfamily}

\ifpdf
\end{flushleft}
\fi

\end{list}
\ifpdf
\subsection*{\large{\textbf{strnSupported}}\normalsize\hspace{1ex}\hrulefill}
\else
\subsection*{strnSupported}
\fi
\label{trstrings-strnSupported}
\index{strnSupported}
\begin{list}{}{
\settowidth{\tmplength}{\textbf{Description}}
\setlength{\itemindent}{0cm}
\setlength{\listparindent}{0cm}
\setlength{\leftmargin}{\evensidemargin}
\addtolength{\leftmargin}{\tmplength}
\settowidth{\labelsep}{X}
\addtolength{\leftmargin}{\labelsep}
\setlength{\labelwidth}{\tmplength}
}
\item[\textbf{Declaration}\hfill]
\ifpdf
\begin{flushleft}
\fi
\begin{ttfamily}
strnSupported='Your Linux distribution is not supported by this package!';\end{ttfamily}

\ifpdf
\end{flushleft}
\fi

\end{list}
\ifpdf
\subsection*{\large{\textbf{strInClose}}\normalsize\hspace{1ex}\hrulefill}
\else
\subsection*{strInClose}
\fi
\label{trstrings-strInClose}
\index{strInClose}
\begin{list}{}{
\settowidth{\tmplength}{\textbf{Description}}
\setlength{\itemindent}{0cm}
\setlength{\listparindent}{0cm}
\setlength{\leftmargin}{\evensidemargin}
\addtolength{\leftmargin}{\tmplength}
\settowidth{\labelsep}{X}
\addtolength{\leftmargin}{\labelsep}
\setlength{\labelwidth}{\tmplength}
}
\item[\textbf{Declaration}\hfill]
\ifpdf
\begin{flushleft}
\fi
\begin{ttfamily}
strInClose='The installer will close now';\end{ttfamily}

\ifpdf
\end{flushleft}
\fi

\end{list}
\ifpdf
\subsection*{\large{\textbf{notifyDevs}}\normalsize\hspace{1ex}\hrulefill}
\else
\subsection*{notifyDevs}
\fi
\label{trstrings-notifyDevs}
\index{notifyDevs}
\begin{list}{}{
\settowidth{\tmplength}{\textbf{Description}}
\setlength{\itemindent}{0cm}
\setlength{\listparindent}{0cm}
\setlength{\leftmargin}{\evensidemargin}
\addtolength{\leftmargin}{\tmplength}
\settowidth{\labelsep}{X}
\addtolength{\leftmargin}{\labelsep}
\setlength{\labelwidth}{\tmplength}
}
\item[\textbf{Declaration}\hfill]
\ifpdf
\begin{flushleft}
\fi
\begin{ttfamily}
notifyDevs='Please notify the developers on http://launchpad.net/listaller';\end{ttfamily}

\ifpdf
\end{flushleft}
\fi

\end{list}
\ifpdf
\subsection*{\large{\textbf{strExtractError}}\normalsize\hspace{1ex}\hrulefill}
\else
\subsection*{strExtractError}
\fi
\label{trstrings-strExtractError}
\index{strExtractError}
\begin{list}{}{
\settowidth{\tmplength}{\textbf{Description}}
\setlength{\itemindent}{0cm}
\setlength{\listparindent}{0cm}
\setlength{\leftmargin}{\evensidemargin}
\addtolength{\leftmargin}{\tmplength}
\settowidth{\labelsep}{X}
\addtolength{\leftmargin}{\labelsep}
\setlength{\labelwidth}{\tmplength}
}
\item[\textbf{Declaration}\hfill]
\ifpdf
\begin{flushleft}
\fi
\begin{ttfamily}
strExtractError='Error while extracting files!';\end{ttfamily}

\ifpdf
\end{flushleft}
\fi

\end{list}
\ifpdf
\subsection*{\large{\textbf{strPkgDM}}\normalsize\hspace{1ex}\hrulefill}
\else
\subsection*{strPkgDM}
\fi
\label{trstrings-strPkgDM}
\index{strPkgDM}
\begin{list}{}{
\settowidth{\tmplength}{\textbf{Description}}
\setlength{\itemindent}{0cm}
\setlength{\listparindent}{0cm}
\setlength{\leftmargin}{\evensidemargin}
\addtolength{\leftmargin}{\tmplength}
\settowidth{\labelsep}{X}
\addtolength{\leftmargin}{\labelsep}
\setlength{\labelwidth}{\tmplength}
}
\item[\textbf{Declaration}\hfill]
\ifpdf
\begin{flushleft}
\fi
\begin{ttfamily}
strPkgDM='The Listaller-package could be damaged or you haven''t enough rights for access some files';\end{ttfamily}

\ifpdf
\end{flushleft}
\fi

\end{list}
\ifpdf
\subsection*{\large{\textbf{strAbLoad}}\normalsize\hspace{1ex}\hrulefill}
\else
\subsection*{strAbLoad}
\fi
\label{trstrings-strAbLoad}
\index{strAbLoad}
\begin{list}{}{
\settowidth{\tmplength}{\textbf{Description}}
\setlength{\itemindent}{0cm}
\setlength{\listparindent}{0cm}
\setlength{\leftmargin}{\evensidemargin}
\addtolength{\leftmargin}{\tmplength}
\settowidth{\labelsep}{X}
\addtolength{\leftmargin}{\labelsep}
\setlength{\labelwidth}{\tmplength}
}
\item[\textbf{Declaration}\hfill]
\ifpdf
\begin{flushleft}
\fi
\begin{ttfamily}
strAbLoad='Loading aborted.';\end{ttfamily}

\ifpdf
\end{flushleft}
\fi

\end{list}
\ifpdf
\subsection*{\large{\textbf{strAlreadyinst}}\normalsize\hspace{1ex}\hrulefill}
\else
\subsection*{strAlreadyinst}
\fi
\label{trstrings-strAlreadyinst}
\index{strAlreadyinst}
\begin{list}{}{
\settowidth{\tmplength}{\textbf{Description}}
\setlength{\itemindent}{0cm}
\setlength{\listparindent}{0cm}
\setlength{\leftmargin}{\evensidemargin}
\addtolength{\leftmargin}{\tmplength}
\settowidth{\labelsep}{X}
\addtolength{\leftmargin}{\labelsep}
\setlength{\labelwidth}{\tmplength}
}
\item[\textbf{Declaration}\hfill]
\ifpdf
\begin{flushleft}
\fi
\begin{ttfamily}
strAlreadyinst='This application is already installed';\end{ttfamily}

\ifpdf
\end{flushleft}
\fi

\end{list}
\ifpdf
\subsection*{\large{\textbf{strInstallAgain}}\normalsize\hspace{1ex}\hrulefill}
\else
\subsection*{strInstallAgain}
\fi
\label{trstrings-strInstallAgain}
\index{strInstallAgain}
\begin{list}{}{
\settowidth{\tmplength}{\textbf{Description}}
\setlength{\itemindent}{0cm}
\setlength{\listparindent}{0cm}
\setlength{\leftmargin}{\evensidemargin}
\addtolength{\leftmargin}{\tmplength}
\settowidth{\labelsep}{X}
\addtolength{\leftmargin}{\labelsep}
\setlength{\labelwidth}{\tmplength}
}
\item[\textbf{Declaration}\hfill]
\ifpdf
\begin{flushleft}
\fi
\begin{ttfamily}
strInstallAgain='Do you want to install it again?';\end{ttfamily}

\ifpdf
\end{flushleft}
\fi

\end{list}
\ifpdf
\subsection*{\large{\textbf{strWelcomeTo}}\normalsize\hspace{1ex}\hrulefill}
\else
\subsection*{strWelcomeTo}
\fi
\label{trstrings-strWelcomeTo}
\index{strWelcomeTo}
\begin{list}{}{
\settowidth{\tmplength}{\textbf{Description}}
\setlength{\itemindent}{0cm}
\setlength{\listparindent}{0cm}
\setlength{\leftmargin}{\evensidemargin}
\addtolength{\leftmargin}{\tmplength}
\settowidth{\labelsep}{X}
\addtolength{\leftmargin}{\labelsep}
\setlength{\labelwidth}{\tmplength}
}
\item[\textbf{Declaration}\hfill]
\ifpdf
\begin{flushleft}
\fi
\begin{ttfamily}
strWelcomeTo='Welcome to the installation of {\%}a';\end{ttfamily}

\ifpdf
\end{flushleft}
\fi

\end{list}
\ifpdf
\subsection*{\large{\textbf{strInstOf}}\normalsize\hspace{1ex}\hrulefill}
\else
\subsection*{strInstOf}
\fi
\label{trstrings-strInstOf}
\index{strInstOf}
\begin{list}{}{
\settowidth{\tmplength}{\textbf{Description}}
\setlength{\itemindent}{0cm}
\setlength{\listparindent}{0cm}
\setlength{\leftmargin}{\evensidemargin}
\addtolength{\leftmargin}{\tmplength}
\settowidth{\labelsep}{X}
\addtolength{\leftmargin}{\labelsep}
\setlength{\labelwidth}{\tmplength}
}
\item[\textbf{Declaration}\hfill]
\ifpdf
\begin{flushleft}
\fi
\begin{ttfamily}
strInstOf='Installation of {\%}a';\end{ttfamily}

\ifpdf
\end{flushleft}
\fi

\end{list}
\ifpdf
\subsection*{\large{\textbf{strTestmode}}\normalsize\hspace{1ex}\hrulefill}
\else
\subsection*{strTestmode}
\fi
\label{trstrings-strTestmode}
\index{strTestmode}
\begin{list}{}{
\settowidth{\tmplength}{\textbf{Description}}
\setlength{\itemindent}{0cm}
\setlength{\listparindent}{0cm}
\setlength{\leftmargin}{\evensidemargin}
\addtolength{\leftmargin}{\tmplength}
\settowidth{\labelsep}{X}
\addtolength{\leftmargin}{\labelsep}
\setlength{\labelwidth}{\tmplength}
}
\item[\textbf{Declaration}\hfill]
\ifpdf
\begin{flushleft}
\fi
\begin{ttfamily}
strTestmode='Testmode';\end{ttfamily}

\ifpdf
\end{flushleft}
\fi

\end{list}
\ifpdf
\subsection*{\large{\textbf{strTestFinished}}\normalsize\hspace{1ex}\hrulefill}
\else
\subsection*{strTestFinished}
\fi
\label{trstrings-strTestFinished}
\index{strTestFinished}
\begin{list}{}{
\settowidth{\tmplength}{\textbf{Description}}
\setlength{\itemindent}{0cm}
\setlength{\listparindent}{0cm}
\setlength{\leftmargin}{\evensidemargin}
\addtolength{\leftmargin}{\tmplength}
\settowidth{\labelsep}{X}
\addtolength{\leftmargin}{\labelsep}
\setlength{\labelwidth}{\tmplength}
}
\item[\textbf{Declaration}\hfill]
\ifpdf
\begin{flushleft}
\fi
\begin{ttfamily}
strTestFinished='Test of the application finished.';\end{ttfamily}

\ifpdf
\end{flushleft}
\fi

\end{list}
\ifpdf
\subsection*{\large{\textbf{strpkgInval}}\normalsize\hspace{1ex}\hrulefill}
\else
\subsection*{strpkgInval}
\fi
\label{trstrings-strpkgInval}
\index{strpkgInval}
\begin{list}{}{
\settowidth{\tmplength}{\textbf{Description}}
\setlength{\itemindent}{0cm}
\setlength{\listparindent}{0cm}
\setlength{\leftmargin}{\evensidemargin}
\addtolength{\leftmargin}{\tmplength}
\settowidth{\labelsep}{X}
\addtolength{\leftmargin}{\labelsep}
\setlength{\labelwidth}{\tmplength}
}
\item[\textbf{Declaration}\hfill]
\ifpdf
\begin{flushleft}
\fi
\begin{ttfamily}
strpkgInval='The package was invalid!';\end{ttfamily}

\ifpdf
\end{flushleft}
\fi

\end{list}
\ifpdf
\subsection*{\large{\textbf{strCouldntSolve}}\normalsize\hspace{1ex}\hrulefill}
\else
\subsection*{strCouldntSolve}
\fi
\label{trstrings-strCouldntSolve}
\index{strCouldntSolve}
\begin{list}{}{
\settowidth{\tmplength}{\textbf{Description}}
\setlength{\itemindent}{0cm}
\setlength{\listparindent}{0cm}
\setlength{\leftmargin}{\evensidemargin}
\addtolength{\leftmargin}{\tmplength}
\settowidth{\labelsep}{X}
\addtolength{\leftmargin}{\labelsep}
\setlength{\labelwidth}{\tmplength}
}
\item[\textbf{Declaration}\hfill]
\ifpdf
\begin{flushleft}
\fi
\begin{ttfamily}
strCouldntSolve='Dependencies couldn''t be solved!';\end{ttfamily}

\ifpdf
\end{flushleft}
\fi

\end{list}
\ifpdf
\subsection*{\large{\textbf{strViewLog}}\normalsize\hspace{1ex}\hrulefill}
\else
\subsection*{strViewLog}
\fi
\label{trstrings-strViewLog}
\index{strViewLog}
\begin{list}{}{
\settowidth{\tmplength}{\textbf{Description}}
\setlength{\itemindent}{0cm}
\setlength{\listparindent}{0cm}
\setlength{\leftmargin}{\evensidemargin}
\addtolength{\leftmargin}{\tmplength}
\settowidth{\labelsep}{X}
\addtolength{\leftmargin}{\labelsep}
\setlength{\labelwidth}{\tmplength}
}
\item[\textbf{Declaration}\hfill]
\ifpdf
\begin{flushleft}
\fi
\begin{ttfamily}
strViewLog='Please view the logfile at {\%}p';\end{ttfamily}

\ifpdf
\end{flushleft}
\fi

\end{list}
\ifpdf
\subsection*{\large{\textbf{strPKGError}}\normalsize\hspace{1ex}\hrulefill}
\else
\subsection*{strPKGError}
\fi
\label{trstrings-strPKGError}
\index{strPKGError}
\begin{list}{}{
\settowidth{\tmplength}{\textbf{Description}}
\setlength{\itemindent}{0cm}
\setlength{\listparindent}{0cm}
\setlength{\leftmargin}{\evensidemargin}
\addtolength{\leftmargin}{\tmplength}
\settowidth{\labelsep}{X}
\addtolength{\leftmargin}{\labelsep}
\setlength{\labelwidth}{\tmplength}
}
\item[\textbf{Declaration}\hfill]
\ifpdf
\begin{flushleft}
\fi
\begin{ttfamily}
strPKGError='Installation package is corrupt';\end{ttfamily}

\ifpdf
\end{flushleft}
\fi

\end{list}
\ifpdf
\subsection*{\large{\textbf{strAppClose}}\normalsize\hspace{1ex}\hrulefill}
\else
\subsection*{strAppClose}
\fi
\label{trstrings-strAppClose}
\index{strAppClose}
\begin{list}{}{
\settowidth{\tmplength}{\textbf{Description}}
\setlength{\itemindent}{0cm}
\setlength{\listparindent}{0cm}
\setlength{\leftmargin}{\evensidemargin}
\addtolength{\leftmargin}{\tmplength}
\settowidth{\labelsep}{X}
\addtolength{\leftmargin}{\labelsep}
\setlength{\labelwidth}{\tmplength}
}
\item[\textbf{Declaration}\hfill]
\ifpdf
\begin{flushleft}
\fi
\begin{ttfamily}
strAppClose='The application will close now ';\end{ttfamily}

\ifpdf
\end{flushleft}
\fi

\end{list}
\ifpdf
\subsection*{\large{\textbf{strStep1}}\normalsize\hspace{1ex}\hrulefill}
\else
\subsection*{strStep1}
\fi
\label{trstrings-strStep1}
\index{strStep1}
\begin{list}{}{
\settowidth{\tmplength}{\textbf{Description}}
\setlength{\itemindent}{0cm}
\setlength{\listparindent}{0cm}
\setlength{\leftmargin}{\evensidemargin}
\addtolength{\leftmargin}{\tmplength}
\settowidth{\labelsep}{X}
\addtolength{\leftmargin}{\labelsep}
\setlength{\labelwidth}{\tmplength}
}
\item[\textbf{Declaration}\hfill]
\ifpdf
\begin{flushleft}
\fi
\begin{ttfamily}
strStep1='Phase 1/4: Resolving dependencies...';\end{ttfamily}

\ifpdf
\end{flushleft}
\fi

\end{list}
\ifpdf
\subsection*{\large{\textbf{strWDLdep}}\normalsize\hspace{1ex}\hrulefill}
\else
\subsection*{strWDLdep}
\fi
\label{trstrings-strWDLdep}
\index{strWDLdep}
\begin{list}{}{
\settowidth{\tmplength}{\textbf{Description}}
\setlength{\itemindent}{0cm}
\setlength{\listparindent}{0cm}
\setlength{\leftmargin}{\evensidemargin}
\addtolength{\leftmargin}{\tmplength}
\settowidth{\labelsep}{X}
\addtolength{\leftmargin}{\labelsep}
\setlength{\labelwidth}{\tmplength}
}
\item[\textbf{Declaration}\hfill]
\ifpdf
\begin{flushleft}
\fi
\begin{ttfamily}
strWDLdep='This application wants to download {\&} install a dependency from {\%}l';\end{ttfamily}

\ifpdf
\end{flushleft}
\fi

\end{list}
\ifpdf
\subsection*{\large{\textbf{strwAllow}}\normalsize\hspace{1ex}\hrulefill}
\else
\subsection*{strwAllow}
\fi
\label{trstrings-strwAllow}
\index{strwAllow}
\begin{list}{}{
\settowidth{\tmplength}{\textbf{Description}}
\setlength{\itemindent}{0cm}
\setlength{\listparindent}{0cm}
\setlength{\leftmargin}{\evensidemargin}
\addtolength{\leftmargin}{\tmplength}
\settowidth{\labelsep}{X}
\addtolength{\leftmargin}{\labelsep}
\setlength{\labelwidth}{\tmplength}
}
\item[\textbf{Declaration}\hfill]
\ifpdf
\begin{flushleft}
\fi
\begin{ttfamily}
strwAllow='Do you want to allow this?';\end{ttfamily}

\ifpdf
\end{flushleft}
\fi

\end{list}
\ifpdf
\subsection*{\large{\textbf{strLiCloseANI}}\normalsize\hspace{1ex}\hrulefill}
\else
\subsection*{strLiCloseANI}
\fi
\label{trstrings-strLiCloseANI}
\index{strLiCloseANI}
\begin{list}{}{
\settowidth{\tmplength}{\textbf{Description}}
\setlength{\itemindent}{0cm}
\setlength{\listparindent}{0cm}
\setlength{\leftmargin}{\evensidemargin}
\addtolength{\leftmargin}{\tmplength}
\settowidth{\labelsep}{X}
\addtolength{\leftmargin}{\labelsep}
\setlength{\labelwidth}{\tmplength}
}
\item[\textbf{Declaration}\hfill]
\ifpdf
\begin{flushleft}
\fi
\begin{ttfamily}
strLiCloseANI='Listaller will close now. Application couldn''t be installed.';\end{ttfamily}

\ifpdf
\end{flushleft}
\fi

\end{list}
\ifpdf
\subsection*{\large{\textbf{strStep2}}\normalsize\hspace{1ex}\hrulefill}
\else
\subsection*{strStep2}
\fi
\label{trstrings-strStep2}
\index{strStep2}
\begin{list}{}{
\settowidth{\tmplength}{\textbf{Description}}
\setlength{\itemindent}{0cm}
\setlength{\listparindent}{0cm}
\setlength{\leftmargin}{\evensidemargin}
\addtolength{\leftmargin}{\tmplength}
\settowidth{\labelsep}{X}
\addtolength{\leftmargin}{\labelsep}
\setlength{\labelwidth}{\tmplength}
}
\item[\textbf{Declaration}\hfill]
\ifpdf
\begin{flushleft}
\fi
\begin{ttfamily}
strStep2='Phase 2/4: Installing files...';\end{ttfamily}

\ifpdf
\end{flushleft}
\fi

\end{list}
\ifpdf
\subsection*{\large{\textbf{strStep3}}\normalsize\hspace{1ex}\hrulefill}
\else
\subsection*{strStep3}
\fi
\label{trstrings-strStep3}
\index{strStep3}
\begin{list}{}{
\settowidth{\tmplength}{\textbf{Description}}
\setlength{\itemindent}{0cm}
\setlength{\listparindent}{0cm}
\setlength{\leftmargin}{\evensidemargin}
\addtolength{\leftmargin}{\tmplength}
\settowidth{\labelsep}{X}
\addtolength{\leftmargin}{\labelsep}
\setlength{\labelwidth}{\tmplength}
}
\item[\textbf{Declaration}\hfill]
\ifpdf
\begin{flushleft}
\fi
\begin{ttfamily}
strStep3='Phase 3/4: Chmod new files...';\end{ttfamily}

\ifpdf
\end{flushleft}
\fi

\end{list}
\ifpdf
\subsection*{\large{\textbf{strStep4}}\normalsize\hspace{1ex}\hrulefill}
\else
\subsection*{strStep4}
\fi
\label{trstrings-strStep4}
\index{strStep4}
\begin{list}{}{
\settowidth{\tmplength}{\textbf{Description}}
\setlength{\itemindent}{0cm}
\setlength{\listparindent}{0cm}
\setlength{\leftmargin}{\evensidemargin}
\addtolength{\leftmargin}{\tmplength}
\settowidth{\labelsep}{X}
\addtolength{\leftmargin}{\labelsep}
\setlength{\labelwidth}{\tmplength}
}
\item[\textbf{Declaration}\hfill]
\ifpdf
\begin{flushleft}
\fi
\begin{ttfamily}
strStep4='Phase 4/4: Registering application...';\end{ttfamily}

\ifpdf
\end{flushleft}
\fi

\end{list}
\ifpdf
\subsection*{\large{\textbf{strAddUpdSrc}}\normalsize\hspace{1ex}\hrulefill}
\else
\subsection*{strAddUpdSrc}
\fi
\label{trstrings-strAddUpdSrc}
\index{strAddUpdSrc}
\begin{list}{}{
\settowidth{\tmplength}{\textbf{Description}}
\setlength{\itemindent}{0cm}
\setlength{\listparindent}{0cm}
\setlength{\leftmargin}{\evensidemargin}
\addtolength{\leftmargin}{\tmplength}
\settowidth{\labelsep}{X}
\addtolength{\leftmargin}{\labelsep}
\setlength{\labelwidth}{\tmplength}
}
\item[\textbf{Declaration}\hfill]
\ifpdf
\begin{flushleft}
\fi
\begin{ttfamily}
strAddUpdSrc='This package tries to add the following update source: ';\end{ttfamily}

\ifpdf
\end{flushleft}
\fi

\end{list}
\ifpdf
\subsection*{\large{\textbf{strQAddUpdSrc}}\normalsize\hspace{1ex}\hrulefill}
\else
\subsection*{strQAddUpdSrc}
\fi
\label{trstrings-strQAddUpdSrc}
\index{strQAddUpdSrc}
\begin{list}{}{
\settowidth{\tmplength}{\textbf{Description}}
\setlength{\itemindent}{0cm}
\setlength{\listparindent}{0cm}
\setlength{\leftmargin}{\evensidemargin}
\addtolength{\leftmargin}{\tmplength}
\settowidth{\labelsep}{X}
\addtolength{\leftmargin}{\labelsep}
\setlength{\labelwidth}{\tmplength}
}
\item[\textbf{Declaration}\hfill]
\ifpdf
\begin{flushleft}
\fi
\begin{ttfamily}
strQAddUpdSrc='Should this source be added to the update-list?';\end{ttfamily}

\ifpdf
\end{flushleft}
\fi

\end{list}
\ifpdf
\subsection*{\large{\textbf{strFinished}}\normalsize\hspace{1ex}\hrulefill}
\else
\subsection*{strFinished}
\fi
\label{trstrings-strFinished}
\index{strFinished}
\begin{list}{}{
\settowidth{\tmplength}{\textbf{Description}}
\setlength{\itemindent}{0cm}
\setlength{\listparindent}{0cm}
\setlength{\leftmargin}{\evensidemargin}
\addtolength{\leftmargin}{\tmplength}
\settowidth{\labelsep}{X}
\addtolength{\leftmargin}{\labelsep}
\setlength{\labelwidth}{\tmplength}
}
\item[\textbf{Declaration}\hfill]
\ifpdf
\begin{flushleft}
\fi
\begin{ttfamily}
strFinished='Finished';\end{ttfamily}

\ifpdf
\end{flushleft}
\fi

\end{list}
\ifpdf
\subsection*{\large{\textbf{strinstAnyway}}\normalsize\hspace{1ex}\hrulefill}
\else
\subsection*{strinstAnyway}
\fi
\label{trstrings-strinstAnyway}
\index{strinstAnyway}
\begin{list}{}{
\settowidth{\tmplength}{\textbf{Description}}
\setlength{\itemindent}{0cm}
\setlength{\listparindent}{0cm}
\setlength{\leftmargin}{\evensidemargin}
\addtolength{\leftmargin}{\tmplength}
\settowidth{\labelsep}{X}
\addtolength{\leftmargin}{\labelsep}
\setlength{\labelwidth}{\tmplength}
}
\item[\textbf{Declaration}\hfill]
\ifpdf
\begin{flushleft}
\fi
\begin{ttfamily}
strinstAnyway='Do you want to install it anyway? (This could be problematical)';\end{ttfamily}

\ifpdf
\end{flushleft}
\fi

\end{list}
\ifpdf
\subsection*{\large{\textbf{strInvarchitecture}}\normalsize\hspace{1ex}\hrulefill}
\else
\subsection*{strInvarchitecture}
\fi
\label{trstrings-strInvarchitecture}
\index{strInvarchitecture}
\begin{list}{}{
\settowidth{\tmplength}{\textbf{Description}}
\setlength{\itemindent}{0cm}
\setlength{\listparindent}{0cm}
\setlength{\leftmargin}{\evensidemargin}
\addtolength{\leftmargin}{\tmplength}
\settowidth{\labelsep}{X}
\addtolength{\leftmargin}{\labelsep}
\setlength{\labelwidth}{\tmplength}
}
\item[\textbf{Declaration}\hfill]
\ifpdf
\begin{flushleft}
\fi
\begin{ttfamily}
strInvarchitecture='This package was not build for this system-architecture';\end{ttfamily}

\ifpdf
\end{flushleft}
\fi

\end{list}
\ifpdf
\subsection*{\large{\textbf{strWillDLFiles}}\normalsize\hspace{1ex}\hrulefill}
\else
\subsection*{strWillDLFiles}
\fi
\label{trstrings-strWillDLFiles}
\index{strWillDLFiles}
\begin{list}{}{
\settowidth{\tmplength}{\textbf{Description}}
\setlength{\itemindent}{0cm}
\setlength{\listparindent}{0cm}
\setlength{\leftmargin}{\evensidemargin}
\addtolength{\leftmargin}{\tmplength}
\settowidth{\labelsep}{X}
\addtolength{\leftmargin}{\labelsep}
\setlength{\labelwidth}{\tmplength}
}
\item[\textbf{Declaration}\hfill]
\ifpdf
\begin{flushleft}
\fi
\begin{ttfamily}
strWillDLFiles='(This program will download the needed files from the internet)';\end{ttfamily}

\ifpdf
\end{flushleft}
\fi

\end{list}
\ifpdf
\subsection*{\large{\textbf{strInvalidDVersion}}\normalsize\hspace{1ex}\hrulefill}
\else
\subsection*{strInvalidDVersion}
\fi
\label{trstrings-strInvalidDVersion}
\index{strInvalidDVersion}
\begin{list}{}{
\settowidth{\tmplength}{\textbf{Description}}
\setlength{\itemindent}{0cm}
\setlength{\listparindent}{0cm}
\setlength{\leftmargin}{\evensidemargin}
\addtolength{\leftmargin}{\tmplength}
\settowidth{\labelsep}{X}
\addtolength{\leftmargin}{\labelsep}
\setlength{\labelwidth}{\tmplength}
}
\item[\textbf{Declaration}\hfill]
\ifpdf
\begin{flushleft}
\fi
\begin{ttfamily}
strInvalidDVersion='Package was not build for this release.';\end{ttfamily}

\ifpdf
\end{flushleft}
\fi

\end{list}
\ifpdf
\subsection*{\large{\textbf{strcnGetDep}}\normalsize\hspace{1ex}\hrulefill}
\else
\subsection*{strcnGetDep}
\fi
\label{trstrings-strcnGetDep}
\index{strcnGetDep}
\begin{list}{}{
\settowidth{\tmplength}{\textbf{Description}}
\setlength{\itemindent}{0cm}
\setlength{\listparindent}{0cm}
\setlength{\leftmargin}{\evensidemargin}
\addtolength{\leftmargin}{\tmplength}
\settowidth{\labelsep}{X}
\addtolength{\leftmargin}{\labelsep}
\setlength{\labelwidth}{\tmplength}
}
\item[\textbf{Declaration}\hfill]
\ifpdf
\begin{flushleft}
\fi
\begin{ttfamily}
strcnGetDep='Problem while downloading the dep-file.';\end{ttfamily}

\ifpdf
\end{flushleft}
\fi

\end{list}
\ifpdf
\subsection*{\large{\textbf{strFTPfailed}}\normalsize\hspace{1ex}\hrulefill}
\else
\subsection*{strFTPfailed}
\fi
\label{trstrings-strFTPfailed}
\index{strFTPfailed}
\begin{list}{}{
\settowidth{\tmplength}{\textbf{Description}}
\setlength{\itemindent}{0cm}
\setlength{\listparindent}{0cm}
\setlength{\leftmargin}{\evensidemargin}
\addtolength{\leftmargin}{\tmplength}
\settowidth{\labelsep}{X}
\addtolength{\leftmargin}{\labelsep}
\setlength{\labelwidth}{\tmplength}
}
\item[\textbf{Declaration}\hfill]
\ifpdf
\begin{flushleft}
\fi
\begin{ttfamily}
strFTPfailed='Problem while downloading the packages. Maybe the login on the FTP-Server failed.';\end{ttfamily}

\ifpdf
\end{flushleft}
\fi

\end{list}
\ifpdf
\subsection*{\large{\textbf{strSuccess}}\normalsize\hspace{1ex}\hrulefill}
\else
\subsection*{strSuccess}
\fi
\label{trstrings-strSuccess}
\index{strSuccess}
\begin{list}{}{
\settowidth{\tmplength}{\textbf{Description}}
\setlength{\itemindent}{0cm}
\setlength{\listparindent}{0cm}
\setlength{\leftmargin}{\evensidemargin}
\addtolength{\leftmargin}{\tmplength}
\settowidth{\labelsep}{X}
\addtolength{\leftmargin}{\labelsep}
\setlength{\labelwidth}{\tmplength}
}
\item[\textbf{Declaration}\hfill]
\ifpdf
\begin{flushleft}
\fi
\begin{ttfamily}
strSuccess='Success!';\end{ttfamily}

\ifpdf
\end{flushleft}
\fi

\end{list}
\ifpdf
\subsection*{\large{\textbf{strMain}}\normalsize\hspace{1ex}\hrulefill}
\else
\subsection*{strMain}
\fi
\label{trstrings-strMain}
\index{strMain}
\begin{list}{}{
\settowidth{\tmplength}{\textbf{Description}}
\setlength{\itemindent}{0cm}
\setlength{\listparindent}{0cm}
\setlength{\leftmargin}{\evensidemargin}
\addtolength{\leftmargin}{\tmplength}
\settowidth{\labelsep}{X}
\addtolength{\leftmargin}{\labelsep}
\setlength{\labelwidth}{\tmplength}
}
\item[\textbf{Declaration}\hfill]
\ifpdf
\begin{flushleft}
\fi
\begin{ttfamily}
strMain='Main';\end{ttfamily}

\ifpdf
\end{flushleft}
\fi

\end{list}
\ifpdf
\subsection*{\large{\textbf{strDetails}}\normalsize\hspace{1ex}\hrulefill}
\else
\subsection*{strDetails}
\fi
\label{trstrings-strDetails}
\index{strDetails}
\begin{list}{}{
\settowidth{\tmplength}{\textbf{Description}}
\setlength{\itemindent}{0cm}
\setlength{\listparindent}{0cm}
\setlength{\leftmargin}{\evensidemargin}
\addtolength{\leftmargin}{\tmplength}
\settowidth{\labelsep}{X}
\addtolength{\leftmargin}{\labelsep}
\setlength{\labelwidth}{\tmplength}
}
\item[\textbf{Declaration}\hfill]
\ifpdf
\begin{flushleft}
\fi
\begin{ttfamily}
strDetails='Details';\end{ttfamily}

\ifpdf
\end{flushleft}
\fi

\end{list}
\ifpdf
\subsection*{\large{\textbf{strInstallation}}\normalsize\hspace{1ex}\hrulefill}
\else
\subsection*{strInstallation}
\fi
\label{trstrings-strInstallation}
\index{strInstallation}
\begin{list}{}{
\settowidth{\tmplength}{\textbf{Description}}
\setlength{\itemindent}{0cm}
\setlength{\listparindent}{0cm}
\setlength{\leftmargin}{\evensidemargin}
\addtolength{\leftmargin}{\tmplength}
\settowidth{\labelsep}{X}
\addtolength{\leftmargin}{\labelsep}
\setlength{\labelwidth}{\tmplength}
}
\item[\textbf{Declaration}\hfill]
\ifpdf
\begin{flushleft}
\fi
\begin{ttfamily}
strInstallation='Installation';\end{ttfamily}

\ifpdf
\end{flushleft}
\fi

\end{list}
\ifpdf
\subsection*{\large{\textbf{strNoLDSources}}\normalsize\hspace{1ex}\hrulefill}
\else
\subsection*{strNoLDSources}
\fi
\label{trstrings-strNoLDSources}
\index{strNoLDSources}
\begin{list}{}{
\settowidth{\tmplength}{\textbf{Description}}
\setlength{\itemindent}{0cm}
\setlength{\listparindent}{0cm}
\setlength{\leftmargin}{\evensidemargin}
\addtolength{\leftmargin}{\tmplength}
\settowidth{\labelsep}{X}
\addtolength{\leftmargin}{\labelsep}
\setlength{\labelwidth}{\tmplength}
}
\item[\textbf{Declaration}\hfill]
\ifpdf
\begin{flushleft}
\fi
\begin{ttfamily}
strNoLDSources='There are no sources available for your Linux-distribution'{\#}13'Try to install common packages?';\end{ttfamily}

\ifpdf
\end{flushleft}
\fi

\end{list}
\ifpdf
\subsection*{\large{\textbf{strUseCompPQ}}\normalsize\hspace{1ex}\hrulefill}
\else
\subsection*{strUseCompPQ}
\fi
\label{trstrings-strUseCompPQ}
\index{strUseCompPQ}
\begin{list}{}{
\settowidth{\tmplength}{\textbf{Description}}
\setlength{\itemindent}{0cm}
\setlength{\listparindent}{0cm}
\setlength{\leftmargin}{\evensidemargin}
\addtolength{\leftmargin}{\tmplength}
\settowidth{\labelsep}{X}
\addtolength{\leftmargin}{\labelsep}
\setlength{\labelwidth}{\tmplength}
}
\item[\textbf{Declaration}\hfill]
\ifpdf
\begin{flushleft}
\fi
\begin{ttfamily}
strUseCompPQ='Use compatible packages?';\end{ttfamily}

\ifpdf
\end{flushleft}
\fi

\end{list}
\ifpdf
\subsection*{\large{\textbf{strNoComp}}\normalsize\hspace{1ex}\hrulefill}
\else
\subsection*{strNoComp}
\fi
\label{trstrings-strNoComp}
\index{strNoComp}
\begin{list}{}{
\settowidth{\tmplength}{\textbf{Description}}
\setlength{\itemindent}{0cm}
\setlength{\listparindent}{0cm}
\setlength{\leftmargin}{\evensidemargin}
\addtolength{\leftmargin}{\tmplength}
\settowidth{\labelsep}{X}
\addtolength{\leftmargin}{\labelsep}
\setlength{\labelwidth}{\tmplength}
}
\item[\textbf{Declaration}\hfill]
\ifpdf
\begin{flushleft}
\fi
\begin{ttfamily}
strNoComp='No compatible packages found!';\end{ttfamily}

\ifpdf
\end{flushleft}
\fi

\end{list}
\ifpdf
\subsection*{\large{\textbf{strUpdSources}}\normalsize\hspace{1ex}\hrulefill}
\else
\subsection*{strUpdSources}
\fi
\label{trstrings-strUpdSources}
\index{strUpdSources}
\begin{list}{}{
\settowidth{\tmplength}{\textbf{Description}}
\setlength{\itemindent}{0cm}
\setlength{\listparindent}{0cm}
\setlength{\leftmargin}{\evensidemargin}
\addtolength{\leftmargin}{\tmplength}
\settowidth{\labelsep}{X}
\addtolength{\leftmargin}{\labelsep}
\setlength{\labelwidth}{\tmplength}
}
\item[\textbf{Declaration}\hfill]
\ifpdf
\begin{flushleft}
\fi
\begin{ttfamily}
strUpdSources='Update sources';\end{ttfamily}

\ifpdf
\end{flushleft}
\fi

\end{list}
\ifpdf
\subsection*{\large{\textbf{strClose}}\normalsize\hspace{1ex}\hrulefill}
\else
\subsection*{strClose}
\fi
\label{trstrings-strClose}
\index{strClose}
\begin{list}{}{
\settowidth{\tmplength}{\textbf{Description}}
\setlength{\itemindent}{0cm}
\setlength{\listparindent}{0cm}
\setlength{\leftmargin}{\evensidemargin}
\addtolength{\leftmargin}{\tmplength}
\settowidth{\labelsep}{X}
\addtolength{\leftmargin}{\labelsep}
\setlength{\labelwidth}{\tmplength}
}
\item[\textbf{Declaration}\hfill]
\ifpdf
\begin{flushleft}
\fi
\begin{ttfamily}
strClose='Close';\end{ttfamily}

\ifpdf
\end{flushleft}
\fi

\end{list}
\ifpdf
\subsection*{\large{\textbf{strDelSrc}}\normalsize\hspace{1ex}\hrulefill}
\else
\subsection*{strDelSrc}
\fi
\label{trstrings-strDelSrc}
\index{strDelSrc}
\begin{list}{}{
\settowidth{\tmplength}{\textbf{Description}}
\setlength{\itemindent}{0cm}
\setlength{\listparindent}{0cm}
\setlength{\leftmargin}{\evensidemargin}
\addtolength{\leftmargin}{\tmplength}
\settowidth{\labelsep}{X}
\addtolength{\leftmargin}{\labelsep}
\setlength{\labelwidth}{\tmplength}
}
\item[\textbf{Declaration}\hfill]
\ifpdf
\begin{flushleft}
\fi
\begin{ttfamily}
strDelSrc='Delete source';\end{ttfamily}

\ifpdf
\end{flushleft}
\fi

\end{list}
\ifpdf
\subsection*{\large{\textbf{strListofSrc}}\normalsize\hspace{1ex}\hrulefill}
\else
\subsection*{strListofSrc}
\fi
\label{trstrings-strListofSrc}
\index{strListofSrc}
\begin{list}{}{
\settowidth{\tmplength}{\textbf{Description}}
\setlength{\itemindent}{0cm}
\setlength{\listparindent}{0cm}
\setlength{\leftmargin}{\evensidemargin}
\addtolength{\leftmargin}{\tmplength}
\settowidth{\labelsep}{X}
\addtolength{\leftmargin}{\labelsep}
\setlength{\labelwidth}{\tmplength}
}
\item[\textbf{Declaration}\hfill]
\ifpdf
\begin{flushleft}
\fi
\begin{ttfamily}
strListofSrc='Here''s a list of all update-sources:';\end{ttfamily}

\ifpdf
\end{flushleft}
\fi

\end{list}
\ifpdf
\subsection*{\large{\textbf{strUninstall}}\normalsize\hspace{1ex}\hrulefill}
\else
\subsection*{strUninstall}
\fi
\label{trstrings-strUninstall}
\index{strUninstall}
\begin{list}{}{
\settowidth{\tmplength}{\textbf{Description}}
\setlength{\itemindent}{0cm}
\setlength{\listparindent}{0cm}
\setlength{\leftmargin}{\evensidemargin}
\addtolength{\leftmargin}{\tmplength}
\settowidth{\labelsep}{X}
\addtolength{\leftmargin}{\labelsep}
\setlength{\labelwidth}{\tmplength}
}
\item[\textbf{Declaration}\hfill]
\ifpdf
\begin{flushleft}
\fi
\begin{ttfamily}
strUninstall='Uninstall';\end{ttfamily}

\ifpdf
\end{flushleft}
\fi

\end{list}
\ifpdf
\subsection*{\large{\textbf{strInstNew}}\normalsize\hspace{1ex}\hrulefill}
\else
\subsection*{strInstNew}
\fi
\label{trstrings-strInstNew}
\index{strInstNew}
\begin{list}{}{
\settowidth{\tmplength}{\textbf{Description}}
\setlength{\itemindent}{0cm}
\setlength{\listparindent}{0cm}
\setlength{\leftmargin}{\evensidemargin}
\addtolength{\leftmargin}{\tmplength}
\settowidth{\labelsep}{X}
\addtolength{\leftmargin}{\labelsep}
\setlength{\labelwidth}{\tmplength}
}
\item[\textbf{Declaration}\hfill]
\ifpdf
\begin{flushleft}
\fi
\begin{ttfamily}
strInstNew='Install new application';\end{ttfamily}

\ifpdf
\end{flushleft}
\fi

\end{list}
\ifpdf
\subsection*{\large{\textbf{strShowSettings}}\normalsize\hspace{1ex}\hrulefill}
\else
\subsection*{strShowSettings}
\fi
\label{trstrings-strShowSettings}
\index{strShowSettings}
\begin{list}{}{
\settowidth{\tmplength}{\textbf{Description}}
\setlength{\itemindent}{0cm}
\setlength{\listparindent}{0cm}
\setlength{\leftmargin}{\evensidemargin}
\addtolength{\leftmargin}{\tmplength}
\settowidth{\labelsep}{X}
\addtolength{\leftmargin}{\labelsep}
\setlength{\labelwidth}{\tmplength}
}
\item[\textbf{Declaration}\hfill]
\ifpdf
\begin{flushleft}
\fi
\begin{ttfamily}
strShowSettings='Listaller settings';\end{ttfamily}

\ifpdf
\end{flushleft}
\fi

\end{list}
\ifpdf
\subsection*{\large{\textbf{strSWCatalogue}}\normalsize\hspace{1ex}\hrulefill}
\else
\subsection*{strSWCatalogue}
\fi
\label{trstrings-strSWCatalogue}
\index{strSWCatalogue}
\begin{list}{}{
\settowidth{\tmplength}{\textbf{Description}}
\setlength{\itemindent}{0cm}
\setlength{\listparindent}{0cm}
\setlength{\leftmargin}{\evensidemargin}
\addtolength{\leftmargin}{\tmplength}
\settowidth{\labelsep}{X}
\addtolength{\leftmargin}{\labelsep}
\setlength{\labelwidth}{\tmplength}
}
\item[\textbf{Declaration}\hfill]
\ifpdf
\begin{flushleft}
\fi
\begin{ttfamily}
strSWCatalogue='Software catalogue';\end{ttfamily}

\ifpdf
\end{flushleft}
\fi

\end{list}
\ifpdf
\subsection*{\large{\textbf{strShow}}\normalsize\hspace{1ex}\hrulefill}
\else
\subsection*{strShow}
\fi
\label{trstrings-strShow}
\index{strShow}
\begin{list}{}{
\settowidth{\tmplength}{\textbf{Description}}
\setlength{\itemindent}{0cm}
\setlength{\listparindent}{0cm}
\setlength{\leftmargin}{\evensidemargin}
\addtolength{\leftmargin}{\tmplength}
\settowidth{\labelsep}{X}
\addtolength{\leftmargin}{\labelsep}
\setlength{\labelwidth}{\tmplength}
}
\item[\textbf{Declaration}\hfill]
\ifpdf
\begin{flushleft}
\fi
\begin{ttfamily}
strShow='Show:';\end{ttfamily}

\ifpdf
\end{flushleft}
\fi

\end{list}
\ifpdf
\subsection*{\large{\textbf{strAll}}\normalsize\hspace{1ex}\hrulefill}
\else
\subsection*{strAll}
\fi
\label{trstrings-strAll}
\index{strAll}
\begin{list}{}{
\settowidth{\tmplength}{\textbf{Description}}
\setlength{\itemindent}{0cm}
\setlength{\listparindent}{0cm}
\setlength{\leftmargin}{\evensidemargin}
\addtolength{\leftmargin}{\tmplength}
\settowidth{\labelsep}{X}
\addtolength{\leftmargin}{\labelsep}
\setlength{\labelwidth}{\tmplength}
}
\item[\textbf{Declaration}\hfill]
\ifpdf
\begin{flushleft}
\fi
\begin{ttfamily}
strAll='All';\end{ttfamily}

\ifpdf
\end{flushleft}
\fi

\end{list}
\ifpdf
\subsection*{\large{\textbf{strEducation}}\normalsize\hspace{1ex}\hrulefill}
\else
\subsection*{strEducation}
\fi
\label{trstrings-strEducation}
\index{strEducation}
\begin{list}{}{
\settowidth{\tmplength}{\textbf{Description}}
\setlength{\itemindent}{0cm}
\setlength{\listparindent}{0cm}
\setlength{\leftmargin}{\evensidemargin}
\addtolength{\leftmargin}{\tmplength}
\settowidth{\labelsep}{X}
\addtolength{\leftmargin}{\labelsep}
\setlength{\labelwidth}{\tmplength}
}
\item[\textbf{Declaration}\hfill]
\ifpdf
\begin{flushleft}
\fi
\begin{ttfamily}
strEducation='Education';\end{ttfamily}

\ifpdf
\end{flushleft}
\fi

\end{list}
\ifpdf
\subsection*{\large{\textbf{strOffice}}\normalsize\hspace{1ex}\hrulefill}
\else
\subsection*{strOffice}
\fi
\label{trstrings-strOffice}
\index{strOffice}
\begin{list}{}{
\settowidth{\tmplength}{\textbf{Description}}
\setlength{\itemindent}{0cm}
\setlength{\listparindent}{0cm}
\setlength{\leftmargin}{\evensidemargin}
\addtolength{\leftmargin}{\tmplength}
\settowidth{\labelsep}{X}
\addtolength{\leftmargin}{\labelsep}
\setlength{\labelwidth}{\tmplength}
}
\item[\textbf{Declaration}\hfill]
\ifpdf
\begin{flushleft}
\fi
\begin{ttfamily}
strOffice='Office';\end{ttfamily}

\ifpdf
\end{flushleft}
\fi

\end{list}
\ifpdf
\subsection*{\large{\textbf{strDevelopment}}\normalsize\hspace{1ex}\hrulefill}
\else
\subsection*{strDevelopment}
\fi
\label{trstrings-strDevelopment}
\index{strDevelopment}
\begin{list}{}{
\settowidth{\tmplength}{\textbf{Description}}
\setlength{\itemindent}{0cm}
\setlength{\listparindent}{0cm}
\setlength{\leftmargin}{\evensidemargin}
\addtolength{\leftmargin}{\tmplength}
\settowidth{\labelsep}{X}
\addtolength{\leftmargin}{\labelsep}
\setlength{\labelwidth}{\tmplength}
}
\item[\textbf{Declaration}\hfill]
\ifpdf
\begin{flushleft}
\fi
\begin{ttfamily}
strDevelopment='Development';\end{ttfamily}

\ifpdf
\end{flushleft}
\fi

\end{list}
\ifpdf
\subsection*{\large{\textbf{strGraphic}}\normalsize\hspace{1ex}\hrulefill}
\else
\subsection*{strGraphic}
\fi
\label{trstrings-strGraphic}
\index{strGraphic}
\begin{list}{}{
\settowidth{\tmplength}{\textbf{Description}}
\setlength{\itemindent}{0cm}
\setlength{\listparindent}{0cm}
\setlength{\leftmargin}{\evensidemargin}
\addtolength{\leftmargin}{\tmplength}
\settowidth{\labelsep}{X}
\addtolength{\leftmargin}{\labelsep}
\setlength{\labelwidth}{\tmplength}
}
\item[\textbf{Declaration}\hfill]
\ifpdf
\begin{flushleft}
\fi
\begin{ttfamily}
strGraphic='Graphic';\end{ttfamily}

\ifpdf
\end{flushleft}
\fi

\end{list}
\ifpdf
\subsection*{\large{\textbf{strNetwork}}\normalsize\hspace{1ex}\hrulefill}
\else
\subsection*{strNetwork}
\fi
\label{trstrings-strNetwork}
\index{strNetwork}
\begin{list}{}{
\settowidth{\tmplength}{\textbf{Description}}
\setlength{\itemindent}{0cm}
\setlength{\listparindent}{0cm}
\setlength{\leftmargin}{\evensidemargin}
\addtolength{\leftmargin}{\tmplength}
\settowidth{\labelsep}{X}
\addtolength{\leftmargin}{\labelsep}
\setlength{\labelwidth}{\tmplength}
}
\item[\textbf{Declaration}\hfill]
\ifpdf
\begin{flushleft}
\fi
\begin{ttfamily}
strNetwork='Network';\end{ttfamily}

\ifpdf
\end{flushleft}
\fi

\end{list}
\ifpdf
\subsection*{\large{\textbf{strGames}}\normalsize\hspace{1ex}\hrulefill}
\else
\subsection*{strGames}
\fi
\label{trstrings-strGames}
\index{strGames}
\begin{list}{}{
\settowidth{\tmplength}{\textbf{Description}}
\setlength{\itemindent}{0cm}
\setlength{\listparindent}{0cm}
\setlength{\leftmargin}{\evensidemargin}
\addtolength{\leftmargin}{\tmplength}
\settowidth{\labelsep}{X}
\addtolength{\leftmargin}{\labelsep}
\setlength{\labelwidth}{\tmplength}
}
\item[\textbf{Declaration}\hfill]
\ifpdf
\begin{flushleft}
\fi
\begin{ttfamily}
strGames='Games';\end{ttfamily}

\ifpdf
\end{flushleft}
\fi

\end{list}
\ifpdf
\subsection*{\large{\textbf{strSystem}}\normalsize\hspace{1ex}\hrulefill}
\else
\subsection*{strSystem}
\fi
\label{trstrings-strSystem}
\index{strSystem}
\begin{list}{}{
\settowidth{\tmplength}{\textbf{Description}}
\setlength{\itemindent}{0cm}
\setlength{\listparindent}{0cm}
\setlength{\leftmargin}{\evensidemargin}
\addtolength{\leftmargin}{\tmplength}
\settowidth{\labelsep}{X}
\addtolength{\leftmargin}{\labelsep}
\setlength{\labelwidth}{\tmplength}
}
\item[\textbf{Declaration}\hfill]
\ifpdf
\begin{flushleft}
\fi
\begin{ttfamily}
strSystem='System';\end{ttfamily}

\ifpdf
\end{flushleft}
\fi

\end{list}
\ifpdf
\subsection*{\large{\textbf{strMultimedia}}\normalsize\hspace{1ex}\hrulefill}
\else
\subsection*{strMultimedia}
\fi
\label{trstrings-strMultimedia}
\index{strMultimedia}
\begin{list}{}{
\settowidth{\tmplength}{\textbf{Description}}
\setlength{\itemindent}{0cm}
\setlength{\listparindent}{0cm}
\setlength{\leftmargin}{\evensidemargin}
\addtolength{\leftmargin}{\tmplength}
\settowidth{\labelsep}{X}
\addtolength{\leftmargin}{\labelsep}
\setlength{\labelwidth}{\tmplength}
}
\item[\textbf{Declaration}\hfill]
\ifpdf
\begin{flushleft}
\fi
\begin{ttfamily}
strMultimedia='Multimedia';\end{ttfamily}

\ifpdf
\end{flushleft}
\fi

\end{list}
\ifpdf
\subsection*{\large{\textbf{strAddidional}}\normalsize\hspace{1ex}\hrulefill}
\else
\subsection*{strAddidional}
\fi
\label{trstrings-strAddidional}
\index{strAddidional}
\begin{list}{}{
\settowidth{\tmplength}{\textbf{Description}}
\setlength{\itemindent}{0cm}
\setlength{\listparindent}{0cm}
\setlength{\leftmargin}{\evensidemargin}
\addtolength{\leftmargin}{\tmplength}
\settowidth{\labelsep}{X}
\addtolength{\leftmargin}{\labelsep}
\setlength{\labelwidth}{\tmplength}
}
\item[\textbf{Declaration}\hfill]
\ifpdf
\begin{flushleft}
\fi
\begin{ttfamily}
strAddidional='Additional';\end{ttfamily}

\ifpdf
\end{flushleft}
\fi

\end{list}
\ifpdf
\subsection*{\large{\textbf{strOther}}\normalsize\hspace{1ex}\hrulefill}
\else
\subsection*{strOther}
\fi
\label{trstrings-strOther}
\index{strOther}
\begin{list}{}{
\settowidth{\tmplength}{\textbf{Description}}
\setlength{\itemindent}{0cm}
\setlength{\listparindent}{0cm}
\setlength{\leftmargin}{\evensidemargin}
\addtolength{\leftmargin}{\tmplength}
\settowidth{\labelsep}{X}
\addtolength{\leftmargin}{\labelsep}
\setlength{\labelwidth}{\tmplength}
}
\item[\textbf{Declaration}\hfill]
\ifpdf
\begin{flushleft}
\fi
\begin{ttfamily}
strOther='Other';\end{ttfamily}

\ifpdf
\end{flushleft}
\fi

\end{list}
\ifpdf
\subsection*{\large{\textbf{strVersion}}\normalsize\hspace{1ex}\hrulefill}
\else
\subsection*{strVersion}
\fi
\label{trstrings-strVersion}
\index{strVersion}
\begin{list}{}{
\settowidth{\tmplength}{\textbf{Description}}
\setlength{\itemindent}{0cm}
\setlength{\listparindent}{0cm}
\setlength{\leftmargin}{\evensidemargin}
\addtolength{\leftmargin}{\tmplength}
\settowidth{\labelsep}{X}
\addtolength{\leftmargin}{\labelsep}
\setlength{\labelwidth}{\tmplength}
}
\item[\textbf{Declaration}\hfill]
\ifpdf
\begin{flushleft}
\fi
\begin{ttfamily}
strVersion='Version';\end{ttfamily}

\ifpdf
\end{flushleft}
\fi

\end{list}
\ifpdf
\subsection*{\large{\textbf{strAuthor}}\normalsize\hspace{1ex}\hrulefill}
\else
\subsection*{strAuthor}
\fi
\label{trstrings-strAuthor}
\index{strAuthor}
\begin{list}{}{
\settowidth{\tmplength}{\textbf{Description}}
\setlength{\itemindent}{0cm}
\setlength{\listparindent}{0cm}
\setlength{\leftmargin}{\evensidemargin}
\addtolength{\leftmargin}{\tmplength}
\settowidth{\labelsep}{X}
\addtolength{\leftmargin}{\labelsep}
\setlength{\labelwidth}{\tmplength}
}
\item[\textbf{Declaration}\hfill]
\ifpdf
\begin{flushleft}
\fi
\begin{ttfamily}
strAuthor='Autor';\end{ttfamily}

\ifpdf
\end{flushleft}
\fi

\end{list}
\ifpdf
\subsection*{\large{\textbf{strUsername}}\normalsize\hspace{1ex}\hrulefill}
\else
\subsection*{strUsername}
\fi
\label{trstrings-strUsername}
\index{strUsername}
\begin{list}{}{
\settowidth{\tmplength}{\textbf{Description}}
\setlength{\itemindent}{0cm}
\setlength{\listparindent}{0cm}
\setlength{\leftmargin}{\evensidemargin}
\addtolength{\leftmargin}{\tmplength}
\settowidth{\labelsep}{X}
\addtolength{\leftmargin}{\labelsep}
\setlength{\labelwidth}{\tmplength}
}
\item[\textbf{Declaration}\hfill]
\ifpdf
\begin{flushleft}
\fi
\begin{ttfamily}
strUsername='Username';\end{ttfamily}

\ifpdf
\end{flushleft}
\fi

\end{list}
\ifpdf
\subsection*{\large{\textbf{strPassword}}\normalsize\hspace{1ex}\hrulefill}
\else
\subsection*{strPassword}
\fi
\label{trstrings-strPassword}
\index{strPassword}
\begin{list}{}{
\settowidth{\tmplength}{\textbf{Description}}
\setlength{\itemindent}{0cm}
\setlength{\listparindent}{0cm}
\setlength{\leftmargin}{\evensidemargin}
\addtolength{\leftmargin}{\tmplength}
\settowidth{\labelsep}{X}
\addtolength{\leftmargin}{\labelsep}
\setlength{\labelwidth}{\tmplength}
}
\item[\textbf{Declaration}\hfill]
\ifpdf
\begin{flushleft}
\fi
\begin{ttfamily}
strPassword='Password';\end{ttfamily}

\ifpdf
\end{flushleft}
\fi

\end{list}
\ifpdf
\subsection*{\large{\textbf{strProxySettings}}\normalsize\hspace{1ex}\hrulefill}
\else
\subsection*{strProxySettings}
\fi
\label{trstrings-strProxySettings}
\index{strProxySettings}
\begin{list}{}{
\settowidth{\tmplength}{\textbf{Description}}
\setlength{\itemindent}{0cm}
\setlength{\listparindent}{0cm}
\setlength{\leftmargin}{\evensidemargin}
\addtolength{\leftmargin}{\tmplength}
\settowidth{\labelsep}{X}
\addtolength{\leftmargin}{\labelsep}
\setlength{\labelwidth}{\tmplength}
}
\item[\textbf{Declaration}\hfill]
\ifpdf
\begin{flushleft}
\fi
\begin{ttfamily}
strProxySettings='Proxy-Settings';\end{ttfamily}

\ifpdf
\end{flushleft}
\fi

\end{list}
\ifpdf
\subsection*{\large{\textbf{strEnableProxy}}\normalsize\hspace{1ex}\hrulefill}
\else
\subsection*{strEnableProxy}
\fi
\label{trstrings-strEnableProxy}
\index{strEnableProxy}
\begin{list}{}{
\settowidth{\tmplength}{\textbf{Description}}
\setlength{\itemindent}{0cm}
\setlength{\listparindent}{0cm}
\setlength{\leftmargin}{\evensidemargin}
\addtolength{\leftmargin}{\tmplength}
\settowidth{\labelsep}{X}
\addtolength{\leftmargin}{\labelsep}
\setlength{\labelwidth}{\tmplength}
}
\item[\textbf{Declaration}\hfill]
\ifpdf
\begin{flushleft}
\fi
\begin{ttfamily}
strEnableProxy='Enable Proxy-Server';\end{ttfamily}

\ifpdf
\end{flushleft}
\fi

\end{list}
\ifpdf
\subsection*{\large{\textbf{strLOKIError}}\normalsize\hspace{1ex}\hrulefill}
\else
\subsection*{strLOKIError}
\fi
\label{trstrings-strLOKIError}
\index{strLOKIError}
\begin{list}{}{
\settowidth{\tmplength}{\textbf{Description}}
\setlength{\itemindent}{0cm}
\setlength{\listparindent}{0cm}
\setlength{\leftmargin}{\evensidemargin}
\addtolength{\leftmargin}{\tmplength}
\settowidth{\labelsep}{X}
\addtolength{\leftmargin}{\labelsep}
\setlength{\labelwidth}{\tmplength}
}
\item[\textbf{Declaration}\hfill]
\ifpdf
\begin{flushleft}
\fi
\begin{ttfamily}
strLOKIError='Can''t load LOKI-Setup information.';\end{ttfamily}

\ifpdf
\end{flushleft}
\fi

\end{list}
\ifpdf
\subsection*{\large{\textbf{strCannotLoadIcon}}\normalsize\hspace{1ex}\hrulefill}
\else
\subsection*{strCannotLoadIcon}
\fi
\label{trstrings-strCannotLoadIcon}
\index{strCannotLoadIcon}
\begin{list}{}{
\settowidth{\tmplength}{\textbf{Description}}
\setlength{\itemindent}{0cm}
\setlength{\listparindent}{0cm}
\setlength{\leftmargin}{\evensidemargin}
\addtolength{\leftmargin}{\tmplength}
\settowidth{\labelsep}{X}
\addtolength{\leftmargin}{\labelsep}
\setlength{\labelwidth}{\tmplength}
}
\item[\textbf{Declaration}\hfill]
\ifpdf
\begin{flushleft}
\fi
\begin{ttfamily}
strCannotLoadIcon='Unable to load the icon of {\%}a. Please notify the developers of Listaller or this application!';\end{ttfamily}

\ifpdf
\end{flushleft}
\fi

\end{list}
\ifpdf
\subsection*{\large{\textbf{strAutoLoadDep}}\normalsize\hspace{1ex}\hrulefill}
\else
\subsection*{strAutoLoadDep}
\fi
\label{trstrings-strAutoLoadDep}
\index{strAutoLoadDep}
\begin{list}{}{
\settowidth{\tmplength}{\textbf{Description}}
\setlength{\itemindent}{0cm}
\setlength{\listparindent}{0cm}
\setlength{\leftmargin}{\evensidemargin}
\addtolength{\leftmargin}{\tmplength}
\settowidth{\labelsep}{X}
\addtolength{\leftmargin}{\labelsep}
\setlength{\labelwidth}{\tmplength}
}
\item[\textbf{Declaration}\hfill]
\ifpdf
\begin{flushleft}
\fi
\begin{ttfamily}
strAutoLoadDep='Load dependencies from included webserver-urls automatically';\end{ttfamily}

\ifpdf
\end{flushleft}
\fi

\end{list}
\ifpdf
\subsection*{\large{\textbf{strReady}}\normalsize\hspace{1ex}\hrulefill}
\else
\subsection*{strReady}
\fi
\label{trstrings-strReady}
\index{strReady}
\begin{list}{}{
\settowidth{\tmplength}{\textbf{Description}}
\setlength{\itemindent}{0cm}
\setlength{\listparindent}{0cm}
\setlength{\leftmargin}{\evensidemargin}
\addtolength{\leftmargin}{\tmplength}
\settowidth{\labelsep}{X}
\addtolength{\leftmargin}{\labelsep}
\setlength{\labelwidth}{\tmplength}
}
\item[\textbf{Declaration}\hfill]
\ifpdf
\begin{flushleft}
\fi
\begin{ttfamily}
strReady='Ready.';\end{ttfamily}

\ifpdf
\end{flushleft}
\fi

\end{list}
\ifpdf
\subsection*{\large{\textbf{strConvertPkg}}\normalsize\hspace{1ex}\hrulefill}
\else
\subsection*{strConvertPkg}
\fi
\label{trstrings-strConvertPkg}
\index{strConvertPkg}
\begin{list}{}{
\settowidth{\tmplength}{\textbf{Description}}
\setlength{\itemindent}{0cm}
\setlength{\listparindent}{0cm}
\setlength{\leftmargin}{\evensidemargin}
\addtolength{\leftmargin}{\tmplength}
\settowidth{\labelsep}{X}
\addtolength{\leftmargin}{\labelsep}
\setlength{\labelwidth}{\tmplength}
}
\item[\textbf{Declaration}\hfill]
\ifpdf
\begin{flushleft}
\fi
\begin{ttfamily}
strConvertPkg='You want to install an {\%}x-Package, but your Linux-Distribution''s package system is {\%}y.'{\#}13'This package can be converted using "alien", but this will take some time and eventually the application won''t work'{\#}13'Do you want to convert the package now?';\end{ttfamily}

\ifpdf
\end{flushleft}
\fi

\end{list}
\ifpdf
\subsection*{\large{\textbf{strConvertPkgQ}}\normalsize\hspace{1ex}\hrulefill}
\else
\subsection*{strConvertPkgQ}
\fi
\label{trstrings-strConvertPkgQ}
\index{strConvertPkgQ}
\begin{list}{}{
\settowidth{\tmplength}{\textbf{Description}}
\setlength{\itemindent}{0cm}
\setlength{\listparindent}{0cm}
\setlength{\leftmargin}{\evensidemargin}
\addtolength{\leftmargin}{\tmplength}
\settowidth{\labelsep}{X}
\addtolength{\leftmargin}{\labelsep}
\setlength{\labelwidth}{\tmplength}
}
\item[\textbf{Declaration}\hfill]
\ifpdf
\begin{flushleft}
\fi
\begin{ttfamily}
strConvertPkgQ='Convert package?';\end{ttfamily}

\ifpdf
\end{flushleft}
\fi

\end{list}
\ifpdf
\subsection*{\large{\textbf{strConvTitle}}\normalsize\hspace{1ex}\hrulefill}
\else
\subsection*{strConvTitle}
\fi
\label{trstrings-strConvTitle}
\index{strConvTitle}
\begin{list}{}{
\settowidth{\tmplength}{\textbf{Description}}
\setlength{\itemindent}{0cm}
\setlength{\listparindent}{0cm}
\setlength{\leftmargin}{\evensidemargin}
\addtolength{\leftmargin}{\tmplength}
\settowidth{\labelsep}{X}
\addtolength{\leftmargin}{\labelsep}
\setlength{\labelwidth}{\tmplength}
}
\item[\textbf{Declaration}\hfill]
\ifpdf
\begin{flushleft}
\fi
\begin{ttfamily}
strConvTitle='Converting {\%}p package...';\end{ttfamily}

\ifpdf
\end{flushleft}
\fi

\end{list}
\ifpdf
\subsection*{\large{\textbf{strFiltering}}\normalsize\hspace{1ex}\hrulefill}
\else
\subsection*{strFiltering}
\fi
\label{trstrings-strFiltering}
\index{strFiltering}
\begin{list}{}{
\settowidth{\tmplength}{\textbf{Description}}
\setlength{\itemindent}{0cm}
\setlength{\listparindent}{0cm}
\setlength{\leftmargin}{\evensidemargin}
\addtolength{\leftmargin}{\tmplength}
\settowidth{\labelsep}{X}
\addtolength{\leftmargin}{\labelsep}
\setlength{\labelwidth}{\tmplength}
}
\item[\textbf{Declaration}\hfill]
\ifpdf
\begin{flushleft}
\fi
\begin{ttfamily}
strFiltering='Filtering...';\end{ttfamily}

\ifpdf
\end{flushleft}
\fi

\end{list}
\ifpdf
\subsection*{\large{\textbf{strFilter}}\normalsize\hspace{1ex}\hrulefill}
\else
\subsection*{strFilter}
\fi
\label{trstrings-strFilter}
\index{strFilter}
\begin{list}{}{
\settowidth{\tmplength}{\textbf{Description}}
\setlength{\itemindent}{0cm}
\setlength{\listparindent}{0cm}
\setlength{\leftmargin}{\evensidemargin}
\addtolength{\leftmargin}{\tmplength}
\settowidth{\labelsep}{X}
\addtolength{\leftmargin}{\labelsep}
\setlength{\labelwidth}{\tmplength}
}
\item[\textbf{Declaration}\hfill]
\ifpdf
\begin{flushleft}
\fi
\begin{ttfamily}
strFilter='Filter:';\end{ttfamily}

\ifpdf
\end{flushleft}
\fi

\end{list}
\ifpdf
\subsection*{\large{\textbf{strLoading}}\normalsize\hspace{1ex}\hrulefill}
\else
\subsection*{strLoading}
\fi
\label{trstrings-strLoading}
\index{strLoading}
\begin{list}{}{
\settowidth{\tmplength}{\textbf{Description}}
\setlength{\itemindent}{0cm}
\setlength{\listparindent}{0cm}
\setlength{\leftmargin}{\evensidemargin}
\addtolength{\leftmargin}{\tmplength}
\settowidth{\labelsep}{X}
\addtolength{\leftmargin}{\labelsep}
\setlength{\labelwidth}{\tmplength}
}
\item[\textbf{Declaration}\hfill]
\ifpdf
\begin{flushleft}
\fi
\begin{ttfamily}
strLoading='Loading...';\end{ttfamily}

\ifpdf
\end{flushleft}
\fi

\end{list}
\ifpdf
\subsection*{\large{\textbf{strDispRootApps}}\normalsize\hspace{1ex}\hrulefill}
\else
\subsection*{strDispRootApps}
\fi
\label{trstrings-strDispRootApps}
\index{strDispRootApps}
\begin{list}{}{
\settowidth{\tmplength}{\textbf{Description}}
\setlength{\itemindent}{0cm}
\setlength{\listparindent}{0cm}
\setlength{\leftmargin}{\evensidemargin}
\addtolength{\leftmargin}{\tmplength}
\settowidth{\labelsep}{X}
\addtolength{\leftmargin}{\labelsep}
\setlength{\labelwidth}{\tmplength}
}
\item[\textbf{Declaration}\hfill]
\ifpdf
\begin{flushleft}
\fi
\begin{ttfamily}
strDispRootApps='Display system applications';\end{ttfamily}

\ifpdf
\end{flushleft}
\fi

\end{list}
\ifpdf
\subsection*{\large{\textbf{strDispOnlyMyApps}}\normalsize\hspace{1ex}\hrulefill}
\else
\subsection*{strDispOnlyMyApps}
\fi
\label{trstrings-strDispOnlyMyApps}
\index{strDispOnlyMyApps}
\begin{list}{}{
\settowidth{\tmplength}{\textbf{Description}}
\setlength{\itemindent}{0cm}
\setlength{\listparindent}{0cm}
\setlength{\leftmargin}{\evensidemargin}
\addtolength{\leftmargin}{\tmplength}
\settowidth{\labelsep}{X}
\addtolength{\leftmargin}{\labelsep}
\setlength{\labelwidth}{\tmplength}
}
\item[\textbf{Declaration}\hfill]
\ifpdf
\begin{flushleft}
\fi
\begin{ttfamily}
strDispOnlyMyApps='Display my applications';\end{ttfamily}

\ifpdf
\end{flushleft}
\fi

\end{list}
\ifpdf
\subsection*{\large{\textbf{strSelMgrMode}}\normalsize\hspace{1ex}\hrulefill}
\else
\subsection*{strSelMgrMode}
\fi
\label{trstrings-strSelMgrMode}
\index{strSelMgrMode}
\begin{list}{}{
\settowidth{\tmplength}{\textbf{Description}}
\setlength{\itemindent}{0cm}
\setlength{\listparindent}{0cm}
\setlength{\leftmargin}{\evensidemargin}
\addtolength{\leftmargin}{\tmplength}
\settowidth{\labelsep}{X}
\addtolength{\leftmargin}{\labelsep}
\setlength{\labelwidth}{\tmplength}
}
\item[\textbf{Declaration}\hfill]
\ifpdf
\begin{flushleft}
\fi
\begin{ttfamily}
strSelMgrMode='Select software-manager mode:';\end{ttfamily}

\ifpdf
\end{flushleft}
\fi

\end{list}
\ifpdf
\subsection*{\large{\textbf{strPackageKitWarning}}\normalsize\hspace{1ex}\hrulefill}
\else
\subsection*{strPackageKitWarning}
\fi
\label{trstrings-strPackageKitWarning}
\index{strPackageKitWarning}
\begin{list}{}{
\settowidth{\tmplength}{\textbf{Description}}
\setlength{\itemindent}{0cm}
\setlength{\listparindent}{0cm}
\setlength{\leftmargin}{\evensidemargin}
\addtolength{\leftmargin}{\tmplength}
\settowidth{\labelsep}{X}
\addtolength{\leftmargin}{\labelsep}
\setlength{\labelwidth}{\tmplength}
}
\item[\textbf{Declaration}\hfill]
\ifpdf
\begin{flushleft}
\fi
\begin{ttfamily}
strPackageKitWarning='Your PackageKit version is {\%}cp. Listaller needs PackageKit {\%}np or higher to work correctly.'{\#}13'Please update PackageKit!';\end{ttfamily}

\ifpdf
\end{flushleft}
\fi

\end{list}
\ifpdf
\subsection*{\large{\textbf{strListallerAlien}}\normalsize\hspace{1ex}\hrulefill}
\else
\subsection*{strListallerAlien}
\fi
\label{trstrings-strListallerAlien}
\index{strListallerAlien}
\begin{list}{}{
\settowidth{\tmplength}{\textbf{Description}}
\setlength{\itemindent}{0cm}
\setlength{\listparindent}{0cm}
\setlength{\leftmargin}{\evensidemargin}
\addtolength{\leftmargin}{\tmplength}
\settowidth{\labelsep}{X}
\addtolength{\leftmargin}{\labelsep}
\setlength{\labelwidth}{\tmplength}
}
\item[\textbf{Declaration}\hfill]
\ifpdf
\begin{flushleft}
\fi
\begin{ttfamily}
strListallerAlien='Listaller uses "alien" to convert foreign packages, bit the tool is not installed'{\#}13'Do you want to install "alien" now to continue?';\end{ttfamily}

\ifpdf
\end{flushleft}
\fi

\end{list}
\ifpdf
\subsection*{\large{\textbf{strInstPkgQ}}\normalsize\hspace{1ex}\hrulefill}
\else
\subsection*{strInstPkgQ}
\fi
\label{trstrings-strInstPkgQ}
\index{strInstPkgQ}
\begin{list}{}{
\settowidth{\tmplength}{\textbf{Description}}
\setlength{\itemindent}{0cm}
\setlength{\listparindent}{0cm}
\setlength{\leftmargin}{\evensidemargin}
\addtolength{\leftmargin}{\tmplength}
\settowidth{\labelsep}{X}
\addtolength{\leftmargin}{\labelsep}
\setlength{\labelwidth}{\tmplength}
}
\item[\textbf{Declaration}\hfill]
\ifpdf
\begin{flushleft}
\fi
\begin{ttfamily}
strInstPkgQ='Install package?';\end{ttfamily}

\ifpdf
\end{flushleft}
\fi

\end{list}
\ifpdf
\subsection*{\large{\textbf{strPkgInstFail}}\normalsize\hspace{1ex}\hrulefill}
\else
\subsection*{strPkgInstFail}
\fi
\label{trstrings-strPkgInstFail}
\index{strPkgInstFail}
\begin{list}{}{
\settowidth{\tmplength}{\textbf{Description}}
\setlength{\itemindent}{0cm}
\setlength{\listparindent}{0cm}
\setlength{\leftmargin}{\evensidemargin}
\addtolength{\leftmargin}{\tmplength}
\settowidth{\labelsep}{X}
\addtolength{\leftmargin}{\labelsep}
\setlength{\labelwidth}{\tmplength}
}
\item[\textbf{Declaration}\hfill]
\ifpdf
\begin{flushleft}
\fi
\begin{ttfamily}
strPkgInstFail='Package {\%}p could not be installed.';\end{ttfamily}

\ifpdf
\end{flushleft}
\fi

\end{list}
\ifpdf
\subsection*{\large{\textbf{strCategory}}\normalsize\hspace{1ex}\hrulefill}
\else
\subsection*{strCategory}
\fi
\label{trstrings-strCategory}
\index{strCategory}
\begin{list}{}{
\settowidth{\tmplength}{\textbf{Description}}
\setlength{\itemindent}{0cm}
\setlength{\listparindent}{0cm}
\setlength{\leftmargin}{\evensidemargin}
\addtolength{\leftmargin}{\tmplength}
\settowidth{\labelsep}{X}
\addtolength{\leftmargin}{\labelsep}
\setlength{\labelwidth}{\tmplength}
}
\item[\textbf{Declaration}\hfill]
\ifpdf
\begin{flushleft}
\fi
\begin{ttfamily}
strCategory='Category:';\end{ttfamily}

\ifpdf
\end{flushleft}
\fi

\end{list}
\ifpdf
\subsection*{\large{\textbf{strWInstallDl}}\normalsize\hspace{1ex}\hrulefill}
\else
\subsection*{strWInstallDl}
\fi
\label{trstrings-strWInstallDl}
\index{strWInstallDl}
\begin{list}{}{
\settowidth{\tmplength}{\textbf{Description}}
\setlength{\itemindent}{0cm}
\setlength{\listparindent}{0cm}
\setlength{\leftmargin}{\evensidemargin}
\addtolength{\leftmargin}{\tmplength}
\settowidth{\labelsep}{X}
\addtolength{\leftmargin}{\labelsep}
\setlength{\labelwidth}{\tmplength}
}
\item[\textbf{Declaration}\hfill]
\ifpdf
\begin{flushleft}
\fi
\begin{ttfamily}
strWInstallDl='Select software you want to download and to install:';\end{ttfamily}

\ifpdf
\end{flushleft}
\fi

\end{list}
\ifpdf
\subsection*{\large{\textbf{strNoInfo}}\normalsize\hspace{1ex}\hrulefill}
\else
\subsection*{strNoInfo}
\fi
\label{trstrings-strNoInfo}
\index{strNoInfo}
\begin{list}{}{
\settowidth{\tmplength}{\textbf{Description}}
\setlength{\itemindent}{0cm}
\setlength{\listparindent}{0cm}
\setlength{\leftmargin}{\evensidemargin}
\addtolength{\leftmargin}{\tmplength}
\settowidth{\labelsep}{X}
\addtolength{\leftmargin}{\labelsep}
\setlength{\labelwidth}{\tmplength}
}
\item[\textbf{Declaration}\hfill]
\ifpdf
\begin{flushleft}
\fi
\begin{ttfamily}
strNoInfo='No information available!';\end{ttfamily}

\ifpdf
\end{flushleft}
\fi

\end{list}
\ifpdf
\subsection*{\large{\textbf{strDLSetUp}}\normalsize\hspace{1ex}\hrulefill}
\else
\subsection*{strDLSetUp}
\fi
\label{trstrings-strDLSetUp}
\index{strDLSetUp}
\begin{list}{}{
\settowidth{\tmplength}{\textbf{Description}}
\setlength{\itemindent}{0cm}
\setlength{\listparindent}{0cm}
\setlength{\leftmargin}{\evensidemargin}
\addtolength{\leftmargin}{\tmplength}
\settowidth{\labelsep}{X}
\addtolength{\leftmargin}{\labelsep}
\setlength{\labelwidth}{\tmplength}
}
\item[\textbf{Declaration}\hfill]
\ifpdf
\begin{flushleft}
\fi
\begin{ttfamily}
strDLSetUp='Downloading set-up...';\end{ttfamily}

\ifpdf
\end{flushleft}
\fi

\end{list}
\ifpdf
\subsection*{\large{\textbf{strErrContactMan}}\normalsize\hspace{1ex}\hrulefill}
\else
\subsection*{strErrContactMan}
\fi
\label{trstrings-strErrContactMan}
\index{strErrContactMan}
\begin{list}{}{
\settowidth{\tmplength}{\textbf{Description}}
\setlength{\itemindent}{0cm}
\setlength{\listparindent}{0cm}
\setlength{\leftmargin}{\evensidemargin}
\addtolength{\leftmargin}{\tmplength}
\settowidth{\labelsep}{X}
\addtolength{\leftmargin}{\labelsep}
\setlength{\labelwidth}{\tmplength}
}
\item[\textbf{Declaration}\hfill]
\ifpdf
\begin{flushleft}
\fi
\begin{ttfamily}
strErrContactMan='Cannot download this package. Please contact the catalogue managers on {\%}h';\end{ttfamily}

\ifpdf
\end{flushleft}
\fi

\end{list}
\ifpdf
\subsection*{\large{\textbf{strInstalling}}\normalsize\hspace{1ex}\hrulefill}
\else
\subsection*{strInstalling}
\fi
\label{trstrings-strInstalling}
\index{strInstalling}
\begin{list}{}{
\settowidth{\tmplength}{\textbf{Description}}
\setlength{\itemindent}{0cm}
\setlength{\listparindent}{0cm}
\setlength{\leftmargin}{\evensidemargin}
\addtolength{\leftmargin}{\tmplength}
\settowidth{\labelsep}{X}
\addtolength{\leftmargin}{\labelsep}
\setlength{\labelwidth}{\tmplength}
}
\item[\textbf{Declaration}\hfill]
\ifpdf
\begin{flushleft}
\fi
\begin{ttfamily}
strInstalling='Running application installation...';\end{ttfamily}

\ifpdf
\end{flushleft}
\fi

\end{list}
\ifpdf
\subsection*{\large{\textbf{strDownloadCTbase}}\normalsize\hspace{1ex}\hrulefill}
\else
\subsection*{strDownloadCTbase}
\fi
\label{trstrings-strDownloadCTbase}
\index{strDownloadCTbase}
\begin{list}{}{
\settowidth{\tmplength}{\textbf{Description}}
\setlength{\itemindent}{0cm}
\setlength{\listparindent}{0cm}
\setlength{\leftmargin}{\evensidemargin}
\addtolength{\leftmargin}{\tmplength}
\settowidth{\labelsep}{X}
\addtolength{\leftmargin}{\labelsep}
\setlength{\labelwidth}{\tmplength}
}
\item[\textbf{Declaration}\hfill]
\ifpdf
\begin{flushleft}
\fi
\begin{ttfamily}
strDownloadCTbase='Downloading catalogue base information...';\end{ttfamily}

\ifpdf
\end{flushleft}
\fi

\end{list}
\ifpdf
\subsection*{\large{\textbf{strOpenPage}}\normalsize\hspace{1ex}\hrulefill}
\else
\subsection*{strOpenPage}
\fi
\label{trstrings-strOpenPage}
\index{strOpenPage}
\begin{list}{}{
\settowidth{\tmplength}{\textbf{Description}}
\setlength{\itemindent}{0cm}
\setlength{\listparindent}{0cm}
\setlength{\leftmargin}{\evensidemargin}
\addtolength{\leftmargin}{\tmplength}
\settowidth{\labelsep}{X}
\addtolength{\leftmargin}{\labelsep}
\setlength{\labelwidth}{\tmplength}
}
\item[\textbf{Declaration}\hfill]
\ifpdf
\begin{flushleft}
\fi
\begin{ttfamily}
strOpenPage='Loading catalogue page...';\end{ttfamily}

\ifpdf
\end{flushleft}
\fi

\end{list}
\ifpdf
\subsection*{\large{\textbf{strctDLAbort}}\normalsize\hspace{1ex}\hrulefill}
\else
\subsection*{strctDLAbort}
\fi
\label{trstrings-strctDLAbort}
\index{strctDLAbort}
\begin{list}{}{
\settowidth{\tmplength}{\textbf{Description}}
\setlength{\itemindent}{0cm}
\setlength{\listparindent}{0cm}
\setlength{\leftmargin}{\evensidemargin}
\addtolength{\leftmargin}{\tmplength}
\settowidth{\labelsep}{X}
\addtolength{\leftmargin}{\labelsep}
\setlength{\labelwidth}{\tmplength}
}
\item[\textbf{Declaration}\hfill]
\ifpdf
\begin{flushleft}
\fi
\begin{ttfamily}
strctDLAbort='Do you really want to abort this download?';\end{ttfamily}

\ifpdf
\end{flushleft}
\fi

\end{list}
\ifpdf
\subsection*{\large{\textbf{strRealUninstQ}}\normalsize\hspace{1ex}\hrulefill}
\else
\subsection*{strRealUninstQ}
\fi
\label{trstrings-strRealUninstQ}
\index{strRealUninstQ}
\begin{list}{}{
\settowidth{\tmplength}{\textbf{Description}}
\setlength{\itemindent}{0cm}
\setlength{\listparindent}{0cm}
\setlength{\leftmargin}{\evensidemargin}
\addtolength{\leftmargin}{\tmplength}
\settowidth{\labelsep}{X}
\addtolength{\leftmargin}{\labelsep}
\setlength{\labelwidth}{\tmplength}
}
\item[\textbf{Declaration}\hfill]
\ifpdf
\begin{flushleft}
\fi
\begin{ttfamily}
strRealUninstQ='Do you really want to uninstall {\%}a?';\end{ttfamily}

\ifpdf
\end{flushleft}
\fi

\end{list}
\ifpdf
\subsection*{\large{\textbf{strUnistSuccess}}\normalsize\hspace{1ex}\hrulefill}
\else
\subsection*{strUnistSuccess}
\fi
\label{trstrings-strUnistSuccess}
\index{strUnistSuccess}
\begin{list}{}{
\settowidth{\tmplength}{\textbf{Description}}
\setlength{\itemindent}{0cm}
\setlength{\listparindent}{0cm}
\setlength{\leftmargin}{\evensidemargin}
\addtolength{\leftmargin}{\tmplength}
\settowidth{\labelsep}{X}
\addtolength{\leftmargin}{\labelsep}
\setlength{\labelwidth}{\tmplength}
}
\item[\textbf{Declaration}\hfill]
\ifpdf
\begin{flushleft}
\fi
\begin{ttfamily}
strUnistSuccess='Application uninstalled successfully!';\end{ttfamily}

\ifpdf
\end{flushleft}
\fi

\end{list}
\ifpdf
\subsection*{\large{\textbf{strRMerror}}\normalsize\hspace{1ex}\hrulefill}
\else
\subsection*{strRMerror}
\fi
\label{trstrings-strRMerror}
\index{strRMerror}
\begin{list}{}{
\settowidth{\tmplength}{\textbf{Description}}
\setlength{\itemindent}{0cm}
\setlength{\listparindent}{0cm}
\setlength{\leftmargin}{\evensidemargin}
\addtolength{\leftmargin}{\tmplength}
\settowidth{\labelsep}{X}
\addtolength{\leftmargin}{\labelsep}
\setlength{\labelwidth}{\tmplength}
}
\item[\textbf{Declaration}\hfill]
\ifpdf
\begin{flushleft}
\fi
\begin{ttfamily}
strRMerror='Error while uninstalling!';\end{ttfamily}

\ifpdf
\end{flushleft}
\fi

\end{list}
\ifpdf
\subsection*{\large{\textbf{strCannotHandleRM}}\normalsize\hspace{1ex}\hrulefill}
\else
\subsection*{strCannotHandleRM}
\fi
\label{trstrings-strCannotHandleRM}
\index{strCannotHandleRM}
\begin{list}{}{
\settowidth{\tmplength}{\textbf{Description}}
\setlength{\itemindent}{0cm}
\setlength{\listparindent}{0cm}
\setlength{\leftmargin}{\evensidemargin}
\addtolength{\leftmargin}{\tmplength}
\settowidth{\labelsep}{X}
\addtolength{\leftmargin}{\labelsep}
\setlength{\labelwidth}{\tmplength}
}
\item[\textbf{Declaration}\hfill]
\ifpdf
\begin{flushleft}
\fi
\begin{ttfamily}
strCannotHandleRM='This application is not a MoJo-Installation and no other package-type Listaller can handle.';\end{ttfamily}

\ifpdf
\end{flushleft}
\fi

\end{list}
\ifpdf
\subsection*{\large{\textbf{strRMUnsdDeps}}\normalsize\hspace{1ex}\hrulefill}
\else
\subsection*{strRMUnsdDeps}
\fi
\label{trstrings-strRMUnsdDeps}
\index{strRMUnsdDeps}
\begin{list}{}{
\settowidth{\tmplength}{\textbf{Description}}
\setlength{\itemindent}{0cm}
\setlength{\listparindent}{0cm}
\setlength{\leftmargin}{\evensidemargin}
\addtolength{\leftmargin}{\tmplength}
\settowidth{\labelsep}{X}
\addtolength{\leftmargin}{\labelsep}
\setlength{\labelwidth}{\tmplength}
}
\item[\textbf{Declaration}\hfill]
\ifpdf
\begin{flushleft}
\fi
\begin{ttfamily}
strRMUnsdDeps='Uninstalling unused deps...';\end{ttfamily}

\ifpdf
\end{flushleft}
\fi

\end{list}
\ifpdf
\subsection*{\large{\textbf{strUninstalling}}\normalsize\hspace{1ex}\hrulefill}
\else
\subsection*{strUninstalling}
\fi
\label{trstrings-strUninstalling}
\index{strUninstalling}
\begin{list}{}{
\settowidth{\tmplength}{\textbf{Description}}
\setlength{\itemindent}{0cm}
\setlength{\listparindent}{0cm}
\setlength{\leftmargin}{\evensidemargin}
\addtolength{\leftmargin}{\tmplength}
\settowidth{\labelsep}{X}
\addtolength{\leftmargin}{\labelsep}
\setlength{\labelwidth}{\tmplength}
}
\item[\textbf{Declaration}\hfill]
\ifpdf
\begin{flushleft}
\fi
\begin{ttfamily}
strUninstalling='Uninstalling...';\end{ttfamily}

\ifpdf
\end{flushleft}
\fi

\end{list}
\ifpdf
\subsection*{\large{\textbf{strRMPkg}}\normalsize\hspace{1ex}\hrulefill}
\else
\subsection*{strRMPkg}
\fi
\label{trstrings-strRMPkg}
\index{strRMPkg}
\begin{list}{}{
\settowidth{\tmplength}{\textbf{Description}}
\setlength{\itemindent}{0cm}
\setlength{\listparindent}{0cm}
\setlength{\leftmargin}{\evensidemargin}
\addtolength{\leftmargin}{\tmplength}
\settowidth{\labelsep}{X}
\addtolength{\leftmargin}{\labelsep}
\setlength{\labelwidth}{\tmplength}
}
\item[\textbf{Declaration}\hfill]
\ifpdf
\begin{flushleft}
\fi
\begin{ttfamily}
strRMPkg='Do you really want to remove "{\%}p", containing {\%}a?'{\#}13'The following package(s) will be removed also: {\%}pl'{\#}13'If you aren''t sure that you won''t need those packages, press "No"!';\end{ttfamily}

\ifpdf
\end{flushleft}
\fi

\end{list}
\ifpdf
\subsection*{\large{\textbf{strRmPkgQ}}\normalsize\hspace{1ex}\hrulefill}
\else
\subsection*{strRmPkgQ}
\fi
\label{trstrings-strRmPkgQ}
\index{strRmPkgQ}
\begin{list}{}{
\settowidth{\tmplength}{\textbf{Description}}
\setlength{\itemindent}{0cm}
\setlength{\listparindent}{0cm}
\setlength{\leftmargin}{\evensidemargin}
\addtolength{\leftmargin}{\tmplength}
\settowidth{\labelsep}{X}
\addtolength{\leftmargin}{\labelsep}
\setlength{\labelwidth}{\tmplength}
}
\item[\textbf{Declaration}\hfill]
\ifpdf
\begin{flushleft}
\fi
\begin{ttfamily}
strRmPkgQ='Really remove?';\end{ttfamily}

\ifpdf
\end{flushleft}
\fi

\end{list}
\ifpdf
\subsection*{\large{\textbf{strWaiting}}\normalsize\hspace{1ex}\hrulefill}
\else
\subsection*{strWaiting}
\fi
\label{trstrings-strWaiting}
\index{strWaiting}
\begin{list}{}{
\settowidth{\tmplength}{\textbf{Description}}
\setlength{\itemindent}{0cm}
\setlength{\listparindent}{0cm}
\setlength{\leftmargin}{\evensidemargin}
\addtolength{\leftmargin}{\tmplength}
\settowidth{\labelsep}{X}
\addtolength{\leftmargin}{\labelsep}
\setlength{\labelwidth}{\tmplength}
}
\item[\textbf{Declaration}\hfill]
\ifpdf
\begin{flushleft}
\fi
\begin{ttfamily}
strWaiting='Waiting...';\end{ttfamily}

\ifpdf
\end{flushleft}
\fi

\end{list}
\ifpdf
\subsection*{\large{\textbf{strRMAppC}}\normalsize\hspace{1ex}\hrulefill}
\else
\subsection*{strRMAppC}
\fi
\label{trstrings-strRMAppC}
\index{strRMAppC}
\begin{list}{}{
\settowidth{\tmplength}{\textbf{Description}}
\setlength{\itemindent}{0cm}
\setlength{\listparindent}{0cm}
\setlength{\leftmargin}{\evensidemargin}
\addtolength{\leftmargin}{\tmplength}
\settowidth{\labelsep}{X}
\addtolength{\leftmargin}{\labelsep}
\setlength{\labelwidth}{\tmplength}
}
\item[\textbf{Declaration}\hfill]
\ifpdf
\begin{flushleft}
\fi
\begin{ttfamily}
strRMAppC='Uninstalling {\%}a';\end{ttfamily}

\ifpdf
\end{flushleft}
\fi

\end{list}
\ifpdf
\subsection*{\large{\textbf{strNoUpdates}}\normalsize\hspace{1ex}\hrulefill}
\else
\subsection*{strNoUpdates}
\fi
\label{trstrings-strNoUpdates}
\index{strNoUpdates}
\begin{list}{}{
\settowidth{\tmplength}{\textbf{Description}}
\setlength{\itemindent}{0cm}
\setlength{\listparindent}{0cm}
\setlength{\leftmargin}{\evensidemargin}
\addtolength{\leftmargin}{\tmplength}
\settowidth{\labelsep}{X}
\addtolength{\leftmargin}{\labelsep}
\setlength{\labelwidth}{\tmplength}
}
\item[\textbf{Declaration}\hfill]
\ifpdf
\begin{flushleft}
\fi
\begin{ttfamily}
strNoUpdates='There are no updates available!';\end{ttfamily}

\ifpdf
\end{flushleft}
\fi

\end{list}
\ifpdf
\subsection*{\large{\textbf{strLogUpdInfo}}\normalsize\hspace{1ex}\hrulefill}
\else
\subsection*{strLogUpdInfo}
\fi
\label{trstrings-strLogUpdInfo}
\index{strLogUpdInfo}
\begin{list}{}{
\settowidth{\tmplength}{\textbf{Description}}
\setlength{\itemindent}{0cm}
\setlength{\listparindent}{0cm}
\setlength{\leftmargin}{\evensidemargin}
\addtolength{\leftmargin}{\tmplength}
\settowidth{\labelsep}{X}
\addtolength{\leftmargin}{\labelsep}
\setlength{\labelwidth}{\tmplength}
}
\item[\textbf{Declaration}\hfill]
\ifpdf
\begin{flushleft}
\fi
\begin{ttfamily}
strLogUpdInfo='Update info:';\end{ttfamily}

\ifpdf
\end{flushleft}
\fi

\end{list}
\ifpdf
\subsection*{\large{\textbf{strFilesChanged}}\normalsize\hspace{1ex}\hrulefill}
\else
\subsection*{strFilesChanged}
\fi
\label{trstrings-strFilesChanged}
\index{strFilesChanged}
\begin{list}{}{
\settowidth{\tmplength}{\textbf{Description}}
\setlength{\itemindent}{0cm}
\setlength{\listparindent}{0cm}
\setlength{\leftmargin}{\evensidemargin}
\addtolength{\leftmargin}{\tmplength}
\settowidth{\labelsep}{X}
\addtolength{\leftmargin}{\labelsep}
\setlength{\labelwidth}{\tmplength}
}
\item[\textbf{Declaration}\hfill]
\ifpdf
\begin{flushleft}
\fi
\begin{ttfamily}
strFilesChanged='{\%}f files will be changed.';\end{ttfamily}

\ifpdf
\end{flushleft}
\fi

\end{list}
\ifpdf
\subsection*{\large{\textbf{strUpdTo}}\normalsize\hspace{1ex}\hrulefill}
\else
\subsection*{strUpdTo}
\fi
\label{trstrings-strUpdTo}
\index{strUpdTo}
\begin{list}{}{
\settowidth{\tmplength}{\textbf{Description}}
\setlength{\itemindent}{0cm}
\setlength{\listparindent}{0cm}
\setlength{\leftmargin}{\evensidemargin}
\addtolength{\leftmargin}{\tmplength}
\settowidth{\labelsep}{X}
\addtolength{\leftmargin}{\labelsep}
\setlength{\labelwidth}{\tmplength}
}
\item[\textbf{Declaration}\hfill]
\ifpdf
\begin{flushleft}
\fi
\begin{ttfamily}
strUpdTo='The application will be updated to version {\%}v';\end{ttfamily}

\ifpdf
\end{flushleft}
\fi

\end{list}
\ifpdf
\subsection*{\large{\textbf{strCheckForUpd}}\normalsize\hspace{1ex}\hrulefill}
\else
\subsection*{strCheckForUpd}
\fi
\label{trstrings-strCheckForUpd}
\index{strCheckForUpd}
\begin{list}{}{
\settowidth{\tmplength}{\textbf{Description}}
\setlength{\itemindent}{0cm}
\setlength{\listparindent}{0cm}
\setlength{\leftmargin}{\evensidemargin}
\addtolength{\leftmargin}{\tmplength}
\settowidth{\labelsep}{X}
\addtolength{\leftmargin}{\labelsep}
\setlength{\labelwidth}{\tmplength}
}
\item[\textbf{Declaration}\hfill]
\ifpdf
\begin{flushleft}
\fi
\begin{ttfamily}
strCheckForUpd='Check for updates';\end{ttfamily}

\ifpdf
\end{flushleft}
\fi

\end{list}
\ifpdf
\subsection*{\large{\textbf{strInstUpd}}\normalsize\hspace{1ex}\hrulefill}
\else
\subsection*{strInstUpd}
\fi
\label{trstrings-strInstUpd}
\index{strInstUpd}
\begin{list}{}{
\settowidth{\tmplength}{\textbf{Description}}
\setlength{\itemindent}{0cm}
\setlength{\listparindent}{0cm}
\setlength{\leftmargin}{\evensidemargin}
\addtolength{\leftmargin}{\tmplength}
\settowidth{\labelsep}{X}
\addtolength{\leftmargin}{\labelsep}
\setlength{\labelwidth}{\tmplength}
}
\item[\textbf{Declaration}\hfill]
\ifpdf
\begin{flushleft}
\fi
\begin{ttfamily}
strInstUpd='Install updates';\end{ttfamily}

\ifpdf
\end{flushleft}
\fi

\end{list}
\ifpdf
\subsection*{\large{\textbf{strShowUpdater}}\normalsize\hspace{1ex}\hrulefill}
\else
\subsection*{strShowUpdater}
\fi
\label{trstrings-strShowUpdater}
\index{strShowUpdater}
\begin{list}{}{
\settowidth{\tmplength}{\textbf{Description}}
\setlength{\itemindent}{0cm}
\setlength{\listparindent}{0cm}
\setlength{\leftmargin}{\evensidemargin}
\addtolength{\leftmargin}{\tmplength}
\settowidth{\labelsep}{X}
\addtolength{\leftmargin}{\labelsep}
\setlength{\labelwidth}{\tmplength}
}
\item[\textbf{Declaration}\hfill]
\ifpdf
\begin{flushleft}
\fi
\begin{ttfamily}
strShowUpdater='Show';\end{ttfamily}

\ifpdf
\end{flushleft}
\fi

\end{list}
\ifpdf
\subsection*{\large{\textbf{strQuitUpdater}}\normalsize\hspace{1ex}\hrulefill}
\else
\subsection*{strQuitUpdater}
\fi
\label{trstrings-strQuitUpdater}
\index{strQuitUpdater}
\begin{list}{}{
\settowidth{\tmplength}{\textbf{Description}}
\setlength{\itemindent}{0cm}
\setlength{\listparindent}{0cm}
\setlength{\leftmargin}{\evensidemargin}
\addtolength{\leftmargin}{\tmplength}
\settowidth{\labelsep}{X}
\addtolength{\leftmargin}{\labelsep}
\setlength{\labelwidth}{\tmplength}
}
\item[\textbf{Declaration}\hfill]
\ifpdf
\begin{flushleft}
\fi
\begin{ttfamily}
strQuitUpdater='Quit';\end{ttfamily}

\ifpdf
\end{flushleft}
\fi

\end{list}
\ifpdf
\subsection*{\large{\textbf{strUpdInstalling}}\normalsize\hspace{1ex}\hrulefill}
\else
\subsection*{strUpdInstalling}
\fi
\label{trstrings-strUpdInstalling}
\index{strUpdInstalling}
\begin{list}{}{
\settowidth{\tmplength}{\textbf{Description}}
\setlength{\itemindent}{0cm}
\setlength{\listparindent}{0cm}
\setlength{\leftmargin}{\evensidemargin}
\addtolength{\leftmargin}{\tmplength}
\settowidth{\labelsep}{X}
\addtolength{\leftmargin}{\labelsep}
\setlength{\labelwidth}{\tmplength}
}
\item[\textbf{Declaration}\hfill]
\ifpdf
\begin{flushleft}
\fi
\begin{ttfamily}
strUpdInstalling='Installing updates...';\end{ttfamily}

\ifpdf
\end{flushleft}
\fi

\end{list}
\chapter{Unit uninstall}
\label{uninstall}
\index{uninstall}
\section{Description}
Window that shows the progress while uninstalling applications
\section{uses}
\begin{itemize}
\item \begin{ttfamily}Classes\end{ttfamily}\item \begin{ttfamily}SysUtils\end{ttfamily}\item \begin{ttfamily}LResources\end{ttfamily}\item \begin{ttfamily}Forms\end{ttfamily}\item \begin{ttfamily}Controls\end{ttfamily}\item \begin{ttfamily}Graphics\end{ttfamily}\item \begin{ttfamily}Dialogs\end{ttfamily}\item \begin{ttfamily}ComCtrls\end{ttfamily}\item \begin{ttfamily}StdCtrls\end{ttfamily}\item \begin{ttfamily}IniFiles\end{ttfamily}\item \begin{ttfamily}LCLType\end{ttfamily}\item \begin{ttfamily}utilities\end{ttfamily}(\ref{utilities})\item \begin{ttfamily}Buttons\end{ttfamily}\item \begin{ttfamily}ExtCtrls\end{ttfamily}\item \begin{ttfamily}process\end{ttfamily}\item \begin{ttfamily}trstrings\end{ttfamily}(\ref{trstrings})\item \begin{ttfamily}FileUtil\end{ttfamily}\item \begin{ttfamily}distri\end{ttfamily}(\ref{distri})\item \begin{ttfamily}ipkhandle\end{ttfamily}(\ref{ipkhandle})\end{itemize}
\section{Overview}
\begin{description}
\item[\texttt{\begin{ttfamily}TRMForm\end{ttfamily} Class}]
\end{description}
\section{Classes, Interfaces, Objects and Records}
\ifpdf
\subsection*{\large{\textbf{TRMForm Class}}\normalsize\hspace{1ex}\hrulefill}
\else
\subsection*{TRMForm Class}
\fi
\label{uninstall.TRMForm}
\index{TRMForm}
\subsubsection*{\large{\textbf{Hierarchy}}\normalsize\hspace{1ex}\hfill}
TRMForm {$>$} TForm
%%%%Description
\subsubsection*{\large{\textbf{Fields}}\normalsize\hspace{1ex}\hfill}
\begin{list}{}{
\settowidth{\tmplength}{\textbf{GetOutPutTimer}}
\setlength{\itemindent}{0cm}
\setlength{\listparindent}{0cm}
\setlength{\leftmargin}{\evensidemargin}
\addtolength{\leftmargin}{\tmplength}
\settowidth{\labelsep}{X}
\addtolength{\leftmargin}{\labelsep}
\setlength{\labelwidth}{\tmplength}
}
\label{uninstall.TRMForm-BitBtn1}
\index{BitBtn1}
\item[\textbf{BitBtn1}\hfill]
\ifpdf
\begin{flushleft}
\fi
\begin{ttfamily}
public BitBtn1: TBitBtn;\end{ttfamily}

\ifpdf
\end{flushleft}
\fi


\par  \label{uninstall.TRMForm-GetOutPutTimer}
\index{GetOutPutTimer}
\item[\textbf{GetOutPutTimer}\hfill]
\ifpdf
\begin{flushleft}
\fi
\begin{ttfamily}
public GetOutPutTimer: TIdleTimer;\end{ttfamily}

\ifpdf
\end{flushleft}
\fi


\par  \label{uninstall.TRMForm-Label1}
\index{Label1}
\item[\textbf{Label1}\hfill]
\ifpdf
\begin{flushleft}
\fi
\begin{ttfamily}
public Label1: TLabel;\end{ttfamily}

\ifpdf
\end{flushleft}
\fi


\par  \label{uninstall.TRMForm-Label2}
\index{Label2}
\item[\textbf{Label2}\hfill]
\ifpdf
\begin{flushleft}
\fi
\begin{ttfamily}
public Label2: TLabel;\end{ttfamily}

\ifpdf
\end{flushleft}
\fi


\par  \label{uninstall.TRMForm-Memo1}
\index{Memo1}
\item[\textbf{Memo1}\hfill]
\ifpdf
\begin{flushleft}
\fi
\begin{ttfamily}
public Memo1: TMemo;\end{ttfamily}

\ifpdf
\end{flushleft}
\fi


\par  \label{uninstall.TRMForm-Process1}
\index{Process1}
\item[\textbf{Process1}\hfill]
\ifpdf
\begin{flushleft}
\fi
\begin{ttfamily}
public Process1: TProcess;\end{ttfamily}

\ifpdf
\end{flushleft}
\fi


\par  \label{uninstall.TRMForm-UProgress}
\index{UProgress}
\item[\textbf{UProgress}\hfill]
\ifpdf
\begin{flushleft}
\fi
\begin{ttfamily}
public UProgress: TProgressBar;\end{ttfamily}

\ifpdf
\end{flushleft}
\fi


\par  \end{list}
\subsubsection*{\large{\textbf{Methods}}\normalsize\hspace{1ex}\hfill}
\paragraph*{BitBtn1Click}\hspace*{\fill}

\label{uninstall.TRMForm-BitBtn1Click}
\index{BitBtn1Click}
\begin{list}{}{
\settowidth{\tmplength}{\textbf{Description}}
\setlength{\itemindent}{0cm}
\setlength{\listparindent}{0cm}
\setlength{\leftmargin}{\evensidemargin}
\addtolength{\leftmargin}{\tmplength}
\settowidth{\labelsep}{X}
\addtolength{\leftmargin}{\labelsep}
\setlength{\labelwidth}{\tmplength}
}
\item[\textbf{Declaration}\hfill]
\ifpdf
\begin{flushleft}
\fi
\begin{ttfamily}
public procedure BitBtn1Click(Sender: TObject);\end{ttfamily}

\ifpdf
\end{flushleft}
\fi

\end{list}
\paragraph*{FormActivate}\hspace*{\fill}

\label{uninstall.TRMForm-FormActivate}
\index{FormActivate}
\begin{list}{}{
\settowidth{\tmplength}{\textbf{Description}}
\setlength{\itemindent}{0cm}
\setlength{\listparindent}{0cm}
\setlength{\leftmargin}{\evensidemargin}
\addtolength{\leftmargin}{\tmplength}
\settowidth{\labelsep}{X}
\addtolength{\leftmargin}{\labelsep}
\setlength{\labelwidth}{\tmplength}
}
\item[\textbf{Declaration}\hfill]
\ifpdf
\begin{flushleft}
\fi
\begin{ttfamily}
public procedure FormActivate(Sender: TObject);\end{ttfamily}

\ifpdf
\end{flushleft}
\fi

\end{list}
\paragraph*{FormClose}\hspace*{\fill}

\label{uninstall.TRMForm-FormClose}
\index{FormClose}
\begin{list}{}{
\settowidth{\tmplength}{\textbf{Description}}
\setlength{\itemindent}{0cm}
\setlength{\listparindent}{0cm}
\setlength{\leftmargin}{\evensidemargin}
\addtolength{\leftmargin}{\tmplength}
\settowidth{\labelsep}{X}
\addtolength{\leftmargin}{\labelsep}
\setlength{\labelwidth}{\tmplength}
}
\item[\textbf{Declaration}\hfill]
\ifpdf
\begin{flushleft}
\fi
\begin{ttfamily}
public procedure FormClose(Sender: TObject; var CloseAction: TCloseAction);\end{ttfamily}

\ifpdf
\end{flushleft}
\fi

\end{list}
\paragraph*{FormCreate}\hspace*{\fill}

\label{uninstall.TRMForm-FormCreate}
\index{FormCreate}
\begin{list}{}{
\settowidth{\tmplength}{\textbf{Description}}
\setlength{\itemindent}{0cm}
\setlength{\listparindent}{0cm}
\setlength{\leftmargin}{\evensidemargin}
\addtolength{\leftmargin}{\tmplength}
\settowidth{\labelsep}{X}
\addtolength{\leftmargin}{\labelsep}
\setlength{\labelwidth}{\tmplength}
}
\item[\textbf{Declaration}\hfill]
\ifpdf
\begin{flushleft}
\fi
\begin{ttfamily}
public procedure FormCreate(Sender: TObject);\end{ttfamily}

\ifpdf
\end{flushleft}
\fi

\end{list}
\paragraph*{FormShow}\hspace*{\fill}

\label{uninstall.TRMForm-FormShow}
\index{FormShow}
\begin{list}{}{
\settowidth{\tmplength}{\textbf{Description}}
\setlength{\itemindent}{0cm}
\setlength{\listparindent}{0cm}
\setlength{\leftmargin}{\evensidemargin}
\addtolength{\leftmargin}{\tmplength}
\settowidth{\labelsep}{X}
\addtolength{\leftmargin}{\labelsep}
\setlength{\labelwidth}{\tmplength}
}
\item[\textbf{Declaration}\hfill]
\ifpdf
\begin{flushleft}
\fi
\begin{ttfamily}
public procedure FormShow(Sender: TObject);\end{ttfamily}

\ifpdf
\end{flushleft}
\fi

\end{list}
\paragraph*{GetOutPutTimerTimer}\hspace*{\fill}

\label{uninstall.TRMForm-GetOutPutTimerTimer}
\index{GetOutPutTimerTimer}
\begin{list}{}{
\settowidth{\tmplength}{\textbf{Description}}
\setlength{\itemindent}{0cm}
\setlength{\listparindent}{0cm}
\setlength{\leftmargin}{\evensidemargin}
\addtolength{\leftmargin}{\tmplength}
\settowidth{\labelsep}{X}
\addtolength{\leftmargin}{\labelsep}
\setlength{\labelwidth}{\tmplength}
}
\item[\textbf{Declaration}\hfill]
\ifpdf
\begin{flushleft}
\fi
\begin{ttfamily}
public procedure GetOutPutTimerTimer(Sender: TObject);\end{ttfamily}

\ifpdf
\end{flushleft}
\fi

\end{list}
\section{Variables}
\ifpdf
\subsection*{\large{\textbf{RMForm}}\normalsize\hspace{1ex}\hrulefill}
\else
\subsection*{RMForm}
\fi
\label{uninstall-RMForm}
\index{RMForm}
\begin{list}{}{
\settowidth{\tmplength}{\textbf{Description}}
\setlength{\itemindent}{0cm}
\setlength{\listparindent}{0cm}
\setlength{\leftmargin}{\evensidemargin}
\addtolength{\leftmargin}{\tmplength}
\settowidth{\labelsep}{X}
\addtolength{\leftmargin}{\labelsep}
\setlength{\labelwidth}{\tmplength}
}
\item[\textbf{Declaration}\hfill]
\ifpdf
\begin{flushleft}
\fi
\begin{ttfamily}
RMForm: TRMForm;\end{ttfamily}

\ifpdf
\end{flushleft}
\fi

\par
\item[\textbf{Description}]
Window that shows the progress of the uninstallation

\end{list}
\chapter{Unit updexec}
\label{updexec}
\index{updexec}
\section{Description}
Executes software updates (GUI based)
\section{uses}
\begin{itemize}
\item \begin{ttfamily}Classes\end{ttfamily}\item \begin{ttfamily}SysUtils\end{ttfamily}\item \begin{ttfamily}LResources\end{ttfamily}\item \begin{ttfamily}Forms\end{ttfamily}\item \begin{ttfamily}Controls\end{ttfamily}\item \begin{ttfamily}Graphics\end{ttfamily}\item \begin{ttfamily}Dialogs\end{ttfamily}\item \begin{ttfamily}StdCtrls\end{ttfamily}\item \begin{ttfamily}ComCtrls\end{ttfamily}\item \begin{ttfamily}HTTPSend\end{ttfamily}(\ref{httpsend})\item \begin{ttfamily}FileUtil\end{ttfamily}\item \begin{ttfamily}AbUnZper\end{ttfamily}\item \begin{ttfamily}AbArcTyp\end{ttfamily}\item \begin{ttfamily}Process\end{ttfamily}\item \begin{ttfamily}utilities\end{ttfamily}(\ref{utilities})\item \begin{ttfamily}IniFiles\end{ttfamily}\item \begin{ttfamily}blcksock\end{ttfamily}\item \begin{ttfamily}trStrings\end{ttfamily}(\ref{trstrings})\item \begin{ttfamily}ipkhandle\end{ttfamily}(\ref{ipkhandle})\end{itemize}
\section{Overview}
\begin{description}
\item[\texttt{\begin{ttfamily}TUExecFm\end{ttfamily} Class}]
\end{description}
\section{Classes, Interfaces, Objects and Records}
\ifpdf
\subsection*{\large{\textbf{TUExecFm Class}}\normalsize\hspace{1ex}\hrulefill}
\else
\subsection*{TUExecFm Class}
\fi
\label{updexec.TUExecFm}
\index{TUExecFm}
\subsubsection*{\large{\textbf{Hierarchy}}\normalsize\hspace{1ex}\hfill}
TUExecFm {$>$} TForm
%%%%Description
\subsubsection*{\large{\textbf{Fields}}\normalsize\hspace{1ex}\hfill}
\begin{list}{}{
\settowidth{\tmplength}{\textbf{ProgressBar1}}
\setlength{\itemindent}{0cm}
\setlength{\listparindent}{0cm}
\setlength{\leftmargin}{\evensidemargin}
\addtolength{\leftmargin}{\tmplength}
\settowidth{\labelsep}{X}
\addtolength{\leftmargin}{\labelsep}
\setlength{\labelwidth}{\tmplength}
}
\label{updexec.TUExecFm-Button1}
\index{Button1}
\item[\textbf{Button1}\hfill]
\ifpdf
\begin{flushleft}
\fi
\begin{ttfamily}
public Button1: TButton;\end{ttfamily}

\ifpdf
\end{flushleft}
\fi


\par  \label{updexec.TUExecFm-ILabel}
\index{ILabel}
\item[\textbf{ILabel}\hfill]
\ifpdf
\begin{flushleft}
\fi
\begin{ttfamily}
public ILabel: TLabel;\end{ttfamily}

\ifpdf
\end{flushleft}
\fi


\par  \label{updexec.TUExecFm-Memo1}
\index{Memo1}
\item[\textbf{Memo1}\hfill]
\ifpdf
\begin{flushleft}
\fi
\begin{ttfamily}
public Memo1: TMemo;\end{ttfamily}

\ifpdf
\end{flushleft}
\fi


\par  \label{updexec.TUExecFm-ProgressBar1}
\index{ProgressBar1}
\item[\textbf{ProgressBar1}\hfill]
\ifpdf
\begin{flushleft}
\fi
\begin{ttfamily}
public ProgressBar1: TProgressBar;\end{ttfamily}

\ifpdf
\end{flushleft}
\fi


\par  \label{updexec.TUExecFm-ProgressBar2}
\index{ProgressBar2}
\item[\textbf{ProgressBar2}\hfill]
\ifpdf
\begin{flushleft}
\fi
\begin{ttfamily}
public ProgressBar2: TProgressBar;\end{ttfamily}

\ifpdf
\end{flushleft}
\fi


\par  \end{list}
\subsubsection*{\large{\textbf{Methods}}\normalsize\hspace{1ex}\hfill}
\paragraph*{Button1Click}\hspace*{\fill}

\label{updexec.TUExecFm-Button1Click}
\index{Button1Click}
\begin{list}{}{
\settowidth{\tmplength}{\textbf{Description}}
\setlength{\itemindent}{0cm}
\setlength{\listparindent}{0cm}
\setlength{\leftmargin}{\evensidemargin}
\addtolength{\leftmargin}{\tmplength}
\settowidth{\labelsep}{X}
\addtolength{\leftmargin}{\labelsep}
\setlength{\labelwidth}{\tmplength}
}
\item[\textbf{Declaration}\hfill]
\ifpdf
\begin{flushleft}
\fi
\begin{ttfamily}
public procedure Button1Click(Sender: TObject);\end{ttfamily}

\ifpdf
\end{flushleft}
\fi

\end{list}
\paragraph*{FormActivate}\hspace*{\fill}

\label{updexec.TUExecFm-FormActivate}
\index{FormActivate}
\begin{list}{}{
\settowidth{\tmplength}{\textbf{Description}}
\setlength{\itemindent}{0cm}
\setlength{\listparindent}{0cm}
\setlength{\leftmargin}{\evensidemargin}
\addtolength{\leftmargin}{\tmplength}
\settowidth{\labelsep}{X}
\addtolength{\leftmargin}{\labelsep}
\setlength{\labelwidth}{\tmplength}
}
\item[\textbf{Declaration}\hfill]
\ifpdf
\begin{flushleft}
\fi
\begin{ttfamily}
public procedure FormActivate(Sender: TObject);\end{ttfamily}

\ifpdf
\end{flushleft}
\fi

\end{list}
\paragraph*{FormClose}\hspace*{\fill}

\label{updexec.TUExecFm-FormClose}
\index{FormClose}
\begin{list}{}{
\settowidth{\tmplength}{\textbf{Description}}
\setlength{\itemindent}{0cm}
\setlength{\listparindent}{0cm}
\setlength{\leftmargin}{\evensidemargin}
\addtolength{\leftmargin}{\tmplength}
\settowidth{\labelsep}{X}
\addtolength{\leftmargin}{\labelsep}
\setlength{\labelwidth}{\tmplength}
}
\item[\textbf{Declaration}\hfill]
\ifpdf
\begin{flushleft}
\fi
\begin{ttfamily}
public procedure FormClose(Sender: TObject; var CloseAction: TCloseAction);\end{ttfamily}

\ifpdf
\end{flushleft}
\fi

\end{list}
\paragraph*{FormCreate}\hspace*{\fill}

\label{updexec.TUExecFm-FormCreate}
\index{FormCreate}
\begin{list}{}{
\settowidth{\tmplength}{\textbf{Description}}
\setlength{\itemindent}{0cm}
\setlength{\listparindent}{0cm}
\setlength{\leftmargin}{\evensidemargin}
\addtolength{\leftmargin}{\tmplength}
\settowidth{\labelsep}{X}
\addtolength{\leftmargin}{\labelsep}
\setlength{\labelwidth}{\tmplength}
}
\item[\textbf{Declaration}\hfill]
\ifpdf
\begin{flushleft}
\fi
\begin{ttfamily}
public procedure FormCreate(Sender: TObject);\end{ttfamily}

\ifpdf
\end{flushleft}
\fi

\end{list}
\paragraph*{FormShow}\hspace*{\fill}

\label{updexec.TUExecFm-FormShow}
\index{FormShow}
\begin{list}{}{
\settowidth{\tmplength}{\textbf{Description}}
\setlength{\itemindent}{0cm}
\setlength{\listparindent}{0cm}
\setlength{\leftmargin}{\evensidemargin}
\addtolength{\leftmargin}{\tmplength}
\settowidth{\labelsep}{X}
\addtolength{\leftmargin}{\labelsep}
\setlength{\labelwidth}{\tmplength}
}
\item[\textbf{Declaration}\hfill]
\ifpdf
\begin{flushleft}
\fi
\begin{ttfamily}
public procedure FormShow(Sender: TObject);\end{ttfamily}

\ifpdf
\end{flushleft}
\fi

\end{list}
\section{Variables}
\ifpdf
\subsection*{\large{\textbf{UExecFm}}\normalsize\hspace{1ex}\hrulefill}
\else
\subsection*{UExecFm}
\fi
\label{updexec-UExecFm}
\index{UExecFm}
\begin{list}{}{
\settowidth{\tmplength}{\textbf{Description}}
\setlength{\itemindent}{0cm}
\setlength{\listparindent}{0cm}
\setlength{\leftmargin}{\evensidemargin}
\addtolength{\leftmargin}{\tmplength}
\settowidth{\labelsep}{X}
\addtolength{\leftmargin}{\labelsep}
\setlength{\labelwidth}{\tmplength}
}
\item[\textbf{Declaration}\hfill]
\ifpdf
\begin{flushleft}
\fi
\begin{ttfamily}
UExecFm: TUExecFm;\end{ttfamily}

\ifpdf
\end{flushleft}
\fi

\end{list}
\chapter{Unit utilities}
\label{utilities}
\index{utilities}
\section{Description}
Here are some global functions which are used everywhere
\section{uses}
\begin{itemize}
\item \begin{ttfamily}Classes\end{ttfamily}\item \begin{ttfamily}SysUtils\end{ttfamily}\item \begin{ttfamily}LResources\end{ttfamily}\item \begin{ttfamily}Forms\end{ttfamily}\item \begin{ttfamily}Controls\end{ttfamily}\item \begin{ttfamily}Graphics\end{ttfamily}\item \begin{ttfamily}Dialogs\end{ttfamily}\item \begin{ttfamily}ComCtrls\end{ttfamily}\item \begin{ttfamily}StdCtrls\end{ttfamily}\item \begin{ttfamily}FileUtil\end{ttfamily}\item \begin{ttfamily}ExtCtrls\end{ttfamily}\item \begin{ttfamily}process\end{ttfamily}\item \begin{ttfamily}Buttons\end{ttfamily}\item \begin{ttfamily}LCLType\end{ttfamily}\item \begin{ttfamily}LCLIntf\end{ttfamily}\item \begin{ttfamily}RegExpr\end{ttfamily}(\ref{RegExpr})\item \begin{ttfamily}trstrings\end{ttfamily}(\ref{trstrings})\item \begin{ttfamily}distri\end{ttfamily}(\ref{distri})\end{itemize}
\section{Overview}
\begin{description}
\item[\texttt{\begin{ttfamily}TListEntry\end{ttfamily} Class}]
\end{description}
\begin{description}
\item[\texttt{DeleteModifiers}]
\item[\texttt{SyblToPath}]
\item[\texttt{SyblToX}]
\item[\texttt{IsSharedFile}]
\item[\texttt{ConfigDir}]
\item[\texttt{CmdResult}]
\item[\texttt{CmdResultCode}]
\item[\texttt{CmdResultList}]
\item[\texttt{CmdFinResult}]
\item[\texttt{FileCopy}]
\item[\texttt{Gtk2LoadStockPixmap}]
\item[\texttt{GetServerName}]
\item[\texttt{GetServerPath}]
\item[\texttt{IsInList}]
\end{description}
\section{Classes, Interfaces, Objects and Records}
\ifpdf
\subsection*{\large{\textbf{TListEntry Class}}\normalsize\hspace{1ex}\hrulefill}
\else
\subsection*{TListEntry Class}
\fi
\label{utilities.TListEntry}
\index{TListEntry}
\subsubsection*{\large{\textbf{Hierarchy}}\normalsize\hspace{1ex}\hfill}
TListEntry {$>$} TPanel
\subsubsection*{\large{\textbf{Description}}\normalsize\hspace{1ex}\hfill}
One entry of Listaller's visual software lists\subsubsection*{\large{\textbf{Fields}}\normalsize\hspace{1ex}\hfill}
\begin{list}{}{
\settowidth{\tmplength}{\textbf{DescLabel}}
\setlength{\itemindent}{0cm}
\setlength{\listparindent}{0cm}
\setlength{\leftmargin}{\evensidemargin}
\addtolength{\leftmargin}{\tmplength}
\settowidth{\labelsep}{X}
\addtolength{\leftmargin}{\labelsep}
\setlength{\labelwidth}{\tmplength}
}
\label{utilities.TListEntry-AppLabel}
\index{AppLabel}
\item[\textbf{AppLabel}\hfill]
\ifpdf
\begin{flushleft}
\fi
\begin{ttfamily}
public AppLabel: TLabel;\end{ttfamily}

\ifpdf
\end{flushleft}
\fi


\par  \label{utilities.TListEntry-DescLabel}
\index{DescLabel}
\item[\textbf{DescLabel}\hfill]
\ifpdf
\begin{flushleft}
\fi
\begin{ttfamily}
public DescLabel: TLabel;\end{ttfamily}

\ifpdf
\end{flushleft}
\fi


\par  \label{utilities.TListEntry-Vlabel}
\index{Vlabel}
\item[\textbf{Vlabel}\hfill]
\ifpdf
\begin{flushleft}
\fi
\begin{ttfamily}
public Vlabel: TLabel;\end{ttfamily}

\ifpdf
\end{flushleft}
\fi


\par  \label{utilities.TListEntry-MnLabel}
\index{MnLabel}
\item[\textbf{MnLabel}\hfill]
\ifpdf
\begin{flushleft}
\fi
\begin{ttfamily}
public MnLabel: TLabel;\end{ttfamily}

\ifpdf
\end{flushleft}
\fi


\par  \label{utilities.TListEntry-Graphic}
\index{Graphic}
\item[\textbf{Graphic}\hfill]
\ifpdf
\begin{flushleft}
\fi
\begin{ttfamily}
public Graphic: TImage;\end{ttfamily}

\ifpdf
\end{flushleft}
\fi


\par  \label{utilities.TListEntry-UnButton}
\index{UnButton}
\item[\textbf{UnButton}\hfill]
\ifpdf
\begin{flushleft}
\fi
\begin{ttfamily}
public UnButton: TBitBtn;\end{ttfamily}

\ifpdf
\end{flushleft}
\fi


\par  \label{utilities.TListEntry-id}
\index{id}
\item[\textbf{id}\hfill]
\ifpdf
\begin{flushleft}
\fi
\begin{ttfamily}
public id: Integer;\end{ttfamily}

\ifpdf
\end{flushleft}
\fi


\par  \label{utilities.TListEntry-sid}
\index{sid}
\item[\textbf{sid}\hfill]
\ifpdf
\begin{flushleft}
\fi
\begin{ttfamily}
public sid: String;\end{ttfamily}

\ifpdf
\end{flushleft}
\fi


\par  \end{list}
\subsubsection*{\large{\textbf{Methods}}\normalsize\hspace{1ex}\hfill}
\paragraph*{SetImage}\hspace*{\fill}

\label{utilities.TListEntry-SetImage}
\index{SetImage}
\begin{list}{}{
\settowidth{\tmplength}{\textbf{Description}}
\setlength{\itemindent}{0cm}
\setlength{\listparindent}{0cm}
\setlength{\leftmargin}{\evensidemargin}
\addtolength{\leftmargin}{\tmplength}
\settowidth{\labelsep}{X}
\addtolength{\leftmargin}{\labelsep}
\setlength{\labelwidth}{\tmplength}
}
\item[\textbf{Declaration}\hfill]
\ifpdf
\begin{flushleft}
\fi
\begin{ttfamily}
public procedure SetImage(AImage: String);\end{ttfamily}

\ifpdf
\end{flushleft}
\fi

\par
\item[\textbf{Description}]
Set an image for the entry

\end{list}
\paragraph*{SetPositions}\hspace*{\fill}

\label{utilities.TListEntry-SetPositions}
\index{SetPositions}
\begin{list}{}{
\settowidth{\tmplength}{\textbf{Description}}
\setlength{\itemindent}{0cm}
\setlength{\listparindent}{0cm}
\setlength{\leftmargin}{\evensidemargin}
\addtolength{\leftmargin}{\tmplength}
\settowidth{\labelsep}{X}
\addtolength{\leftmargin}{\labelsep}
\setlength{\labelwidth}{\tmplength}
}
\item[\textbf{Declaration}\hfill]
\ifpdf
\begin{flushleft}
\fi
\begin{ttfamily}
public procedure SetPositions;\end{ttfamily}

\ifpdf
\end{flushleft}
\fi

\par
\item[\textbf{Description}]
Correct positions

\end{list}
\paragraph*{Create}\hspace*{\fill}

\label{utilities.TListEntry-Create}
\index{Create}
\begin{list}{}{
\settowidth{\tmplength}{\textbf{Description}}
\setlength{\itemindent}{0cm}
\setlength{\listparindent}{0cm}
\setlength{\leftmargin}{\evensidemargin}
\addtolength{\leftmargin}{\tmplength}
\settowidth{\labelsep}{X}
\addtolength{\leftmargin}{\labelsep}
\setlength{\labelwidth}{\tmplength}
}
\item[\textbf{Declaration}\hfill]
\ifpdf
\begin{flushleft}
\fi
\begin{ttfamily}
public constructor Create(AOwner: TComponent); override;\end{ttfamily}

\ifpdf
\end{flushleft}
\fi

\end{list}
\paragraph*{Destroy}\hspace*{\fill}

\label{utilities.TListEntry-Destroy}
\index{Destroy}
\begin{list}{}{
\settowidth{\tmplength}{\textbf{Description}}
\setlength{\itemindent}{0cm}
\setlength{\listparindent}{0cm}
\setlength{\leftmargin}{\evensidemargin}
\addtolength{\leftmargin}{\tmplength}
\settowidth{\labelsep}{X}
\addtolength{\leftmargin}{\labelsep}
\setlength{\labelwidth}{\tmplength}
}
\item[\textbf{Declaration}\hfill]
\ifpdf
\begin{flushleft}
\fi
\begin{ttfamily}
public destructor Destroy; override;\end{ttfamily}

\ifpdf
\end{flushleft}
\fi

\end{list}
\section{Functions and Procedures}
\ifpdf
\subsection*{\large{\textbf{DeleteModifiers}}\normalsize\hspace{1ex}\hrulefill}
\else
\subsection*{DeleteModifiers}
\fi
\label{utilities-DeleteModifiers}
\index{DeleteModifiers}
\begin{list}{}{
\settowidth{\tmplength}{\textbf{Description}}
\setlength{\itemindent}{0cm}
\setlength{\listparindent}{0cm}
\setlength{\leftmargin}{\evensidemargin}
\addtolength{\leftmargin}{\tmplength}
\settowidth{\labelsep}{X}
\addtolength{\leftmargin}{\labelsep}
\setlength{\labelwidth}{\tmplength}
}
\item[\textbf{Declaration}\hfill]
\ifpdf
\begin{flushleft}
\fi
\begin{ttfamily}
function DeleteModifiers(s: String): String;\end{ttfamily}

\ifpdf
\end{flushleft}
\fi

\par
\item[\textbf{Description}]
Remove modifiers from a string \par
\item[\textbf{Returns}]Cleaned string


\end{list}
\ifpdf
\subsection*{\large{\textbf{SyblToPath}}\normalsize\hspace{1ex}\hrulefill}
\else
\subsection*{SyblToPath}
\fi
\label{utilities-SyblToPath}
\index{SyblToPath}
\begin{list}{}{
\settowidth{\tmplength}{\textbf{Description}}
\setlength{\itemindent}{0cm}
\setlength{\listparindent}{0cm}
\setlength{\leftmargin}{\evensidemargin}
\addtolength{\leftmargin}{\tmplength}
\settowidth{\labelsep}{X}
\addtolength{\leftmargin}{\labelsep}
\setlength{\labelwidth}{\tmplength}
}
\item[\textbf{Declaration}\hfill]
\ifpdf
\begin{flushleft}
\fi
\begin{ttfamily}
function SyblToPath(s: String): String;\end{ttfamily}

\ifpdf
\end{flushleft}
\fi

\par
\item[\textbf{Description}]
Replaces placeholders (like {\$}INSt or {\$}APP) with their current paths \par
\item[\textbf{Returns}]Final path as string


\end{list}
\ifpdf
\subsection*{\large{\textbf{SyblToX}}\normalsize\hspace{1ex}\hrulefill}
\else
\subsection*{SyblToX}
\fi
\label{utilities-SyblToX}
\index{SyblToX}
\begin{list}{}{
\settowidth{\tmplength}{\textbf{Description}}
\setlength{\itemindent}{0cm}
\setlength{\listparindent}{0cm}
\setlength{\leftmargin}{\evensidemargin}
\addtolength{\leftmargin}{\tmplength}
\settowidth{\labelsep}{X}
\addtolength{\leftmargin}{\labelsep}
\setlength{\labelwidth}{\tmplength}
}
\item[\textbf{Declaration}\hfill]
\ifpdf
\begin{flushleft}
\fi
\begin{ttfamily}
function SyblToX(s: String): String;\end{ttfamily}

\ifpdf
\end{flushleft}
\fi

\par
\item[\textbf{Description}]
Removes every symbol or replace it an simpla dummy path \par
\item[\textbf{Returns}]Cleaned string


\end{list}
\ifpdf
\subsection*{\large{\textbf{IsSharedFile}}\normalsize\hspace{1ex}\hrulefill}
\else
\subsection*{IsSharedFile}
\fi
\label{utilities-IsSharedFile}
\index{IsSharedFile}
\begin{list}{}{
\settowidth{\tmplength}{\textbf{Description}}
\setlength{\itemindent}{0cm}
\setlength{\listparindent}{0cm}
\setlength{\leftmargin}{\evensidemargin}
\addtolength{\leftmargin}{\tmplength}
\settowidth{\labelsep}{X}
\addtolength{\leftmargin}{\labelsep}
\setlength{\labelwidth}{\tmplength}
}
\item[\textbf{Declaration}\hfill]
\ifpdf
\begin{flushleft}
\fi
\begin{ttfamily}
function IsSharedFile(s: String): Boolean;\end{ttfamily}

\ifpdf
\end{flushleft}
\fi

\par
\item[\textbf{Description}]
Check if file is a shared one \par
\item[\textbf{Returns}]States as bool


\end{list}
\ifpdf
\subsection*{\large{\textbf{ConfigDir}}\normalsize\hspace{1ex}\hrulefill}
\else
\subsection*{ConfigDir}
\fi
\label{utilities-ConfigDir}
\index{ConfigDir}
\begin{list}{}{
\settowidth{\tmplength}{\textbf{Description}}
\setlength{\itemindent}{0cm}
\setlength{\listparindent}{0cm}
\setlength{\leftmargin}{\evensidemargin}
\addtolength{\leftmargin}{\tmplength}
\settowidth{\labelsep}{X}
\addtolength{\leftmargin}{\labelsep}
\setlength{\labelwidth}{\tmplength}
}
\item[\textbf{Declaration}\hfill]
\ifpdf
\begin{flushleft}
\fi
\begin{ttfamily}
function ConfigDir: String;\end{ttfamily}

\ifpdf
\end{flushleft}
\fi

\par
\item[\textbf{Description}]
Creates Listaller's config dir \par
\item[\textbf{Returns}]Current config dir


\end{list}
\ifpdf
\subsection*{\large{\textbf{CmdResult}}\normalsize\hspace{1ex}\hrulefill}
\else
\subsection*{CmdResult}
\fi
\label{utilities-CmdResult}
\index{CmdResult}
\begin{list}{}{
\settowidth{\tmplength}{\textbf{Description}}
\setlength{\itemindent}{0cm}
\setlength{\listparindent}{0cm}
\setlength{\leftmargin}{\evensidemargin}
\addtolength{\leftmargin}{\tmplength}
\settowidth{\labelsep}{X}
\addtolength{\leftmargin}{\labelsep}
\setlength{\labelwidth}{\tmplength}
}
\item[\textbf{Declaration}\hfill]
\ifpdf
\begin{flushleft}
\fi
\begin{ttfamily}
function CmdResult(cmd:String):String;\end{ttfamily}

\ifpdf
\end{flushleft}
\fi

\par
\item[\textbf{Description}]
Executes a command{-}line application \par
\item[\textbf{Returns}]The application's last output string


\end{list}
\ifpdf
\subsection*{\large{\textbf{CmdResultCode}}\normalsize\hspace{1ex}\hrulefill}
\else
\subsection*{CmdResultCode}
\fi
\label{utilities-CmdResultCode}
\index{CmdResultCode}
\begin{list}{}{
\settowidth{\tmplength}{\textbf{Description}}
\setlength{\itemindent}{0cm}
\setlength{\listparindent}{0cm}
\setlength{\leftmargin}{\evensidemargin}
\addtolength{\leftmargin}{\tmplength}
\settowidth{\labelsep}{X}
\addtolength{\leftmargin}{\labelsep}
\setlength{\labelwidth}{\tmplength}
}
\item[\textbf{Declaration}\hfill]
\ifpdf
\begin{flushleft}
\fi
\begin{ttfamily}
function CmdResultCode(cmd:String):Integer;\end{ttfamily}

\ifpdf
\end{flushleft}
\fi

\par
\item[\textbf{Description}]
Executes a command{-}line application \par
\item[\textbf{Returns}]The application's exit code


\end{list}
\ifpdf
\subsection*{\large{\textbf{CmdResultList}}\normalsize\hspace{1ex}\hrulefill}
\else
\subsection*{CmdResultList}
\fi
\label{utilities-CmdResultList}
\index{CmdResultList}
\begin{list}{}{
\settowidth{\tmplength}{\textbf{Description}}
\setlength{\itemindent}{0cm}
\setlength{\listparindent}{0cm}
\setlength{\leftmargin}{\evensidemargin}
\addtolength{\leftmargin}{\tmplength}
\settowidth{\labelsep}{X}
\addtolength{\leftmargin}{\labelsep}
\setlength{\labelwidth}{\tmplength}
}
\item[\textbf{Declaration}\hfill]
\ifpdf
\begin{flushleft}
\fi
\begin{ttfamily}
procedure CmdResultList(cmd:String;Result: TStringList);\end{ttfamily}

\ifpdf
\end{flushleft}
\fi

\par
\item[\textbf{Description}]
Executes a command{-}line application  \par
\item[\textbf{Parameters}]
\begin{description}
\item[cmd] Command that should be executed
\item[Result] TStringList to recive the output
\end{description}


\end{list}
\ifpdf
\subsection*{\large{\textbf{CmdFinResult}}\normalsize\hspace{1ex}\hrulefill}
\else
\subsection*{CmdFinResult}
\fi
\label{utilities-CmdFinResult}
\index{CmdFinResult}
\begin{list}{}{
\settowidth{\tmplength}{\textbf{Description}}
\setlength{\itemindent}{0cm}
\setlength{\listparindent}{0cm}
\setlength{\leftmargin}{\evensidemargin}
\addtolength{\leftmargin}{\tmplength}
\settowidth{\labelsep}{X}
\addtolength{\leftmargin}{\labelsep}
\setlength{\labelwidth}{\tmplength}
}
\item[\textbf{Declaration}\hfill]
\ifpdf
\begin{flushleft}
\fi
\begin{ttfamily}
function CmdFinResult(cmd: String): String;\end{ttfamily}

\ifpdf
\end{flushleft}
\fi

\par
\item[\textbf{Description}]
Executes a command{-}line application \par
\item[\textbf{Returns}]The final application output string (without breaks in line and invisible codes)


\end{list}
\ifpdf
\subsection*{\large{\textbf{FileCopy}}\normalsize\hspace{1ex}\hrulefill}
\else
\subsection*{FileCopy}
\fi
\label{utilities-FileCopy}
\index{FileCopy}
\begin{list}{}{
\settowidth{\tmplength}{\textbf{Description}}
\setlength{\itemindent}{0cm}
\setlength{\listparindent}{0cm}
\setlength{\leftmargin}{\evensidemargin}
\addtolength{\leftmargin}{\tmplength}
\settowidth{\labelsep}{X}
\addtolength{\leftmargin}{\labelsep}
\setlength{\labelwidth}{\tmplength}
}
\item[\textbf{Declaration}\hfill]
\ifpdf
\begin{flushleft}
\fi
\begin{ttfamily}
function FileCopy(source,dest: String): Boolean;\end{ttfamily}

\ifpdf
\end{flushleft}
\fi

\par
\item[\textbf{Description}]
Advanced file copy method \par
\item[\textbf{Returns}]Success of the command


\end{list}
\ifpdf
\subsection*{\large{\textbf{Gtk2LoadStockPixmap}}\normalsize\hspace{1ex}\hrulefill}
\else
\subsection*{Gtk2LoadStockPixmap}
\fi
\label{utilities-Gtk2LoadStockPixmap}
\index{Gtk2LoadStockPixmap}
\begin{list}{}{
\settowidth{\tmplength}{\textbf{Description}}
\setlength{\itemindent}{0cm}
\setlength{\listparindent}{0cm}
\setlength{\leftmargin}{\evensidemargin}
\addtolength{\leftmargin}{\tmplength}
\settowidth{\labelsep}{X}
\addtolength{\leftmargin}{\labelsep}
\setlength{\labelwidth}{\tmplength}
}
\item[\textbf{Declaration}\hfill]
\ifpdf
\begin{flushleft}
\fi
\begin{ttfamily}
function Gtk2LoadStockPixmap(StockId: PChar; IconSize: integer): HBitmap;\end{ttfamily}

\ifpdf
\end{flushleft}
\fi

\par
\item[\textbf{Description}]
Loads an GTK2 stock icon \par
\item[\textbf{Returns}]Handle to th bitmap


\end{list}
\ifpdf
\subsection*{\large{\textbf{GetServerName}}\normalsize\hspace{1ex}\hrulefill}
\else
\subsection*{GetServerName}
\fi
\label{utilities-GetServerName}
\index{GetServerName}
\begin{list}{}{
\settowidth{\tmplength}{\textbf{Description}}
\setlength{\itemindent}{0cm}
\setlength{\listparindent}{0cm}
\setlength{\leftmargin}{\evensidemargin}
\addtolength{\leftmargin}{\tmplength}
\settowidth{\labelsep}{X}
\addtolength{\leftmargin}{\labelsep}
\setlength{\labelwidth}{\tmplength}
}
\item[\textbf{Declaration}\hfill]
\ifpdf
\begin{flushleft}
\fi
\begin{ttfamily}
function GetServerName(url:string):string;\end{ttfamily}

\ifpdf
\end{flushleft}
\fi

\par
\item[\textbf{Description}]
Get server name from an url \par
\item[\textbf{Returns}]The server name


\end{list}
\ifpdf
\subsection*{\large{\textbf{GetServerPath}}\normalsize\hspace{1ex}\hrulefill}
\else
\subsection*{GetServerPath}
\fi
\label{utilities-GetServerPath}
\index{GetServerPath}
\begin{list}{}{
\settowidth{\tmplength}{\textbf{Description}}
\setlength{\itemindent}{0cm}
\setlength{\listparindent}{0cm}
\setlength{\leftmargin}{\evensidemargin}
\addtolength{\leftmargin}{\tmplength}
\settowidth{\labelsep}{X}
\addtolength{\leftmargin}{\labelsep}
\setlength{\labelwidth}{\tmplength}
}
\item[\textbf{Declaration}\hfill]
\ifpdf
\begin{flushleft}
\fi
\begin{ttfamily}
function GetServerPath(url:string):string;\end{ttfamily}

\ifpdf
\end{flushleft}
\fi

\par
\item[\textbf{Description}]
Path on an server (from an url) \par
\item[\textbf{Returns}]The path on the server


\end{list}
\ifpdf
\subsection*{\large{\textbf{IsInList}}\normalsize\hspace{1ex}\hrulefill}
\else
\subsection*{IsInList}
\fi
\label{utilities-IsInList}
\index{IsInList}
\begin{list}{}{
\settowidth{\tmplength}{\textbf{Description}}
\setlength{\itemindent}{0cm}
\setlength{\listparindent}{0cm}
\setlength{\leftmargin}{\evensidemargin}
\addtolength{\leftmargin}{\tmplength}
\settowidth{\labelsep}{X}
\addtolength{\leftmargin}{\labelsep}
\setlength{\labelwidth}{\tmplength}
}
\item[\textbf{Declaration}\hfill]
\ifpdf
\begin{flushleft}
\fi
\begin{ttfamily}
function IsInList(nm: String;list: TStringList): Boolean;\end{ttfamily}

\ifpdf
\end{flushleft}
\fi

\par
\item[\textbf{Description}]
Fast check if entry is in a list  \par
\item[\textbf{Parameters}]
\begin{description}
\item[nm] Name of the entry that has to be checked
\item[list] The string list that has to be searched
\end{description}


\end{list}
\section{Constants}
\ifpdf
\subsection*{\large{\textbf{LiVersion}}\normalsize\hspace{1ex}\hrulefill}
\else
\subsection*{LiVersion}
\fi
\label{utilities-LiVersion}
\index{LiVersion}
\begin{list}{}{
\settowidth{\tmplength}{\textbf{Description}}
\setlength{\itemindent}{0cm}
\setlength{\listparindent}{0cm}
\setlength{\leftmargin}{\evensidemargin}
\addtolength{\leftmargin}{\tmplength}
\settowidth{\labelsep}{X}
\addtolength{\leftmargin}{\labelsep}
\setlength{\labelwidth}{\tmplength}
}
\item[\textbf{Declaration}\hfill]
\ifpdf
\begin{flushleft}
\fi
\begin{ttfamily}
LiVersion='0.1.93a';\end{ttfamily}

\ifpdf
\end{flushleft}
\fi

\par
\item[\textbf{Description}]
Version of the Listaller toolset

\end{list}
\section{Variables}
\ifpdf
\subsection*{\large{\textbf{Testmode}}\normalsize\hspace{1ex}\hrulefill}
\else
\subsection*{Testmode}
\fi
\label{utilities-Testmode}
\index{Testmode}
\begin{list}{}{
\settowidth{\tmplength}{\textbf{Description}}
\setlength{\itemindent}{0cm}
\setlength{\listparindent}{0cm}
\setlength{\leftmargin}{\evensidemargin}
\addtolength{\leftmargin}{\tmplength}
\settowidth{\labelsep}{X}
\addtolength{\leftmargin}{\labelsep}
\setlength{\labelwidth}{\tmplength}
}
\item[\textbf{Declaration}\hfill]
\ifpdf
\begin{flushleft}
\fi
\begin{ttfamily}
Testmode: Boolean=false;\end{ttfamily}

\ifpdf
\end{flushleft}
\fi

\par
\item[\textbf{Description}]
True if Listaller is in testmode

\end{list}
\chapter{Unit xtypefm}
\label{xtypefm}
\index{xtypefm}
\section{Description}
This unit provides an formular that lets the user choose between program runlevels (root / not as root / in app{-}testmode)
\section{uses}
\begin{itemize}
\item \begin{ttfamily}Classes\end{ttfamily}\item \begin{ttfamily}SysUtils\end{ttfamily}\item \begin{ttfamily}LResources\end{ttfamily}\item \begin{ttfamily}Forms\end{ttfamily}\item \begin{ttfamily}Controls\end{ttfamily}\item \begin{ttfamily}Graphics\end{ttfamily}\item \begin{ttfamily}Dialogs\end{ttfamily}\item \begin{ttfamily}StdCtrls\end{ttfamily}\item \begin{ttfamily}process\end{ttfamily}\item \begin{ttfamily}ExtCtrls\end{ttfamily}\item \begin{ttfamily}utilities\end{ttfamily}(\ref{utilities})\item \begin{ttfamily}LCLType\end{ttfamily}\item \begin{ttfamily}Buttons\end{ttfamily}\item \begin{ttfamily}distri\end{ttfamily}(\ref{distri})\item \begin{ttfamily}gtk2\end{ttfamily}\item \begin{ttfamily}trstrings\end{ttfamily}(\ref{trstrings})\end{itemize}
\section{Overview}
\begin{description}
\item[\texttt{\begin{ttfamily}TimdFrm\end{ttfamily} Class}]
\end{description}
\section{Classes, Interfaces, Objects and Records}
\ifpdf
\subsection*{\large{\textbf{TimdFrm Class}}\normalsize\hspace{1ex}\hrulefill}
\else
\subsection*{TimdFrm Class}
\fi
\label{xtypefm.TimdFrm}
\index{TimdFrm}
\subsubsection*{\large{\textbf{Hierarchy}}\normalsize\hspace{1ex}\hfill}
TimdFrm {$>$} TForm
%%%%Description
\subsubsection*{\large{\textbf{Fields}}\normalsize\hspace{1ex}\hfill}
\begin{list}{}{
\settowidth{\tmplength}{\textbf{btnInstallAll}}
\setlength{\itemindent}{0cm}
\setlength{\listparindent}{0cm}
\setlength{\leftmargin}{\evensidemargin}
\addtolength{\leftmargin}{\tmplength}
\settowidth{\labelsep}{X}
\addtolength{\leftmargin}{\labelsep}
\setlength{\labelwidth}{\tmplength}
}
\label{xtypefm.TimdFrm-btnTest}
\index{btnTest}
\item[\textbf{btnTest}\hfill]
\ifpdf
\begin{flushleft}
\fi
\begin{ttfamily}
public btnTest: TBitBtn;\end{ttfamily}

\ifpdf
\end{flushleft}
\fi


\par  \label{xtypefm.TimdFrm-btnHome}
\index{btnHome}
\item[\textbf{btnHome}\hfill]
\ifpdf
\begin{flushleft}
\fi
\begin{ttfamily}
public btnHome: TBitBtn;\end{ttfamily}

\ifpdf
\end{flushleft}
\fi


\par  \label{xtypefm.TimdFrm-btnInstallAll}
\index{btnInstallAll}
\item[\textbf{btnInstallAll}\hfill]
\ifpdf
\begin{flushleft}
\fi
\begin{ttfamily}
public btnInstallAll: TBitBtn;\end{ttfamily}

\ifpdf
\end{flushleft}
\fi


\par  \label{xtypefm.TimdFrm-Image1}
\index{Image1}
\item[\textbf{Image1}\hfill]
\ifpdf
\begin{flushleft}
\fi
\begin{ttfamily}
public Image1: TImage;\end{ttfamily}

\ifpdf
\end{flushleft}
\fi


\par  \label{xtypefm.TimdFrm-Image2}
\index{Image2}
\item[\textbf{Image2}\hfill]
\ifpdf
\begin{flushleft}
\fi
\begin{ttfamily}
public Image2: TImage;\end{ttfamily}

\ifpdf
\end{flushleft}
\fi


\par  \label{xtypefm.TimdFrm-Label1}
\index{Label1}
\item[\textbf{Label1}\hfill]
\ifpdf
\begin{flushleft}
\fi
\begin{ttfamily}
public Label1: TLabel;\end{ttfamily}

\ifpdf
\end{flushleft}
\fi


\par  \label{xtypefm.TimdFrm-Label13}
\index{Label13}
\item[\textbf{Label13}\hfill]
\ifpdf
\begin{flushleft}
\fi
\begin{ttfamily}
public Label13: TLabel;\end{ttfamily}

\ifpdf
\end{flushleft}
\fi


\par  \label{xtypefm.TimdFrm-Label2}
\index{Label2}
\item[\textbf{Label2}\hfill]
\ifpdf
\begin{flushleft}
\fi
\begin{ttfamily}
public Label2: TLabel;\end{ttfamily}

\ifpdf
\end{flushleft}
\fi


\par  \end{list}
\subsubsection*{\large{\textbf{Methods}}\normalsize\hspace{1ex}\hfill}
\paragraph*{btnHomeClick}\hspace*{\fill}

\label{xtypefm.TimdFrm-btnHomeClick}
\index{btnHomeClick}
\begin{list}{}{
\settowidth{\tmplength}{\textbf{Description}}
\setlength{\itemindent}{0cm}
\setlength{\listparindent}{0cm}
\setlength{\leftmargin}{\evensidemargin}
\addtolength{\leftmargin}{\tmplength}
\settowidth{\labelsep}{X}
\addtolength{\leftmargin}{\labelsep}
\setlength{\labelwidth}{\tmplength}
}
\item[\textbf{Declaration}\hfill]
\ifpdf
\begin{flushleft}
\fi
\begin{ttfamily}
public procedure btnHomeClick(Sender: TObject);\end{ttfamily}

\ifpdf
\end{flushleft}
\fi

\end{list}
\paragraph*{btnInstallAllClick}\hspace*{\fill}

\label{xtypefm.TimdFrm-btnInstallAllClick}
\index{btnInstallAllClick}
\begin{list}{}{
\settowidth{\tmplength}{\textbf{Description}}
\setlength{\itemindent}{0cm}
\setlength{\listparindent}{0cm}
\setlength{\leftmargin}{\evensidemargin}
\addtolength{\leftmargin}{\tmplength}
\settowidth{\labelsep}{X}
\addtolength{\leftmargin}{\labelsep}
\setlength{\labelwidth}{\tmplength}
}
\item[\textbf{Declaration}\hfill]
\ifpdf
\begin{flushleft}
\fi
\begin{ttfamily}
public procedure btnInstallAllClick(Sender: TObject);\end{ttfamily}

\ifpdf
\end{flushleft}
\fi

\end{list}
\paragraph*{btnTestClick}\hspace*{\fill}

\label{xtypefm.TimdFrm-btnTestClick}
\index{btnTestClick}
\begin{list}{}{
\settowidth{\tmplength}{\textbf{Description}}
\setlength{\itemindent}{0cm}
\setlength{\listparindent}{0cm}
\setlength{\leftmargin}{\evensidemargin}
\addtolength{\leftmargin}{\tmplength}
\settowidth{\labelsep}{X}
\addtolength{\leftmargin}{\labelsep}
\setlength{\labelwidth}{\tmplength}
}
\item[\textbf{Declaration}\hfill]
\ifpdf
\begin{flushleft}
\fi
\begin{ttfamily}
public procedure btnTestClick(Sender: TObject);\end{ttfamily}

\ifpdf
\end{flushleft}
\fi

\end{list}
\paragraph*{FormClose}\hspace*{\fill}

\label{xtypefm.TimdFrm-FormClose}
\index{FormClose}
\begin{list}{}{
\settowidth{\tmplength}{\textbf{Description}}
\setlength{\itemindent}{0cm}
\setlength{\listparindent}{0cm}
\setlength{\leftmargin}{\evensidemargin}
\addtolength{\leftmargin}{\tmplength}
\settowidth{\labelsep}{X}
\addtolength{\leftmargin}{\labelsep}
\setlength{\labelwidth}{\tmplength}
}
\item[\textbf{Declaration}\hfill]
\ifpdf
\begin{flushleft}
\fi
\begin{ttfamily}
public procedure FormClose(Sender: TObject; var CloseAction: TCloseAction);\end{ttfamily}

\ifpdf
\end{flushleft}
\fi

\end{list}
\paragraph*{FormCreate}\hspace*{\fill}

\label{xtypefm.TimdFrm-FormCreate}
\index{FormCreate}
\begin{list}{}{
\settowidth{\tmplength}{\textbf{Description}}
\setlength{\itemindent}{0cm}
\setlength{\listparindent}{0cm}
\setlength{\leftmargin}{\evensidemargin}
\addtolength{\leftmargin}{\tmplength}
\settowidth{\labelsep}{X}
\addtolength{\leftmargin}{\labelsep}
\setlength{\labelwidth}{\tmplength}
}
\item[\textbf{Declaration}\hfill]
\ifpdf
\begin{flushleft}
\fi
\begin{ttfamily}
public procedure FormCreate(Sender: TObject);\end{ttfamily}

\ifpdf
\end{flushleft}
\fi

\end{list}
\paragraph*{FormShow}\hspace*{\fill}

\label{xtypefm.TimdFrm-FormShow}
\index{FormShow}
\begin{list}{}{
\settowidth{\tmplength}{\textbf{Description}}
\setlength{\itemindent}{0cm}
\setlength{\listparindent}{0cm}
\setlength{\leftmargin}{\evensidemargin}
\addtolength{\leftmargin}{\tmplength}
\settowidth{\labelsep}{X}
\addtolength{\leftmargin}{\labelsep}
\setlength{\labelwidth}{\tmplength}
}
\item[\textbf{Declaration}\hfill]
\ifpdf
\begin{flushleft}
\fi
\begin{ttfamily}
public procedure FormShow(Sender: TObject);\end{ttfamily}

\ifpdf
\end{flushleft}
\fi

\end{list}
\section{Variables}
\ifpdf
\subsection*{\large{\textbf{imdFrm}}\normalsize\hspace{1ex}\hrulefill}
\else
\subsection*{imdFrm}
\fi
\label{xtypefm-imdFrm}
\index{imdFrm}
\begin{list}{}{
\settowidth{\tmplength}{\textbf{Description}}
\setlength{\itemindent}{0cm}
\setlength{\listparindent}{0cm}
\setlength{\leftmargin}{\evensidemargin}
\addtolength{\leftmargin}{\tmplength}
\settowidth{\labelsep}{X}
\addtolength{\leftmargin}{\labelsep}
\setlength{\labelwidth}{\tmplength}
}
\item[\textbf{Declaration}\hfill]
\ifpdf
\begin{flushleft}
\fi
\begin{ttfamily}
imdFrm: TimdFrm;\end{ttfamily}

\ifpdf
\end{flushleft}
\fi

\end{list}
\end{document}
